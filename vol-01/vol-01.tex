\documentclass{book}

% Packages
\usepackage{blindtext}
\usepackage{csquotes}
\usepackage{verbatim}

% Font Setup
\usepackage[T1]{fontenc}
\usepackage{fontspec, newunicodechar}
\setmainfont{Crimson}
%\newfontfamily\myenglishfont{Crimson}
%\newfontfamily\myjapanesefont{ipamp.ttf}

% CJK Setup
\usepackage{xeCJK}
%\usepackage{ruby}
%\renewcommand{\rubysize}{0.5}
%\renewcommand{\rubysep}{-0.2ex}  
\setCJKmainfont{ipamp.ttf}
\setCJKsansfont{ipagp.ttf}

% Fallback font for missing ○ character
\newfontfamily{\fallbackfont}{Unifont}
\DeclareTextFontCommand{\textfallback}{\fallbackfont}
\newunicodechar{○}{\textfallback{○}}



\begin{document}
\chapter*{I}
I am a cat. As yet I have no name. I've no idea where I was born. All I
remember is that I was miaowing in a dampish dark place when, for the
first time, I saw a human being. This human being, I heard afterwards,
was a member of the most ferocious human species; a shosei, one of those
students who, in return for board and lodging, perform small chores
about the house. I hear that, on occasion, this species catches, boils,
and eats us. However as at that time I lacked all knowledge of such
creatures, I did not feel particularly frightened. I simply felt myself
floating in the air as I was lifted up lightly on his palm. When I
accustomed myself to that position, I looked at his face. This must have
been the very first time that ever I set eyes on a human being. The
impression of oddity, which I then received, still remains today. First
of all, the face that should be decorated with hair is as bald as a
kettle. Since that day I have met many a cat but never have I come
across such deformity. The center of the face protrudes excessively and
sometimes, from the holes in that protuberance, smoke comes out in
little puffs. I was originally somewhat troubled by such exhalations for
they made me choke, but I learnt only recently that it was the smoke of
burnt tobacco which humans like to breathe.

For a little while I sat comfortably in that creature's palm, but things
soon developed at a tremendous speed. I could not tell whether the
shosei was in movement or whether it was only I that moved; but anyway I
began to grow quite giddy, to feel sick. And just as I was thinking that
the giddiness would kill me, I heard a thud and saw a million stars.
Thus far I can remember but, however hard I try, I cannot recollect
anything thereafter.

When I came to myself, the creature had gone. I had at one time had a
basketful of brothers, but now not one could be seen. Even my precious
mother had disappeared. Moreover I now found myself in a painfully
bright place most unlike that nook where once I'd sheltered. It was in
fact so bright that I could hardly keep my eyes open. Sure that there
was something wrong, I began to crawl about. Which proved painful. I had
been snatched away from softest straw only to be pitched with violence
into a prickly clump of bamboo grass.

After a struggle, I managed to scramble clear of the clump and emerged
to find a wide pond stretching beyond it. I sat at the edge of the pond
and wondered what to do. No helpful thought occurred. After a while it
struck me that, if I cried, perhaps the shosei might come back to fetch
me. I tried some feeble mewing, but no one came. Soon a light wind blew
across the pond and it began to grow dark. I felt extremely hungry. I
wanted to cry, but I was too weak to do so. There was nothing to be
done. However, having decided that I simply must find food, I turned,
very, very slowly, left around the pond. It was extremely painful going.
Nevertheless, I persevered and crawled on somehow until at long last I
reached a place where my nose picked up some trace of human presence. I
slipped into a property through a gap in a broken bamboo fence, thinking
that something might turn up once I got inside. It was sheer chance; if
the bamboo fence had not been broken just at that point, I might have
starved to death at the roadside. I realize now how true the adage is
that what is to be will be. To this very day that gap has served as my
shortcut to the neighbor's tortoiseshell.

Well, though I had managed to creep into the property, I had no idea
what to do next. Soon it got really dark. I was hungry, it was cold and
rain began to fall. I could not afford to lose any more time. I had no
choice but to struggle toward a place which seemed, since brighter,
warmer. I did not know it then, but I was in fact already inside the
house where I now had a chance to observe further specimens of
humankind. The first one that I met was O-san, the servant-woman, one of
a species yet more savage than the shosei. No sooner had she seen me
than she grabbed me by the scruff of the neck and flung me out of the
house. Accepting that I had no hope, I lay stone-still, my eyes shut
tight and trusting to Providence. But the hunger and the cold were more
than I could bear. Seizing a moment when O-san had relaxed her watch, I
crawled up once again to flop into the kitchen. I was soon flung out
again. I crawled up yet again, only to be flung out yet again. I
remember that the process was several times repeated. Ever since that
time, I have been utterly disgusted with this O-san person. The other
day I managed at long last to rid myself of my sense of grievance, for I
squared accounts by stealing her dinner of mackerel-pike. As I was about
to be flung out for the last time, the master of the house appeared,
complaining of the noise and demanding an explanation. The servant
lifted me up, turned my face to the master and said, ``This little stray
kitten is being a nuisance. I keep putting it out and it keeps crawling
back into the kitchen.'' The master briefly studied my face, twisting
the black hairs under his nostrils. Then, ``In that case, let it stay,''
he said; and turned and went inside. The master seemed to be a person of
few words. The servant resentfully threw me down in the kitchen. And it
was thus that I came to make this house my dwelling.

My master seldom comes face-to-face with me. I hear he is a
schoolteacher. As soon as he comes home from school, he shuts himself up
in the study for the rest of the day; and he seldom emerges. The others
in the house think that he is terribly hard-working. He himself pretends
to be hard-working. But actually he works less hard than any of them
think. Sometimes I tiptoe to his study for a peep and find him taking a
snooze. Occasionally his mouth is drooling onto some book he has begun
to read. He has a weak stomach and his skin is of a pale yellowish
color, inelastic and lacking in vitality. Nevertheless he is an enormous
gormandiser. After eating a great deal, he takes some taka-diastase for
his stomach and, after that, he opens a book. When he has read a few
pages, he becomes sleepy. He drools onto the book. This is the routine
religiously observed each evening. There are times when even I, a mere
cat, can put two thoughts together. ``Teachers have it easy. If you are
born a human, it's best to become a teacher. For if it's possible to
sleep this much and still to be a teacher, why, even a cat could
teach.'' However, according to the master, there's nothing harder than a
teacher's life and every time his friends come round to see him, he
grumbles on and on.

During my early days in the house, I was terribly unpopular with
everyone except the master. Everywhere I was unwelcome, and no one would
have anything to do with me. The fact that nobody, even to this day, has
given me a name indicates quite clearly how very little they have
thought about me. Resigned, I try to spend as much of my time as
possible with the master, the man who had taken me in. In the morning,
while he reads the newspaper, I jump to curl up on his knees. Throughout
his afternoon siesta, I sit upon his back. This is not because I have
any particular fondness for the master, but because I have no other
choice; no one else to turn to. Additionally, and in the light of other
experiments, I have decided to sleep on the boiled-rice container, which
stays warm through the morning, on the quilted foot-warmer during the
evening, and out on the veranda when it is fine. But what I find
especially agreeable is to creep into the children's bed and snuggle
down between them. There are two children, one of five and one of three:
they sleep in their own room, sharing a bed. I can always find a space
between their bodies, and I manage somehow to squeeze myself quietly in.
But if, by great ill-luck, one of the children wakes, then I am in
trouble. For the children have nasty natures, especially the younger
one. They start to cry out noisily, regardless of the time, even in the
middle of the night, shouting, ``Here's the cat!'' Then invariably the
neurotic dyspeptic in the next room wakes and comes rushing in. Why,
only the other day, my master beat my backside black and blue with a
wooden ruler.

Living as I do with human beings, the more that I observe them, the more
I am forced to conclude that they are selfish. Especially those
children. I find my bedmates utterly unspeakable. When the fancy takes
them, they hang me upside-down, they stuff my face into a paper-bag,
they fling me about, they ram me into the kitchen range. Furthermore, if
I do commit so much as the smallest mischief, the entire household
unites to chase me around and persecute me. The other day when I
happened to be sharpening my claws on some straw floor-matting, the
mistress of the house became so unreasonably incensed that now it is
only with the greatest reluctance that she'll even let me enter a matted
room. Though I'm shivering on the wooden floor in the kitchen,
heartlessly she remains indifferent. Miss Blanche, the white cat who
lives opposite and whom I much admire, tells me whenever I see her that
there is no living creature quite so heartless as a human. The other
day, she gave birth to four beautiful kittens. But three days later, the
shosei of her house removed all four and tossed them away into the
backyard pond. Miss Blanche, having given through her tears a complete
account of this event, assured me that, to maintain our own parental
love and to enjoy our beautiful family life, we, the cat-race, must
engage in total war upon all humans. We have no choice but to
exterminate them. I think it is a very reasonable proposition. And the
three-colored tomcat living next door is especially indignant that human
beings do not understand the nature of proprietary rights. Among our
kind it is taken for granted that he who first finds something, be it
the head of a dried sardine or a gray mullet's navel, acquires thereby
the right to eat it. And if this rule be flouted, one may well resort to
violence. But human beings do not seem to understand the rights of
property. Every time we come on something good to eat, invariably they
descend and take it from us. Relying on their naked strength, they
coolly rob us of things which are rightly ours to eat. Miss Blanche
lives in the house of a military man, and the tomcat's master is a
lawyer. But since I live in a teacher's house, I take matters of this
sort rather more lightly than they. I feel that life is not unreasonable
so long as one can scrape along from day to day. For surely even human
beings will not flourish forever. I think it best to wait in patience
for the Day of the Cats.

Talking of selfishness reminds me that my master once made a fool of
himself by reason of this failing. I'll tell you all about it. First you
must understand that this master of mine lacks the talent to be more
than average at anything at all; but nonetheless he can't refrain from
trying his hand at everything and anything. He's always writing haiku
and submitting them to Cuckoo; he sends off new-style poetry to Morning
Star; he has a shot at English prose peppered with gross mistakes; he
develops a passion for archery; he takes lessons in chanting No
play-texts; and sometimes he devotes himself to making hideous noises
with a violin. But I am sorry to say that none of these activities has
led to anything whatsoever. Yet, though he is dyspeptic, he gets
terribly keen once he has embarked upon a project. He once got himself
nicknamed ``The Maestro of the Water-closet'' through chanting in the
lavatory, but he remains entirely unconcerned and can still be heard
there chanting ``I am Taira-no-Munemori.'' We all say, ``There goes
Munemori,'' and titter at his antics. I do not know why it happened, but
one fine day (a payday roughly four weeks after I'd taken up residence)
this master of mine came hurrying home with a large parcel under his
arm. I wondered what he'd bought. It turned out that he'd purchased
watercolor paints, brushes, and some special ``Whatman'' paper. It
looked to me as if haiku-writing, and mediaeval chanting were going to
be abandoned in favor of watercolor painting. Sure enough, from the next
day on and every day for some long time, he did nothing but paint
pictures in his study. He even went without his afternoon siestas.
However, no one could tell what he had painted by looking at the result.
Possibly he himself thought little of his work; for, one day when his
friend who specializes in matters of aesthetics came to visit him, I
heard the following conversation.

``Do you know it's quite difficult? When one sees someone else painting,
it looks easy enough; but not till one takes a brush oneself, does one
realize just how difficult it is.'' So said my noble master, and it was
true enough.

His friend, looking at my master over his gold-rimmed spectacles,
observed, ``It's only natural that one cannot paint particularly well
the moment one starts. Besides, one cannot paint a picture indoors by
force of the imagination only. The Italian Master, Andrea del Sarto,
remarked that if you want to paint a picture, always depict nature as
she is. In the sky, there are stars. On earth, there are sparkling dews.
Birds are flying. Animals are running. In a pond there are goldfish. On
an old tree one sees winter crows. Nature herself is one vast living
picture. D'you understand? If you want to paint a picturesque picture,
why not try some preliminary sketching?''

``Oh, so Andrea del Sarto said that? I didn't know that at all. Come to
think of it, it's quite true. Indeed, it's very true.'' The master was
unduly impressed. I saw a mocking smile behind the gold-rimmed glasses.

The next day when, as always, I was having a pleasant nap on the
veranda, the master emerged from his study (an act unusual in itself)
and began behind my back to busy himself with something. At this point I
happened to wake up and, wondering what he was up to, opened my eyes
just one slit the tenth of an inch. And there he was, fairly killing
himself at being Andrea del Sarto. I could not help but laugh. He's
starting to sketch me just because he's had his leg pulled by a friend.
I have already slept enough, and I'm itching to yawn. But seeing my
master sketching away so earnestly, I hadn't the heart to move: so I
bore it all with resignation. Having drawn my outline, he's started
painting the face. I confess that, considering cats as works of art, I'm
far from being a collector's piece. I certainly do not think that my
figure, my fur, or my features are superior to those of other cats. But
however ugly I may be, there's no conceivable resemblance between myself
and that queer thing which my master is creating. First of all, the
coloring is wrong. My fur, like that of a Persian, bears tortoiseshell
markings on a ground of a yellowish pale grey. It is a fact beyond all
argument. Yet the color which my master has employed is neither yellow
nor black; neither grey nor brown; nor is it any mixture of those four
distinctive colors. All one can say is that the color used is a sort of
color. Furthermore, and very oddly, my face lacks eyes. The lack might
be excused on the grounds that the sketch is a sketch of a sleeping cat;
but, all the same, since one cannot find even a hint of an eye's
location, it is not all clear whether the sketch is of a sleeping cat or
of a blind cat. Secretly I thought to myself that this would never do,
even for Andrea del Sarto. However, I could not help being struck with
admiration for my master's grim determination. Had it been solely up to
me, I would gladly have maintained my pose for him, but Nature has now
been calling for some time. The muscles in my body are getting pins and
needles. When the tingling reached a point where I couldn't stand it
another minute, I was obliged to claim my liberty. I stretched my front
paws far out in front of me, pushed my neck out low and yawned
cavernously. Having done all that, there's no further point in trying to
stay still. My master's sketch is spoilt anyway, so I might as well pad
round to the backyard and do my business. Moved by these thoughts, I
start to crawl sluggishly away. Immediately, ``You fool'' came shouted
in my master's voice, a mixture of wrath and disappointment, out of the
inner room. He has a fixed habit of saying, ``You fool'' whenever he
curses anyone. He cannot help it since he knows no other swear words.
But I thought it rather impertinent of him thus unjustifiably to call me
``a fool.'' After all, I had been very patient up to this point. Of
course, had it been his custom to show even the smallest pleasure
whenever I jump on his back, I would have tamely endured his
imprecations: but it is a bit thick to be called ``a fool'' by someone
who has never once with good grace done me a kindness just because I get
up to go and urinate. The prime fact is that all humans are puffed up by
their extreme self-satisfaction with their own brute power. Unless some
creatures more powerful than humans arrive on earth to bully them,
there's just no knowing to what dire lengths their fool presumptuousness
will eventually carry them.

One could put up with this degree of selfishness, but I once heard a
report concerning the unworthiness of humans, which is several times
more ugly and deplorable.

At the back of my house there is a small tea-plantation, perhaps some
six yards square. Though certainly not large, it is a neat and
pleasantly sunny spot. It is my custom to go there whenever my morale
needs strengthening; when, for instance, the children are making so much
noise that I cannot doze in peace, or when boredom has disrupted my
digestion. One day, a day of Indian summer, at about two o'clock in the
afternoon, I woke from a pleasant after-luncheon nap and strolled out to
this tea-plantation by way of taking exercise. Sniffing, one after
another, at the roots of the tea plants, I came to the cypress fence at
the western end; and there I saw an enormous cat fast asleep on a bed of
withered chrysanthemums, which his weight had flattened down. He did not
seem to notice my approach. Perhaps he noticed but did not care. Anyway,
there he was, stretched out at full length and snoring loudly. I was
amazed at the daring courage that permitted him, a trespasser, to sleep
so unconcernedly in someone else's garden. He was a pure black cat. The
sun of earliest afternoon was pouring its most brilliant rays upon him,
and it seemed as if invisible flames were blazing out from his glossy
fur. He had a magnificent physique; the physique, one might say, of the
Emperor of Catdom. He was easily twice my size. Filled with admiration
and curiosity, I quite forgot myself. I stood stock-still, entranced,
all eyes in front of him. The quiet zephyrs of that Indian summer set
gently nodding a branch of Sultan's Parasol, which showed above the
cypress fence, and a few leaves pattered down upon the couch of crushed
chrysanthemums. The Emperor suddenly opened his huge round eyes. I
remember that moment to this day. His eyes gleamed far more beautifully
than that dull amber stuff which humans so inordinately value. He lay
dead still. Focussing the piercing light that shone from his eyes'
interior upon my dwarfish forehead, he remarked, ``And who the hell are
you?''

I thought his turn of phrase a shade inelegant for an Emperor, but
because the voice was deep and filled with a power that could suppress a
bulldog, I remained dumb-struck with pure awe. Reflecting, however, that
I might get into trouble if I failed to exchange civilities, I answered
frigidly, with a false sang froid as cold as I could make it, ``I, sir,
am a cat. I have as yet no name.'' My heart at that moment was beating a
great deal faster than usual.

In a tone of enormous scorn, the Emperor observed, ``You\ldots{} a cat?
Well, I'm damned. Anyway, where the devil do you hang out?'' I thought
this cat excessively blunt-spoken.

``I live here, in the teacher's house.''

``Huh, I thought as much. 'Orrible scrawny aren't you.'' Like a true
Emperor, he spoke with great vehemence.

Judged by his manner of speech, he could not be a cat of respectable
background. On the other hand, he seemed well fed and positively
prosperous, almost obese, in his oily glossiness. I had to ask him ``And
you, who on earth are you?''

``Me? I'm Rickshaw Blacky.'' He gave his answer with spirit and some
pride: for Rickshaw Blacky is well-known in the neighborhood as a real
rough customer. As one would expect of those brought up in a
rickshaw-garage, he's tough but quite uneducated. Hence very few of us
mix with him, and it is our common policy to ``keep him at a respectful
distance.'' Consequently when I heard his name, I felt a trifle jittery
and uneasy but at the same time a little disdainful of him. Accordingly,
and in order to establish just how illiterate he was, I pursued the
conversation by enquiring, ``Which do you think is superior, a
rickshaw-owner or a teacher?''

``Why, a rickshaw-owner, of course. He's the stronger. Just look at your
master, almost skin and bones.''

``You, being the cat of a rickshaw-owner, naturally look very tough. I
can see that one eats well at your establishment.''

``Ah well, as far as I'm concerned, I never want for decent grub
wherever I go. You too, instead of creeping around in a tea-plantation,
why not follow along with me? Within a month, you'd get so fat nobody'd
recognize you.''

``In due course I might come and ask to join you. But it seems that the
teacher's house is larger than your boss's.''

``You dimwit! A house, however big it is, won't help fill an empty
belly.'' He looked quite huffed. Savagely twitching his ears, ears as
sharp as slant-sliced stems of the solid bamboo, he took off rowdily.

This was how I first made the acquaintance of Rickshaw Blacky, and since
that day I've run across him many times. Whenever we meet he talks big,
as might be expected from a rickshaw-owner's cat; but that deplorable
incident which I mentioned earlier was a tale he told me.

One day Blacky and I were lying as usual, sunning ourselves in the
tea-garden. We were chatting about this and that when, having made his
usual boasts as if they were all brandnew, he asked me, ``How many rats
have you caught so far?''

While I flatter myself that my general knowledge is wider and deeper
than Blacky's, I readily admit that my physical strength and courage are
nothing compared with his. All the same, his point-blank question
naturally left me feeling a bit confused. Nevertheless, a fact's a fact,
and one should face the truth. So I answered ``Actually, though I'm
always thinking of catching one, I've never yet caught any.''

Blacky laughed immoderately, quivering the long whiskers, which stuck
out stiffly round his muzzle. Blacky, like all true braggarts, is
somewhat weak in the head. As long as you purr and listen attentively,
pretending to be impressed by his rhodomontade, he is a more or less
manageable cat. Soon after getting to know him, I learnt this way to
handle him. Consequently on this particular occasion I also thought it
would be unwise to further weaken my position by trying to defend
myself, and that it would be more prudent to dodge the issue by inducing
him to brag about his own successes. So without making a fuss, I sought
to lead him on by saying, ``You, judging by your age, must have caught a
notable number of rats?'' Sure enough, he swallowed the bait with gusto.

``Well, not too many, but I must've caught thirty or forty,'' was his
triumphant answer. ``I can cope,'' he went on, ``with a hundred or two
hundred rats, any time and by myself. But a weasel, no. That I just
can't take. Once I had a hellish time with a weasel.''

``Did you really?'' I innocently offered. Blacky blinked his saucer eyes
but did not discontinue.

``It was last year, the day for the general housecleaning. As my master
was crawling in under the floorboards with a bag of lime, suddenly a
great, dirty weasel came whizzing out.''

``Really?'' I make myself look impressed.

``I say to myself, `So what's a weasel? Only a wee bit bigger than a
rat.' So I chase after it, feeling quite excited and finally I got it
cornered in a ditch.''

``That was well done,'' I applaud him.

``Not in the least. As a last resort it upped its tail and blew a filthy
fart. Ugh! The smell of it! Since that time, whenever I see a weasel, I
feel poorly.'' At this point, he raised a front paw and stroked his
muzzle two or three times as if he were still suffering from last year's
stench.

I felt rather sorry for him and, in an effort to cheer him up, said,
``But when it comes to rats, I expect you just pin them down with one
hypnotic glare. And I suppose that it's because you're such a marvelous
ratter, a cat well nourished by plenty of rats, that you are so
splendidly fat and have such a good complexion.'' Though this speech was
meant to flatter Blacky, strangely enough it had precisely the opposite
effect. He looked distinctly cast down and replied with a heavy sigh.

``It's depressing,'' he said, ``when you come to think of it. However
hard one slaves at catching rats\ldots{} In the whole wide world there's
no creature more brazen-faced than a human being. Every rat I catch they
confiscate, and they tote them off to the nearest police-box. Since the
copper can't tell who caught the rats, he just pays up a penny a tail to
anyone that brings them in. My master, for instance, has already earned
about half a crown purely through my efforts, but he's never yet stood
me a decent meal. The plain fact is that humans, one and all, are merely
thieves at heart.''

Though Blacky's far from bright, one cannot fault him in this
conclusion. He begins to look extremely angry and the fur on his back
stands up in bristles. Somewhat disturbed by Blacky's story and
reactions, I made some vague excuse and went off home. But ever since
then I've been determined never to catch a rat. However, I did not take
up Blacky's invitation to become his associate in prowling after
dainties other than rodents. I prefer the cozy life, and it's certainly
easier to sleep than to hunt for titbits. Living in a teacher's house,
it seems that even a cat acquires the character of teachers. I'd best
watch out lest, one of these days, I, too, become dyspeptic.

Talking of teachers reminds me that my master seems to have recently
realized his total incapacity as a painter of watercolors; for under the
date of December 1st his diary contains the following passage: At
today's gathering I met for the first time a man who shall be nameless.
He is said to have led a fast life. Indeed he looks very much a man of
the world. Since women like this type of person, it might be more
appropriate to say that he has been forced to lead, rather than that he
has led, a fast life. I hear his wife was originally a geisha. He is to
be envied. For the most part, those who carp at rakes are those
incapable of debauchery. Further, many of those who fancy themselves as
rakehells are equally incapable of debauchery. Such folk are under no
obligation to live fast lives, but do so of their own volition. So I in
the matter of watercolors. Neither of us will ever make the grade. And
yet this type of debauchee is calmly certain that only he is truly a man
of the world. If it is to be accepted that a man can become a man of the
world by drinking saké in restaurants, or by frequenting houses of
assignation, then it would seem to follow that I could acquire a name as
a painter of watercolors. The notion that my watercolor pictures will be
better if I don't actually paint them leads me to conclude that a
boorish country-bumpkin is in fact far superior to such foolish men of
the world.

His observations about men of the world strike me as somewhat
unconvincing. In particular his confession of envy in respect of that
wife who'd worked as a geisha is positively imbecile and unworthy of a
teacher. Nevertheless his assessment of the value of his own watercolor
painting is certainly just. Indeed my master is a very good judge of his
own character but still manages to retain his vanity. Three days later,
on December 4th, he wrote in his diary:

Last night I dreamt that someone picked up one of my watercolor
paintings which I, thinking it worthless, had tossed aside. This person
in my dream put the painting in a splendid frame and hung it up on a
transom. Staring at my work thus framed, I realized that I have suddenly
become a true artist. I feel exceedingly pleased. I spend whole days
just staring at my handiwork, happy in the conviction that the picture
is a masterpiece. Dawn broke and I woke up, and in the morning sunlight
it was obvious that the picture was still as pitiful an object as when I
painted it.

The master, even in his dreams, seems burdened with regrets about his
watercolors. And men who accept the burdens of regret, whether in
respect of watercolors or of anything else, are not the stuff that men
of the world are made of.

The day after my master dreamt about the picture, the aesthete in the
gold-rimmed spectacles paid a call upon him. He had not visited for some
long time. As soon as he was seated he inquired, ``And how is the
painting coming along?''

My master assumed a nonchalant air and answered, ``Well, I took your
advice and I am now busily engaged in sketching. And I must say that
when one sketches one seems to apprehend those shapes of things, those
delicate changes of color, which hitherto had gone unnoticed. I take it
that sketching has developed in the West to its present remarkable
condition solely as the result of the emphasis which, historically, has
always there been placed upon the essentiality thereof. Precisely as
Andrea del Sarto once observed.'' Without even so much as alluding to
the passage in his diary, he speaks approvingly of Andrea del Sarto.

The aesthete scratched his head, and remarked with a laugh, ``Well
actually that bit about del Sarto was my own invention.''

``What was?'' My master still fails to grasp that he's been tricked into
making a fool of himself.

``Why, all that stuff about Andrea del Sarto whom you so particularly
admire, I made it all up. I never thought you'd take it seriously.'' He
laughed and laughed, enraptured with the situation.

I overheard their conversation from my place on the veranda and I could
not help wondering what sort of entry would appear in the diary for
today. This aesthete is the sort of man whose sole pleasure lies in
bamboozling people by conversation consisting entirely of humbug. He
seems not to have thought of the effect his twaddle about Andrea del
Sarto must have on my master's feelings, for he rattled on proudly,
``Sometimes I cook up a little nonsense and people take it seriously,
which generates an aesthetic sensation of extreme comicality which I
find interesting. The other day, I told a certain undergraduate that
Nicholas Nickleby had advised Gibbon to cease using French for the
writing of his masterpiece, The History of the French Revolution, and
had indeed persuaded Gibbon to publish it in English. Now this
undergraduate was a man of almost eidetic memory, and it was especially
amusing to hear him repeating what I told him, word for word and in all
seriousness, to a debating session of the Japan Literary Society. And
d'you know, there were nearly a hundred in his audience, and all of them
sat listening to his drivel with the greatest enthusiasm! In fact, I've
another, even better, story. The other day, when I was in the company of
some men of letters, one of them happened to mention Theofano,
Ainsworth's historical novel of the Crusades. I took the occasion to
remark that it was a quite outstanding romantic monograph and added the
comment that the account of the heroine's death was the epitome of the
spectral. The man sitting opposite to me, one who has never uttered the
three words `I don't know,' promptly responded that those particular
paragraphs were indeed especially fine writing. From which observation I
became aware that he, no more than I, had ever read the book.''

Wide-eyed, my poor dyspeptic master asked him, ``Fair enough, but what
would you do if the other party had in fact read the book?'' It appears
that my master is not worried about the dishonesty of the deception,
merely about the possible embarrassment of being caught out in a lie.
The question leaves the aesthete utterly unfazed.

``Well, if that should happen, I'd say I'd mistaken the title or
something like that,'' and again, quite unconcerned, he gave himself to
laughter.

Though nattily tricked out in gold-rimmed spectacles, his nature is
uncommonly akin to that of Rickshaw Blacky. My master said nothing, but
blew out smoke rings as if in confession of his own lack of such
audacity. The aesthete (the glitter of whose eyes seemed to be
answering, ``and no wonder; you, being you, could not even cope with
watercolors'') went on aloud. ``But, joking apart, painting a picture's
a difficult thing. Leonardo da Vinci is supposed to have once told his
pupils to make drawings of a stain on the Cathedral wall. The words of a
great teacher. In a lavatory for instance, if absorbedly one studies the
pattern of the rain leaks on the wall, a staggering design, a natural
creation, invariably emerges. You should keep your eyes open and try
drawing from nature. I'm sure you could make something interesting.''

``Is this another of your tricks?''

``No; this one, I promise, is seriously meant. Indeed, I think that that
image of the lavatory wall is really rather witty, don't you? Quite the
sort of thing da Vinci would have said.''

``Yes, it's certainly witty,'' my master somewhat reluctantly conceded.
But I do not think he has so far made a drawing in a lavatory.

Rickshaw Blacky has recently gone lame. His glossy fur has thinned and
gradually grown dull. His eyes, which I once praised as more beautiful
than amber, are now bleared with mucus. What I notice most is his loss
of all vitality and his sheer physical deterioration. When last I saw
him in the tea garden and asked him how he was, the answer was
depressingly precise: ``I've had enough of being farted at by weasels
and crippled with side-swipes from the fishmonger's pole.''

The autumn leaves, arranged in two or three scarlet terraces among the
pine trees, have fallen like ancient dreams. The red and white
sasan-quas near the garden's ornamental basin, dropping their petals,
now a white and now a red one, are finally left bare. The wintry sun
along the ten-foot length of the southwards-facing veranda goes down
daily earlier than yesterday. Seldom a day goes by but a cold wind
blows. So my snoozes have been painfully curtailed.

The master goes to school every day and, as soon as he returns, shuts
himself up in the study. He tells all visitors that he's tired of being
a teacher. He seldom paints. He's stopped taking his taka-diastase,
saying it does no good. The children, dear little things, now trot off,
day after day, to kindergarten: but on their return, they sing songs,
bounce balls and sometimes hang me up by the tail.

Since I do not receive any particularly nourishing food, I have not
grown particularly fat; but I struggle on from day to day keeping myself
more or less fit and, so far, without getting crippled. I catch no rats.
I still detest that O-san. No one has yet named me but, since it's no
use crying for the moon, I have resolved to remain for the rest of my
life a nameless cat in the house of this teacher.
\chapter*{II}
SINCE New Year's Day I have acquired a certain modest celebrity: so
that, though only a cat, I am feeling quietly proud of myself. Which is
not unpleasing.

On the morning of New Year's Day, my master received a picture-postcard,
a card of New Year greetings from a certain painter-friend of his. The
upper part was painted red, the lower deep green; and right in the
center was a crouching animal painted in pastel. The master, sitting in
his study, looked at this picture first one way up and then the other.
``What fine coloring!'' he observed. Having thus expressed his
admiration, I thought he had finished with the matter. But no, he
continued studying it, first sideways and then longways. In order to
examine the object he twists his body, then stretches out his arms like
an ancient studying the Book of Divinations and then, turning to face
the window, he brings it in to the tip of his nose. I wish he would soon
terminate this curious performance, for the action sets his knees asway
and I find it hard to keep my balance. When at long last the wobbling
began to diminish, I heard him mutter in a tiny voice, ``I wonder what
it is.'' Though full of admiration for the colors on the
picture-postcard, he couldn't identify the animal painted in its center.
Which explained his extraordinary antics. Could it perhaps really be a
picture more difficult to interpret than my own first glance had
suggested? I half-opened my eyes and looked at the painting with an
imperturbable calmness. There could be no shadow of a doubt: it was a
portrait of myself. I do not suppose that the painter considered himself
an Andrea del Sarto, as did my master; but, being a painter, what he had
painted, both in respect of form and of color, was perfectly harmonious.
Any fool could see it was a cat. And so skillfully painted that anyone
with eyes in his head and the mangiest scrap of discernment would
immediately recognize that it was a picture of no other cat but me. To
think that anyone should need to go to such painful lengths over such a
blatantly simple matter\ldots{} I felt a little sorry for the human
race. I would have liked to have let him know that the picture is of me.
Even if it were too difficult for him to grasp that particularity, I
would still have liked to help him see that the painting is of a cat.
But since heaven has not seen fit to dower the human animal with an
ability to understand cat language, I regret to say that I let the
matter be.

Incidentally, I would like to take the occasion of this incident to
advise my readers that the human habit of referring to me in a scornful
tone of voice as some mere trifling ``cat'' is an extremely bad one.
Humans appear to think that cows and horses are constructed from
rejected human material, and that cats are constructed from cow pats and
horse dung. Such thoughts, objectively regarded, are in very poor taste
though they are no doubt not uncommon among teachers who, ignorant even
of their ignorance, remain self-satisfied with their quaint puffed-up
ideas of their own unreal importance. Even cats must not be treated
roughly or taken for granted. To the casual observer it may appear that
all cats are the same, facsimiles in form and substance, as
indistinguishable as peas in a pod; and that no cat can lay claim to
individuality. But once admitted to feline society, that casual observer
would very quickly realize that things are not so simple, and that the
human saying that ``people are freaks'' is equally true in the world of
cats. Our eyes, noses, fur, paws---all of them differ. From the tilt of
one's whiskers to the set of one's ears, down to the very hang of one's
tail, we cats are sharply differentiated. In our good looks and our poor
looks, in our likes and dislikes, in our refinement and our
coarsenesses, one may fairly say that cats occur in infinite variety.
Despite the fact of such obvious differentiation, humans, their eyes
turned up to heaven by reason of the elevation of their minds or some
such other rubbish, fail to notice even obvious differences in our
external features, that our characters might be characteristic is beyond
their comprehension. Which is to be pitied. I understand and endorse the
thought behind such sayings as, the cobbler should stick to his last,
that birds of a feather flock together, that rice-cakes are for
rice-cake makers. For cats, indeed, are for cats. And should you wish to
learn about cats, only a cat can tell you. Humans, however advanced, can
tell you nothing on this subject. And inasmuch as humans are, in fact,
far less advanced than they fancy themselves, they will find it
difficult even to start learning about cats. And for an unsympathetic
man like my master there's really no hope at all. He does not even
understand that love can never grow unless there is at least a complete
and mutual understanding. Like an ill-natured oyster, he secretes
himself in his study and has never once opened his mouth to the outside
world. And to see him there, looking as though he alone has truly
attained enlightenment, is enough to make a cat laugh. The proof that he
has not attained enlightenment is that, although he has my portrait
under his nose, he shows no sign of comprehension but coolly offers such
crazy comment as, ``perhaps, this being the second year of the war
against the Russians, it is a painting of a bear.''

As, with my eyes closed, I sat thinking these thoughts on my master's
knees, the servant-woman brought in a second picture-postcard. It is a
printed picture of a line of four or five European cats all engaged in
study, holding pens or reading books. One has broken away from the line
to perform a simple Western dance at the corner of their common desk.
Above this picture ``I am a cat'' is written thickly in Japanese black
ink. And down the right-hand side there is even a haiku stating that
``on spring days cats read books or dance.'' The card is from one of the
master's old pupils and its meaning should be obvious to anyone. However
my dimwitted master seems not to understand, for he looked puzzled and
said to himself, ``Can this be a Year of the Cat?'' He just doesn't seem
to have grasped that these postcards are manifestations of my growing
fame.

At that moment the servant brought in yet a third postcard. This time
the postcard has no picture, but alongside the characters wishing my
master a happy New Year, the correspondent has added those for, ``Please
be so kind as to give my best regards to the cat.'' Bone-headed though
he is, my master does appear to get the message when it's written out
thus unequivocally: for he glanced down at my face and, as if he really
had at last comprehended the situation, said, ``hmm.'' And his glance,
unlike his usual ones, did seem to contain a new modicum of respect.
Which was quite right and proper considering the fact that it is
entirely due to me that my master, hitherto a nobody, has suddenly begun
to get a name and to attract attention.

Just then the gate-bell sounded: tinkle-tinkle, possibly even ting-ting.
Probably a visitor. If so, the servant will answer. Since I never go out
of my way to investigate callers, except the fishmonger's errand-boy, I
remained quietly on my master's knees. The master, however, peered
worriedly toward the entrance as if duns were at the door. I deduce that
he just doesn't like receiving New Year's callers and sharing a
convivial tot. What a marvellous way to be. How much further can pure
bigotry go? If he doesn't like visitors, he should have gone out
himself, but he lacks even that much enterprise. The inaudacity of his
clam-like character grows daily more apparent. A few moments later the
servant comes in to say that Mr.~Coldmoon has called. I understand that
this Coldmoon person was also once a pupil of my master's and that,
after leaving school, he so rose in the world to be far better known
than his teacher. I don't know why, but this fellow often comes round
for a chat. On every such visit he babbles on, with a dreadful sort of
coquettishness, about being in love or not in love with somebody or
other; about how much he enjoys life or how desperately he is tired of
it. And then he leaves. It is quaint enough that to discuss such matters
he should seek the company of a withered old nut like my master, but
it's quainter still to see my mollusk opening up to comment, now and
again, on Coldmoon's mawkish maunderings.

``I'm afraid I haven't been round for quite some time. Actually, I've
been as busy as, busy since the end of last year, and, though I've
thought of going out often enough, somehow shanks' pony has just not
headed here.'' Thus, twisting and untwisting the fastening-strings of
his short surcoat, Coldmoon babbled on.

``Where then did shanks' pony go?'' my master enquired with a serious
look as he tugged at the cuffs of his worn, black, crested surcoat. It
is a cotton garment unduly short in the sleeves, and some of its
nonde-script, thin, silk lining sticks out about a half an inch at the
cuffs.

``As it were in various directions,'' Coldmoon answered, and then
laughed. I notice that one of his front teeth is missing.

``What's happened to your teeth?'' asks my master, changing the subject.

``Well, actually, at a certain place I ate mushrooms.''

``What did you say you ate?''

``A bit of mushroom. As I tried to bite off a mushroom's umbrella with
my front teeth, a tooth just broke off.''

``Breaking teeth on a mushroom sounds somewhat senile. An image possibly
appropriate to a haiku but scarcely appropriate to the pursuit of
love,'' remarked my master as he tapped lightly on my head with the palm
of his hand.

``Ah! Is that the cat? But he's quite plump! Sturdy as that, not even
Rickshaw Blacky could beat him up. He certainly is a most splendid
beast.'' Coldmoon offers me his homage.

``He's grown quite big lately,'' responds my master, and proudly smacks
me twice upon the head. I am flattered by the compliment but my head
feels slightly sore.

``The night before last, what's more, we had a little concert,'' said
Coldmoon going back to his story.

``Where?''

``Surely you don't have to know where. But it was quite interesting,
three violins to a piano accompaniment. However unskilled, when there
are three of them, violins sound fairly good. Two of them were women and
I managed to place myself between them. And I myself, I thought, played
rather well.''

``Ah, and who were the women?'' enviously my master asks. At first
glance my master usually looks cold and hard; but, to tell the truth, he
is by no means indifferent to women. He once read in a Western novel of
a man who invariably fell partially in love with practically every woman
that he met. Another character in the book somewhat sarcastically
observed that, as a rough calculation, that fellow fell in love with
just under seven-tenths of the women he passed in the street. On reading
this, my master was struck by its essential truth and remained deeply
impressed. Why should a man so impressionable lead such an oysterish
existence? A mere cat such as I cannot possibly understand it. Some say
it is the result of a love affair that went wrong; some say it is due to
his weak stomach; while others simply state that it's because he lacks
both money and audacity. Whatever the truth, it doesn't much matter
since he's a person of insufficient importance to affect the history of
his period. What is certain is that he did enquire enviously about
Coldmoon's female fiddlers. Coldmoon, looking amused, picked up a sliver
of boiled fishpaste in his chopsticks and nipped at it with his
remaining front teeth. I was worried lest another should fall out. But
this time it was all right.

``Well, both of them are daughters of good families. You don't know
them,'' Coldmoon coldly answered.

The master drawled ``Is---th-a-t---,'' but omitted the final ``so''
which he'd intended.

Coldmoon probably considered it was about time to be off, for he said,
``What marvellous weather. If you've nothing better to do, shall we go
out for a walk? As a result of the fall of Port Arthur,'' he added
encouragingly, ``the town's unusually lively.''

My master, looking as though he would sooner discuss the identity of the
female fiddlers than the fall of Port Arthur, hesitated for a moment's
thought. But he seemed finally to reach a decision, for he stood up
resolutely and said, ``All right, let's go out.'' He continues to wear
his black cotton crested surcoat and, thereunder, a quilted kimono of
hand-woven silk which, supposedly a keep-sake of his elder brother, he
has worn continuously for twenty years. Even the most strongly woven
silk, cannot survive such unremitting, such preternaturally, perennial
wear. The material has been worn so thin that, held against the light,
one can see the patches sewn on here and there from the inner side. My
master wears the same clothes throughout December and January, not
bothering to observe the traditional New Year change. He makes, indeed,
no distinction between workaday and Sunday clothes. When he leaves the
house he saunters out in whatever dress he happens to have on. I do not
know whether this is because he has no other clothes to wear or whether,
having such clothes, he finds it too much of a bore to change into them.
Whatever the case, I can't conceive that these uncouth habits are in any
way connected with disappointment in love.

After the two men left, I took the liberty of eating such of the boiled
fishpaste as Coldmoon had not already devoured. I am, these days, no
longer just a common, old cat. I consider myself at least as good as
those celebrated in the tales of Momokawa Joen or as that cat of Thomas
Gray's, which trawled for goldfish. Brawlers such as Rickshaw Blacky are
now beneath my notice. I don't suppose anyone will make a fuss if I
sneak a bit of fishpaste. Besides, this habit of taking secret snacks
between meals is by no means a purely feline custom. O-san, for
instance, is always pinching cakes and things, which she gobbles down
whenever the mistress leaves the house. Nor is O-san the only offender:
even the children, of whose refined upbringing the mistress is
continually bragging, display the selfsame tendency. Only a few days ago
that precious pair woke at some ungodly hour, and, though their parents
were still sound asleep, took it upon themselves to sit down,
face-to-face, at the dining-table. Now it is my master's habit every
morning to consume most of a loaf of bread, and to give the children
scraps thereof which they eat with a dusting of sugar. It so happened
that on this day the sugar basin was already on the table, even a spoon
stuck in it. Since there was no one there to dole them out their sugar,
the elder child scooped up a spoonful and dumped it on her plate. The
younger followed her elder's fine example and spooned an equal pile of
sugar onto another plate. For a brief while these charming creatures
just sat and glared at each other. Then the elder girl scooped a second
spoonful onto her plate, and the younger one proceeded to equalize the
position. The elder sister took a third spoonful and the younger, in a
splendid spirit of rivalry, followed suit. And so it went on until both
plates were piled high with sugar and not one single grain remained in
the basin. My master thereupon emerged from his bedroom rubbing
half-sleepy eyes and proceeded to return the sugar, so laboriously
extracted by his daughters, back into the sugar-basin. This incident
suggests that, though egotistical egalitarianism may be more highly
developed among humans than among cats, cats are the wiser creatures. My
advice to the children would have been to lick the sugar up quickly
before it became massed into such senseless pyramids, but, because they
cannot understand what I say, I merely watched them in silence from my
warm, morning place on top of the container for boiled rice.

My master came home late last night from his expedition with Coldmoon.
God knows where he went, but it was already past nine before he sat down
at the breakfast table. From my same old place I watched his morose
consumption of a typical New Year's breakfast of rice-cakes boiled with
vegetables, all served up in soup. He takes endless helpings. Though the
rice-cakes are admittedly small, he must have eaten some six or seven
before leaving the last one floating in the bowl. ``I'll stop now,'' he
remarked and laid his chopsticks down. Should anyone else behave in such
a spoilt manner, he could be relied upon to put his foot down: but, vain
in the exercise of his petty authority as master of the house, he seems
quite unconcerned by the sight of the corpse of a scorched rice-cake
drowning in turbid soup. When his wife took taka-diastase from the back
of a small cupboard and put it on the table, my master said, ``I won't
take it, it does me no good.''

``But they say it's very good after eating starchy things. I think you
should take some.'' His wife wants him to take it.

``Starchy or not, the stuff's no good.'' He remains stubborn.

``Really, you are a most capricious man,'' the mistress mutters as
though to herself.

``I'm not capricious, the medicine doesn't work.''

``But until the other day you used to say it worked very well and you
used to take it every day, didn't you?''

``Yes, it did work until that other day, but it hasn't worked since
then,'' an antithetical answer.

``If you continue in these inconsistencies, taking it one day and
stopping it the next, however efficacious the medicine may be, it will
never do you any good. Unless you try to be a little more patient,
dyspepsia, unlike other illnesses, won't get cured, will it?'' and she
turns to O-san who was serving at the table.

``Quite so, madam. Unless one takes it regularly, one cannot find out
whether a medicine is a good one or a bad one.'' O-san readily sides
with the mistress.

``I don't care. I don't take it because I don't take it. How can a mere
woman understand such things? Keep quiet.''

``All right. I'm merely a woman,'' she says pushing the taka-diastase
toward him, quite determined to make him see he is beaten. My master
stands up without saying a word and goes off into his study. His wife
and servant exchange looks and giggle. If on such occasions I follow him
and jump up onto his knees, experience tells me that I shall pay dearly
for my folly. Accordingly, I go quietly round through the garden and hop
up onto the veranda outside his study. I peeped through the slit between
the paper sliding doors and found my master examining a book by somebody
called Epictetus. If he could actually understand what he's reading,
then he would indeed be worthy of praise. But within five or six minutes
he slams the book down on the table, which is just what I'd suspected.
As I sat there watching him, he took out his diary and made the
following entry.

Took a stroll with Coldmoon round Nezu, Ueno, Ikenohata and Kanda. At
Ikenohata, geishas in formal spring kimono were playing battledore and
shuttlecock in front of a house of assignation. Their clothes beautiful,
but their faces extremely plain. It occurs to me that they resemble the
cat at home.

I don't see why he should single me out as an example of plain features.
If I went to a barber and had my face shaved, I wouldn't look much
different from a human. But, there you are, humans are conceited and
that's the trouble with them.

As we turned at Hotan's corner another geisha appeared. She was slim,
well-shaped and her shoulders were most beautifully sloped. The way she
wore her mauve kimono gave her a genuine elegance. ``Sorry about last
night, Gen-chan---I was so busy\ldots{}'' She laughed and one glimpsed
white teeth. Her voice was so harsh, as harsh as that of a roving crow,
that her otherwise fine appearance diminished in enchantment. So much so
that I didn't even bother to turn around to see what sort of person this
Gen-chan was, but sauntered on toward Onarimachi with my hands tucked
inside the breast-fold of my kimono. Coldmoon, however, seemed to have
become a trifle fidgety.

There is nothing more difficult than understanding human mentality. My
master's present mental state is very far from clear; is he feeling
angry or lighthearted, or simply seeking solace in the scribblings of
some dead philosopher? One just can't tell whether he's mocking the
world or yearning to be accepted into its frivolous company; whether he
is getting furious over some piddling little matter or holding himself
aloof from worldly things. Compared with such complexities, cats are
truly simple. If we want to eat, we eat; if we want to sleep, we sleep;
when we are angry, we are angry utterly; when we cry, we cry with all
the desperation of extreme commitment to our grief. Thus we never keep
things like diaries. For what would be the point? No doubt human beings
like my two-faced master find it necessary to keep diaries in order to
display in a darkened room that true character so assiduously hidden
from the world. But among cats both our four main occupations (walking,
standing, sitting, and lying down) and such incidental activities as
excreting waste are pursued quite openly. We live our diaries, and
consequently have no need to keep a daily record as a means of
maintaining our real characters. Had I the time to keep a diary, I'd use
that time to better effect; sleeping on the veranda.

We dined somewhere in Kanda. Because I allowed myself one or two cups of
saké (which I had not tasted for quite a time), my stomach this morning
feels extremely well. I conclude that the best remedy for a stomach
ailment is saké at suppertime. Taka-diastase just won't do. Whatever
claims are made for it, it's just no good. That which lacks effect will
continue to lack effect.

Thus with his brush he smears the good name of taka-diastase. It is as
though he quarreled with himself, and in this entry one can see a last
flash of this morning's ugly mood. Such entries are perhaps most
characteristic of human mores.

The other day, Mr.~X claimed that going without one's breakfast helped
the stomach. So I took no breakfast for two or three days but the only
effect was to make my stomach grumble. Mr.~Y strongly advised me to
refrain from eating pickles. According to him, all disorders of the
stomach originate in pickles. His thesis was that abstinence from
pickles so dessicates the sources of all stomach trouble, that a
complete cure must follow. For at least a week no pickle crossed my
lips, but, since that banishment produced no noticeable effect, I have
resumed consuming them. According to Mr.~Z, the one true remedy is
ventral massage. But no ordinary massage of the stomach would suffice.
It must be massage in accordance with the old-world methods of the
Minagawa School. Massaged thus once, or at most twice, the stomach would
be rid of every ill. The wisest scholars, such as Yasui Sokuken, and the
most resourceful heroes, such as Sakamoto Ryoma, all relied upon this
treatment. So off I went to Kaminegishi for an immediate massage. But
the methods used were of inordinate cruelty. They told me, for instance,
that no good could be hoped for unless one's bones were massaged; that
it would be difficult properly to eradicate my troubles unless, at least
once, my viscera were totally inverted. At all events, a single session
reduced my body to the condition of cotton-wool and I felt as though I
had become a lifelong sufferer from sleeping sickness. I never went
there again. Once was more than enough. Then Mr.~A assured me that one
shouldn't eat solids. So I spent a whole day drinking nothing but milk.
My bowels gave forth heavy plopping noises as though they had been
swamped, and I could not sleep all night. Mr.~B states that exercising
one's intestines by diaphragmic breathing produces a naturally healthy
stomach and he counsels me to follow his advice. And I did try. For a
time. But it proved no good for it made my bowels queasy. Besides,
though every now and again I strive with all my heart and soul to
control my breathing with the diaphragm, in five or six minutes I forget
to discipline my muscles. And if I concentrate on maintaining that
discipline I get so midriff-minded that I can neither read nor write.
Waverhouse, my aesthete friend, once found me thus breathing in pursuit
of a naturally healthy stomach and, rather unkindly, urged me, as a man,
to terminate my labor-pangs. So diaphragmic breathing is now also a
thing of the past. Dr.~C recommends a diet of buckwheat noodles. So
buckwheat noodles it was, alternately in soup and served cold after
boiling. It did nothing, except loosen my bowels. I have tried every
possible means to cure my ancient ailment, but all of them are useless.
But those three cups of saké which I drank last night with Coldmoon have
certainly done some good. From now on, I will drink two or three cups
each evening.

I doubt whether this saké treatment will be kept up very long. My
master's mind exhibits the same incessant changeability as can be seen
in the eyes of cats. He has no sense of perseverance. It is, moreover,
idiotic that, while he fills his diary with lamentation over his stomach
troubles, he does his best to present a brave face to the world; to grin
and bear it.

The other day his scholar friend, Professor Whatnot, paid a visit and
advanced the theory that it was at least arguable that every illness is
the direct result of both ancestral and personal malefaction. He seemed
to have studied the matter pretty deeply for the sequence of his logic
was clear, consistent, and orderly. Altogether it was a fine theory. I
am sorry to say that my master has neither the brain nor the erudition
to rebut such theories. However, perhaps because he himself was actually
suffering from stomach trouble, he felt obliged to make all sorts of
face-saving excuses. He irrelevantly retorted, ``Your theory is
interesting, but are you aware that Carlyle was dyspeptic?'' as if
claiming that because Carlyle was dyspeptic his own dyspepsia was an
intellectual honor. His friend replied,

``It does not follow that because Carlyle was a dyspeptic, all
dyspeptics are Carlyles.'' My master, reprimanded, held his tongue, but
the incident revealed his curious vanity. It's all the more amusing when
one recalls that he would probably prefer not to be dyspeptic, for just
this morning he recorded in his diary an intention to take treatment by
saké as from tonight. Now that I've come to think of it, his inordinate
consumption of rice-cakes this morning must have been the effect of last
night's saké session with Coldmoon. I could have eaten those cakes
myself.

Though I am a cat, I eat practically anything. Unlike Rickshaw Blacky, I
lack the energy to go off raiding fishshops up distant alleys. Further,
my social status is such that I cannot expect the luxury enjoyed by
Tortoiseshell whose mistress teaches the idle rich to play on the
two-stringed harp. Therefore I don't, as others can, indulge myself in
likes and dislikes. I eat small bits of bread left over by the children,
and I lick the jam from bean-jam cakes. Pickles taste awful, but to
broaden my experience I once tried a couple of slices of pickled radish.
It's a strange thing but once I've tried it, almost anything turns out
edible. To say, ``I don't like that'' or ``I don't like this'' is mere
extravagant willfulness, and a cat that lives in a teacher's house
should eschew such foolish remarks.

According to my master, there was once a novelist whose name was Balzac
and he lived in France. He was an extremely extravagant man. I do not
mean an extravagant eater but that, being a novelist, he was extravagant
in his writing. One day he was trying to find a suitable name for a
character in the novel he was writing, but, for whatever reason, could
not think of a name that pleased him. Just then one of his friends
called by, and Balzac suggested they should go out for a walk. This
friend had, of course, no idea why, still less that Balzac was
determined to find the name he needed. Out on the streets, Balzac did
nothing but stare at shop signboards, but still he couldn't find a
suitable name. He marched on endlessly, while his puzzled friend, still
ignorant of the object of the expedition, tagged along behind him.
Having fruitlessly explored Paris from morning till evening, they were
on their way home when Balzac happened to notice a tailor's signboard
bearing the name ``Marcus.'' He clapped his hands. ``This is it,'' he
shouted. ``It just has to be this. Marcus is a good name, but with a Z
in front of Marcus it becomes a perfect name. It has to be a Z. Z.
Marcus is remarkably good. Names that I invent are never good. They
sound unnatural however cleverly constructed. But now, at long, long
last, I've got the name I like.'' Balzac, extremely pleased with
himself, was totally oblivious to the inconvenience he had caused his
friend. It would seem unduly troublesome that one should have to spend a
whole day trudging around Paris merely to find a name for a character in
a novel. Extravagance of such enormity acquires a certain splendor, but
folk like me, a cat kept by a clam-like introvert, cannot even envisage
such inordinate behavior. That I should not much care what, so long as
it's edible, I eat is probably an inevitable result of my circumstances.
Thus it was in no way as an expression of extravagance that I expressed
just now my feeling of wishing to eat a rice-cake. I simply thought that
I'd better eat while the chance offered, and I then remembered that the
piece of rice-cake which my master had left in his breakfast bowl was
possibly still in the kitchen. So round to the kitchen I went.

The rice-cake was stuck, just as I saw it this morning, at the bottom of
the bowl and its color was still as I remembered it. I must confess that
I've never previously tasted rice-cake. Yet, though I felt a shade
uncertain, it looks quite good to eat. With a tentative front paw I rake
at the green vegetables adhering to the rice-cake. My claws, having
touched the outer part of the rice-cake, become sticky. I sniff at them
and recognize the smell that can be smelt when rice stuck at the bottom
of a cooking-pot is transferred into the boiled-rice container. I look
around, wondering, ``Shall I eat it, shall I not?'' Fortunately, or
unfortunately, there's nobody about. O-san, with a face that shows no
change between year end and the spring, is playing battledore and
shuttlecock. The children in the inner room are singing something about
a rabbit and a tortoise. If I am to eat this New Year speciality, now's
the moment. If I miss this chance I shall have to spend a whole, long
year not knowing how a rice-cake tastes. At this point, though a mere
cat, I perceived a truth: that golden opportunity makes all animals
venture to do even those things they do not want to do. To tell the
truth, I do not particularly want to eat the rice-cake. In fact the more
I examined the thing at the bottom of the bowl the more nervous I became
and the more keenly disinclined to eat it. If only O-san would open the
kitchen door, or if I could hear the children's footsteps coming toward
me, I would unhesitatingly abandon the bowl; not only that, I would have
put away all thought of rice-cakes for another year. But no one comes.
I've hesitated long enough. Still no one comes. I feel as if someone
were hotly urging me on, someone whispering, ``Eat it, quickly!'' I
looked into the bowl and prayed that someone would appear. But no one
did. I shall have to eat the rice-cake after all. In the end, lowering
the entire weight of my body into the bottom of the bowl, I bit about an
inch deep into a corner of the rice-cake.

Most things that I bite that hard come clean off in my mouth. But what a
surprise! For I found when I tried to reopen my jaw that it would not
budge. I try once again to bite my way free, but find I'm stuck. Too
late I realize that the rice-cake is a fiend. When a man who has fallen
into a marsh struggles to escape, the more he thrashes about trying to
extract his legs, the deeper in he sinks. Just so, the harder I clamp my
jaws, the more my mouth grows heavy and my teeth immobilized. I can feel
the resistance to my teeth, but that's all. I cannot dispose of it.
Waverhouse, the aesthete, once described my master as an aliquant man
and I must say it's rather a good description. This rice-cake too, like
my master, is aliquant. It looked to me that, however much I continued
biting, nothing could ever result: the process could go on and on
eternally like the division of ten by three. In the middle of this
anguish I found my second truth: that all animals can tell by instinct
what is or is not good for them. Although I have now discovered two
great truths, I remain unhappy by reason of the adherent rice-cake. My
teeth are being sucked into its body, and are becoming excruciatingly
painful. Unless I can complete my bite and run away quickly, O-san will
be on me. The children seem to have stopped singing, and I'm sure
they'll soon come running into the kitchen. In an extremity of anguish,
I lashed about with my tail, but to no effect. I made my ears stand up
and then lie flat, but this didn't help either. Come to think of it, my
ears and tail have nothing to do with the rice-cake. In short, I had
indulged in a waste of wagging, a waste of ear-erection, and a waste of
ear-flattening. So I stopped.

At long last it dawned on me that the best thing to do is to force the
rice-cake down by using my two front paws. First I raised my right paw
and stroked it around my mouth. Naturally, this mere stroking brought no
relief whatsoever. Next, I stretched out my left paw and with it scraped
quick circles around my mouth. These ineffectual passes failed to
exorcize the fiend in the rice-cake. Realizing that it was essential to
proceed with patience, I scraped alternatively with my right and left
paws, but my teeth stayed stuck in the rice-cake. Growing impatient, I
now used both front paws simultaneously. Then, only then, I found to my
amazement that I could actually stand up on my hind legs. Somehow I feel
un-catlike. But not caring whether I am a cat or not, I scratch away
like mad at my whole face in frenzied determination to keep on
scratching until the fiend in the rice-cake has been driven out. Since
the movements of my front paws are so vigorous I am in danger of losing
my balance and falling down. To keep my equilibrium I find myself
marking time with my hind legs. I begin to tittup from one spot to
another, and I finish up prancing madly all over the kitchen. It gives
me great pride to realize that I can so dextrously maintain an upright
position, and the revelation of a third great truth is thus vouchsafed
me: that in conditions of exceptional danger one can surpass one's
normal level of achievement. This is the real meaning of Special
Providence.

Sustained by Special Providence, I am fighting for dear life against
that demonic rice-cake when I hear footsteps. Someone seems to be
approaching. Thinking it would be fatal to be caught in this
predicament, I redouble my efforts and am positively running around the
kitchen. The footsteps come closer and closer. Alas, that Special
Providence seems not to last forever. In the end I am discovered by the
children who loudly shout, ``Why look! The cat's been eating rice-cakes
and is dancing.'' The first to hear their announcement was that O-san
person. Abandoning her shuttlecock and battledore, she flew in through
the kitchen door crying, ``Gracious me!'' Then the mistress, sedate in
her formal silk kimono, deigns to remark, ``What a naughty cat.'' And my
master, drawn from his study by the general hubbub, shouts, ``You
fool!'' The children find me funniest, but by general agreement the
whole household is having a good old laugh. It is annoying, it is
painful, it is impossible to stop dancing. Hell and damnation! When at
long last the laughter began to die down, the dear, little five-year-old
piped up with an, ``Oh what a comical cat,'' which had the effect of
renewing the tide of their ebbing laughter. They fairly split their
sides. I have already heard and seen quite a lot of heartless human
behavior, but never before have I felt so bitterly critical of their
conduct. Special Providence having vanished into thin air, I was back in
my customary position on all fours, finally at my wit's end, and, by
reason of giddiness, cutting a quite ridiculous figure. My master seems
to have felt it would be perhaps a pity to let me die before his very
eyes, for he said to O-san, ``Help him get rid of that rice-cake.''
O-san looks at the mistress as if to say, ``Why not make him go on
dancing?'' The mistress would gladly see my minuet continued, but, since
she would not go so far as wanting me to dance myself to death, says
nothing. My master turned somewhat sharply to the servant and ordered,
``Hurry it up, if you don't help quickly the cat will be dead.'' O-san,
with a vacant look on her face, as though she had been roughly wakened
from some peculiarly delicious dream, took a firm grip on the rice-cake
and yanked it out of my mouth. I am not quite as feeble-fanged as
Coldmoon, but I really did think my entire front toothwork was about to
break off. The pain was indescribable. The teeth embedded in the
rice-cake are being pitilessly wrenched. You can't imagine what it was
like. It was then that the fourth enlightenment burst upon me: that all
comfort is achieved through hardship. When at last I came to myself and
looked around at a world restored to normality, all the members of the
household had disappeared into the inner room.

Having made such a fool of myself, I feel quite unable to face such
hostile critics as O-san. It would, I think, unhinge my mind. To restore
my mental tranquillity, I decided to visit Tortoiseshell, so I left the
kitchen and set off through the backyard to the house of the
two-stringed harp. Tortoiseshell is a celebrated beauty in our district.
Though I am undoubtedly a cat, I possess a wide general knowledge of the
nature of compassion and am deeply sensitive to affection,
kind-heartedness, tenderness, and love. Merely to observe the bitterness
in my master's face, just to be snubbed by O-san, leaves me out of
sorts. At such times I visit this fair, lady friend of mine and our
conversation ranges over many things. Then, before I am aware of it, I
find myself refreshed. I forget my worries, hardships, everything. I
feel as if reborn. Female influence is indeed a most potent thing.
Through a gap in the cedar-hedge, I peer to see if she is anywhere
about. Tortoiseshell, wearing a smart new collar in celebration of the
season, is sitting very neatly on her veranda. The rondure of her back
is indescribably beautiful. It is the most beautiful of all curved
lines. The way her tail curves, the way she folds her legs, the
charmingly lazy shake of her ears---all these are quite beyond
description. She looks so warm sitting there so gracefully in the very
sunniest spot. Her body holds an attitude of utter stillness and
correctness. And her fur, glossy as velvet that reflects the rays of
spring, seems suddenly to quiver although the air is still. For a while
I stood, completely enraptured, gazing at her. Then as I came to myself,
I softly called, ``Miss Tortoiseshell, Miss Tortoiseshell,'' and
beckoned with my paw.

``Why, Professor,'' she greeted me as she stepped down from the veranda.
A tiny bell attached to her scarlet collar made little tinkling sounds.
I say to myself, ``Ah, it's for the New Year that she's wearing a
bell,'' and, while I am still admiring its lively tinkle, find she has
arrived beside me. ``A happy NewYear, Professor,'' and she waves her
tail to the left; for when cats exchange greetings one first holds one's
tail upright like a pole, then twists it round to the left. In our
neighborhood it is only Tortoiseshell who calls me Professor. Now, I
have already mentioned that I have, as yet, no name; it is
Tortoiseshell, and she alone, who pays me the respect due to one that
lives in a teacher's house. Indeed, I am not altogether displeased to be
addressed as a Professor, and respond willingly to her apostrophe.

``And a happy New Year to you,'' I say. ``How beautifully you're done
up!''

``Yes, the mistress bought it for me at the end of last year. Isn't it
nice?'' and she makes it tinkle for me.

``Yes indeed, it has a lovely sound. I've never seen such a wonderful
thing in my life.''

``No! Everyone's using them,'' and she tinkle-tinkles. ``But isn't it a
lovely sound? I'm so happy.'' She tinkle-tinkle-tinkles continuously.

``I can see your mistress loves you very dearly.'' Comparing my lot with
hers, I hinted at my envy of a pampered life.

Tortoiseshell is a simple creature. ``Yes,'' she says, ``that's true;
she treats me as if I were her own child.'' And she laughs innocently.
It is not true that cats never laugh. Human beings are mistaken in their
belief that only they are capable of laughter. When I laugh my nostrils
grow triangular and my Adam's apple trembles. No wonder human beings
fail to understand it.

``What is your master really like?''

``My master? That sounds strange. Mine is a mistress. A mistress of the
two stringed harp.''

``I know that. But what is her background? I imagine she's a person of
high birth?''

``Ah, yes.''

A small Princess-pine

While waiting for you\ldots{}

Beyond the sliding paper-door the mistress begins to play on her
two-stringed harp.

``Isn't that a splendid voice?'' Tortoiseshell is proud of it.

``It seems extremely good, but I don't understand what she's singing.
What's the name of the piece?''

``That? Oh, it's called something or other. The mistress is especially
fond of it. D'you know, she's actually sixty-two? But in excellent
condition, don't you think?''

I suppose one has to admit that she's in excellent condition if she's
still alive at sixty-two. So I answered, ``Yes.'' I thought to myself
that I'd given a silly answer, but I could do no other since I couldn't
think of anything brighter to say.

``You may not think so, but she used to be a person of high standing.
She always tells me so.''

``What was she originally?''

``I understand that she's the thirteenth Shogun's widowed wife's
private-secretary's younger sister's husband's mother's nephew's
daughter.''

``What?''

``The thirteenth Shogun's widowed wife's private-secretary's younger
sister's\ldots{}''

``Ah! But, please, not quite so fast. The thirteenth Shogun's widowed
wife's younger sister's private-secretary's\ldots{}''

``No, no, no. The thirteenth Shogun's widowed wife's private-secretary's
younger sister's\ldots{}''

``The thirteenth Shogun's widowed wife's\ldots{}''

``Right.''

``Private-secretary's. Right?''

``Right.''

``Husband's\ldots{}''

``No, younger sister's husband's.''

``Of course. How could I? Younger sister's husband's\ldots{}''

``Mother's nephew's daughter. There you are.''

``Mother's nephew's daughter?''

``Yes, you've got it.''

``Not really. It's so terribly involved that I still can't get the hang
of it.

What exactly is her relation to the thirteenth Shogun's widowed wife?''

``Oh, but you are so stupid! I've just been telling you what she is.

She's the thirteenth Shogun's widowed wife's private-secretary's younger
sister's husband's mother's\ldots{}''

``That much I've followed, but\ldots{}''

``Then, you've got it, haven't you?''

``Yes.'' I had to give in. There are times for little white lies.

Beyond the sliding paper-door the sound of the two-stringed harp came to
a sudden stop and the mistress' voice called, ``Tortoiseshell,
Tortoiseshell, your lunch is ready.'' Tortoiseshell looked happy and
remarked, ``There, she's calling, so I must go home. I hope you'll
forgive me?'' What would be the good of my saying that I mind? ``Come
and see me again,'' she said; and she ran off through the garden
tinkling her bell. But suddenly she turned and came back to ask me
anxiously, ``You're looking far from well. Is anything wrong?'' I
couldn't very well tell her that I'd eaten a rice-cake and gone dancing;
so, ``No,'' I said, ``nothing in particular. I did some weighty
thinking, which brought on something of a headache. Indeed I called
today because I fancied that just to talk with you would help me to feel
better.''

``Really? Well, take good care of yourself. Good-bye now.'' She seemed a
tiny bit sorry to leave me, which has completely restored me to the
liveliness I'd felt before the rice-cake bit me. I now felt wonderful
and decided to go home through that tea-plantation where one could have
the pleasure of treading down lumps of half-melted frost. I put my face
through the broken bamboo hedge, and there was Rickshaw Blacky, back
again on the dry chrysanthemums, yawning his spine into a high, black
arch. Nowadays I'm no longer scared of Blacky, but, since any
conversation with him involves the risk of trouble, I endeavor to pass,
cutting him off. But it's not in Blacky's nature to contain his feelings
if he believes himself looked down upon. ``Hey you, Mr.~No-name. You're
very stuck-up these days, now aren't you? You may be living in a
teacher's house, but don't go giving yourself such airs. And stop, I
warn you, trying to make a fool of me.'' Blacky doesn't seem to know
that I am now a celebrity. I wish I could explain the situation to him,
but, since he's not the kind who can understand such things, I decide
simply to offer him the briefest of greetings and then to take my leave
as soon as I decently can.

``A happy New Year, Mr.~Blacky. You do look well, as usual.'' And I lift
up my tail and twist it to the left. Blacky, keeping his tail straight
up, refused to return my salutation.

``What! Happy? If the New Year's happy, then you should be out of your
tiny mind the whole year round. Now push off sharp, you back-end of a
bellows.''

That turn of phrase about the back-end of a bellows sounds distinctly
derogatory, but its semantic content happened to escape me. ``What,'' I
enquired, ``do you mean by the back-end of a bellows?''

``You're being sworn at and you stand there asking its meaning. I give
up! I really do! You really are a New Year's nit.''

A New Year's nit sounds somewhat poetic, but its meaning is even more
obscure than that bit about the bellows. I would have liked to ask the
meaning for my future reference, but, as it was obvious I'd get no clear
answer, I just stood facing him without a word. I was actually feeling
rather awkward, but just then the wife of Blacky's master suddenly
screamed out, ``Where in hell is that cut of salmon I left here on the
shelf? My God, I do declare that hellcat's been here and snitched it
once again! That's the nastiest cat I've ever seen. See what he'll get
when he comes back!'' Her raucous voice unceremoniously shakes the mild
air of the season, vulgarizing its natural peacefulness. Blacky puts on
an impudent look as if to say, ``If you want to scream your head off,
scream away,'' and he jerked his square chin forward at me as if to say,
``Did you hear that hullaballoo?'' Up to this point I've been too busy
talking to Blacky to notice or think about anything else; but now,
glancing down, I see between his legs a mud-covered bone from the
cheapest cut of salmon.

``So you've been at it again!'' Forgetting our recent exchanges, I
offered Blacky my usual flattering exclamation. But it was not enough to
restore him to good humor.

``Been at it! What the hell d'you mean, you saucy blockhead? And what do
you mean by saying `again' when this is nothing but a skinny slice of
the cheapest fish? Don't you know who I am! I'm Rickshaw Blacky, damn
you.'' And, having no shirtsleeves to roll up, he lifts an aggressive
right front-paw as high as his shoulder.

``I've always known you were Mr.~Rickshaw Blacky.''

``If you knew, why the hell did you say I'd been at it again? Answer
me!'' And he blows out over me great gusts of oven breath. Were we
humans, I would be shaken by the collar of my coat. I am somewhat taken
aback and am indeed wondering how to get out of the situation, when that
woman's fearful voice is heard again.

``Hey! Mr.~Westbrook. You there, Westbrook, can you hear me? Listen, I
got something to say. Bring me a pound of beef, and quick. O.K.?
Understand? Beef that isn't tough. A pound of it. See?'' Her
beef-demanding tones shatter the peace of the whole neighborhood.

``It's only once a year she orders beef and that's why she shouts so
loud. She wants the entire neighborhood to know about her marvellous
pound of beef. What can one do with a woman like that!'' asked Blacky
jeeringly as he stretched all four of his legs. As I can find nothing to
say in reply, I keep silent and watch.

``A miserable pound just simply will not do. But I reckon it can't be
helped. Hang on to that beef. I'll have it later.'' Blacky communes with
himself as though the beef had been ordered specially for him.

``This time you're in for a real treat. That's wonderful!'' With these
words I'd hoped to pack him off to his home.

But Blacky snarled, ``That's nothing to do with you. Just shut your big
mouth, you!'' and using his strong hind-legs, he suddenly scrabbles up a
torrent of fallen icicles which thuds down on my head. I was taken
completely aback, and, while I was still busy shaking the muddy debris
off my body, Blacky slid off through the hedge and disappeared.
Presumably to possess himself of Westbrook's beef.

When I get home I find the place unusually springlike and even the
master is laughing gaily. Wondering why, I hopped onto the veranda, and,
as I padded to sit beside the master, noticed an unfamiliar guest. His
hair is parted neatly and he wears a crested cotton surcoat and a
duck-cloth hakama. He looks like a student and, at that, an extremely
serious one. Lying on the corner of my master's small hand-warming
brazier, right beside the lacquer cigarette-box, there's a visiting card
on which is written, ``To introduce Mr.~Beauchamp Blowlamp: from
Coldmoon.''

Which tells me both the name of this guest and the fact that he's a
friend of Coldmoon. The conversation going on between host and guest
sounds enigmatic because I missed the start of it. But I gather that it
has something to do with Waverhouse, the aesthete whom I have had
previous occasion to mention.

``And he urged me to come along with him because it would involve an
ingenious idea, he said.'' The guest is talking calmly.

``Do you mean there was some ingenious idea involved in lunching at
aWestern style restaurant?'' My master pours more tea for the guest and
pushes the cup toward him.

``Well, at the time I did not understand what this ingenious idea could
be, but, since it was his idea, I thought it bound to be something
interesting and\ldots{}''

``So you accompanied him. I see.''

``Yes, but I got a surprise.''

The master, looking as if to say, ``I told you so,'' gives me a whack on
the head. Which hurts a little. ``I expect it proved somewhat farcical.
He's rather that way inclined.'' Clearly, he has suddenly remembered
that business with Andrea del Sarto.

``Ah yes? Well, as he suggested we would be eating something
special\ldots{}''

``What did you have?''

``First of all, while studying the menu, he gave me all sorts of
information about food.''

``Before ordering any?''

``Yes.''

``And then?''

``And then, turning to a waiter, he said, `There doesn't seem to be
anything special on the card.' The waiter, not to be outdone, suggested
roast duck or veal chops. Whereupon Waverhouse remarked quite sharply
that we hadn't come a very considerable distance just for common or
garden fare. The waiter, who didn't understand the significance of
common or garden, looked puzzled and said nothing.''

``So I would imagine.''

``Then, turning to me, Waverhouse observed that in France or in England
one can obtain any amount of dishes cooked à la Tenmei or à la Manyō but
that in Japan, wherever you go, the food is all so stereotyped that one
doesn't even feel tempted to enter a restaurant of the so-called Western
style. And so on and so on. He was in tremendous form. But has he ever
been abroad?''

``Waverhouse abroad? Of course not. He's got the money and the time. If
he wanted to, he could go off anytime. He probably just converted his
future intention to travel into the past tense of widely traveled
experience as a sort of joke.'' The master flatters himself that he has
said something witty and laughs invitingly. His guest looks largely
unimpressed.

``I see. I wondered when he'd been abroad. I took everything he said
quite seriously. Besides, he described such things as snail soup and
stewed frogs as though he'd really seen them with his own two eyes.''

``He must have heard about them from someone. He's adept at such
terminological inexactitudes.''

``So it would seem,'' and Beauchamp stares down at the narcissus in a
vase. He seems a little disappointed.

``So, that then was his ingenious idea, I take it?'' asks the master
still in quest of certainties.

``No, that was only the beginning. The main part's still to come.''

``Ah!'' The master utters an interjection mingled with curiosity.

``Having finished his dissertation on matters gastronomical and
European, he proposed `since it's quite impossible to obtain snails or
frogs, however much we may desire them, let's at least have moat-bells.

What do you say?' And without really giving the matter any thought at
all, I answered, `Yes, that would be fine.'''

``Moat-bells sound a little odd.''

``Yes, very odd, but because Waverhouse was speaking so seriously, I
didn't then notice the oddity.'' He seems to be apologizing to my master
for his carelessness.

``What happened next?'' asks my master quite indifferently and without
any sign of sympathetic response to his guest's implied apology.

``Well, then he told the waiter to bring moat-bells for two. The waiter
said, `Do you mean meatballs, sir?' but Waverhouse, assuming an ever
more serious expression, corrected him with gravity. `No, not meatballs,
moat-bells.'\,''

``Really? But is there any such dish as moat-bells?''

``Well I thought it sounded somewhat strange, but as Waverhouse was so
calmly sure and is so great an authority on all things
Occidental---remember it was then my firm belief that he was widely
traveled---I too joined in and explained to the waiter, `Moat-bells, my
good man, moat-bells.'\,''

``What did the waiter do?''

``The waiter---it's really rather funny now one comes to think back on
it---looked thoughtful for a while and then said, `I'm terribly sorry
sir, but today, unfortunately, we have no moat-bells. Though should you
care for meatballs we could serve you, sir, immediately.' Waverhouse
thereupon looked extremely put out and said, `So we've come all this
long way for nothing. Couldn't you really manage moat-bells? Please do
see what can be done,' and he slipped a small tip to the waiter. The
waiter said he would ask the cook again and went off into the kitchen.''

``He must have had his mind dead set on eating moat-bells.''

``After a brief interval the waiter returned to say that if moat-bells
were ordered specially they could be provided, but that it would take a
long time. Waverhouse was quite composed. He said, `It's the New Year
and we are in no kind of hurry. So let's wait for it?' He drew a cigar
from the inside of his Western suit and lighted up in the most leisurely
manner. I felt called upon to match his cool composure so, taking the
Japan News from my kimono pocket, I started reading it. The waiter
retired for further consultations.''

``What a business!'' My master leans forward, showing quite as much
enthusiasm as he does when reading war news in the dailies.

``The waiter re-emerged with apologies and the confession that, of late,
the ingredients of moat-bells were in such short supply that one could
not get them at Kameya's nor even down at No.~15 in Yokohama.

He expressed regret, but it seemed certain that the material for
moat-bells would not be back in stock for some considerable time.

Waverhouse then turned to me and repeated, over and over again,

`What a pity, and we came especially for that dish.' I felt that I had
to say something, so I joined him in saying, `Yes, it's a terrible
shame! Really, a great, great pity!'''

``Quite so,'' agrees my master, though I myself don't follow his
reasoning.

``These observations must have made the waiter feel quite sorry, for he
said, `When, one of these days, we do have the necessary ingredients,
we'd be happy if you would come, sir, and sample our fare.' But when
Waverhouse proceeded to ask him what ingredients the restaurant did use,
the waiter just laughed and gave no answer. Waverhouse then pressingly
enquired if the key-ingredient happened to be Tochian (who, as you know,
is a haiku poet of the Nihon School); and d'you know, the waiter
answered, `Yes, it is, sir, and that is precisely why none is currently
available even in Yokohama. I am indeed,' he added, `most regretful,
sir.'\,''

``Ha-ha-ha! So that's the point of the story? How very funny!'' and the
master, quite unlike his usual self, roars with laughter. His knees
shake so much that I nearly tumble off. Paying no regard to my
predicament, the master laughs and laughs. He seems suddenly deeply
pleased to realize that he is not alone in being gulled by Andrea del
Sarto.

``And then, as soon as we were out in the street, he said `You see,
we've done well. That ploy about the moat-bells was really rather good,
wasn't it?' and he looked as pleased as punch. I let it be known that I
was lost in admiration, and so we parted. However, since by then it was
well past the lunch-hour, I was nearly starving.''

``That must have been very trying for you.'' My master shows, for the
first time, a sympathy to which I have no objection. For a while there
was a pause in the conversation and my purring could be heard by host
and guest.

Mr.~Beauchamp drains his cup of tea, now quite cold, in one quick gulp
and with some formality remarks, ``As a matter-of-fact I've come today
to ask a favor from you.''

``Yes? And what can I do for you?'' My master, too, assumes a formal
face.

``As you know, I am a devotee of literature and art\ldots{}''

``That's a good thing,'' replies my master quite encouragingly.

``Since a little while back, I and a few like-minded friends have got
together and organized a reading group. The idea is to meet once a month
for the purpose of continued studying in this field. In fact, we've
already had the first meeting at the end of last year.''

``May I ask you a question? When you say, like that, a reading group, it
suggests that you engage in reading poetry and prose in a singsong tone.
But in what sort of manner do you, in fact, proceed?''

``Well, we are beginning with ancient works but we intend to consider
the works of our fellow members.''

``When you speak of ancient works, do you mean something like Po Chu-i's
Lute Song?''

``No.''

``Perhaps things like Buson's mixture of haiku and Chinese verse?''

``No.''

``What kinds of thing do you then do?''

``The other day, we did one of Chikamatsu's lovers' suicides.''

``Chikamatsu? You mean the Chikamatsu who wrote jōruri plays?''

There are not two Chikamatsus. When one says Chikamatsu, one does indeed
mean Chikamatsu the playwright and could mean nobody else. I thought my
master really stupid to ask so fool a question. However, oblivious to my
natural reactions, he gently strokes my head. I calmly let him go on
stroking me, justifying my compliance with the reflection that so small
a weakness is permissible when there are those in the world who admit to
thinking themselves under loving observation by persons who merely
happen to be cross-eyed.

Beauchamp answers, ``Yes,'' and tries to read the reaction on my
master's face.

``Then is it one person who reads or do you allot parts among you?''

``We allot parts and each reads out the appropriate dialogue. The idea
is to empathize with the characters in the play and, above all, to bring
out their individual personalities. We do gestures as well. The main
thing is to catch the essential character of the era of the play.
Accordingly, the lines are read out as if spoken by each character,
which may perhaps be a young lady or possibly an errand-boy.''

``In that case it must be like a play.''

``Yes, almost the only things missing are the costumes and the
scenery.''

``May I ask if your reading was a success?''

``For a first attempt, I think one might claim that it was, if anything,
a success.''

``And which lovers' suicide play did you perform on the last occasion?''

``We did a scene in which a boatman takes a fare to the red light
quarter of Yoshiwara.''

``You certainly picked on a most irregular incident, didn't you?'' My
master, being a teacher, tilts his head a little sideways as if
regarding something slightly doubtful. The cigarette smoke drifting from
his nose passes up by his ear and along the side of his head.

``No, it isn't that irregular. The characters are a passenger, a
boatman, a high-class prostitute, a serving-girl, an ancient crone of a
brothel-attendant, and, of course, a geisha-registrar. But that's all.''
Beauchamp seems utterly unperturbed. My master, on hearing the words ``a
high-class prostitute,'' winces slightly but probably only because he's
not well up in the meanings of such technical terms as nakai, yarite,
and kemban.

He seeks to clear the ground with a question. ``Does not nakai signify
something like a maid-servant in a brothel?''

``Though I have not yet given the matter my full attention, I believe
that nakai signifies a serving-girl in a teahouse and that yarite is
some sort of an assistant in the women's quarters.'' Although Beauchamp
recently claimed that his group seeks to impersonate the actual voices
of the characters in the plays, he does not seem to have fully grasped
the real nature of yarite and nakai.

``I see, nakai belong to a teahouse while yarite live in a brothel.
Next, are kemban human beings or is it the name of a place? If human,
are they men or women?''

``Kemban, I rather think, is a male human being.''

``What is his function?''

``I've not yet studied that far. But I'll make inquiries, one of these
days.''

Thinking, in the light of these revelations, that the play-readings must
be affairs extraordinarily ill-conducted, I glance up at my master's
face.

Surprisingly, I find him looking serious. ``Apart from yourself, who
were the other readers taking part?''

``A wide variety of people. Mr.~K, a Bachelor of Law, played the
high-class prostitute, but his delivery of that woman's sugary dialogue
through his very male mustache did, I confess, create a slightly queer
impression. And then there was a scene in which this oiran was seized
with spasms\ldots{}''

``Do your readers extend their reading activities to the simulation of
spasms?'' asked my master anxiously.

``Yes indeed; for expression is, after all, important.'' Beauchamp
clearly considers himself a literary artist à l'outrance.

``Did he manage to have his spasms nicely?'' My master has made a witty
remark.

``The spasms were perhaps the only thing beyond our capability at such a
first endeavor.'' Beauchamp, too, is capable of wit.

``By the way,'' asks my master, ``what part did you take?''

``I was the boatman.''

``Really? You, the boatman!'' My master's tone was such as to suggest
that, if Beauchamp could be a boatman, he himself could be a
geisha-registrar. Switching his tone to one of simple candor, he then
asks: ``Was the role of the boatman too much for you?''

Beauchamp does not seem particularly offended. Maintaining the same calm
voice, he replies, ``As a matter of fact, it was because of this boatman
that our precious gathering, though it went up like a rocket, came down
like a stick. It so happened that four or five girl students are living
in the boarding house next door to our meeting hall. I don't know how,
but they found out when our reading was to take place. Anyway, it
appears that they came and listened to us under the window of the hall.

I was doing the boatman's voice, and, just when I had warmed up nicely
and was really getting into the swing of it---perhaps my gestures were a
little over-exaggerated---the girl students, all of whom had managed to
control their feelings up to that point, thereupon burst out into
simultaneous cachinnations. I was of course surprised, and I was of
course embarrassed: indeed, thus dampened, I could not find it in me to
continue. So our meeting came to an end.''

If this were considered a success, even for a first meeting, what would
failure have been like? I could not help laughing. Involuntarily, my
Adam's apple made a rumbling noise. My master, who likes what he takes
to be purring, strokes my head ever more and more gently. I'm thankful
to be loved just because I laugh at someone, but at the same time I feel
a bit uneasy.

``What very bad luck!'' My master offers condolences despite the fact
that we are still in the congratulatory season of the New Year.

``As for our second meeting, we intend to make a great advance and
manage things in the grand style. That, in fact, is the very reason for
my call today: we'd like you to join our group and help us.''

``I can't possibly have spasms.'' My negative-minded master is already
poised to refuse.

``No, you don't have to have spasms or anything like that. Here's a list
of the patron members.'' So saying, Beauchamp very carefully produced a
small notebook from a purple-colour carrying-wrapper. He opened the
notebook and placed it in front of my master's knees. ``Will you please
sign and make your seal-mark here?'' I see that the book contains the
names of distinguished Doctors of Literature and Bachelors of Arts of
this present day, all neatly mustered in full force.

``Well, I wouldn't say I object to becoming a supporter, but what sort
of obligations would I have to meet?'' My oyster-like master displays
his apprehensions\ldots{}

``There's hardly any obligation. We ask nothing from you except a
signature expressing your approval.''

``Well, in that case, I'll join.'' As he realizes that there is no real
obligation involved, he suddenly becomes lighthearted. His face assumes
the expression of one who would sign even a secret commitment to engage
in rebellion, provided it was clear that the signature carried no
binding obligation. Besides, it is understandable that he should assent
so eagerly: for to be included, even by name only, among so many names
of celebrated scholars is a supreme honor for one who has never before
had such an opportunity. ``Excuse me,'' and my master goes off to the
study to fetch his seal. I am tipped to fall unceremoniously onto the
matting.

Beauchamp helps himself to a slice of sponge cake from the cake-bowl and
crams it into his mouth. For a while he seems to be in pain, mumbling.
Just for a second I am reminded of my morning experience with the
rice-cake. My master reappears with his seal just as the sponge cake
settles down in Beauchamp's bowels. My master does not seem to notice
that a piece of sponge cake is missing from the cake-bowl. If he does, I
shall be the first to be suspected.

Mr.~Beauchamp having taken his departure, my master reenters the study
where he finds on his desk a letter from friend Waverhouse.

``I wish you a very happy New Year\ldots{}''

My master considers the letter to have started with an unusual
seriousness. Letters from Waverhouse are seldom serious. The other day,
for instance, he wrote: ``Of late, as I am not in love with any woman, I
receive no love letters from anywhere. As I am more or less alive,
please set your mind at ease.'' Compared with which, this New Year's
letter is exceptionally matter-of-fact:

I would like to come and see you, but I am so very extremely busy every
day because, contrary to your negativism, I am planning to greet this
New Year, a year unprecedented in all history, with as positive an
attitude as is possible. Hoping you will understand\ldots{}

My master quite understands, thinking that Waverhouse, being Waverhouse,
must be busy having fun during the New Year season.

Yesterday, finding a minute to spare, I sought to treat Mr.~Beauchamp to
a dish of moat-bells. Unfortunately, due to a shortage of their
ingredients, I could not carry out my intention. It was most
regrettable\ldots{}

My master smiles, thinking that the letter is falling more into the
usual pattern.

Tomorrow there will be a card party at a certain Baron's house; the day
after tomorrow a New Year's banquet at the Society of Aesthetes; and the
day after that, a welcoming party for Professor Toribe; and on the day
thereafter\ldots{} My master, finding it rather a bore, skips a few
lines.

So you see, because of these incessant parties--- nō song parties, haiku
parties, tanka parties, even parties for New Style Poetry, and so on and
so on, I am perpetually occupied for quite some time. And that is why I
am obliged to send you this New Year's letter instead of calling on you
in person. I pray you will forgive me\ldots{}

``Of course you do not have to call on me.'' My master voices his answer
to the letter.

Next time that you are kind enough to visit me, I would like you to stay
and dine. Though there is no special delicacy in my poor larder, at
least I hope to be able to offer you some moat-bells, and I am indeed
looking forward to that pleasure\ldots{}

``He's still brandishing his moat-bells,'' muttered my master, who,
thinking the invitation an insult, begins to feel indignant.

However, because the ingredients necessary for the preparation of
moat-bells are currently in rather short supply, it may not be possible
to arrange it. In which case, I will offer you some peacocks'
tongues\ldots{}

``Aha! So he's got two strings to his bow,'' thinks my master and cannot
resist reading the rest of the letter.

As you know, the tongue meat per peacock amounts to less than half the
bulk of the small finger. Therefore, in order to satisfy your gluttonous
stomach\ldots{}

``What a pack of lies,'' remarks my master in a tone of resignation.

I think one needs to catch at least twenty or thirty peacocks. However,
though one sees an occasional peacock, maybe two, at the zoo or at the
Asakusa Amusement Center, there are none to be found at my poulterer's,
which is occasioning me pain, great pain\ldots{}

``You're having that pain of your own free will.'' My master shows no
evidence of gratitude.

The dish of peacocks' tongues was once extremely fashionable in Rome
when the Roman Empire was in the full pride of its prosperity. How I
have always secretly coveted after peacocks' tongues, that acme of
gastronomical luxury and elegance, you may well imagine\ldots{}

``I may well imagine, may I? How ridiculous.'' My master is extremely
cold.

From that time forward until about the sixteenth century, peacock was an
indispensable delicacy at all banquets. If my memory serves me, when the
Earl of Leicester invited Queen Elizabeth to Kenilworth, peacocks'
tongues were on the menu. And in one of Rembrandt's banquet scenes, a
peacock is clearly to be seen, lying in its pride upon the table\ldots{}

My master grumbles that if Waverhouse can find time to compose a history
of the eating of peacocks, he cannot really be so busy.

Anyway, if I go on eating good food as I have been doing recently, I
will doubtless end up one of these days with a stomach weak as
yours\ldots{}

``\,`Like yours' is quite unnecessary. He has no need to establish me as
the prototypical dyspeptic,'' grumbles my master.

According to historians, the Romans held two or three banquets every
day. But the consumption of so much good food, while sitting at a large
table two or three times a day, must produce in any man, however sturdy
his stomach, disorders in the digestive functions. Thus nature has, like
you\ldots{}

``\,`Like you,' again, what impudence!''

But they, who studied long and hard simultaneously to enjoy both luxury
and exuberant health, considered it vital not only to devour
disproportionately large quantities of delicacies, but also to maintain
the bowels in full working order. They accordingly devised a secret
formula\ldots{}

``Really?'' My master suddenly becomes enthusiastic.

They invariably took a post-prandial bath. After the bath, utilizing
methods whose secret has long been lost, they proceeded to vomit up
everything they had swallowed before the bath. Thus were the insides of
their stomachs kept scrupulously clean. Having so cleansed their
stomachs, they would sit down again at the table and there savor to the
uttermost the delicacies of their choice. Then they took a bath again
and vomited once more. In this way, though they gorged on their favorite
dishes to their hearts' content, none of their internal organs suffered
the least damage. In my humble opinion, this was indeed a case of having
one's cake and eating it.

``They certainly seem to have killed two or more birds with one stone.''
My master's expression is one of envy.

Today, this twentieth century, quite apart from the heavy traffic and
the increased number of banquets, when our nation is in the second year
of a war against Russia, is indeed eventful. I, consequently, firmly
believe that the time has come for us, the people of this victorious
country, to bend our minds to study of the truly Roman art of bathing
and vomiting. Otherwise, I am afraid that even the precious people of
this mighty nation will, in the very near future, become, like you,
dyspeptic\ldots{}

``What, again like me? An annoying fellow,'' thinks my master.

Now suppose that we, who are familiar with all things Occidental, by
study of ancient history and legend contrive to discover the secret
formula that has long been lost; then to make use of it now in our Meiji
Era would be an act of virtue. It would nip potential misfortune in the
bud, and, moreover, it would justify my own everyday life which has been
one of constant indulgence in pleasure.

My master thinks all this a trifle odd.

Accordingly, I have now, for some time, been digging into the relevant
works of Gibbon, Mommsen, and Goldwin Smith, but I am extremely sorry to
report that, so far, I have gained not even the slightest clue to the
secret. However, as you know, I am a man who, once set upon a course,
will not abandon it until my object is achieved. Therefore my belief is
that a rediscovery of the vomiting method is not far off. I will let you
know when it happens. Incidentally, I would prefer postponing that feast
of moat-bells and peacocks' tongues, which I've mentioned above, until
the discovery has actually been made. Which would not only be convenient
to me, but also to you who suffer from a weak stomach.

``So, he's been pulling my leg all along. The style of writing was so
sober that I have read it all, and took the whole thing seriously.

Waverhouse must indeed be a man of leisure to play such a practical joke
on me,'' said my master through his laughter.

Several days then passed without any particular event. Thinking it too
boring to spend one's time just watching the narcissus in a white vase
gradually wither, and the slow blossoming of a branch of the
blue-stemmed plum in another vase, I have gone around twice to look for
Tortoiseshell, but both times unsuccessfully. On the first occasion I
thought she was just out, but on my second visit I learnt that she was
ill.

Hiding myself behind the aspidistra beside a wash-basin, I heard the
following conversation which took place between the mistress and her
maid on the other side of the sliding paper-door.

``Is Tortoiseshell taking her meal?''

``No, madam, she's eaten nothing this morning. I've let her sleep on the
quilt of the foot-warmer, well wrapped up.'' It does not sound as if
they spoke about a cat. Tortoiseshell is being treated as if she were a
human.

As I compare this situation with my own lot, I feel a little envious but
at the same time I am not displeased that my beloved cat should be
treated with such kindness.

``That's bad. If she doesn't eat she will only get weaker.''

``Yes indeed, madam. Even me, if I don't eat for a whole day, I couldn't
work at all the next day.''

The maid answers as though she recognized the cat as an animal superior
to herself. Indeed, in this particular household the cat may well be
more important than the maid.

``Have you taken her to see a doctor?''

``Yes, and the doctor was really strange. When I went into his
consulting room carrying Tortoiseshell in my arms, he asked me if I'd
caught a cold and tried to take my pulse. I said `No, Doctor, it is not
I who am the patient, this is the patient,' and I placed Tortoiseshell
on my knees.

The doctor grinned and said he had no knowledge of the sicknesses of
cats, and that if I just left it, perhaps it would get better. Isn't he
too terrible? I was so angry that I told him, `Then, please don't bother
to examine her, she happens to be our precious cat.' And I snuggled
Tortoiseshell back into the breast of my kimono and came straight
home.''

``Truly so.''

``Truly so'' is one of those elegant expressions that one would never
hear in my house. One has to be the thirteenth Shogun's widowed wife's
somebody's something to be able to use such a phrase. I was much
impressed by its refinement.

``She seems to be sniffling\ldots{}''

``Yes, I'm sure she's got a cold and a sore throat; whenever one has a
cold, one suffers from an honorable cough.''

As might be expected from the maid of the thirteenth Shogun's somebody's
something, she's quick with honorifics.

``Besides, recently, there's a thing they call consumption\ldots{}''

``Indeed these days one cannot be too careful. What with the increase in
all these new diseases like tuberculosis and the black plague.''

``Things that did not exist in the days of the Shogunate are all no good
to anyone. So you be careful too!''

``Is that so, madam?''

The maid is much moved.

``I don't see how she could have caught a cold, she hardly ever went
out\ldots{}''

``No, but you see she's recently acquired a bad friend.''

The maid is as highly elated as if she were telling a State secret.

``A bad friend?''

``Yes, that tatty-looking tom at the teacher's house in the main
street.''

``D'you mean that teacher who makes rude noises every morning?''

``Yes, the one who makes the sounds like a goose being strangled every
time he washes his face.''

The sound of a goose being strangled is a clever description. Every
morning when my master gargles in the bathroom he has an odd habit of
making a strange, unceremonious noise by tapping his throat with his
toothbrush. When he is in a bad temper he croaks with a vengeance; when
he is in a good temper, he gets so pepped up that he croaks even more
vigorously. In short, whether he is in a good or a bad temper, he croaks
continually and vigorously. According to his wife, until they moved to
this house he never had the habit; but he's done it every day since the
day he first happened to do it. It is rather a trying habit. We cats
cannot even imagine why he should persist in such behavior. Well, let
that pass. But what a scathing remark that was about ``a tatty-looking
tom.'' I continue to eavesdrop.

``What good can he do making that noise! Under the Shogunate even a
lackey or a sandal-carrier knew how to behave; and in a residential
quarter there was no one who washed his face in such a manner.''

``I'm sure there wasn't, madam.''

That maid is all too easily influenced, and she uses ``madam'' far too
often.

``With a master like that what's to be expected from his cat? It can
only be a stray. If he comes round here again, beat him.''

``Most certainly I'll beat him. It must be all his fault that
Tortoiseshell's so poorly. I'll take it out on him, that I will.''

How false these accusations laid against me! But judging it rash to
approach too closely, I came home without seeing Tortoiseshell.

When I return, my master is in the study meditating in the middle of
writing something. If I told him what they say about him in the house of
the two-stringed harp, he would be very angry; but, as the saying goes,
ignorance is bliss. There he sits, posing like a sacred poet, groaning.

Just then, Waverhouse, who has expressly stated in his New Year letter
that he would be too busy to call for some long time, dropped in.

``Are you composing a new-style poem or something? Show it to me if it's
interesting.''

``I considered it rather impressive prose, so I thought I'd translate
it,'' answers my master somewhat reluctantly.

``Prose? Whose prose?''

``Don't know whose.''

``I see, an anonymous author. Among anonymous works, there are indeed
some extremely good ones. They are not to be slighted. Where did you
find it?''

``The Second Reader,'' answers my master with imperturbable calmness.

``The Second Reader? What's this got to do with the Second Reader?''

``The connection is that the beautifully written article which I'm now
translating appears in the Second Reader.''

``Stop talking rubbish. I suppose this is your idea of a last minute
squaring of accounts for the peacocks' tongues?''

``I'm not a braggart like you,'' says my master and twists his mustache.
He is perfectly composed.

``Once when someone asked Sanyo whether he'd lately seen any fine pieces
of prose, that celebrated scholar of the Chinese classics produced a
dunning letter from a packhorse man and said, `This is easily the finest
piece of prose that has recently come to my attention.' Which implies
that your eye for the beautiful might, contrary to one's expectations,
actually be accurate. Read your piece aloud. I'll review it for you,''
says Waverhouse as if he were the originator of all aesthetic theories
and practice. My master starts to read in the voice of a Zen priest,
reading that injunction left by the Most Reverend Priest Daitō.
``\,`Giant Gravitation,'\,'' he intoned.

``What on earth is giant gravitation?''

``\,`Giant Gravitation' is the title.''

``An odd title. I don't quite understand.''

``The idea is that there's a giant whose name is Gravitation.''

``A somewhat unreasonable idea but, since it's a title, I'll let that
pass.

All right, carry on with the text. You have a good voice. Which makes it
rather interesting.''

``Right, but no more interruptions.'' My master, having laid down his
prior conditions, begins to read again.

Kate looks out of the window. Children are playing ball. They throw the
ball high up in the sky. The ball rises up and up. After a while the
ball comes down. They throw it high again: twice, three times. Every
time they throw it up, the ball comes down. Kate asks why it comes down
instead of rising up and up. ``It is because a giant lives in the
earth,'' replies her mother. ``He is the Giant Gravitation. He is
strong. He pulls everything toward him. He pulls the houses to the
earth. If he didn't they would fly away. Children, too, would fly away.
You've seen the leaves fall, haven't you? That's because the Giant
called them. Sometimes you drop a book. It's because the Giant
Gravitation asks for it. A ball goes up in the sky. The giant calls for
it. Down it falls.

``Is that all?''

``Yes, isn't it good?''

``All right, you win. I wasn't expecting such a present in return for
the moat-bells.''

``It wasn't meant as a return present, or anything like that. I
translated it because I thought it was good. Don't you think it's
good?'' My master stares deep into the gold-rimmed spectacles.

``What a surprise! To think that you of all people had this
talent\ldots{} Well, well! I've certainly been taken in right and proper
this time. I take my hat off to you.'' He is alone in his understanding.
He's talking to himself. The situation is quite beyond my master's
grasp.

``I've no intention of making you doff your cap. I translated this text
simply because I thought it was an interesting piece of writing.''

``Indeed, yes! Most interesting! Quite as it should be! Smashing! I feel
small.''

``You don't have to feel small. Since I recently gave up painting in
watercolors, I've been thinking of trying my hand at writing.''

``And compared with your watercolors, which showed no sense of
perspective, no appreciation of differences in tone, your writings are
superb. I am lost in admiration.''

``Such encouraging words from you are making me positively enthusiastic
about it,'' says my master, speaking from under his continuing
mis-apprehension.

Just then Mr.~Coldmoon enters with the usual greeting.

``Why, hello,'' responds Waverhouse, ``I've just been listening to a
terrifically fine article and the curtain has been rung down upon my
moat-bells.'' He speaks obliquely about something incomprehensible.

``Have you really?'' The reply is equally incomprehensible. It is only
my master who seems not to be in any particularly light humor.

``The other day,'' he remarked, ``a man called Beauchamp Blowlamp came
to see me with an introduction from you.''

``Ah, did he? Beauchamp's an uncommonly honest person, but, as he is
also somewhat odd, I was afraid that he might make himself a nuisance to
you. However, since he had pressed me so hard to be introduced to
you\ldots{}''

``Not especially a nuisance\ldots{}''

``Didn't he, during his visit, go on at length about his name?''

``No, I don't recall him doing so.''

``No? He's got a habit at first meeting of expatiating upon the
singularity of his name.''

``What is the nature of that singularity?'' butts in Waverhouse, who has
been waiting for something to happen.

``He gets terribly upset if someone pronounces Beauchamp as Beecham.''

``Odd!'' said Waverhouse, taking a pinch of tobacco from his
gold-painted, leather tobacco pouch.

``Invariably he makes the immediate point that his name is not Beecham
Blowlamp but Bo-champ Blowlamp.''

``That's strange,'' and Waverhouse inhales pricey tobacco-smoke deep
into his stomach.

``It comes entirely from his craze for literature. He likes the effect
and is inexplicably proud of the fact that his personal name and his
family name can be made to rhyme with each other. That's why when one
pronounces Beauchamp incorrectly, he grumbles that one does not
appreciate what he is trying to get across.''

``He certainly is extraordinary.'' Getting more and more interested,
Waverhouse hauls back the pipe smoke from the bottom of his stomach to
let it loose at his nostrils. The smoke gets lost en route and seems to
be snagged in his gullet. Transferring the pipe to his hand, he coughs
chokingly.

``When he was here the other day, he said he'd taken the part of a
boatman at a meeting of his Reading Society, and that he'd gotten
himself laughed at by a gaggle of schoolgirls,'' says my master with a
laugh.

``Ah, that's it, I remember.'' Waverhouse taps his pipe upon his knees.

This strikes me as likely to prove dangerous, so I move a little way
farther off. ``That Reading Society, now. The other day when I treated
him to moat-bells, he mentioned it. He said they were going to make
their second meeting a grand affair by inviting well-known literary men,
and he cordially invited me to attend. When I asked him if they would
again try another of Chikamatsu's dramas of popular life, he said no and
that they'd decided on a fairly modern play, The Golden Demon. I asked
him what role he would take and he said, `I'm going to play O-miya.'
Beauchamp as O-miya would certainly be worth seeing. I'm determined to
attend the meeting in his support.''

``It's going to be interesting, I think,'' says Coldmoon and he laughs
in an odd way.

``But he is so thoroughly sincere, which is good, and has no hint of
frivolousness about him. Quite different from Waverhouse, for
instance.'' My master is revenged for Andrea del Sarto, for peacocks'
tongues, and for moat-bells all in one go. Waverhouse appears to take no
notice of the remark.

``Ah well, when all's said and done, I'm nothing but a chopping board at
Gyōtoku.''

``Yes, that's about it,'' observes my master, although in fact he does
not understand Waverhouse's involved method of describing himself as a
highly sophisticated simpleton. But not for nothing has he been so many
years a schoolteacher. He is skilled in prevarication, and his long
experience in the classrooms can be usefully applied at such awkward
moments in his social life.

``What is a chopping board at Gyōtoku?'' asks the guileless Coldmoon.

My master looks toward the alcove and pulverizes that chopping board at
Gyōtoku by saying, ``Those narcissi are lasting well. I bought them on
my way home from the public baths toward the end of last year.''

``Which reminds me,'' says Waverhouse, twirling his pipe, ``that at the
end of last year I had a really most extraordinary experience.''

``Tell us about it.'' My master, confident that the chopping board is
now safely back in Gyōtoku, heaves a sigh of relief. The extraordinary
experience of Mr.~Waverhouse fell thus upon our ears:

``If I remember correctly, it was on the twenty-seventh of December.

Beauchamp had said he would like to come and hear me talk upon matters
literary, and had asked me to be sure to be in. Accordingly, I waited
for him all the morning but he failed to turn up. I had lunch and was
seated in front of the stove reading one of Pain's humorous books, when
a letter arrived from my mother in Shizuoka. She, like all old women,
still thinks of me as a child. She gives me all sorts of advice; that I
mustn't go out at night when the weather's cold; that unless the room is
first well-heated by a stove, I'll catch my death of cold every time I
take a bath. We owe much to our parents. Who but a parent would think of
me with such solicitude? Though normally I take things lightly and as
they come, I confess that at that juncture the letter affected me
deeply. For it struck me that to idle my life away, as indeed I do, was
rather a waste. I felt that I must win honor for my family by producing
a masterwork of literature or something like that. I felt I would like
the name of Doctor Waverhouse to become renowned, that I should be
acclaimed as a leading figure in Meiji literary circles, while my mother
is still alive.

Continuing my perusal of the letter, I read, `You are indeed lucky.
While our young people are suffering great hardships for the country in
the war against Russia, you are living in happy-go-lucky idleness as if
life were one long New Year's party organized for your particular
benefit!'

Actually, I'm not as idle as my mother thinks. But she then proceeded to
list the names of my classmates at elementary school who had either died
or had been wounded in the present war. As, one after another, I read
those names, the world grew hollow, all human life quite futile.

And she ended her letter by saying, `since I am getting old, perhaps
this NewYear's rice-cakes will be my last\ldots{}' You will understand
that, as she wrote so very dishearteningly, I grew more and more
depressed. I began to yearn for Beauchamp to come soon, but somehow he
didn't. And at last it was time for supper. I thought of writing in
reply to my mother, and I actually wrote about a dozen lines. My
mother's letter was more than six feet long, but, unable myself to match
such a prodigious performance, I usually excuse myself after writing
some ten lines. As I had been sitting down for the whole of the day, my
stomach felt strange and heavy. Thinking that if Beauchamp did turn up
he could jolly well wait, I went out for a walk to post my letter.
Instead of going toward Fujimicho, which is my usual course, I went,
without my knowing it, out toward the third embankment. It was a little
cloudy that evening and a dry wind was blowing across from the other
side of the moat. It was terribly cold. A train coming from the
direction of Kagurazaka passed with a whistle along the lower part of
the bank. I felt very lonely. The end of the year, those deaths on the
battlefield, senility, life's insecurity, that time and tide wait for no
man, and other thoughts of a similar nature ran around in my head. One
often talks about hanging oneself.

But I was beginning to think that one could be tempted to commit suicide
just at such a time as this. It so happened that at that moment I raised
my head slightly, and, as I looked up to the top of the bank, I found
myself standing right below that very pine tree.''

``That very pine tree? What's that?'' cuts in my master.

``The pine for hanging heads,'' says Waverhouse ducking his noddle.

``Isn't the pine for hanging heads that one at Ko-nodai?'' Coldmoon
amplifies the ripple.

``The pine at Kōnodai is the pine for hanging temple bells. The pine at
Dotesambanchō is the one for hanging heads. The reason why it has
acquired this name is that an old legend says that anyone who finds
himself under this pine tree is stricken with a desire to hang himself.
Though there are several dozen pine trees on the bank, every time
someone hangs himself, it is invariably on this particular tree that the
body is found dangling. I can assure you there are at least two or three
such danglings every year. It would be unthinkable to go and dangle on
any other pine. As I stared at the tree I noted that a branch stuck out
conveniently toward the pavement. Ah! What an exquisitely fashioned
branch. It would be a real pity to leave it as it is. I wish so much
that I could arrange for some human body to be suspended there. I look
around to see if anyone is coming. Unfortunately, no one comes. It can't
be helped.

Shall I hang myself? No, no, if I hang myself, I'll lose my life. I
won't because it's dangerous. But I've heard a story that an ancient
Greek used to entertain banquet parties by giving demonstrations of how
to hang oneself. A man would stand on a stool and the very second that
he put his head through a noose, a second man would kick the stool from
under him. The trick was that the first man would loosen the knot in the
rope just as his stool was kicked away, and so drop down unharmed. If
this story is really true, I've no need to be frightened. So thinking I
might try the trick myself, I place my hand on the branch and find it
bends in a manner precisely appropriate. Indeed the way it bends is
positively aesthetic. I feel extraordinarily happy as I try to picture
myself floating on this branch. I felt I simply must try it, but then I
began to think that it would be inconsiderate if Beauchamp were waiting
for me. Right, I would first see Beauchamp and have the chat I'd
promised; thereafter I could come out again. So thinking, I went home.''

``And is that the happy ending to your story?'' asks my master.

``Very interesting,'' says Coldmoon with a broad grin.

``When I got home, Beauchamp had not arrived. Instead, I found a
postcard from him saying that he was sorry he could not keep our
appointment because of some pressing but unexpected happening, and that
he was looking forward to having a long interview with me in the near
future. I was relieved, and I felt happy, for now I could hang myself
with an easy mind. Accordingly, I hurry back to the same spot, and
then\ldots{}'' Waverhouse, assuming a nonchalant air, gazes at Coldmoon
and my master.

``And then, what happened?'' My master is becoming a little impatient.

``We've now come to the climax,'' says Coldmoon as he twists the strings
of his surcoat.

``And then, somebody had beaten me to it and had already hanged himself.
I'm afraid I missed the chance just by a second. I see now that I had
been in the grip of the God of Death. William James, that eminent
philosopher, would no doubt explain that the region of the dead in the
world of one's subliminal consciousness and the real world in which I
actually exist, must have interacted in mutual response in accordance
with some kind of law of cause and effect. But it really was
extraordinary, wasn't it?'' Waverhouse looks quite demure.

My master, thinking that he has again been taken in, says nothing but
crams his mouth with bean-jam cake and mumbles incoherently.

Coldmoon carefully rakes smooth the ashes in the brazier and casts down
his eyes, grinning; eventually he opens his mouth. He speaks in an
extremely quiet tone.

``It is indeed so strange that it does not seem a thing likely to
happen. On the other hand, because I myself have recently had a similar
kind of experience, I can readily believe it.''

``What! Did you too want to stretch your neck?''

``No, mine wasn't a hanging matter. It seems all the more strange in
that it also happened at the end of last year, at about the same time
and on the same day as the extraordinary experience of Mr.~Waverhouse.''

``That's interesting,'' says Waverhouse. And he, too, stuffs his mouth
with bean-jam cake.

``On that day, there was a year-end party combined with a concert given
at the house of a friend of mine at Mūkōjima. I went there taking my
violin with me. It was a grand affair with fifteen or sixteen young or
married ladies. Everything was so perfectly arranged that one felt it
was the most brilliant event of recent times. When the dinner and the
concert were over, we sat and talked late, and as I was about to take my
leave, the wife of a certain doctor came up to me and asked in whisper
if I knew that Miss O was unwell. A few days earlier, when last I saw
Miss O, she had been looking well and normal. So I was surprised to hear
this news, and my immediate questions elicited the information that she
had become feverish on the very evening of the day when I'd last seen
her, and that she was saying all sorts of curious things in her
delirium. What was worse, every now and again in that delirium, she was
calling my name.''

Not only my master but even Waverhouse refrain from making any such
hackneyed remark as ``you lucky fellow.'' They just listen in silence.

``They fetched a doctor who examined her. According to the doctor's
diagnosis, though the name of the disease was unknown, the high fever
affecting the brain made her condition dangerous unless the
administration of soporifics worked as effectively as was to be hoped
for. As soon as I heard this news, a feeling of something awful grew
within me. It was a heavy feeling, as though one were having a
nightmare, and all the surrounding air seemed suddenly to be solidifying
like a clamp upon my body. On my way home, moreover, I found I could
think of nothing else, and it hurt. That beautiful, that gay, that so
healthy Miss O\ldots{}''

``Just a minute, please. You've mentioned Miss O about two times. If
you've no objection, we'd like to know her name wouldn't we?'' asks
Waverhouse turning to look at my master. The latter evades the question
and says, ``Hmm.''

``No, I won't tell you her name since it might compromise the person in
question.''

``Do you then propose to recount your entire story in such vague,
ambiguous, equivocal, and noncommittal terms?''

``You mustn't sneer. This is a serious story. Anyway, the thought of
that young lady suffering from so odd an ailment filled my heart with
mournful emotion, and my mind with sad reflections on the ephemerality
of life. I felt suddenly depressed beyond all saying, as if every last
ounce of my vitality had, just like that, evaporated from my body. I
staggered on, tottering and wobbling, until I came to the Azuma Bridge.
As I looked down, leaning on the parapet, the black waters---at neap or
ebb, I don't know which---seemed to be coagulating, only just barely
moving. A rickshaw coming from the direction of Hanakawado ran over the
bridge. I watched its lamp grow smaller and smaller until it disappeared
at the Sapporo Beer factory. Again, I looked down at the water.

And at that moment I heard a voice from upstream calling my name. It is
most improbable that anyone should be calling after me at this unlikely
time of night, and, wondering whom it could possibly be, I peered down
to the surface of the water, but I could see nothing in the darkness.
Thinking it must have been my imagination, I had decided to go home,
when I again heard the voice calling my name. I stood dead-still and
listened. When I heard it calling me for the third time, though I was
gripping the parapet firmly, my knees began to tremble uncontrollably.

The voice seemed to be coming either from far away or from the bottom of
the river, but it was unmistakably the voice of Miss O. In spite of
myself I answered, `Yes.' My answer was so loud that it echoed back from
the still water, and, surprised by my own voice, I looked around me in a
startled manner. There was no one to be seen. No dog. No moon. Nothing.
At this very second I experienced a sudden urge to immerse myself in
that total darkness from which the voice had summoned me. And, once
again, the voice of Miss O pierced my ears painfully, appealingly, as if
begging for help. This time I cried, `I'm coming now,' and, leaning well
out over the parapet, I looked down into the somber depths. For, it
seemed to me that the summoning voice was surging powerfully up from
beneath the waves. Thinking that the source of the pleading must lie in
the water directly below me, I at last managed to clamber onto the
parapet. I was determined that, next time the voice called out to me, I
would dive straight in; and, as I stood watching the stream, once again
the thin thread of that pitiful voice came floating up to me. This, I
thought, is it; jumping high with all my strength, I came dropping down
without regret like a pebble, or something.''

``So, you actually did dive in?'' asks my master, blinking his eyes.

``I never thought you'd go as far as that,'' says Waverhouse pinching
the tip of his nose.

``After my dive I became unconscious, and for a while I seemed to be
living in a dream. But eventually I woke up, and, though I felt cold, I
was not at all wet and did not feel as if I had swallowed any water. Yet
I was sure that I had dived. How very strange! Realizing that something
peculiar must have taken place, I looked around me and received a real
shock.

I'd meant to dive into the water but apparently I'd accidentally landed
in the middle of the bridge itself. I felt abysmally regretful. Having,
by sheer mistake, jumped backwards instead of forwards, I'd lost my
chance to answer the summons of the voice.'' Coldmoon smirks and fiddles
with the strings of his surcoat as if they were in some way irksome.

``Ha-ha-ha, how very comical. It's odd that your experience so much
resembles mine. It, too, could be adduced in support of the theories of
Professor James. If you were to write it up in an article entitled `The
Human Response,' it would astound the whole literary world. But what,''
persisted Waverhouse, ``became of the ailing Miss O?''

``When I called at her house a few days ago, I saw her just inside the
gate playing battledore and shuttlecock with her maid. So I expect she
has completely recovered from her illness.''

My master, who for some time has been deep in thought, finally opens his
mouth, and, in a spirit of unnecessary rivalry, remarks, ``I too have a
strange experience to relate.''

``You've got what?'' In Waverhouse's view, my master counts for so
little that he is scarcely entitled to have experiences.

``Mine also occurred at the end of last year.''

``It's queer,'' observed Coldmoon, ``that all last year,'' and he
sniggers. A piece of bean-jam cake adheres to the corner of his chipped
front tooth.

``And it took place, doubtless,'' added Waverhouse, ``at the very same
time on the very same day.''

``No, I think the date is different: it was about the 20th. My wife had
earlier asked me, as a year's-end present to herself, to take her to
hear Settsu Daijō. I'd replied that I wouldn't say no, and asked her the
nature of the program for that day. She consulted the newspapers and
answered that it was one of Chikamatsu's suicide dramas, Unagidani.
`Let's not go today, I don't like Unagidani,' said I. So we did not go
that day. The next day my wife, bringing out the newspaper again, said,
`Today he's doing the Monkey Man at Horikawa, so, let's go.' I said
let's not, because Horikawa was so frivolous, just samisen-playing with
no meat in it. My wife went away looking discontented. The following
day, she stated almost as a demand, `Today's program is The Temple With
Thirty-Three Pillars. You may dislike the Temple quite as strongly as
you disliked all the others, but since the treat is intended to be for
me, surely you won't object to taking me there.' I responded, `If you've
set your heart on it so firmly, then we'll go, but since the performance
has been announced as Settsu's farewell appearance on the stage, the
house is bound to be packed full, and since we haven't booked in
advance, it will obviously be impossible to get in. To start with, in
order to attend such performances there's an established procedure to be
observed. You have to go to the theatre-teahouse and there negotiate for
seat reservations. It would be hopeless to try going about it in the
wrong way. You just can't dodge this proper procedure. So, sorry though
I am, we simply cannot go today.' My wife's eyes glittered fiercely.
`Since I am a mere woman, I do not understand your complicated
procedures, but both Ohare's mother and Kimiyo of the Suzuki family
managed to get in without observance of any such formalities, and they
heard everything very well. I realize that you are a teacher, but surely
you don't have to go through all that troublesome rigmarole just to
visit a theatre? It's too bad\ldots{} you are so\ldots{}' and her voice
became tearful. I gave in. `All right. We'll go to the theatre even if
we can't get into it. After an early supper we'll take the tram.' She
suddenly became quite lively. `If we're going, we must be there by four
o'clock, so we mustn't dilly-dally.' When I asked her why one had to be
there by four o'clock, she explained that Kimiyo had told her that, if
one arrived any later, all the seats would be taken. I asked her again,
to make quite sure, if it would be fruitless to turn up later than four
o'clock; and she answered briskly, `Of course it would be no good.'
Then, d'you know, at that very moment the shivering set in.''

``Do you mean your wife?'' asks Coldmoon.

``Oh, no, my wife was as fit as a fiddle. It was me. I had a sudden
feeling that I was shriveling like a pricked balloon. Then I grew giddy
and unable even to move.''

``You were taken ill with a most remarkable suddenness,'' commented
Waverhouse.

``This is terrible. What shall I do? I'd like so much to grant my wife
her wish, her one and only request in the whole long year. All I ever do
is scold her fiercely, or not speak to her, or nag her about household
expenses, or insist that she cares more carefully for the children; yet
I have never rewarded her for all her efforts in the domestic field.
Today, luckily, I have the time and the money available. I could easily
take her on some little outing. And she very much wants to go. Just as I
very much want to take her. But much indeed as I want to take her, this
icy shivering and frightful giddiness make it impossible for me even to
step down from the entrance of my own house, let alone to climb up into
a tram. The more I think how deeply I grieve for her, the poor thing,
the worse my shivering grows and the more giddy I become. I thought if I
consulted a doctor and took some medicine, I might get well before four
o'clock. I discussed the matter with my wife and sent for Mr.~Amaki,
Bachelor of Medicine. Unfortunately, he had been on night duty at the
university hospital and hadn't yet come home. However, we received every
assurance that he was expected home by about two o'clock and that he
would hurry round to see me the minute he returned. What a nuisance. If
only I could get some sedative, I know I could be cured before four. But
when luck is running against one, nothing goes well.

Here I am, just this once in a long, long time, looking forward to
seeing my wife's happy smile, and to be sharing in that happiness. My
expectations seem sadly unlikely to be fulfilled. My wife, with a most
reproachful look, enquires whether it really is impossible for me to go
out. `I'll go; certainly I'll go. Don't worry. I'm sure I'll be all
right by four. Wash your face, get ready to go out and wait for me.'
Though I uttered all these reassurance, my mind was shaken with profound
emotions. The shivering strengthens and accelerates, and my giddiness
grows worse and worse. Unless I do get well by four o'clock and
implement my promise, one can never tell what such a pusillanimous woman
might do.

What a wretched business. What should I do? As I thought it possible
that the very worst could happen, I began to consider whether perhaps it
might be my duty as a husband to explain to my wife, now while I was
still in possession of my faculties, the dread truths concerning
mortality and the vicissitudes of life. For if the worst should happen,
she would then at least be prepared and less liable to be overcome by
the paroxysms of her grief. I accordingly summoned my wife to come
immediately to my study. But when I began by saying, `Though but a woman
you must be aware of that Western proverb which states that there is
many a slip ``twixt the cup and the lip,''\,' she flew into a fury. `How
should I know anything at all about such sideways-written words? You're
deliberately making a fool of me by choosing to speak English when you
know perfectly well that I don't understand a word of it. All right. So
I can't understand English. But if you're so besotted about English, why
didn't you marry one of those girls from the mission schools? I've never
come across anyone quite so cruel as you.' In the face of this tirade,
my kindly feelings, my husbandly anxiety to prepare her for extremities,
were naturally damped down. I'd like you two to understand that it was
not out of malice that I spoke in English. The words sprang solely from
a sincere sentiment of love for my wife. Consequently, my wife's malign
interpretation of my motives left me feeling helpless. Besides, my brain
was somewhat disturbed by reason of the cold shivering and the
giddiness; on top of all that, I was understandably distraught by the
effort of trying quickly to explain to her the truths of mortality and
the nature of the vicissitudes of life. That was why, quite
unconsciously and forgetting that my wife could not understand the
tongue, I spoke in English. I immediately realized I was in the wrong.
It was all entirely my fault. But as a result of my blunder, the cold
shivering intensified its violence and my giddiness grew ever more
viciously vertiginous. My wife, in accordance with my instructions,
proceeds to the bathroom and, stripping herself to the waist, completes
her make-up. Then, taking a kimono from a drawer, she puts it on. Her
attitudes make it quite clear that she is now ready to go out any time,
and is simply waiting for me. I begin to get nervous. Wishing that
Mr.~Amaki would arrive quickly, I look at my watch. It's already three
o'clock. Only one hour to go. My wife slides open the study door and
putting her head in, asks, `Shall we go now?' It may sound silly to
praise one's own wife, but I had never thought her quite so beautiful as
she was at that moment. Her skin, thoroughly polished with soap, gleams
deliciously and makes a marvelous contrast with the blackness of her
silken surcoat. Her face has a kind of radiance both externally and
shining from within; partly because of the soap and partly because of
her intense longing to listen to Settsu Daijō. I feel I must, come what
may, take her out to satisfy that yearning. All right, perhaps I will
make the awful effort to go out. I was smoking and thinking along these
lines when at long last Mr.~Amaki arrived. Excellent. Things are turning
out as one would wish. However, when I told him about my condition,
Amaki examined my tongue, took my pulse, tapped my chest, stroked my
back, turned my eyelids inside out, patted my skull, and thereafter sank
into deep thought for quite some time. I said to him, `It is my
impression that there may be some danger\ldots{}' but he replied, `No, I
don't think there's anything seriously wrong.'

`I imagine it would be perfectly all right for him to go out for a
little while?' asked my wife.

`Let me think.' Amaki sank back into the profundities of thought,
reemerging to remark, `Well, so long as he doesn't feel unwell\ldots{}'

`Oh, but I do feel unwell,' I said.

`In that case I'll give you a mild sedative and some liquid
medicine\ldots{}'

`Yes please. This is going to be something serious, isn't it?'

`Oh no, there's nothing to worry about. You mustn't get nervous,' said
Amaki, and thereupon departed. It is now half past three.

The maid was sent to fetch the medicine. In accordance with my wife's
imperative instructions, the wretched girl not only ran the whole way
there, but also the whole way back. It is now a quarter to four. Fifteen
minutes still to go. Then, quite suddenly, just about that time, I began
to feel sick. It came on with a quite extraordinary suddenness. All
totally unexpected. My wife had poured the medicine into a teacup and
placed it in front of me, but as soon as I tried to lift the teacup,
some keck-keck thing stormed up from within the stomach. I am compelled
to put the teacup down. `Drink it up quickly' urges my wife. Yes,
indeed, I must drink it quickly and go out quickly. Mustering all my
courage to imbibe the potion I bring the teacup to my lips, when again
that insuppressible keck-keck thing prevents my drinking it. While this
process of raising the cup and putting it down is being several times
repeated, the minutes crept on until the wall clock in the living room
struck four o'clock.

Ting-ting-ting-ting. Four o'clock it is. I can no longer dilly-dally and
I raise the teacup once again. D'you know, it really was most strange.
I'd say that it was certainly the uncanniest thing I've ever
experienced. At the fourth stroke my sickliness just vanished, and I was
able to take the medicine without any trouble at all. And, by about ten
past four---here I must add that I now realized for the first time how
truly skilled a physician we have in Dr.~Amaki---the shivering of my
back and the giddiness in my head both disappeared like a dream. Up to
that point I had expected that I was bound to be laid up for days, but
to my great pleasure the illness proved to have been completely cured.''

``And did you two then go out to the theatre?'' asks Waverhouse with the
puzzled expression of one who cannot see the point of a story.

``We certainly both wanted to go, but since it had been my wife's
reiterated view that there was no hope of getting in after four o'clock,
what could we do? We didn't go. If only Amaki had arrived fifteen
minutes earlier, I could have kept my promise and my wife would have
been satisfied. Just that fifteen minute difference. I was indeed
distressed. Even now, when I think how narrow the margin was, I am again
distressed.''

My master, having told his shabby tale, contrives to look like a person
who has done his duty. I imagine he feels he's gotten even with the
other two.

``How very vexing,'' says Coldmoon. His laugh, as usual, displays his
broken tooth.

Waverhouse, with a false naivety, remarks as if to himself. ``Your wife,
with a husband so thoughtful and kind-hearted, is indeed a lucky
woman.'' Behind the sliding paper-door, we heard the master's wife make
an harumphing noise as though clearing her throat.

I had been quietly listening to the successive stories of these three
precious humans, but I was neither amused nor saddened by what I'd
heard. I merely concluded that human beings were good for nothing,
except for the strenuous employment of their mouths for the purpose of
whiling away their time in laughter at things which are not funny, and
in the enjoyment of amusements which are not amusing. I have long known
of my master's selfishness and narrowmindedness, but, because he usually
has little to say, there was always something about him which I could
not understand. I'd felt a certain caution, a certain fear, even a
certain respect toward him on account of that aspect of his nature that
I did not understand. But having heard his story, my uncertainties
suddenly coalesced into a mere contempt for him. Why can't he listen to
the stories of the other two in silence? What good purpose can he serve
by talking such utter rubbish just because his competitive spirit has
been roused? I wonder if, in his portentous writings, Epictetus
advocated any such course of action. In short, my master, Waverhouse,
and Coldmoon are all like hermits in a peaceful reign. Though they adopt
a nonchalant attitude, keeping themselves aloof from the crowd,
segegrated like so many snake-gourds swayed lightly by the wind, in
reality they, too, are shaken by just the same greed and worldly
ambition as their fellow men.

The urge to compete and their anxiety to win are revealed flickeringly
in their everyday conversation, and only a hair's breadth separates them
from the Philistines whom they spend their idle days denouncing. They
are all animals from the same den. Which fact, from a feline viewpoint,
is infinitely regrettable. Their only moderately redeeming feature is
that their speech and conduct are less tediously uninventive than those
of less subtle creatures.

As I thus summed up the nature of the human race, I suddenly felt the
conversation of these specimens to be intolerably boring, so I went
around to the garden of the mistress of the two-stringed harp to see how
Tortoiseshell was getting on. Already the pine tree decorations for the
New Year and that season's sacred festoons have been taken down. It is
the 10th of January. From a deep sky containing not even a single streak
of cloud the glorious springtime sun shines down upon the lands and seas
of the whole wide world, so that even her tiny garden seems yet more
brilliantly lively than when it saw the dawn of New Year's Day.

There is a cushion on the veranda, the sliding paper-door is closed, and
there's nobody about. Which probably means that the mistress has gone
off to the public baths. I'm not at all concerned if the mistress should
be out, but I do very much worry about whether Tortoiseshell is any
better. Since everything's so quiet and not a sign of a soul, I hop up
onto the veranda with my muddy paws and curl up right in the middle of
the cushion, which I find comfortable. A drowsiness came over me, and,
forgetting all about Tortoiseshell, I was about to drop off into a doze
when suddenly I heard voices beyond the paper-door.

``Ah, thanks. Was it ready?'' The mistress has not gone out after all.

``Yes, madam. I'm sorry to have taken such a long time. When I got
there, the man who makes Buddhist altar furniture told me he'd only just
finished it.''

``Well, let me see it. Ah, but it's beautifully done. With this,
Tortoiseshell can surely rest in peace. Are you sure the gold won't peel
away?''

``Yes, I've made sure of it. They said that as they had used the very
best quality, it would last longer than most human memorial tablets.
They also said that the character for `honor' in Tortoiseshell's
posthumous name would look better if written in the cursive style, so
they had added the appropriate strokes.''

``Is that so? Well, let's put Myōyoshinnyo's tablet in the family shrine
and offer incense sticks.''

Has anything happened to Tortoiseshell? Thinking something must be
wrong, I stand up on the cushion. Ting! ``Amen! Myōyoshinnyo. Save us,
merciful Buddha! May she rest in peace.'' It is the voice of the
mistress.

``You, too, say prayers for her.''

Ting! ``Amen! Myōyoshinnyo. Save us, merciful Buddha! May she rest in
peace.'' Suddenly my heart throbs violently. I stand dead-still upon the
cushion, like a wooden cat; not even my eyes are moving.

``It really was a pity. It was only a cold at first.''

``Perhaps if Dr.~Amaki had given her some medicine, it might have
helped.''

``It was indeed Amaki's fault. He paid too little regard to
Tortoiseshell.''

``You must not speak ill of other persons. After all, everyone dies when
their allotted span is over.''

It seems thatTortoiseshell was also attended by that skilled physician,
Dr.~Amaki.

``When all's said and done, I believe the root cause was that the stray
cat at the teacher's in the main street took her out too often.''

``Yes, that brute.''

I would like to exculpate myself, but realizing that at this juncture it
behoves me to be patient, I swallow hard and continue listening. There
is a pause in the conversation.

``Life does not always turn out as one wishes. A beauty like
Tortoiseshell dies young. That ugly stray remains healthy and flourishes
in devilment\ldots{}''

``It is indeed so, Madam. Even if one searched high and low for a cat as
charming as Tortoiseshell, one would never find another person like
her.''

She didn't say `another cat,' she said `another person.' The maid seems
to think that cats and human beings are of one race. Which reminds me
that the face of this particular maid is strangely like a cat's.

``If only instead of our dear Tortoiseshell\ldots{}''

``\ldots{}that wretched stray at the teacher's had been taken. Then,
Madam, how perfectly everything would have gone\ldots{}''

If everything had gone that perfectly, I would have been in deep
trouble. Since I have not yet had the experience of being dead, I cannot
say whether or not I would like it. But the other day, it happening to
be unpleasantly chilly, I crept into the tub for conserving half-used
charcoal and settled down upon its still-warm contents. The maid, not
realizing I was in there, popped on the lid. I shudder even now at the
mere thought of the agony I then suffered. According to Miss Blanche,
the cat across the road, one dies if that agony continues for even a
very short stretch. I wouldn't complain if I were asked to substitute
for Tortoiseshell; but if one cannot die without going through that kind
of agony, I frankly would not care to die on anyone's behalf.

``Though a cat, she had her funeral service conducted by a priest and
now she's been given a posthumous Buddhist name. I don't think she would
expect us to do more.''

``Of course not, madam. She is indeed thrice blessed. The only comment
that one might make is that the funeral service read by the priest was,
perhaps, a little wanting in gravity.''

``Yes, and I thought it rather too brief. But when I remarked to the
priest from the Gekkei Temple `you've finished very quickly, haven't
you?' he answered `I've done sufficient of the effective parts, quite
enough to get a kitty into Paradise.'\,''

``Dear me! But if the cat in question were that unpleasant
stray\ldots{}''

I have pointed out often enough that I have no name, but this maid keeps
calling me ``that stray.'' She is a vulgar creature.

``So very sinful a creature. Madam, would never be able to rest in
peace, however many edifying texts were read for its salvation.''

I do not know how many hundreds of times I was thereafter stigmatized as
a stray. I stopped listening to their endless babble while it was still
only half-run, and, slipping down from the cushion, I jumped off the
veranda. Then, simultaneously erecting every single one of my
eighty-eight thousand, eight hundred, eighty hairs, I shook my whole
body.

Since that day I have not ventured near the mistress of the two-stringed
harp. No doubt by now she herself is having texts of inadequate gravity
read on her behalf by the priest from the Gekkei Temple.

Nowadays I haven't even energy to go out. Somehow life seems weary. I
have become as indolent a cat as my master is an indolent human. I have
come to understand that it is only natural that people should so often
explain my master's self-immurement in his study as the result of a love
affair gone wrong.

As I have never caught a rat, that O-san person once proposed that I
should be expelled; but my master knows that I'm no ordinary common or
garden cat, and that is why I continue to lead an idle existence in this
house. For that understanding I am deeply grateful to my master. What's
more, I take every opportunity to show the respect due to his
perspicacity. I do not get particularly angry with O-san's ill-treatment
of me, for she does not understand why I am as I now am. But when, one
of these days, some master sculptor, some regular Hidari Jingorō, comes
and carves my image on a temple gate; when some Japanese equivalent of
the French master portraitist, Steinlein, immortalizes my features on a
canvas, then at last will the silly purblind beings in shame regret
their lack of insight.
\chapter*{III}
TORTOISESHELL is dead, one cannot consort with Rickshaw Blacky, and I
feel a little lonely. Luckily I have made acquaintances among humankind
so I do not suffer from any real sense of boredom. Someone wrote
recently asking my master to have my photograph taken and the picture
sent to him. And then the other day somebody else presented some millet
dumplings, that speciality of Okayama, specifically addressed to me. The
more that humans show me sympathy, the more I am inclined to forget that
I am a cat. Feeling that I am now closer to humans than to cats, the
idea of rallying my own race in an effort to wrest supremacy from the
bipeds no longer has the least appeal. Moreover, I have developed,
indeed evolved, to such an extent that there are now times when I think
of myself as just another human in the human world; which I find very
encouraging. It is not that I look down on my own race, but it is no
more than natural to feel most at ease among those whose attitudes are
similar to one's own. I would consequently feel somewhat piqued if my
growing penchant for mankind were stigmatized as fickleness or flippancy
or treachery. It is precisely those who sling such words about in
slanderous attacks on others who are usually both drearily
straight-laced and born unlucky. Having thus graduated from felinity to
humanity, I find myself no longer able to confine my interests to the
world of Tortoiseshell and Blacky. With a haughtiness not less prideful
than that of human beings, I, too, now like to judge and criticize their
thoughts and words and deeds. This, surely, is equally natural. Yet,
though I have become thus proudly conscious of my own dignity, my master
still regards me as a cat only slightly superior to any other common or
garden moggy. For, as if they were his own and without so much as a
by-your-leave to me, he has eaten all the millet dumplings; which is, I
find, regrettable. Nor does he seem yet to have dispatched my
photograph. I suppose I would be justified if I made this fact a cause
for grumbling, but after all, if our opinions---my master's and
mine---are naturally at difference, the consequences of that difference
cannot be helped. Since I am seeking to behave with total humanity, I'm
finding it increasingly difficult to write about the activities of cats
with whom I no longer associate. I must accordingly seek the indulgence
of my readers if I now confine my writing to reports about such
respected figures as Waverhouse and Coldmoon.

Today is a Sunday and the weather fine. The master has therefore crept
out of his study, and, placing a brush, an ink stone, and a writing pad
in a row before him, he now lies flat on his belly beside me, and is
groaning hard. I watch him, thinking that he is perhaps making this
peculiar noise in the birth pangs of some literary effort. After a
while, and in thick black strokes, he wrote, ``Burn incense.'' Is it
going to be a poem or a haiku? Just when I was thinking that the phrase
was rather too witty for my master, he abandons it, and, his brush
running quickly over the paper, writes an entirely new line: ``Now for
some little time I have been thinking of writing an article about
Mr.~the-late-and-sainted Natural Man.'' At this point the brush stops
dead. My master, brush in hand, racks his brains, but no bright notions
seem to emerge for he now starts licking the head of his brush. I
watched his lips acquire a curious inkiness. Then, underneath what he
had just written, he drew a circle, put in two dots as eyes, added a
nostrilled nose in the center, and finally drew a single sideways line
for a mouth. One could not call such creations either haiku or prose.
Even my master must have been disgusted with himself, for he quickly
smeared away the face. He then starts a new line. He seems to have some
vague notion that, provided he himself produces a new line, maybe some
kind of a Chinese poem will evolve itself.

After further moonings, he suddenly started writing briskly in the
colloquial style. ``Mr.~the-late-and-sainted Natural Man is one who
studies Infinity, reads the Analects of Confucius, eats baked yams, and
has a runny nose.'' A somewhat muddled phrase. He thereupon read the
phrase aloud in a declamatory manner and, quite unlike his usual self,
laughed. ``Ha-ha-ha. Interesting! But that `runny nose' is a shade
cruel, so I'll cross it out,'' and he proceeds to draw lines across that
phrase.

``Though a single line would clearly have sufficed, he draws two lines
and then three lines. He goes on drawing more and more lines regardless
of their crowding into the neighboring line of writing. When he has
drawn eight such obliterations, he seems unable to think of anything to
add to his opening outburst. So he takes to twirling his mustache,
determined to wring some telling sentence from his whiskers. He is still
twisting them up and twirling them down when his wife appears from the
living room, and sitting herself down immediately before my master's
nose, remarks, ``My dear.''

``What is it?'' My master's voice sounds dully like a gong struck under
water. His wife seems not to like the answer, for she starts all over
again.

``My dear!'' she says.

``Well, what is it?''

This time, cramming a thumb and index finger into a nostril, he yanks
out nostril hairs.

``We are a bit short this month\ldots{}''

``Couldn't possibly be short. We've settled the doctor's fee and we paid
off the bookshop's bill last month. So this month, there ought in fact
to be something left over.'' He coolly examines his uprooted nostril
hairs as though they were some wonder of the world.

``But because you, instead of eating rice, have taken to bread and
jam\ldots{}''

``Well, how many tins of jam have I gone through?''

``This month, eight tins were emptied.''

``Eight? I certainly haven't eaten that much.''

``It wasn't only you. The children also lick it.''

``However much one licks, one couldn't lick more than two or three
shillings worth.'' My master calmly plants his nostril hairs, one by
one, on the writing pad. The sticky-rooted bristles stand upright on the
paper like a little copse of needles. My master seems impressed by this
unexpected discovery and he blows upon them. Being so sticky, they do
not fly away.

``Aren't they obstinate?'' he says and blows upon them frantically.

``It is not only the jam. There's other things we have to buy.'' The
lady of the house expresses her extreme dissatisfaction by pouting
sulkily.

``Maybe.'' Again inserting his thumb and finger, he extracts some hairs
with a jerk. Among these hairs of various hue, red ones and black ones,
there is a single pure white bristle. My master who, with a look of
great surprise, has been staring at this object, proceeds to show it to
his wife, holding it up between his fingers right in front of her face.

``No, don't.'' She pushes his hand away with a grimace of distaste.

``Look at it! A white hair from the nostrils.'' My master seems to be
immensely impressed. His wife, resigned, went back into the living room
with a laugh. She seems to have given up hope of getting any answer to
her problems of domestic economy. My master resumes his consideration of
the problems of Natural Man.

Having succeeded in driving off his wife with his scourge of nostril
hair, he appears to feel relieved, and, while continuing that
depilation, struggles to get on with his article. But his brush remains
unmoving.

``That `eats baked yams' is also superfluous. Out with it.'' He deletes
the phrase. ``And `incense burns' is somewhat over-abrupt, so let's
cross that out too.'' His exuberant self-criticism leaves nothing on the
paper but the single sentence ``Mr.~the-late-and-sainted Natural Man is
one who studies Infinity and reads the Analects of Confucius.'' My
master thinks this statement a trifle over-simplified. ``Ah well, let's
not be bothered: let's abandon prose and just make it an inscription.''
Brandishing the brush crosswise, he paints vigorously on the writing pad
in that watercolor style so common among literary men and produces a
very poor study of an orchid. Thus all his precious efforts to write an
article have come down to this mere nothing. Turning the sheet, he
writes something that makes no sense. ``Born in Infinity, studied
Infinity, and died into Infinity. Mr.~the-late-and-sainted Natural Man.
Infinity.'' At this moment Waverhouse drifts into the room in his usual
casual fashion. He appears to make no distinction between his own and
other people's houses; unannounced and unceremoniously, he enters any
house and, what's more, will sometimes float in unexpectedly through a
kitchen door. He is one of those who, from the moment of their birth,
discaul themselves of all such tiresome things as worry, reserve,
scruple, and concern.

``\,`Giant Gravitation again?'\,'' asks Waverhouse still standing.

``How could I be always writing only about `Giant Gravitation?' I'm
trying to compose an epitaph for the tombstone of
Mr.~the-late-and-sainted Natural Man,'' replied my master with
considerable exaggeration.

``Is that some sort of posthumous Buddhist name like Accidental Child?''
inquires Waverhouse in his usual irrelevant style.

``Is there then someone called Accidental Child?''

``No, of course there isn't, but I take it that you're working on
something like that.''

``I don't think Accidental Child is anyone I know. But
Mr.~the-late-and-sainted Natural Man is a person of your own
acquaintance.''

``Who on earth could get a name like that?''

``It's Sorosaki. After he graduated from the University, he took a
post-graduate course involving study of the `theory of infinity.' But he
over-worked, got peritonitis, and died of it. Sorosaki happened to be a
very close friend of mine.''

``All right, so he was your very close friend. I'm far from criticizing
that fact. But who was responsible for converting Sorosaki into
Mr.~the-late-and-sainted Natural Man?''

``Me. I created that name. For there is really nothing more philistine
than the posthumous names conferred by Buddhist priests.'' My master
boasts as if his nomination of Natural Man were a feat of artistry.

``Anyway, let's see the epitaph,'' says Waverhouse laughingly. He picks
up my master's manuscript and reads it out aloud. ``Eh\ldots{} `Born
into infinity, studied infinity, and died into infinity.
Mr.~the-late-and-sainted Natural Man. Infinity.' I see. This is fine.
Quite appropriate for poor old Sorosaki.''

``Good, isn't it?'' says my master obviously very pleased.

``You should have this epitaph engraved on a weight-stone for pickles
and then leave it at the back of the main hall of some temple for the
practice-benefit of passing weight lifters. It's good. It's most
artistic. Mr.~the-late-and-sainted may now well rest in peace.''

``Actually, I'm thinking of doing just that,'' answers my master quite
seriously. ``But you'll have to excuse me,'' he went on, ``I won't be
long. Just play with the cat. Don't go away.'' And my master departed
like the wind without even waiting for Waverhouse to answer.

Being thus unexpectedly required to entertain the culture-vulture
Waverhouse, I cannot very well maintain my sour attitude. Accordingly, I
mew at him encouragingly and sidle up on to his knees. ``Hello,'' says
Waverhouse, ``you've grown distinctly chubby. Let's take a look at
you.''

Grabbing me impolitely by the scruff of my neck he hangs me up in
midair. ``Cats like you that let their hind legs dangle are cats that
catch no mice\ldots{} Tell me,'' he said, turning to my master's wife in
the next room, ``has he ever caught anything?''

``Far from catching so much as a single mouse, he eats rice-cakes and
then dances.'' The lady of the house unexpectedly probes my old wound,
which embarrassed me. Especially when Waverhouse still held me in midair
like a circus-performer.

``Indeed, with such a face, it's not surprising that he dances. Do you
know, this cat possesses a truly insidious physionomy. He looks like one
of those goblin-cats illustrated in the old storybooks.'' Waverhouse,
babbling whatever comes into his head, tries to make conversation with
the mistress. She reluctantly interrupts her sewing and comes into the
room.

``I do apologize. You must be bored. He won't be long now.'' And she
poured fresh tea for him.

``I wonder where he's gone.''

``Heaven only knows. He never explains where he's going. Probably to see
his doctor.''

``You mean Dr.~Amaki? What a misfortune for Amaki to be involved with
such a patient.''

Perhaps finding this comment difficult to answer, she answers briefly:

``Well, yes.''

Waverhouse takes not the slightest notice, but goes on to ask, ``How is
he lately? Is his weak stomach any better?''

``It's impossible to say whether it's better or worse. However carefully
Dr.~Amaki may look after him, I don't see how his health can ever
improve if he continues to consume such vast quantities of jam.'' She
thus works off on Waverhouse her earlier grumblings to my master.

``Does he eat all that much jam? It sounds like a child.''

``And not just jam. He's recently taken to guzzling grated radish on the
grounds that it's a sovereign cure for dyspepsia.''

``You surprise me,'' marvels Waverhouse.

``It all began when he read in some rag that grated radish contains
diastase.''

``I see. I suppose he reckons that grated radish will repair the ravages
of jam. It's certainly an ingenious equation.'' Waverhouse seems vastly
diverted by her recital of complaint.

``Then only the other day he forced some on the baby.''

``He made the baby eat jam?''

``No, grated radish! Would you believe it? He said, `Come here, my
little babykin, father'll give you something good\ldots{}' Whenever,
once in a rare while, he shows affection for the children, he always
does remarkably silly things. A few days ago he put our second daughter
on top of a chest of drawers.''

``What ingenious scheme was that?'' Waverhouse looks to discover
ingenuities in everything.

``There was no question of any ingenious scheme. He just wanted the
child to make the jump when it's quite obvious that a little girl of
three or four is incapable of such tomboy feats.''

``I see. Yes, that proposal does indeed seem somewhat lacking in
ingenuity. Still, he's a good man without an ill wish in his heart.''

``Do you think that I could bear it if, on top of everything else, he
were ill-natured?'' She seems in uncommonly high spirits.

``Surely you don't have cause for such vehement complaint? To be as
comfortably off as you are is, after all, the best way to be. Your
husband neither leads the fast life nor squanders money on dandified
clothing. He's a born family man of quiet taste.'' Waverhouse fairly
lets himself go in unaccustomed laud of an unknown way of life.

``On the contrary, he's not at all like that\ldots{}''

``Indeed? So he has secret vices? Well, one cannot be too careful in
this world.'' Waverhouse offers a nonchalantly fluffy comment.

``He has no secret vices, but he is totally abandoned in the way he buys
book after book, never to read a single one. I wouldn't mind if he used
his head and bought in moderation, but no. Whenever the mood takes him,
he ambles off to the biggest bookshop in the city and brings back home
as many books as chance to catch his fancy. Then, at the end of the
month, he adopts an attitude of complete detachment. At the end of last
year, for instance, I had a terrible time coping with the bill that had
been accumulating month after month.''

``It doesn't matter that he should bring home however many books he may
like. If, when the bill collector comes, you just say that you'll pay
some other time, he'll go away.''

``But one cannot put things off indefinitely.'' She looks cast down.

``Then you should explain the matter to your husband and ask him to cut
down expenditure on books.''

``And do you really believe he would listen to me? Why, only the other
day, he said, `You are so unlike a scholar's wife: you lack the least
understanding of the value of books. Listen carefully to this story from
ancient Rome. It will give you beneficial guidance for your future
conduct.'\,''

``That sounds interesting. What sort of story was it?'' Waverhouse
becomes enthusiastic, though he appears less sympathetic to her
predicament than prompted by sheer curiosity.

``It seems there was in ancient Rome a king named Tarukin.''

``Tarukin? That sounds odd in Japanese.''

``I can never remember the names of foreigners. It's all too difficult.
Maybe he was a barrel of gold. He was, at any rate, the seventh king of
Rome.''

``Really? The seventh barrel of gold certainly sounds queer. But, tell
me, what then happened to this seventh Tarukin.''

``You mustn't tease me like that. You quite embarrass me. If you know
this king's true name, you should teach me it. Your attitude,'' she
snaps at him, ``is really most unkind.''

``I tease you? I wouldn't dream of doing such an unkind thing. It was
simply that the seventh barrel of gold sounded so wonderful. Let's
see\ldots{} a Roman, the seventh king\ldots{} I can't be absolutely
certain but I rather think it must have been Tarquinius Superbus,
Tarquin the Proud. Well, it doesn't really matter who it was. What did
this monarch do?''

``I understand that some woman, Sibyl by name, went to this king with
nine books and invited him to buy them.''

``I see.''

``When the king asked her how much she wanted, she stated a very high
price, so high that the king asked for a modest reduction. Whereupon the
woman threw three of the nine books into the fire where they were
quickly burnt to ashes.''

``What a pity!''

``The books were said to contain prophecies, predictions, things like
that of which there was no other record anywhere.''

``Really?''

``The king, believing that six books were bound to be cheaper than nine,
asked the price of the remaining volumes. The price proved to be exactly
the same; not one penny less. When the king complained of this
outrageous development, the women threw another three books into the
fire. The king apparently still hankered for the books and he
accordingly asked the price of the last three left. The woman again
demanded the same price as she had asked for the original nine. Nine
books had shrunk to six, and then to three, but the price remained
unaltered even by a farthing. Suspecting that any attempt to bargain
would merely lead the woman to pitch the last three volumes into the
flames, the king bought them at the original staggering price. My
husband appeared confident that, having heard this story, I would begin
to appreciate the value of books, but I don't at all see what it is that
I'm supposed to have learnt to appreciate.''

Having thus stated her own position, she as good as challenges
Waverhouse to contravert her. Even the resourceful Waverhouse seems to
be at a loss. He draws a handkerchief from the sleeve of his kimono and
tempts me to play with it. Then, in a loud voice as if an idea had
suddenly struck him, he remarked, ``But you know, Mrs.~Sneaze, it is
precisely because your husband buys so many books and fills his head
with wild notions that he is occasionally mentioned as a scholar, or
something of that sort. Only the other day a comment on your husband
appeared in a literary magazine.''

``Really?'' She turns around. After all, it's only natural that his wife
should feel anxiety about comments on my master.

``What did it say?''

``Oh, only a few lines. It said that Mr.~Sneaze's prose was like a cloud
that passes in the sky, like water flowing in a stream.''

``Is that,'' she asks smiling, ``all that it said?''

``Well, it also said `it vanishes as soon as it appears and, when it
vanishes, it is forever forgetful to return.'\,''

The lady of the house looks puzzled and asks anxiously ``Was that
praise?''

``Well, yes, praise of a sort,'' says Waverhouse coolly as he jiggles
his handkerchief in front of me.

``Since books are essential to his work, I suppose one shouldn't
complain, but his eccentricity is so pronounced that\ldots{}''

Waverhouse assumes that she's adopting a new line of attack. ``True,''
he interrupts, ``he is a little eccentric, but any man who pursues
learning tends to get like that.'' His answer, excellently noncommittal,
contrives to combine ingratiation and special pleading.

``The other day, when he had to go somewhere soon after he got home from
school, he found it too troublesome to change his clothes. So do you
know, he sat down on his low desk without even taking off his overcoat
and ate his dinner just as he was. He had his tray put on the footwarmer
while I sat on the floor holding the rice container. It was really very
funny\ldots{}''

``It sounds like the old-time custom when generals sat down to identify
the severed heads of enemies killed in battle. But that would be quite
typical of Mr.~Sneaze. At any rate he's never boringly conventional.''

Waverhouse offers a somewhat strained compliment.

``A woman cannot say what's conventional or unconventional, but I do
think his conduct is often unduly odd.''

``Still, that's better than being conventional.'' As Waverhouse moves
firmly to the support of my master, her dissatisfaction deepens.

``People are always saying this or that is conventional, but would you
please tell what makes a thing conventional?'' Adopting a defiant
attitude, she demands a definition of conventionality.

``Conventional? When one says something is conventional\ldots{} It's a
bit difficult to explain\ldots{}''

``If it's so vague a thing, surely there's nothing wrong with being
conventional.'' She begins to corner Waverhouse with typically feminine
logic.

``No, it isn't vague, it's perfectly clear-cut. But it's hard to
explain.''

``I expect you call everything you don't like conventional.'' Though
totally uncalculated, her words land smack on target. Waverhouse is now
indeed cornered and can no longer dodge defining the conventional.

``I'll give you an example. A conventional man is one who would yearn
after a girl of sixteen or eighteen but, sunk in silence, never do
anything about it; a man who, whenever the weather's fine, would do no
more than stroll along the banks of the Sumida taking, of course, a
flask of saké with him.''

``Are there really such people?'' Since she cannot make heads or tails
of the twaddle vouchsafed by Waverhouse, she begins to abandon her
position, which she finally surrenders by saying, ``It's all so
complicated that it's really quite beyond me.''

``You think that complicated? Imagine fitting the head of Major
Pendennis onto Bakin's torso, wrapping it up and leaving it all for one
or two years exposed to European air.''

``Would that produce a conventional man?'' Waverhouse offers no reply
but merely laughs.

``In fact it could be produced without going to quite so much trouble.
If you added a shop assistant from a leading store to any middle school
student and divided that sum by two, then indeed you'd have a fine
example of a conventional man.''

``Do you really think so?'' She looks puzzled but certainly unconvinced.

``Are you still here?'' My master sits himself down on the floor beside
Waverhouse. We had not noticed his return.

``\,`Still here' is a bit hard. You said you wouldn't be long and you
yourself invited me to wait for you.''

``You see, he's always like that,'' remarks the lady of the house
leaning toward Waverhouse.

``While you were away I heard all sorts of tales about you.''

``The trouble with women is that they talk too much. It would be good if
human beings would keep as silent as this cat.'' And the master strokes
my head.

``I hear you've been cramming grated radish into the baby.''

``Hum,'' says my master and laughs. He then added ``Talking of the baby,
modern babies are quite intelligent. Since that time when I gave our
baby grated radish, if you ask him `where is the hot place?' he
invariably sticks out his tongue. Isn't it strange?''

``You sound as if you were teaching tricks to a dog. It's positively
cruel. By the way, Coldmoon ought to have arrived by now.''

``Is Coldmoon coming?'' asks my master in a puzzled voice.

``Yes. I sent him a postcard telling him to be here not later than one
o'clock.''

``How very like you! Without even asking us if it happened to be
convenient. What's the idea of asking Coldmoon here?''

``It's not really my idea, but Coldmoon's own request. It seems he is
going to give a lecture to the Society of Physical Science. He said he
needed to rehearse his speech and asked me to listen to it. Well, I
thought it would be obliging to let you hear it, too. Accordingly, I
suggested he should come to your house. Which should be quite convenient
since you are a man of leisure. I know you never have any engagements.
You'd do well to listen.'' Waverhouse thinks he knows how to handle the
situation.

``I wouldn't understand a lecture on physical science,'' says my master
in a voice betraying his vexation at his friend's high-handed action.

``On the contrary, his subject is no such dry-as-dust matter as, for
example, the magnetized nozzle. The transcendentally extraordinary
subject of his discourse is `The Mechanics of Hanging.' Which should be
worth listening to.''

``Inasmuch as you once only just failed to hang yourself, I can
understand your interest in the subject, but I'm\ldots{}''

``\ldots{}The man who got cold shivers over going to the theatre, so you
cannot expect not to listen to it.'' Waverhouse interjects one of his
usual flippant remarks and Mrs.~Sneaze laughs. Glancing back at her
husband, she goes off into the next room. My master, keeping silent,
strokes my head. This time, for once, he stroked me with delicious
gentleness.

Some seven minutes later in comes the anticipated Coldmoon. Since he's
due to give his lecture this same evening, he is not wearing his usual
get-up. In a fine frock-coat and with a high and exceedingly white clean
collar, he looks twenty per cent more handsome than himself. ``Sorry to
be late.'' He greets his two seated friends with perfect composure.

``It's ages that we've now been waiting for you. So we'd like you to
start right away. Wouldn't we?'' says Waverhouse, turning to look at my
master. The latter, thus forced to respond, somewhat reluctantly says,

``Hmm.'' But Coldmoon's in no hurry. He remarks, ``I think I'll have a
glass of water, please.''

``I see you are going to do it in real style. You'll be calling next for
a round of applause.'' Waverhouse, but he alone, seems to be enjoying
himself.

Coldmoon produced his text from an inside pocket and observed,

``Since it is the established practice, may I say I would welcome
criticism.'' That invitation made, he at last begins to deliver his
lecture.

``Hanging as a death penalty appears to have originated among the
Anglo-Saxons. Previously, in ancient times, hanging was mainly a method
of committing suicide. I understand that among the Hebrews it was
customary to execute criminals by stoning them to death. Study of the
Old Testament reveals that the word `hanging' is there used to mean
`suspending a criminal's body after death for wild beasts and birds of
prey to devour it.' According to Herodotus, it would seem that the Jews,
even before they departed from Egypt, abominated the mere thought that
their dead bodies might be left exposed at night. The Egyptians used to
behead a criminal, nail the torso to a cross and leave it exposed during
the night. The Persians\ldots{}''

``Steady on, Coldmoon,'' Waverhouse interrupts. ``You seem to be
drifting farther and farther away from the subject of hanging. Do you
think that wise?''

``Please be patient. I am just coming to the main subject. Now, with
respect to the Persians. They, too, seemed to have used crucifixion as a
method of criminal execution. However, whether the nailing took place
while the criminal was alive or simply after his death is not
incontrovertibly established.''

``Who cares? Such details are really of little importance,'' yawned my
master as from boredom.

``There are still many matters of which I'd like to inform you but, as
it will perhaps prove tedious for you\ldots{}''

``\,`As it might prove' would sound better than `as it will perhaps
prove.' What d'you think, Sneaze?'' Waverhouse starts carping again but
my master answers coldly, ``What difference could it make?''

``I have now come to the main subject, and will accordingly recite my
piece.''

``A storyteller `recites a piece.' An orator should use more elegant
diction.'' Waverhouse again interrupts.

``If to `recite my piece' sounds vulgar, what words should I use?'' asks
Coldmoon in a voice that showed he was somewhat nettled.

``It is never clear, when one is dealing with Waverhouse, whether he's
listening or interrupting. Pay no attention to his heckling, Coldmoon,
just keep going.'' My master seeks to find a way through the difficulty
as quickly as possible.

``So, having made your indignant recitation, now I suppose you've found
the willow tree?'' With a pun on a little known haiku, Waverhouse, as
usual, comes up with something odd. Coldmoon, in spite of himself, broke
into laughter.

``My researches reveal that the first account of the employment of
hanging as a deliberate means of execution occurs in the Odyssey, volume
twenty-two. The relevant passage records how Telemachus arranged the
execution by hanging of Penelope's twelve ladies-in-waiting. I could
read the passage aloud in its original Greek, but, since such an act
might be regarded as an affectation, I will refrain from doing so. You
will, however, find the passage between lines 465 and 473.''

``You'd better cut out all that Hellenic stuff. It sounds as if you are
just showing off your knowledge of Greek. What do you think, Sneaze?''

``On that point, I agree with you. It would be more modest, altogether
an improvement, to avoid such ostentation.'' Quite unusually my master
immediately sides with Waverhouse. The reason is, of course, that
neither can read a word of Greek.

``Very well, I will this evening omit those references. And now I will
recite\ldots{} that is to say, I will now continue. Let us consider,
then, how a hanging is actually carried out. One can envisage two
methods. The first method is that adopted by Telemachus who, with the
help of Eumaeus and Philoetios, tied one end of a rope to the top of a
pillar: next, having made several loose loops in the rope, he forced a
woman's head through each such loop, and finally hauled up hard on the
other end of the rope.''

``In short, he had the women dangling in a row like shirts hung out at a
laundry. Right?''

``Exactly. Now the second method is, as in the first case, to tie one
end of a rope to the top of a pillar and similarly to secure the other
end of the rope somewhere high up on the ceiling. Thereafter, several
other short ropes are attached to the main rope, and in each of these
subsidiary ropes a slip-knot is then tied. The women's heads are then
inserted in the slipknots. The idea is that at the crucial moment you
remove the stools on which the women have been stood.''

``They would then look something like those ball-shaped paper-lanterns
one sometimes sees suspended from the end-tips of rope curtains,
wouldn't they?'' hazarded Waverhouse.

``That I cannot say,'' answered Coldmoon cautiously. ``I have never seen
any such ball as a paper-lantern-ball, but if such balls exist, the
resemblance may be just. Now, the first method as described in the
Odyssey is, in fact, mechanically impossible; and I shall proceed, for
your benefit, to substantiate that statement.''

``How interesting,'' says Waverhouse.

``Indeed, most interesting,'' adds my master.

``Let us suppose that the women are to be hanged at intervals of an
equal distance, and that the rope between the two women nearest the
ground stretches out horizontally, right? Now α1, α2 up to α6 become the
angles between the rope and the horizon. T1, T2, and so on up to T6
represent the force exerted on each section of the rope, so that T7 = X
is the force exerted on the lowest part of the rope. W is, of course,
the weight of the women. So far so good. Are you with me?''

My master and Waverhouse exchange glances and say, ``Yes, more or
less.'' I need hardly point out that the value of this ``more or less''
is singular to Waverhouse and my master. It could possibly have a
different value for other people.

``Well, in accordance with the theory of averages as applied to the
polygon, a theory with which you must of course be well acquainted, the
following twelve equations can, in this particular case, be established:
T1 cos αl=T2 cos α2\ldots{}\ldots{}(1), T2 cos α2=T3 cos
α3\ldots{}\ldots{}(2).''

``I think that's enough of the equations,'' my master irresponsibly
remarks.

``But these equations are the very essence of my lecture.'' Coldmoon
really seems reluctant to be parted from them.

``In that case, let's hear those particular parts of its very essence at
some other time.'' Waverhouse, too, seems out of his depth.

``But if I omit the full detail of the equations, it becomes impossible
to substantiate the mechanical studies to which I have devoted so much
effort\ldots{}''

``Oh, never mind that. Cut them all out,'' came the cold-blooded comment
of my master.

``That's most unreasonable. However, since you insist, I will omit
them.''

``That's good,'' says Waverhouse, unexpectedly clapping his hands.

``Now we come to England where, in Beowulf, we find the word `gallows':
that is to say `galga.' It follows that hanging as a penalty must have
been in use as early as the period with which the book is concerned.
According to Blackstone, a convicted person who is not killed at his
first hanging by reason of some fault in the rope should simply be
hanged again. But, oddly enough, one finds it stated in The Vision of
Piers Plowman that even a murderer should not be strung up twice. I do
not know which statement is correct, but there are many melancholy
instances of victims failing to be killed outright. In 1786 the
authorities attempted to hang a notorious villain named Fitzgerald, but
when the stool was removed, by some strange chance the rope broke. At
the next attempt the rope proved so long that his legs touched ground
and he again survived. In the end, at the third attempt, he was enabled
to die with the help of the spectators.''

``Well, well,'' says Waverhouse becoming, as was only to be expected,
re-enlivened.

``A true thanatophile.'' Even my master shows signs of jollity.

``There is one other interesting fact. A hanged person grows taller by
about an inch. This is perfectly true. Doctors have measured it.''

``That's a novel notion. How about it, Sneaze?'' says Waverhouse turning
to my master. ``Try getting hanged. If you were an inch taller, you
might acquire the appearance of an ordinary human being.'' The reply,
however, was delivered with an unexpected gravity.

``Tell me, Coldmoon, is there any chance of surviving that process of
extension by one inch?''

``Absolutely none. The point is that it is the spinal cord which gets
stretched in hanging. It's more a matter of breaking than of growing
taller.''

``In that case, I won't try.'' My master abandons hope.

There was still a good deal of the lecture left to deliver and Coldmoon
had clearly been anxious to deal with the question of the physiological
function of hanging. But Waverhouse made so many and such
capriciously-phrased interjections and my master yawned so rudely and so
frequently that Coldmoon finally broke off his rehearsal in mid-flow and
took his leave. I cannot tell you what oratorical triumphs he achieved,
still less what gestures he employed that evening, because the lecture
took place miles away from me.

A few days passed uneventfully by. Then, one day about two in the
afternoon, Waverhouse dropped in with his usual casual manners and
looking as totally uninhibited as his own concept of the ``Accidental
Child.'' The minute he sat down he asked abruptly, ``Have you heard
about Beauchamp Blowlamp and the Takanawa Incident?'' He spoke
excitedly, in a tone of voice appropriate to an announcement of the fall
of Port Arthur.

``No, I haven't seen him lately.'' My master is his usual cheerless
self.

``I've come today, although I'm busy, especially to inform you of the
frightful blunder which Beauchamp has committed.''

``You're exaggerating again. Indeed you're quite impossible.''

``Impossible, never: improbable, perhaps. I must ask you to make a
distinction on this point, for it affects my honor.''

``It's the same thing,'' replied my master assuming an air of provoking
indifference. He is the very image of a Mr.~the-late-and-sainted Natural
Man.

``Last Sunday, Beauchamp went to the Sengaku Temple at Takanawa, which
was silly in this cold weather, especially when to make such a visit
nowadays stamps one as a country bumpkin out to see the sights.''

``But Beauchamp's his own master. You've no right to stop him going.''

``True, I haven't got the right, so let's not bother about that. The
point is that the temple yard contains a showroom displaying relics of
the forty-seven ronin. Do you know it?''

``N-no.''

``You don't? But surely you've been to the Temple?''

``No.''

``Well, I am surprised. No wonder you so ardently defended Beauchamp.
But it's positively shameful that a citizen of Tokyo should never have
visited the Sengaku Temple.''

``One can contrive to teach without trailing out to the ends of the
city.'' My master grows more and more like his blessed Natural Man.

``All right. Anyway, Beauchamp was examining the relics when a married
couple, Germans as it happened, entered the showroom. They began by
asking him questions in Japanese, but, as you know, Beauchamp is always
aching to practice his German so he naturally responded by rattling off
a few words in that language. Apparently he did it rather well. Indeed,
when one thinks back over the whole deplorable incident, his very
fluency was the root cause of the trouble.''

``Well, what happened?'' My master finally succumbs.

``The Germans pointed out a gold-lacquered pill-box which had belonged
to Otaka Gengo and, saying they wished to buy it, asked Beauchamp if the
object were for sale. Beauchamp's reply was not uninteresting. He said
such a purchase would be quite impossible because all Japanese people
were true gentlemen of the sternest integrity. Up to that point he was
doing fine. However, the Germans, thinking that they'd found a useful
interpreter, thereupon deluged him with questions.''

``About what?''

``That's just it. If he had understood their questions, there would have
been no trouble. But you see he was subjected to floods of such
questions, all delivered in rapid German, and he simply couldn't make
head or tail of what was being asked. When at last he chanced to
understand part of their outpourings, it was something about a fireman's
axe or a mallet---some word he couldn't translate---so again, naturally,
he was completely at a loss how to reply.''

``That I can well imagine,'' sympathizes my master, thinking of his own
difficulties as a teacher.

``Idle onlookers soon began to gather around and eventually Beauchamp
and the Germans were totally surrounded by staring eyes. In his
confusion Beauchamp fell to blushing. In contrast to his earlier
self-confidence he was now at his wit's end.''

``How did it all turn out?''

``In the end Beauchamp could stand it no longer, shouted sainara in
Japanese and came rushing home. I pointed out to him that sainara was an
odd phrase to use and inquired whether, in his home-district, people
used sainara rather than sayonara. He replied they would say sayonara
but, since he was talking to Europeans, he had used sainara in order to
maintain harmony. I must say I was much impressed to find him a man
mindful of harmony even when in difficulties.''

``So that's the bit about sainara. What did the Europeans do?''

``I hear that the Europeans looked utterly flabbergasted.'' And
Waverhouse gave vent to laughter. ``Interesting, eh?''

``Frankly, no. I really can't find anything particularly interesting in
your story. But that you should have come here specially to tell me the
tale, that I do find much more interesting.'' My master taps his
cigarette's ash into the brazier. Just at that moment the bell on the
lattice door at the entrance rang with an alarming loudness, and a
piercing woman's voice declared, ``Excuse me.'' Waverhouse and my master
look at each other in silence.

Even while I am thinking that it is unusual for my master's house to
have a female visitor, the owner of that piercing voice enters the room.

She is wearing two layers of silk crepe kimono, and looks to be a little
over forty. Her forelock towers up above the bald expanse of her brow
like the wall of a dyke and sticks out toward heaven for easily one half
the length of her face. Her eyes, set at an angle like a road cut
through a mountain, slant up symmetrically in straight lines. I speak,
of course, metaphorically. Her eyes, in fact, are even narrower than
those of a whale. But her nose is exceedingly large. It gives the
impression that it has been stolen from someone else and thereafter
fastened in the center of her face. It is as if a large, stone lantern
from some major shrine had been moved to a tiny ten-square-meter garden.

It certainly asserts its own importance, but yet looks out of place. It
could almost be termed hooked: it begins by jutting sharply out, but
then, halfway along its length, it suddenly turns shy so that its tip,
bereft of the original vigour, hangs limply down to peer into the mouth
below.

Her nose is such that, when she speaks, it is the nose rather than the
mouth which seems to be in action. Indeed, in homage to the enormity of
that organ, I shall refer hence forward to its owner as Madam Conk.

When the ceremonials of her self-introduction had been completed, she
glared around the room and remarked, ``What a nice house.''

``What a liar,'' says my master to himself, and concentrates upon his
smoking. Waverhouse studies the ceiling. ``Tell me,'' he says, ``is that
odd pattern the result of a rain leak or is it inherent in the grain of
the wood?''

``Rain leak, naturally'' replies my master. To which Waverhouse coolly
answers, ``Wonderful.''

Madam Conk clearly regards them as unsociable persons and boils quietly
with suppressed annoyance. For a time the three of them just sit there
in a triangle without saying a word.

``I've come to ask you about a certain matter.'' Madam Conk starts up
again.

``Ah.'' My master's response lacks warmth.

Madam Conk, dissatisfied with this development, bestirs herself again.
``I live nearby. In fact, at the residence on the corner of the block
across the road.''

``That large house in the European style, the one with a godown? Ah,
yes. Of course. Have I not seen `Goldfield' on the nameplate of that
dwelling?'' My master, at last, seems ready to take cognizance of
Goldfield's European house and his incorporated godown, but his attitude
toward Madam Conk displays no deepening of respect.

``Of course my husband should call upon you and seek your valued advice,
but he is always so busy with his company affairs.'' She puts on a
``that ought to shift them'' face, but my master remains entirely
unimpressed. He is, in fact, displeased by her manner of speaking,
finding it too direct in a woman met for the first time. ``And not of
just one company either. He is connected with two or three of them and
is a director of them all, as I expect you already know.'' She looks as
if saying to herself, ``Now surely he should feel small.'' In point of
fact, the master of this house behaves most humbly toward anyone who
happens to be a doctor or a professor, but, oddly enough, he offers
scant respect toward businessmen. He considers a middle school teacher
to be a more elevated person than any businessman. Even if he doesn't
really believe this, he is quite resigned, being of an unadaptable
nature, to the fact that he can never hope to be smiled upon by
businessmen or millionaires.

For he feels nothing but indifference toward any person, no matter how
rich or influential, from whom he has ceased to hope for benefits. He
consequently pays not the faintest attention to anything extraneous to
the society of scholars, and is almost actively disinterested in the
goings-on of the business world. Had he even the vaguest knowledge of
the activities of businessmen, he still could never muster the slightest
feeling of awe or respect for such abysmal persons. While, for her part,
Madam Conk could never stretch her imagination to the point of
considering that any being so eccentric as my master could actually
exist, that any corner of the world might harbor such an oddity. Her
experience has included meetings with many people and invariably, as
soon as she declares that she is wife to Goldfield, their attitude
towards her never fails immediately to alter. At any party whatsoever
and no matter how lofty the social standing of any man before whom she
happens to find herself, she has always found that Mrs.~Goldfield is
eminently acceptable. How then could she fail to impress such an obscure
old teacher? She had expected that the mere mention of the fact that her
house was the corner residence of the opposite block would startle my
master even before she added information about Mr.~Goldfield's notable
activities in the world of business.

``Do you know anyone called Goldfield?'' my master inquires of
Waverhouse with the utmost nonchalance.

``Of course I know him. He's a friend of my uncle. Only the other day he
was present at our garden party.'' Waverhouse answers in a serious
manner.

``Really?'' said my master. ``And who, may I ask, is your uncle?''

``Baron Makiyama,'' replied Waverhouse in even graver tones. My master
is obviously about to say something, but before he can bring himself to
words, Madam Conk turns abruptly toward Waverhouse and subjects him to a
piercing stare. Waverhouse, secure in a kimono of the finest silk,
remains entirely unperturbed.

``Oh, you are Baron Makiyama's\ldots{} That I didn't know. I hope you'll
excuse me\ldots{} I've heard so much about Baron Makiyama from my
husband. He tells me that the Baron has always been so helpful\ldots{}''
Madam Conk's manner of speech has suddenly become polite. She even bows.

``Ah yes,'' observes Waverhouse who is inwardly laughing. My master,
quite astonished, watches the two in silence.

``I understand he has even troubled the Baron about our daughter's
marriage\ldots{}''

``Has he indeed?'' exclaims Waverhouse as if surprised. Even Waverhouse
seems somewhat taken aback by this unexpected development.

``We are, in fact, receiving proposal after proposal in respect of
marriage to our daughter. They flood in from all over the place. You
will appreciate that, having to think seriously of our social position,
we cannot rashly marry off our daughter to just anyone\ldots{}''

``Quite so.'' Waverhouse feels relieved.

``I have, in point of fact, made this visit precisely to raise with you
a question about this marriage matter.'' Madam Conk turns back to my
master and reverts to her earlier vulgar style of speech. ``I hear that
a certain Avalon Coldmoon pays you frequent visits. What sort of a man
is he?''

``Why do you want to know about Coldmoon?'' replies my master in a
manner revealing his displeasure.

``Perhaps it is in connection with your daughter's marriage that you
wish to know something about the character of Coldmoon,'' puts in
Waverhouse tactfully.

``If you could tell me about his character, it would indeed be
helpful.''

``Then is it that you want to give your daughter in marriage to
Coldmoon?''

``It's not a question of my wanting to give her.'' Madam Conk
immediately squashes my master. ``Since there will be innumerable
proposals, we couldn't care less if he doesn't marry her.''

``In that case, you don't need any information about Coldmoon,'' my
master replies with matching heat.

``But you've no reason to withhold information.'' Madam Conk adopts an
almost defiant attitude.

Waverhouse, sitting between the two and holding his silver pipe as if it
were an umpire's instrument of office, is secretly beside himself with
glee. His gloating heart urges them on to yet more extravagant
exchanges.

``Tell me, did Coldmoon actually say he wanted to marry her?'' My master
fires a broadside pointblank.

``He didn't actually say he wanted to, but\ldots{}''

``You just think it likely that he might want to?'' My master seems to
have realized that broadsides are best in dealing with this woman.

``The matter is not yet so far advanced, but\ldots{} well, I don't think
Mr.~Coldmoon is wholly averse to the idea.'' Madam Conk rallies well in
her extremity.

``Is there any concrete evidence whatsoever that Coldmoon is enamored of
this daughter of yours?'' My master, as if to say, ``now answer me if
you can,'' sticks out his chest belligerently.

``Well, more or less, yes.'' This time my master's militance has failed
in its effect. Waverhouse has hitherto been so delighted with his
self-appointed role of umpire that he has simply sat and watched the
scrap, but now his curiosity seems suddenly to have been aroused. He
puts down the pipe and leans forward. ``Has Coldmoon sent your daughter
a love letter? What fun! One more new event since the New Year and, at
that, a splendid subject for debate.'' Waverhouse alone is pleased.

``Not a love letter. Something much more ardent than that. Are you two
really so much in the dark?'' Madam Conk adopts a disbelieving attitude.

``Are you aware of anything?'' My master, looking nonplussed, addresses
himself to Waverhouse.

Waverhouse takes refuge in banter. ``I know nothing. If anyone should
know, it would be you.'' His reaction is disappointingly modest.

``But the two of you know all about it,'' Madam Conk triumphs over both
of them.

``Oh!'' The sound expressed their simultaneous astonishment.

``In case you've forgotten, let me remind you of what happened. At the
end of last year Mr.~Coldmoon went to a concert at the Abe residence in
Mukōjima, right? That evening, on his way home, something happened at
Azuma Bridge. You remember? I won't repeat the details since that might
compromise the person in question, but what I've said is surely proof
enough. What do you think?'' She sits bolt-upright with her
diamond-ringed fingers in her lap. Her magnificent nose looks more
resplendent than ever, so much so that Waverhouse and my master seem
practically obliterated.

My master, naturally, but Waverhouse also, appear dumbfounded by this
surprise attack. For a while they just sit there in bewilderment, like
patients whose fits of ague have suddenly ceased. But as the first shock
of their astonishment subsides and they come slowly back to normality,
their sense of humor irrepressibly asserts itself and they burst into
gales of laughter. Madam Conk, baulked in her expectations and,
ill-prepared for this reaction of rude laughing, glares at both of them.

``Was that your daughter? Isn't it wonderful! You're quite right. Indeed
Coldmoon must be mad about her. I say, Sneaze, there's no point now in
trying to keep it secret. Let's make a clean breast of everything.''

My master just says ``Hum.''

``There's certainly no point in your trying to keep it secret. The cat's
already out of the bag.'' Madam Conk is once more cock-a-hoop.

``Yes, indeed, we're cornered. We'll have to make a true statement on
everything concerning Coldmoon for this lady's information. Sneaze!
you're the host here. Pull yourself together, man. Stop grinning like
that or we'll never get this business sorted out. It's extraordinary.
Secretiveness is a most mysterious matter. However well one guards a
secret, sooner or later it's bound to come out. Indeed, when you come to
think of it, it really is most extraordinary. Tell us, Mrs.~Goldfield,
how did you ever discover this secret? I am truly amazed.'' Waverhouse
rattles on.

``I've a nose for these things.'' Madam Conk declares with some
self-satisfaction.

``You must indeed be very well informed. Who on earth has told you about
this matter?''

``The wife of the rickshawman who lives just there at the back.''

``Do you mean that man who owns that vile black cat?'' My master is
wide-eyed.

``Yes, your Mr.~Coldmoon has cost me a pretty penny. Every time he comes
here I want to know what he talks about, so I've arranged for the wife
of the rickshawman to learn what happens and to report it all to me.''

``But that's terrible!'' My master raises his voice.

``Don't worry, I don't give a damn what you do or say. I'm not in the
least concerned with you, only with Mr.~Coldmoon.''

``Whether with Coldmoon or with anyone else\ldots{} Really, that
rickshaw woman is a quite disgusting creature.'' My master begins to get
angry.

``But surely she is free to stand outside your hedge. If you don't want
your conversations overheard, you should either talk less loudly or live
in a larger house.'' Madam Conk is clearly not the least ashamed of
herself. ``And that's not my only source. I've also heard a deal of
stuff from the Mistress of the two-stringed harp.''

``You mean about Coldmoon?''

``Not solely about Coldmoon.'' This sounds menacing but, far from
retreating in embarrassment, my master retorts. ``That woman gives
herself such airs. Acting as though she and she alone were the only
person of any standing in this neighborhood. A vain, an idiotic
fellow\ldots{}''

``Pardon me! It's a woman you're describing. A fellow, did you say?
Believe me, you're talking out of the back of your neck.'' Her language
more and more betrays her vulgar origin. Indeed, it now appears as if
she has only come in order to pick a quarrel. But Waverhouse, typically,
just sits listening to the quarrel as if it were being conducted for his
amusement. Indeed, he looks like a Chinese sage at a cockfight: cool and
above it all.

My master at last realizes that he can never match Madam Conk in the
exchange of scurrilities, and he lapses into a forced silence. But
eventually a bright idea occurs to him.

``You've been speaking as though it were Coldmoon who was besotted with
your daughter, but from what I've heard, the situation is quite
different. Isn't that so, Waverhouse?''

``Certainly. As we heard it, your daughter fell ill and then, we
understand, began babbling in delirium.''

``No.~You've got it all wrong.'' Madam Conk gives the lie direct.

``But Coldmoon undoubtedly said that that was what he had been told by
Dr.~O's wife.''

``That was our trap. We'd asked the Doctor's wife to play that trick on
Coldmoon precisely in order to see how he'd react.''

``Did the doctor's wife agree to this deception in full knowledge that
it was a trick?''

``Yes. Of course we couldn't expect her to help us purely for
affection's sake. As I've said, we've had to lay out a very pretty penny
on one thing and another.''

``You are quite determined to impose yourself upon us and quiz us in
detail about Coldmoon, eh?'' Even Waverhouse seems to be getting annoyed
for he uses some sharpish turns of phrase quite unlike his usual manner.

``Ah well, Sneaze,'' he continues, ``what do we lose if we talk? Let's
tell her everything. Now, Mrs.~Goldfield, both Sneaze and I will tell
you anything within reason about Coldmoon. But it would be more
convenient for us if you'd present your questions one at a time.''

Madam Conk was thus at last brought to see reason. And when she began to
pose her questions, her style of speech, only recently so coarsely
violent, acquired a certain civil polish, at least when she spoke to
Waverhouse. ``I understand,'' she opens, ``that Mr.~Coldmoon is a
bachelor of science. Now please tell me in what sort of subject has he
specialized?''

``In his post-graduate course, he's studying terrestrial magnetism,''
answers my master seriously.

Unfortunately, Madam Conk does not understand this answer.

Therefore, though she says, ``Ah,'' she looks dubious and asks: ``If one
studies that, could one obtain a doctor's degree?''

``Are you seriously suggesting that you wouldn't allow your daughter to
marry him unless he held a doctorate?'' The tone of my master's inquiry
discloses his deep displeasure.

``That's right. After all, if it's just a bachelor's degree, there are
so many of them!'' Madam Conk replies with complete unconcern.

My master's glance at Waverhouse reveals a deepening disgust.

``Since we cannot be sure whether or not he'll gain a doctorate, you'll
have to ask us something else.'' Waverhouse seems equally displeased.

``Is he still just studying that terrestrial something?''

``A few days ago,'' my master quite innocently offers, ``he made a
speech on the results of his investigation of the mechanics of
hanging.''

``Hanging? How dreadful! He must be peculiar. I don't suppose he could
ever become a doctor by devoting himself to hanging.''

``It would of course be difficult for him to gain a doctorate if he
actually hanged himself, but it is not impossible to become a doctor
through study of the mechanics of hanging.''

``Is that so?'' she answers, trying to read my master's expression. It's
a sad, sad thing but, since she does not know what mechanics are, she
cannot help feeling uneasy. She probably thinks that to ask the meaning
of such a trifling matter might involve her in loss of face. Like a
fortuneteller, she tries to guess the truth from facial expressions. My
master's face is glum. ``Is he studying anything else, something more
easy to understand?''

``He once wrote a treatise entitled `A Discussion of the Stability of
Acorns in Relation to the Movements of Heavenly Bodies.'\,''

``Does one really study such things as acorns at a university?''

``Not being a member of any university, I cannot answer your question
with complete certainty, but since Coldmoon is engaged in such studies,
the subject must undoubtedly be worth studying.'' With a dead-pan face,
Waverhouse makes fun of her.

Madam Conk seems to have realized that her questions about matters of
scholarship have carried her out of her depth, for she changes the
subject. ``By the way,'' she says, ``I hear that he broke two of his
front teeth when eating mushrooms during the New Year season.''

``True, and a rice-cake became fixed on the broken part.''

Waverhouse, feeling that this question is indeed up his street, suddenly
becomes light-hearted.

``How unromantic! I wonder why he doesn't use a toothpick!''

``Next time I see him, I'll pass on your sage advice,'' says my master
with a chuckle.

``If his teeth can be snapped on mushrooms, they must be in very poor
condition. What do you think?''

``One could hardly say such teeth were good. Could one, Waverhouse?''

``Of course they can't be good, but they do provide a certain humor.
It's odd that he hasn't had them filled. It really is an extraordinary
sight when a man just leaves his teeth to become mere hooks for snagging
rice-cakes.''

``Is it because he lacks the money to get them filled or because he's
just so odd that he leaves them unattended to?''

``Ah, you needn't worry. I don't suppose he will continue as Mr.~Broken
Front Tooth for any long time.'' Waverhouse is evidently regaining his
usual bouyancy.

Madam Conk again changes the subject. ``If you should have some letter
or anything which he's written, I'd like to see it.''

``I have masses of postcards from him. Please have a look at them,'' and
my master produces some thirty or forty postcards from his study.

``Oh, I don't have to look at so many of them\ldots{} perhaps two or
three would do\ldots{}''

``Let me choose some for you,'' offers Waverhouse, adding as he selects
a picture postcard, ``Here's an interesting one.''

``Gracious! So he paints pictures as well? Rather clever that,'' she
exclaims. But after examining the picture she remarks ``How very silly!
It's a badger! Why on earth does he have to paint a badger of all
things! Strange. But it does indeed look like a badger.'' She is, albeit
reluctantly, mildly impressed.

``Read what he's written beside it,'' suggests my master with a laugh.

Madam Conk begins to read aloud like a servant-girl deciphering a
newspaper.

``On New Year's Eve, as calculated under the ancient calendar, the
mountain badgers hold a garden party at which they dance excessively.

Their song says, `This evening, being New Year's Eve, no mountain hikers
will come this way.' And bom-bom-bom they thump upon their bellies. What
is he writing about? Is he not being a trifle frivolous?'' Madam Conk
seems seriously dissatisfied.

``Doesn't this heavenly maiden please you?'' Waverhouse picks out
another card on which a kind of angel in celestial raiment is depicted
as playing upon a lute.

``The nose of this heavenly maiden seems rather too small.''

``Oh no, that's about the average size for an angel. But forget the nose
for the moment and read what it says,'' urges Waverhouse.

``It says `Once upon a time there was an astronomer. One night he went
as was his wont high up into his observatory, and, as he was intently
watching the stars, a beautiful heavenly maiden appeared in the sky and
began to play some music; music too delicate ever to be heard on earth.
The astronomer was so entranced by the music that he quite forgot the
dark night's bitter cold. Next morning the dead body of the astronomer
was found covered with pure white frost. An old man, a liar, told me
that this story was all true.' What the hell is this? It makes no sense,
no nothing. Can Coldmoon really be a bachelor of science? Perhaps he
should read a few literary magazines.'' Thus mercilessly does Madam Conk
lambaste the defenseless Coldmoon.

Waverhouse for fun selects a third postcard and says, ``Well then, what
about this one?'' The card has a sailing boat printed on it and, as
usual, there is something scribbled underneath the picture.

Last night a tiny whore of sixteen summers

Declared she had no parents.

Like a plover on a reefy coast,

She wept on waking in the early morning.

Her parents, sailors both, lie at the bottom of the sea.

``Oh, that's good. How very clever! He's got real feeling,'' erupted
Madam Conk.

``Feeling?'' says Waverhouse.

``Oh yes,'' says Madam Conk. ``That would go well on a samisen.

``If it could be played on the samisen, then it's the real McCoy. Well,
how about these?'' asks Waverhouse picking out postcard after postcard.

``Thank you, but I've seen enough. For now, at least I know that
Coldmoon's not a straight-laced prude.'' She thinks she has achieved
some real understanding and appears to have no more queries about
Coldmoon, for she remarks, ``I'm sorry to have troubled you. Please do
not report my visit to Mr.~Coldmoon.'' Her request reflects her selfish
nature in that she seems to feel entitled to make a thorough
investigation of Coldmoon whilst expecting that none of her activities
should be revealed to him. Both Waverhouse and my master concede a
half-hearted ``Y-es,'' but as Madam Conk gets up to leave, she
consolidates their assent by saying, ``I shall, of course, at some later
date repay you for your services.''

The two men showed her out and, as they resumed their seats, Waverhouse
exclaimed, ``What on earth is that?'' At the very same moment my master
also ejaculated, ``Whatever's that?'' I suppose my master's wife could
not restrain her laughter any longer, for we heard her gurgling in the
inner-room.

Waverhouse thereupon addressed her in a loud voice through the sliding
door. ``That, Mrs.~Sneaze, was a remarkable specimen of all that is
conventional, of all that is `common or garden.' But when such
characteristics become developed to that incredible degree the result is
positively staggering. Such quintessence of the common approximates to
the unique. Don't seek to restrain yourself. Laugh to your heart's
content.''

With evident disgust my master speaks in tones of the deepest revulsion.
``To begin with,'' he says, ``her face is unattractive.''

Waverhouse immediately takes the cue. ``And that nose, squatting, as it
were, in the middle of that phiz, seems affectedly unreal.''

``Not only that, it's crooked.''

``Hunchbacked, one might say. A hunchbacked nose! Quite extraordinary.''
And Waverhouse laughs in genuine delight.

``It is the face of a woman who keeps her husband under her bottom.''

My master still looks resentful.

``It is a sort of physiognomy that, left unsold in the nineteenth
century, becomes in the twentieth shop-soiled.'' Waverhouse produces
another of his invariably bizarre remarks. At which juncture my master's
wife emerges from the inner-room and, being a woman and thus aware of
the ways of women, quietly warns them, ``If you talk such scandal, the
rickshaw-owner's wife will snitch on you again.''

``But, Mrs.~Sneaze, to hear such tattle will do that Goldfield woman no
end of good.''

``But it's self-demeaning to calumniate a person's face. No one sports
that sort of nose as a matter of choice. Besides, she is a woman. You're
going a little too far.'' Her defense of the nose of Madam Conk is
simultaneously an indirect defense of her own indifferent looks.

``We're not unkind at all. That creature isn't a woman. She's just an
oaf. Waverhouse, am I not right?''

``Maybe an oaf, but a formidable character nonetheless. She gave you
quite a tousling, didn't she just?''

``What does she take a teacher for, anyway?''

``She ranks a teacher on roughly the same level as a rickshaw-owner.

To earn the respect of such viragoes one needs to have at least a
doctor's degree. You were ill-advised not to have taken your doctorate.
Don't you agree, Mrs.~Sneaze?'' Waverhouse looks at her with a smile.

``A doctorate? Quite impossible.'' Even his wife despairs of my master.

``You never know. I might become one, one of these days. You mustn't
always doubt my worth. You may well be ignorant of the fact, but in
ancient times a certain Greek, Isocrates, produced major literary works
at the age of ninety-four. Similarly, Sophocles was almost a centenarian
when he shook the world with his masterpiece. Simonides was writing
wonderful poetry in his eighties. I, too\ldots{}''

``Don't be silly. How can you possibly expect, you with your stomach
troubles, to live that long.'' Mrs.~Sneaze has already determined my
master's span of life.

``How dare you! Just go and talk to Dr.~Amaki. Anyway, it's all your
fault. It's because you make me wear this crumpled black cotton surcoat
and this patched-up kimono that I am despised by women like
Mrs.~Goldfield. Very well then. From tomorrow I shall rig myself out in
such fineries as Waverhouse is wearing. So get them ready.''

``You may well say `get them ready,' but we don't possess any such
elegant clothes. Anyway, Mrs.~Goldfield only grew civil to Waverhouse
after he'd mentioned his uncle's name. Her attitude was in no way
conditioned by the ill-condition of your kimono.'' Mrs.~Sneaze has
neatly dodged the charge against her.

The mention of that uncle appears to trigger my master's memory, for he
turns to Waverhouse and says, ``That was the first I ever heard of your
uncle. You never spoke of him before. Does he, in fact, exist?''

Waverhouse has obviously been expecting this question, and he jumps to
answer it. ``Yes, that uncle of mine, a remarkably stubborn man. He,
too, is a survival from the nineteenth century.'' He looks at husband
and wife.

``You do say the quaintest things. Where does this uncle live?'' asks
Mr.

Sneaze with a titter.

``In Shizuoka. But he doesn't just live. He lives with a top-knot still
on his head. Can you beat it? When we suggest he should wear a hat, he
proudly answers that he has never found the weather cold enough to don
such gear. And when we hint that he might be wise to stay abed when the
weather's freezing, he replies that four hour's sleep is sufficient for
any man. He is convinced that to sleep more than four hours is sheer
extravagance, so he gets up while it's still pitch-dark. It is his boast
that it took many long years of training so to minimize his sleeping
hours. `When I was young,' he says, `it was indeed hard because I felt
sleepy, but recently I have at last achieved that wonderful condition
where I can sleep or wake, anywhere, anytime, just as I happen to wish.'
It is of course natural that a man of sixty-seven should need less
sleep. It has nothing to do with early training, but my uncle is happy
in the belief that he has succeeded in attaining his present condition
entirely as a result of rigorous self-discipline. And when he goes out,
he always carries an iron fan.''

``Whatever for?'' asks my master.

``I haven't the faintest idea. He just carries it. Perhaps he prefers a
fan to a walking stick. As a matter of fact an odd thing happened only
the other day.'' Waverhouse speaks to Mrs.~Sneaze.

``Ah yes?'' she noncommittally responds.

``In the spring this year he wrote to me out of the blue with a request
that I should send him a bowler hat and a frock-coat. I was somewhat
surprised and wrote back asking for further clarification. I received an
answer stating that the old man himself intended to wear both items on
the occasion of the Shizuoka celebration of the war victory, and that I
should therefore send them quickly. It was an order. But the quaintness
of his letter was that it enjoined me `to choose a hat of suitable size
and, as for the suit, to go and order one from Daimaru of whatever size
you think appropriate.'\,''

``Can one get suits made at Daimaru?''

``No.~I think he'd got confused and meant to say at Shirokiya's.''

``Isn't it a little unhelpful to say `of whatever size you think
appropriate'?''

``That's just my uncle all over.''

``What did you do?''

``What could I do? I ordered a suit which I thought appropriate and sent
it to him.''

``How very irresponsible! And did it fit?''

``More or less, I think. For I later noticed in my home-town newspaper
that the venerable Mr.~Makiyama had created something of a sensation by
appearing at the said celebration in a frock coat carrying, as usual,
his famous iron fan.''

``It seems difficult to part him from that object.''

``When he's buried, I shall ensure that the fan is placed within the
coffin.''

``Still it was fortunate that the coat and bowler fitted him.''

``But they didn't. Just when I was congratulating myself that everything
had gone off smoothly, a parcel came from Shizuoka. I opened it
expecting some token of his gratitude, but it proved only to contain the
bowler. An accompanying letter stated, `Though you have taken the
trouble of making this purchase for me, I find the hat too large. Please
be so kind as to take it back to the hatter's and have it shrunk. I will
of course defray your consequent expenses by postal order.'\,''

``Peculiar, one must admit.'' My master seems greatly pleased to
discover that there is someone even more peculiar than himself. ``So
what did you do?'' he asks.

``What did I do? I could do nothing. I'm wearing the hat myself.''

``And is that the very hat?'' says my master with a smirk.

``And he's a Baron?'' asks my master's wife from her mystification.

``Is who?''

``Your uncle with the iron fan.''

``Oh, no. He's a scholar of the Chinese classics. When he was young he
studied at that shrine dedicated to Confucius in Yushima and became so
absorbed in the teachings of Chu-Tzu that, most reverentially, he
continues to wear a top-knot in these days of the electric light.
There's nothing one can do about it.'' Waverhouse rubs his chin.

``But I have the impression that in speaking just now to that awful
woman you mentioned a Baron Makiyama.''

``Indeed you did. I heard you quite distinctly, even in the other
room.''

Mrs.~Sneaze for once supports her husband.

``Oh, did I?'' Waverhouse permits himself a snigger. ``Fancy that. Well,
it wasn't true. Had I a Baron for an uncle I would by now be a senior
civil servant.'' Waverhouse is not in the least embarrassed.

``I thought it was somehow queer,'' says my master with an expression
half-pleased, half-worried.

``It's astonishing how calmly you can lie. I must say you're a past
master at the game.'' Mrs.~Sneaze is deeply impressed.

``You flatter me. That woman quite outclasses me.''

``I don't think she could match you.''

``But, Mrs.~Sneaze, my lies are merely tarrydiddles. That woman's lies,
every one of them, have hooks inside them. They're tricky lies. Lies
loaded with malice aforethought. They are the spawn of craftiness.
Please never confuse such calculated monkey-minded wickedness with my
heaven-sent taste for the comicality of things. Should such confusion
prevail, the God of Comedy would have no choice but to weep for
mankind's lack of perspicacity.''

``I wonder,'' says my master, lowering his eyes, while Mrs.~Sneaze,
still laughing, remarks that it all comes down to the same thing in the
end.

Up until now I have never so much as crossed the road to investigate the
block opposite. I have never clapped eyes on the Goldfield's corner
residence so I naturally have no idea what it looks like. Indeed today
is the first time that I've even heard of its existence. No one in this
house has ever previously talked about a businessman and consequently I,
who am my master's cat, have shared his total disinterest in the world
of business and his equally total indifference to businessmen. However,
having just been present during the colloquy with Madam Conk, having
overheard her talk, having imagined her daughter's beauty and charm, and
also having given some thought to that family's wealth and power, I have
come to realize that, though no more than a cat, I should not idle all
my days away lying on the veranda. Nor only that, I cannot help but feel
deep sympathy with Coldmoon. His opponent has already bribed a doctor's
wife, bribed the wife of the rickshaw-owner, bribed even that
high-falutin mistress of the two-stringed harp. She has so spied upon
poor Coldmoon that even his broken teeth have been disclosed, while he
has done no more than fiddle with the fastenings of his surcoat and, on
occasion, grin. He is guileless even for a bachelor of science just out
of the university. And it's not just anyone who can cope with a woman
equipped with such a jut of nose. My master not only lacks the heart for
dealing with matters of this sort, but he lacks the money, too.

Waverhouse has sufficient money, but is such an inconsequential being
that he'd never go out of his way merely to help Coldmoon. How isolated,
then, is that unfortunate person who lectures on the mechanics of
hanging. It would be less than fair if I failed at least to try and
insinuate myself into the enemy fortress and, for Coldmoon's sake, pick
up news of their activities. Though but a cat, I am not quite as other
cats. I differ from the general run of idiot cats and stupid cats. I am
a cat that lodges in the house of a scholar who, having read it, can
bang down any book by Epictetus on his desk. Concentrated in the tip of
my tail there is sufficient of the spirit of chivalry for me to take it
upon myself to venture upon knight-errantry. It is not that I am in any
way beholden to poor Coldmoon, nor am I engaging in foolhardy action for
the sake of any single individual. If I may be allowed to blow my own
trumpet, I am proposing to take magnificent unself-interested action
simply in order to realize the will of Heaven that smiles upon
impartiality and blesses the happy medium. Since Madam Conk makes
impermissible use of such things as the happenings at Azuma Bridge;
since she hires underlings to spy and eavesdrop on us; since she
triumphantly retails to all and sundry the products of her espionage;
since by the employment of rickshaw-folk, mere grooms, plain rogues,
student riff-raff, crone daily-help, midwives, witches, masseurs, and
other trouble-makers she seeks to trouble a man of talent; for all these
reasons even a cat must do what can be done to prevent her getting away
with it.

The weather, fortunately, is fine. The thaw is something of nuisance,
but one must be prepared to sacrifice one's life in the cause of
justice. If my feet get muddy and stamp plum blossom patterns on the
veranda, O-san may be narked but that won't worry me. For I have come to
the superlatively courageous, firm decision that I will not put off
until tomorrow what needs to be done today. Accordingly, I whisk off
around to the kitchen, but, having arrived there, pause for further
thought.

``Softly, softly,'' I say to myself. It's not simply that I've attained
the highest degree of evolution that can occur in cats, but I make bold
to believe my brain is as well-developed as that of any boy in his third
year at a middle school. Nevertheless, alas, the construction of my
throat is still only that of a cat, and I cannot therefore speak the
babbles of mankind. Thus, even if I succeed in sneaking into the
Goldfield's citadel and there discovering matters of moment, I shall
remain unable to communicate my discoveries to that Coldmoon who so
needs them. Neither shall I be able to communicate my gleanings to my
master or to Waverhouse. Such incommunicable knowledge would, like a
buried diamond, be denied its brilliance and my hard-won wisdom would
all be won for nothing.

Which would be stupid. Perhaps I should scrap my plan. So thinking, I
hesitated on the very doorstep.

But to abandon a project halfway through breeds a kind of regret, that
sense of unfulfillment which one feels when the slower one had so
confidently expected drifts away under inky clouds into some other part
of the countryside. Of course, to persist when one is in the wrong is an
altogether different matter, but to press on for the sake of so-called
justice and humanity, even at the risk of death uncrowned by success,
that, for a man who knows his duty, can be a source of the deepest
satisfaction. Accordingly, to engage in fruitless effort and to muddy
one's paws on a fool's errand would seem about right for a cat. Since it
is my misfortune to have been born a cat, I cannot by turns of the tip
of my tail convey, as I can to cats, my thinking to such scholars as
Coldmoon, Sneaze, and Waverhouse. However, by virtue of felinity, I can,
better than all such bookmen, make myself invisible. To do what no one
else can do is, of itself, delightful. That I alone should know the
inner workings of the Goldfield household is better than if nobody
should know.

Though I cannot pass my knowledge on, it is still cause for delight that
I may make the Goldfields conscious that someone knows their secrets.

In the light of this succession of delights, I boldly make to believe my
brain is as delightful as well. All right then. I will go.

Coming to the side street in the opposite block, there, sure enough, I
find a Western-style house dominating the crossroads as if it owned the
whole area. Thinking that the master of such a house must be no less
stuck up than his building, I slide past the gate and examine the
edifice. Its construction has no merit. Its two stories rear up into the
air for no purpose whatever but to impress, even to coerce, the
passersby. This, I suppose, is what Waverhouse means when he calls
things common or garden. I slink through some bushes, take note of the
main entrance to my right, and so find my way round to the kitchen. As
might be expected, the kitchen is large---at least ten times as large as
that in my master's dwelling.

Everything is in such apple pie order, all so clean and shining, that it
cannot be less splendid than that fabulous kitchen of Count Okuma so
ful-somely described in a recent product of the national press. I tell
myself, as I slip inside on silent muddy paws, that this must be ``a
model kitchen.''

On the plastered part of its floor the wife of the rickshaw-owner is
standing in earnest discussion with a kitchen-maid and a
rickshaw-runner.

Realizing the dangers of this situation, I hide behind a water-tub.

``That teacher, doesn't he really even know our master's name?'' the
kitchen-maid demands.

``Of course he knows it. Anyone in this district who doesn't know the
Goldfield residence must be a deaf cripple without eyes,'' snaps the man
who pulls the Goldfield's private rickshaw.

``Well, you never know. That teacher's one of those cranks who know
nothing at all except what it says in books. If he knew even the least
little thing about Mr.~Goldfield he might be scared out of his wits. But
he hasn't the wits to be scared out of. Why,'' snorts Blacky's
bloody-minded mistress, ``he doesn't even know the ages of his own
mis-managed children.''

``So he's not afraid of our Mr.~Goldfield! What a cussed clot he is!
There's no call to show him the least consideration. Let's go around and
give him something to be scared about.''

``Good idea! He says such dreadful things. He was telling his crackpot
cronies that, since Madam's nose is far too big for her face, he finds
her unattractive. No doubt he thinks himself a proper picture, but his
mug's the spitting image of a terra-cotta badger. What can be done, I
ask you, with such an animal?''

``And it isn't only his face. The way he saunters down to the public
bathhouse carrying a hand towel is far too high and mighty. He thinks
he's the cat's whiskers.'' My master Sneaze seems notably unpopular,
even with this kitchen-maid.

``Let's all go and call him names as loud as we can from just outside
his hedge.''

``That'll bring him down a peg.''

``But we mustn't let ourselves be seen. We must spoil his studying just
with shouting, getting him riled as much as we can. Those are Madam's
latest orders.''

``I know all that,'' says the rickshaw wife in a voice that makes it
clear that she's only too ready to undertake one-third of their
scurrilous assignment. Thinking to myself, ``So that's the gang who're
going to ridicule my master,'' I drift quietly past the noisesome trio
and penetrate yet further into the enemy fortress.

Cat's paws are as if they do not exist. Wheresoever they may go, they
never make clumsy noises. Cats walk as if on air, as if they trod the
clouds, as quietly as a stone going light-tapped under water, as an
ancient Chinese harp touched in a sunken cave. The walking of a cat is
the instinctive realization of all that is most delicate. For such as I
am concerned, this vulgar Western house simply is not there. Nor do I
take cognizance of the rickshaw-woman, manservant, kitchen-maids, the
daughter of the house, Madam Conk, her parlor-maids or even her ghastly
husband. For me they do not exist. I go where I like and I listen to
whatever talk it interests me to hear. Thereafter, sticking out my
tongue and frisking my tail, I walk home self-composedly with my
whiskers proudly stiff. In this particular field of endeavor there's not
a cat in all Japan so gifted as am I. Indeed, I sometimes think I really
must be blood-kin to that monster cat one sees in ancient picture books.
They say that every toad carries in its forehead a gem that in the
darkness utters light, but packed within my tail I carry not only the
power of God, Buddha, Confucius, Love, and even Death, but also an
infallible panacea for all ills that could bewitch the entire human
race. I can as easily move unnoticed through the corridors of
Goldfield's awful mansion as a giant god of stone could squash a
milk-blancmange.

At this point, I become so impressed by my own powers and so conscious
of the reverence I consequently owe to my own most precious tail that I
feel unable to withhold immediate recognition of its divinity. I desire
to pray for success in war by worshiping my honored Great Tail Gracious
Deity, so I lower my head a little, only to find I am not facing in the
right direction. When I make the three appropriate obeisances I should,
of course, as far as it is possible, be facing toward my tail. But as I
turn my body to fulfill that requirement, my tail moves away from me.

In an effort to catch up with myself, I twist my neck. But still my tail
eludes me. Being a thing so sacred, containing as it does the entire
universe in its three-inch length, my tail is inevitably beyond my power
to control. I spun round in pursuit of it seven and a half times but,
feeling quite exhausted, I finally gave up. I feel a trifle giddy. For a
moment I lose all sense of where I am and, deciding that my whereabouts
are totally unimportant, I start to walk about at random. Then I hear
the voice of Madam Conk. It comes from the far side of a paper-window.
My ears prick up in sharp diagonals and, once more fully alert, I hold
my breath.

This is the place which I set out to find.

``He's far too cocky for a penny-pinching usher,'' she's screaming in
that parrot's voice.

``Sure, he's a cocky fellow. I'll have a bit of the bounce taken out of
him, just to teach him a lesson. There are one or two fellows I know,
fellows from my own province, teaching at his school.''

``What fellows are those?''

``Well, there's Tsuki Pinsuke and Fukuchi Kishago for a start. I'll
arrange with them for him to be ragged in class.''

I don't know from what province old man Goldfield comes, but I'm rather
surprised to find it stiff with such outlandish names.

``Is he a teacher of English?'' her husband asks.

``Yes. According to the wife of the rickshaw-owner, his teaching
specializes in an English Reader or something like that.''

``In any case, he's gotta be a rotten teacher.''

I'm also struck by the vulgarity of that ``gotta be'' phraseology.

``When I saw Pinsuke the other day he mentioned that there was some
crackpot at his school. When asked the English word for bancha, this
fathead answered that the English called it, not `coarse tea' as they
actually do, but `savage tea.' He's now the laughing stock of all his
teaching colleagues. Pinsuke added that all the other teachers suffer
for this one's follies. Very likely it's the self-same loon.''

``It's bound to be. He's got the face you'd expect on a fool who thinks
that tea can be savage. And to think he has the nerve to sport such a
dashing mustache!''

``Saucy bastard.''

If whiskers establish sauciness, every cat is impudent.

``As for that man Waverhouse---Staggering Drunk I'd call him---he's an
obstreperous freak if ever I saw one. Baron Makiyama, his uncle indeed!
I was sure that no one with a face like his could have a baron for an
uncle.''

``You, too, are at fault for believing anything which a man of such
dubious origins might say.''

``Maybe I was at fault. But really there's a limit and he's gone much
too far.'' Madam Conk sounds singularly vexed. The odd thing is that
neither mentions Coldmoon. I wonder if they concluded their discussion
about him before I sneaked up on them or whether perhaps they had
earlier decided to block his marriage suit and had therefore already
forgotten all about him. I remain disturbed about this question, but
there's nothing I can do about it. For a little while I lay crouched
down in silence but then I heard a bell ring at the far end of the
corridor. What's up down there? Determined this time not to be late on
the scene, I set out smartly in the direction of the sound.

I arrived to find some female yattering away by herself in a loud
unpleasant voice. Since her tones resemble those of Madam Conk, I deduce
that this must be that darling daughter, that delicious charmer for
whose sake Coldmoon has already risked death by drowning.

Unfortunately, the paper-windows between us make it impossible for me to
observe her beauty and I cannot therefore be sure whether she, too, has
a massive nose plonked down in the center of her face. But I infer from
her mannerisms, such as the way she sounds to be turning up her nose
when she talks, that that organ is unlikely to be an inconspicuous
pug-nose. Though she talks continuously, nobody seems to be answering,
and I deduce that she must be using one of those modern telephones.

``Is that the Yamato? I want to reserve, for tomorrow, the third box in
the lower gallery. All right? Got it? What's that? You can't? But you
must. Why should I be joking? Don't be such a fool. Who the devil are
you? Chōkichi? Well, Chōkichi, you're not doing very well. Ask the
proprietress to come to the phone. What's that? Did you say you were
able to cope with any possible inquiries? How dare you speak to me like
that? D'you know who I am? This is Miss Goldfield speaking. Oh, you're
well aware of that, are you? You really are a fathead. Don't you
understand, this is the Goldfield. Again? You thank us for being regular
patrons? I don't want your stupid thanks. I want the third box in the
lower gallery. Don't laugh, you idiot. You must be terribly stupid. You
are, you say? If you don't stop being insolent, I shall just ring off.
You understand? I can promise you you'll be sorry. Hello. Are you still
there? Hello, hello. Speak up. Answer me. Hello, hello, hello.''
Chōkichi seems to have hung up, for no answer is forthcoming. The girl
is now in something of a tizzy and she grinds away at the telephone
handle as though she's gone off her head. A lapdog somewhere around her
feet suddenly starts to yap, and, realizing I'd better keep my wits
about me, I quickly hop off the veranda and creep in under the house.

Just then I hear approaching footsteps and the sound of a paper-door
being slid aside. I tilt my head to listen.

``Your father and mother are asking for you, Miss.'' It sounds like a
parlor-maid.

``So who cares?'' was the vulgar answer.

``They sent me to fetch you because they've something they want to tell
you.''

``You're being a nuisance. I said I just don't care.'' She snubs the
maid once more.

``They said it's something to do with Mr.~Coldmoon.'' The maid tries
tactfully to put this young vixen into a better humor.

``I couldn't care less if they want to talk about Coldmoon or
Piddlemoon. I abominate that man with his daft face looking like a
bewildered gourd.'' Her third sour outburst is directed at the absent
Coldmoon. ``Hello,'' she suddenly goes on, ``when did you start dressing
your hair in the Western style?''

The parlor-maid gulps and then replies as briefly as she can ``Today.''

``What sauce. A mere parlor-maid, what's more.'' Her fourth attack comes
in from a different direction. ``And isn't that a brand new collar
you've got on?''

``Yes, it's the one you gave me recently. I've been keeping it in my box
because it seemed too good for the likes of me, but my other collar
became so grubby I thought I'd make the change.''

``When did I give it you?''

``It was January you bought it. At Shirokiya's. It's got the ranks of
sumō wrestlers set out as decoration on the greeny-brown material. You
said it was too somber for your style. So you gave it me.''

``Did I? Well, it certainly looks nice on you. How very provoking!''

``I'm much obliged!''

``I didn't intend a compliment. I'm very much put out.''

``Yes, Miss.''

``Why did you accept something which so very much becomes you without
letting me know that it would?''

``But Miss\ldots{}''

``Since it looks that nice on you, it couldn't fail, could it, to look
more nice on me?''

``I'm sure it would have looked delightful on you.''

``Then why didn't you say so? Instead of that, you just stand there
wearing it when you know I'd like it back. You little beast.'' Her
vituperations seem to have no end. I was wondering what would happen
when, from the room at the other end of the house, old man Goldfield
himself suddenly roared out for his daughter. ``Opula,'' he bellowed.

``Opula, come here.'' She had no choice but to obey and mooched sulkily
out of the room containing the telephone. Her lapdog, slightly bigger
than myself with its eyes and mouth all bunched together in the middle
of its revolting mug, slopped along behind her.

Thereupon, with my usual stealthy steps, I tiptoed back to the kitchen
and, through the kitchen-door, found my way to the street, and so back
home. My expedition has been notably successful.

Coming thus suddenly from a beautiful mansion to our dirty little
dwelling, I felt as though I had descended from a sunlit mountaintop to
some dark dismal grot. Whilst on my spying mission. I'd been far too
busy to take any notice of the ornaments in the rooms, of the decoration
of the sliding-doors and paper-windows or of any similar features, but
as soon as I returned and became conscious of the shabbiness of home, I
found myself yearning for what Waverhouse claims to despise. I am
inclined to think that, after all, there's a good deal more to a
businessman than there is to a teacher. Uncertain of the soundness of
this thinking, I consult my infallible tail. The oracle confirms that my
thinking is correct.

I am surprised to find Waverhouse still sitting in my master's room.

His cigarette stubs, stuffed into the brazier, make it look like a
beehive.

Comfortably cross-legged on the floor, he is, as usual, talking. It
appears, moreover, that during my absence Coldmoon has dropped in.

My master, his head pillowed on his arms, lies flat on his back rapt in
contemplation of the pattern of the rainmarks on the ceiling. It is
another of those meetings of hermits in a peaceful reign.

``Coldmoon, my dear fellow, I seem to remember that you insisted upon
maintaining as the darkest of dark secrets the name of that young lady
who called your name from the depths of her delirium. But surely the
time has now come when you could reveal her identity?''

Waverhouse begins to niggle Coldmoon.

``Were it just solely my concern, I wouldn't mind telling you, but since
any such disclosure might compromise the other party\ldots{}''

``So you still won't tell?''

``Besides, I promised the Doctor's wife\ldots{}''

``Promised never to tell anyone?''

``Yes,'' says Coldmoon back at his usual fiddling with the strings of
his surcoat. The strings are a bright purple, objects of a color one
could never nowadays find in any shop.

``The color of those strings is early nineteenth century'' remarks my
supine master. He is genuinely quite indifferent to anything that
concerns the Goldfields.

``Quite. It couldn't possibly belong to these times of the
Russo-Japanese War. That kind of string would be appropriate only to the
garments worn by the rank and file of soldiers under the Shogunate. It
is said that on the occasion of his marriage, nearly four hundred years
ago, Oda Nobunaga dressed his hair back in the fashion of a tea whisk,
and I have no doubt his projecting top-knot was bound with precisely
such a string.'' Waverhouse goes, as usual, all around the houses to
make his little point.

``As a matter of fact, my grandfather wore these strings at the time,
not forty years back, when the Tokugawa were putting down the last
rebellion before the restoration of the Emperor.'' Coldmoon takes it all
dead seriously.

``Isn't it then about time you presented those strings to a museum? For
that well-known lecturer on the mechanics of hanging, that leading
bachelor of science, Mr.~Avalon Coldmoon to go around looking like a
relic of mediaevalism would scarcely help his reputation.''

``I myself would be only too ready to follow your advice. However,
there's a certain person who says that these strings do specially become
me\ldots{}''

``Who on earth could have made such an imperceptive comment?''

asks my master in a loud voice as he rolls over onto his side.

``A person not of your acquaintance.''

``Never mind that. Who was it?''

``A certain lady.''

``Gracious me, what delicacy! Shall I guess who it is? I think it's the
lady who whimpered for you from the bottom of the Sumida River. Why
don't you tie up your surcoat with those nice purple strings and go on
out and get drowned again?'' Waverhouse offers a helpful suggestion.

Coldmoon laughs at the sally. ``As a matter of fact she no longer calls
me from the riverbed. She is now, as it were, in the Pure Land, a little
northwest from here\ldots{}''

``Don't hope for too much purity. That ghastly nose looks singularly
unwholesome.''

``Eh?'' says Coldmoon, looking puzzled.

``The Archnose from over the way has just been round to see us. Yes,
right here. I can tell you we had quite a surprise. Hadn't we, Sneaze?''

``We had,'' replies my master still lying on his side but now sipping
tea.

``Whom do you mean by the Archnose?''

``We mean the honorable mother of your ever-darling lady.''

``Oh!''

``A woman calling herself Mrs.~Goldfield came round here asking all
sorts of questions about you.'' My master, clarifying the situation,
speaks quite seriously.

I watch poor Coldmoon, wondering if he will be surprised or pleased or
embarrassed, but in fact he looks exactly as he always does. And in his
accustomed quiet tones he comments ``I suppose she's asking if I'll
marry the daughter? Was that it?'' and he goes on twisting and
untwisting his purple strings.

``Far from it! That mother happens to own the most enormous
nose\ldots{}'' But before Waverhouse could finish his sentence my master
interrupted him with a sudden irrelevance.

``Listen,'' he chirps, ``I've been trying to compose a new-style haiku
on that snout of hers.'' Mrs.~Sneaze begins to giggle in the next room.

``You're taking it all extremely lightly! And have you composed your
poem?''

``I've made a start. The first line goes `A Conker Festival takes place
in this face.'\,''

``And then?''

``\,`At which one offers sacred wine.'\,''

``And the concluding line?''

``I've not yet got to that.''

``Interesting,'' says Coldmoon with a grin.

``How about this for the missing line?'' improvises Waverhouse. ``\,`Two
orifices dim.'\,''

Whereupon Coldmoon offers, ``\,`So deep no hairs appear.'\,''

They were thus thoroughly enjoying themselves by proposing wilder and
wilder lines when from the street beyond the hedge came the voices of
several people shouting ``Where's that terra-cotta badger? Come on out,
you terra-cotta badger. Terra-cotta badger! Yah!''

Both my master and Waverhouse look somewhat startled and they peer out
through the hedge. Loud hoots of derisive laughter are followed by the
sound of footsteps running away.

``Whatever can they mean by a terra-cotta badger?'' Waverhouse asks in
puzzled tones.

``I've no idea,'' replies my master.

``An unusual occurrence,'' says Coldmoon.

Waverhouse suddenly gets to his feet as if he had remembered something.
``For some several lustra,'' he declaims in parody of the style of
public lecturers, ``I have devoted myself to the study of aesthetic
nasofrontology and I would accordingly now like to trespass on your time
and patience in order to present certain interim conclusions at which I
have arrived.'' His initiative has been so suddenly taken that my master
just stares up at him in silent blank amazement.

Coldmoon's tiniest voice observes, ``I'd love to hear your interim
conclusions.''

``Though I have made a thorough study of this matter, the origin of the
nose remains, alas, still deeply obfuscated. The first question that
arises reflects the assumption that the nose is intended for use. The
functional approach. If that premise is valid, would not two mere
vent-holes meet the case? There is no obvious need either for such
arrogant profusion or for the nasal arrogation of a median position in
the human physiognomy. Why then should the nasal organ thus,'' and he
paused to pinch his own, ``thrust itself forward?''

``Yours doesn't stick out much,'' cuts in my master rather rudely.

``At any rate it has no indentations, no incurvations; still less could
it be described as countersunk or infundibular. I draw your attention to
these facts because if you fail to make the necessary distinction
between having two holes in the medio-frontal area of the face and
having two such holes in some form of protuberance, you will inevitably
be unable to follow the quintessential drift of my dissertation. Now, it
is at least my own, albeit humble, opinion that it is by an accumulation
of human actions trifling in themselves, for who could attach major
importance to the blowing of one's nose, that the organ in question has
developed into its present phenomenal form.''

``How very humbly you do hold your humble views,'' interjects my master.

``As you will know, the act of blowing the nose involves the coarctation
of that organ. Such stenosis of the nose, such astrictive and, one might
even venture to say, pleonastic stimulation of so localized an area
results, by response to that stimulus and in accordance with the
well-established principles of Lamarckian evolutionary theory, in the
development of that specific area to a degree disproportionate to the
development of other areas. The epidermis of the affected area
inevitably indurates and the subcutaneous material so coagulates as
eventually to ossify.''

``That's a bit extreme. Surely you can't turn flesh to bone just by
blowing your nose.'' Coldmoon, as behoves a bachelor of science, lodges
a protest. Waverhouse continues to deliver his speech with the utmost
nonchalance.

``I can well appreciate your natural dubieties, but the proof of the
pudding is the eating. For, behold, there is bone there, and that bone
has demonstrably been molded. Nevertheless, and despite that bone, one
snivels. If one snivels one has to blow the nose, and in the course of
that action both sides of the bone get worn away until the nose itself
acquires the shape of a high and narrow bulge. It is indeed a terrifying
process. But just as little taps of dropping water will eventually bore
through granite, so has the high, straight ridge of the nasal organ been
smithied by incessant nose-blows. Thus painfully was fangled the hard
straight line on one's face.''

``But yours is flabby.''

``I deliberately refrain from any discussion of this particular feature
as it may be observed in the physiognomy of the lecturer himself; for
such a purely personal approach involves the dangers of
self-exculpation, the temptation to gloss over, even to defend, one's
individual defects or deficiencies. But the nose of the honorable
Mrs.~Goldfield is such that I would wish to bring it to your attention
as the most highly developed of its kind, the most egregiously rare
object, in the world.''

Coldmoon cries out in spontaneous admiration. ``Hear, hear.''

``But anything whatever that develops to an extreme degree becomes
thereby intimidating. Even terrifying. Spectacular it may be, but
simultaneously awesome, unapproachable. Thus the bridge of that lady's
nose, though certainly magnificent, appears to me unduly rigid,
unacceptably steep. If one pauses to consider the nature of the noses of
the ancients, it seems probable that those of Socrates, Oliver
Goldsmith, and William Thackeray were strikingly imperfect from the
structural point of view, but those very imperfections had their own
peculiar charms. This is, no doubt, the intellection behind the saying
that a nose, like a mountain, is not significant because it is high but
because it is odd. Similarly, the popular catch-phrase that `dumplings
are better than nosegays' is no doubt a corruption of some yet more
ancient adage to the effect that dumplings are better than noses. From
which it follows that, viewed aesthetically, the nose of Citizen
Waverhouse is just about right.''

Coldmoon and my master greet this fantastication with peals of
appreciative laughter, and even Waverhouse joins in.

``Now, the piece I have just been reciting\ldots{}''

``Distinguished speaker, I must object to your use of the phrase
`reciting a piece': a somewhat vulgar word one would only expect from a
storyteller.'' Coldmoon, catching Waverhouse in the use of language
which only recently Waverhouse had criticized, feels himself revenged.

``In which case, sir, and having with your gracious permission purged
myself of error, I would now like to touch upon the matter of the proper
proportion between the nose and its associated face. If I were simply to
discuss noses in disregard of their relation to other entities, then I
would declare without fear of contradiction that the nose of Mrs.

Goldfield is superb, superlative, and, though possibly supervacaneous,
one well-placed to win first prize at any exhibition of nasal
development which might be organized by the long-nosed goblins on Mount
Kurama.

But alas! And even alack! That nose appears to have been formed,
fashioned, dare I say fabricated, without any regard for the
configuration of such other major items as the eyes and mouth. Julius
Caesar was undoubtedly dowered with a very fine nose. But what do you
think would be the result if one scissored off that Julian beak and
fixed it on the face of this cat here? Cats' foreheads are proverbially
diminutive. To raise the tower of Caesar's boned proboscis on such a
tiny site would be like plonking down on a chessboard the giant image of
Buddha now to be seen at Nara. The juxtaposing of disproportionate
elements destroys aesthetic value. Mrs.~Goldfield's nose, like that of
Caesar's, is, as a thing in itself, a most dignified and majestic
protuberance. But how does it appear in relation to its surroundings? Of
course those circumjacent areas are not quite so barren of aesthetic
merit as the face of this cat.

Nevertheless, it is a bloated face, the face of an epileptic skivvy
whose eyebrows meet in a sharp-pitched gable above thin tilted eyes.

Gentlemen, I ask you, what sort of nose could ever survive so lamentable
a face?''

As Waverhouse paused, a voice could be heard from the back of the house.
``He's still going on about noses. What a spiteful bore he is.''

``That's the wife of the rickshaw-owner,'' my master explains to
Waverhouse.

Waverhouse resumes. ``It is a great, if unexpected, honor for this
present lecturer to discover at, as it were, the back of the hall an
interested listener of the gentle sex. I am especially gratified that a
gleam of charm should be added to my arid lecture by the bell-sweet
voice of this new participant. It is, indeed, a happiness unlooked for,
a serendipity. To be worthy of our beautiful lady's patronage I would
gladly alter the academic style of this discourse into a more popular
mode, but, as I am just about to discuss a problem in mechanics, the
unavoidably technical terminology may prove a trifle difficult for the
ladies to comprehend. I must therefore beg them to be patient.''

Coldmoon responds to the mention of mechanics with his usual grin.

``The point I wish to establish is that such a nose and such a face will
never harmonize. In brief, they cannot conform to Zeising's rule of the
Golden Section, a fact which I propose to prove by use of a mechanical
formula. We should first designate H as the height of the nose, and α as
the angle between the nose and the level surface of the face. Please
note that W is, of course, the weight of the nose. Are you with me thus
far?''

``Hardly,'' breathes my master.

``Coldmoon, what about you?''

``I, too, am slightly at a loss.''

``You distress me, Coldmoon. Sneaze doesn't matter, but I'm shocked that
you, a bachelor of science, should fail to understand. This formula is a
key part of my lecture. To abandon this portion of my argument must
render the whole endeavor pointless. However, such things can't be
helped. I'll omit the formula and merely deliver the peroration.''

``Is there a peroration?'' asks my master in genuine curiosity.

``Why, naturally! A lecture without a peroration is like a Western
dinner shorn of the dessert. Now, listen, both of you, carefully. I am
launching on my peroration. Gentlemen, if one reflects upon the theory
which I have advanced on this occasion and gives due weight to the
related theories of Virchow and of Wisemen, one is bound also to take
appropriate account of the problem of the heredity of congenital form.
Furthermore, though there is a substantial body of evidence to support
the contention that acquired characteristics are not hereditarily
transmissible, one cannot lightly dismiss the view that the mental
conditions associated with hereditarily transmissible forms are
themselves also transmissible. It is consequently reasonable to assume
that a child born to the possessor of a nose of such enormity will have
an abnormal nose. Because Coldmoon is still young, he has not noticed
any particular abnormality in the structure of Miss Goldfield's nasal
organ. But the genes lurk. The products of heredity take long to
incubate. One never knows. Perhaps it would need no more than a sharp
change of climate for the daughter's snout suddenly to germinate and, in
a mere instant, to tumesce into a replica of that of her most honorable
mother. In sum, I believe that in the light of my theoretical
demonstration, it would seem prudent to forswear any idea of this
marriage. Now, while it is still possible to do so. I would go so far as
to claim that, quite apart from the master of this house, even his
monstrous cat asleep among us, would not dissent my conclusions.''

My master sits up at last. ``Of course,'' he says ``no one in his senses
would ever marry a daughter of that creature! Really, my dear
Coldmoon,'' he insists in real earnest, ``you simply must not marry
her.''

I seek in my own humble way to second all these sentiments by mewing
twice. Coldmoon, however, does not seem to be particularly alarmed. ``If
you two sages share that opinion, I would be prepared to give her up,
but it would be cruel if the consequent distress brought the person in
question into poor health.''

``That,'' burbled Waverhouse happily, ``might even be regarded as a sort
of sex crime.''

Only my master continues to take the matter seriously. ``Don't joke
about such things. That girl wouldn't wither away, not if she's the
daughter of that forward and presumptuous creature who strove to
humiliate me from the moment she set an uninvited foot in my house.'' My
master again works himself up into a great huff.

At which point there is a further outbreak of laughter from, by the
sound of it, three or four people on the far side of the hedge. A voice
says, ``You're a stuck-up blockhead.'' Another jeers, ``I bet you'd like
to live in a bigger house.'' A third loud voice announces, ``Ain't it a
pity! You swagger around but you're only a silly old windbag.''

My master goes out on to the veranda and shouts with matching violence,
``Hold your tongues. What do you think you're doing making this sort of
disturbance so close to my property?''

The laughter gets even louder. ``Hark at him. It's silly old Savage Tea.
Savage Tea. Savage Tea.'' They set up an abusive chant.

My master, looking furious, turns abruptly, snatches up his stick and
rushes out into the street.

Waverhouse claps his hands in pure delight. ``Up guards and at 'em'' he
shouts, urging my master on.

Coldmoon sits and grins, twisting his purple fastening-strings.

I follow my master and, as I crawl out through a gap in the hedge, find
him standing in the middle of the street with his stick held awkwardly
in his hand. Apart from him, the street is empty. I cannot help but feel
that he's been made to make a ninny of himself.
%\hypertarget{ux4e00}{%
%\paragraph{}%\label{ux4e00}}
\chapter*{一}
吾輩(わがはい)は猫である。名前はまだ無い。
どこで生れたかとんと見当(けんとう)がつかぬ。何でも薄暗いじめじめした所でニャーニャー泣いていた事だけは記憶している。吾輩はここで始めて人間というものを見た。しかもあとで聞くとそれは書生という人間中で一番獰悪(どうあく)な種族であったそうだ。この書生というのは時々我々を捕(つかま)えて煮(に)て食うという話である。しかしその当時は何という考もなかったから別段恐しいとも思わなかった。ただ彼の掌(てのひら)に載せられてスーと持ち上げられた時何だかフワフワした感じがあったばかりである。掌の上で少し落ちついて書生の顔を見たのがいわゆる人間というものの見始(みはじめ)であろう。この時妙なものだと思った感じが今でも残っている。第一毛をもって装飾されべきはずの顔がつるつるしてまるで薬缶(やかん)だ。その後(ご)猫にもだいぶ逢(あ)ったがこんな片輪(かたわ)には一度も出会(でく)わした事がない。のみならず顔の真中があまりに突起している。そうしてその穴の中から時々ぷうぷうと煙(けむり)を吹く。どうも咽(む)せぽくて実に弱った。これが人間の飲む煙草(たばこ)というものである事はようやくこの頃知った。
この書生の掌の裏(うち)でしばらくはよい心持に坐っておったが、しばらくすると非常な速力で運転し始めた。書生が動くのか自分だけが動くのか分らないが無暗(むやみ)に眼が廻る。胸が悪くなる。到底(とうてい)助からないと思っていると、どさりと音がして眼から火が出た。それまでは記憶しているがあとは何の事やらいくら考え出そうとしても分らない。
ふと気が付いて見ると書生はいない。たくさんおった兄弟が一疋(ぴき)も見えぬ。肝心(かんじん)の母親さえ姿を隠してしまった。その上今(いま)までの所とは違って無暗(むやみ)に明るい。眼を明いていられぬくらいだ。はてな何でも容子(ようす)がおかしいと、のそのそ這(は)い出して見ると非常に痛い。吾輩は藁(わら)の上から急に笹原の中へ棄てられたのである。
ようやくの思いで笹原を這い出すと向うに大きな池がある。吾輩は池の前に坐ってどうしたらよかろうと考えて見た。別にこれという分別(ふんべつ)も出ない。しばらくして泣いたら書生がまた迎に来てくれるかと考え付いた。ニャー、ニャーと試みにやって見たが誰も来ない。そのうち池の上をさらさらと風が渡って日が暮れかかる。腹が非常に減って来た。泣きたくても声が出ない。仕方がない、何でもよいから食物(くいもの)のある所まであるこうと決心をしてそろりそろりと池を左(ひだ)りに廻り始めた。どうも非常に苦しい。そこを我慢して無理やりに這(は)って行くとようやくの事で何となく人間臭い所へ出た。ここへ這入(はい)ったら、どうにかなると思って竹垣の崩(くず)れた穴から、とある邸内にもぐり込んだ。縁は不思議なもので、もしこの竹垣が破れていなかったなら、吾輩はついに路傍(ろぼう)に餓死(がし)したかも知れんのである。一樹の蔭とはよく云(い)ったものだ。この垣根の穴は今日(こんにち)に至るまで吾輩が隣家(となり)の三毛を訪問する時の通路になっている。さて邸(やしき)へは忍び込んだもののこれから先どうして善(い)いか分らない。そのうちに暗くなる、腹は減る、寒さは寒し、雨が降って来るという始末でもう一刻の猶予(ゆうよ)が出来なくなった。仕方がないからとにかく明るくて暖かそうな方へ方へとあるいて行く。今から考えるとその時はすでに家の内に這入っておったのだ。ここで吾輩は彼(か)の書生以外の人間を再び見るべき機会に遭遇(そうぐう)したのである。第一に逢ったのがおさんである。これは前の書生より一層乱暴な方で吾輩を見るや否やいきなり頸筋(くびすじ)をつかんで表へ抛(ほう)り出した。いやこれは駄目だと思ったから眼をねぶって運を天に任せていた。しかしひもじいのと寒いのにはどうしても我慢が出来ん。吾輩は再びおさんの隙(すき)を見て台所へ這(は)い上(あが)った。すると間もなくまた投げ出された。吾輩は投げ出されては這い上り、這い上っては投げ出され、何でも同じ事を四五遍繰り返したのを記憶している。その時におさんと云う者はつくづくいやになった。この間おさんの三馬(さんま)を偸(ぬす)んでこの返報をしてやってから、やっと胸の痞(つかえ)が下りた。吾輩が最後につまみ出されようとしたときに、この家(うち)の主人が騒々しい何だといいながら出て来た。下女は吾輩をぶら下げて主人の方へ向けてこの宿(やど)なしの小猫がいくら出しても出しても御台所(おだいどころ)へ上(あが)って来て困りますという。主人は鼻の下の黒い毛を撚(ひね)りながら吾輩の顔をしばらく眺(なが)めておったが、やがてそんなら内へ置いてやれといったまま奥へ這入(はい)ってしまった。主人はあまり口を聞かぬ人と見えた。下女は口惜(くや)しそうに吾輩を台所へ抛(ほう)り出した。かくして吾輩はついにこの家(うち)を自分の住家(すみか)と極(き)める事にしたのである。
吾輩の主人は滅多(めった)に吾輩と顔を合せる事がない。職業は教師だそうだ。学校から帰ると終日書斎に這入ったぎりほとんど出て来る事がない。家のものは大変な勉強家だと思っている。当人も勉強家であるかのごとく見せている。しかし実際はうちのものがいうような勤勉家ではない。吾輩は時々忍び足に彼の書斎を覗(のぞ)いて見るが、彼はよく昼寝(ひるね)をしている事がある。時々読みかけてある本の上に涎(よだれ)をたらしている。彼は胃弱で皮膚の色が淡黄色(たんこうしょく)を帯びて弾力のない不活溌(ふかっぱつ)な徴候をあらわしている。その癖に大飯を食う。大飯を食った後(あと)でタカジヤスターゼを飲む。飲んだ後で書物をひろげる。二三ページ読むと眠くなる。涎を本の上へ垂らす。これが彼の毎夜繰り返す日課である。吾輩は猫ながら時々考える事がある。教師というものは実に楽(らく)なものだ。人間と生れたら教師となるに限る。こんなに寝ていて勤まるものなら猫にでも出来ぬ事はないと。それでも主人に云わせると教師ほどつらいものはないそうで彼は友達が来る度(たび)に何とかかんとか不平を鳴らしている。
吾輩がこの家へ住み込んだ当時は、主人以外のものにははなはだ不人望であった。どこへ行っても跳(は)ね付けられて相手にしてくれ手がなかった。いかに珍重されなかったかは、今日(こんにち)に至るまで名前さえつけてくれないのでも分る。吾輩は仕方がないから、出来得る限り吾輩を入れてくれた主人の傍(そば)にいる事をつとめた。朝主人が新聞を読むときは必ず彼の膝(ひざ)の上に乗る。彼が昼寝をするときは必ずその背中(せなか)に乗る。これはあながち主人が好きという訳ではないが別に構い手がなかったからやむを得んのである。その後いろいろ経験の上、朝は飯櫃(めしびつ)の上、夜は炬燵(こたつ)の上、天気のよい昼は椽側(えんがわ)へ寝る事とした。しかし一番心持の好いのは夜(よ)に入(い)ってここのうちの小供の寝床へもぐり込んでいっしょにねる事である。この小供というのは五つと三つで夜になると二人が一つ床へ入(はい)って一間(ひとま)へ寝る。吾輩はいつでも彼等の中間に己(おの)れを容(い)るべき余地を見出(みいだ)してどうにか、こうにか割り込むのであるが、運悪く小供の一人が眼を醒(さ)ますが最後大変な事になる。小供は――ことに小さい方が質(たち)がわるい――猫が来た猫が来たといって夜中でも何でも大きな声で泣き出すのである。すると例の神経胃弱性の主人は必(かなら)ず眼をさまして次の部屋から飛び出してくる。現にせんだってなどは物指(ものさし)で尻ぺたをひどく叩(たた)かれた。
吾輩は人間と同居して彼等を観察すればするほど、彼等は我儘(わがまま)なものだと断言せざるを得ないようになった。ことに吾輩が時々同衾(どうきん)する小供のごときに至っては言語同断(ごんごどうだん)である。自分の勝手な時は人を逆さにしたり、頭へ袋をかぶせたり、抛(ほう)り出したり、\emph{へっつい}の中へ押し込んだりする。しかも吾輩の方で少しでも手出しをしようものなら家内(かない)総がかりで追い廻して迫害を加える。この間もちょっと畳で爪を磨(と)いだら細君が非常に怒(おこ)ってそれから容易に座敷へ入(い)れない。台所の板の間で他(ひと)が顫(ふる)えていても一向(いっこう)平気なものである。吾輩の尊敬する筋向(すじむこう)の白君などは逢(あ)う度毎(たびごと)に人間ほど不人情なものはないと言っておらるる。白君は先日玉のような子猫を四疋産(う)まれたのである。ところがそこの家(うち)の書生が三日目にそいつを裏の池へ持って行って四疋ながら棄てて来たそうだ。白君は涙を流してその一部始終を話した上、どうしても我等猫族(ねこぞく)が親子の愛を完(まった)くして美しい家族的生活をするには人間と戦ってこれを剿滅(そうめつ)せねばならぬといわれた。一々もっともの議論と思う。また隣りの三毛(みけ)君などは人間が所有権という事を解していないといって大(おおい)に憤慨している。元来我々同族間では目刺(めざし)の頭でも鰡(ぼら)の臍(へそ)でも一番先に見付けたものがこれを食う権利があるものとなっている。もし相手がこの規約を守らなければ腕力に訴えて善(よ)いくらいのものだ。しかるに彼等人間は毫(ごう)もこの観念がないと見えて我等が見付けた御馳走は必ず彼等のために掠奪(りゃくだつ)せらるるのである。彼等はその強力を頼んで正当に吾人が食い得べきものを奪(うば)ってすましている。白君は軍人の家におり三毛君は代言の主人を持っている。吾輩は教師の家に住んでいるだけ、こんな事に関すると両君よりもむしろ楽天である。ただその日その日がどうにかこうにか送られればよい。いくら人間だって、そういつまでも栄える事もあるまい。まあ気を永く猫の時節を待つがよかろう。
我儘(わがまま)で思い出したからちょっと吾輩の家の主人がこの我儘で失敗した話をしよう。元来この主人は何といって人に勝(すぐ)れて出来る事もないが、何にでもよく手を出したがる。俳句をやって\emph{ほととぎす}へ投書をしたり、新体詩を\emph{明星}へ出したり、間違いだらけの英文をかいたり、時によると弓に凝(こ)ったり、謡(うたい)を習ったり、またあるときはヴァイオリンなどをブーブー鳴らしたりするが、気の毒な事には、どれもこれも物になっておらん。その癖やり出すと胃弱の癖にいやに熱心だ。後架(こうか)の中で謡をうたって、近所で後架先生(こうかせんせい)と渾名(あだな)をつけられているにも関せず一向(いっこう)平気なもので、やはりこれは平(たいら)の宗盛(むねもり)にて候(そうろう)を繰返している。みんながそら宗盛だと吹き出すくらいである。この主人がどういう考になったものか吾輩の住み込んでから一月ばかり後(のち)のある月の月給日に、大きな包みを提(さ)げてあわただしく帰って来た。何を買って来たのかと思うと水彩絵具と毛筆とワットマンという紙で今日から謡や俳句をやめて絵をかく決心と見えた。果して翌日から当分の間というものは毎日毎日書斎で昼寝もしないで絵ばかりかいている。しかしそのかき上げたものを見ると何をかいたものやら誰にも鑑定がつかない。当人もあまり甘(うま)くないと思ったものか、ある日その友人で美学とかをやっている人が来た時に下(しも)のような話をしているのを聞いた。
「どうも甘(うま)くかけないものだね。人のを見ると何でもないようだが自(みずか)ら筆をとって見ると今更(いまさら)のようにむずかしく感ずる」これは主人の述懐(じゅっかい)である。なるほど詐(いつわ)りのない処だ。彼の友は金縁の眼鏡越(めがねごし)に主人の顔を見ながら、「そう初めから上手にはかけないさ、第一室内の想像ばかりで画(え)がかける訳のものではない。昔(むか)し以太利(イタリー)の大家アンドレア・デル・サルトが言った事がある。画をかくなら何でも自然その物を写せ。天に星辰(せいしん)あり。地に露華(ろか)あり。飛ぶに禽(とり)あり。走るに獣(けもの)あり。池に金魚あり。枯木(こぼく)に寒鴉(かんあ)あり。自然はこれ一幅の大活画(だいかつが)なりと。どうだ君も画らしい画をかこうと思うならちと写生をしたら」
「へえアンドレア・デル・サルトがそんな事をいった事があるかい。ちっとも知らなかった。なるほどこりゃもっともだ。実にその通りだ」と主人は無暗(むやみ)に感心している。金縁の裏には嘲(あざ)けるような笑(わらい)が見えた。
その翌日吾輩は例のごとく椽側(えんがわ)に出て心持善く昼寝(ひるね)をしていたら、主人が例になく書斎から出て来て吾輩の後(うし)ろで何かしきりにやっている。ふと眼が覚(さ)めて何をしているかと一分(いちぶ)ばかり細目に眼をあけて見ると、彼は余念もなくアンドレア・デル・サルトを極(き)め込んでいる。吾輩はこの有様を見て覚えず失笑するのを禁じ得なかった。彼は彼の友に揶揄(やゆ)せられたる結果としてまず手初めに吾輩を写生しつつあるのである。吾輩はすでに十分(じゅうぶん)寝た。欠伸(あくび)がしたくてたまらない。しかしせっかく主人が熱心に筆を執(と)っているのを動いては気の毒だと思って、じっと辛棒(しんぼう)しておった。彼は今吾輩の輪廓をかき上げて顔のあたりを色彩(いろど)っている。吾輩は自白する。吾輩は猫として決して上乗の出来ではない。背といい毛並といい顔の造作といいあえて他の猫に勝(まさ)るとは決して思っておらん。しかしいくら不器量の吾輩でも、今吾輩の主人に描(えが)き出されつつあるような妙な姿とは、どうしても思われない。第一色が違う。吾輩は波斯産(ペルシャさん)の猫のごとく黄を含める淡灰色に漆(うるし)のごとき斑入(ふい)りの皮膚を有している。これだけは誰が見ても疑うべからざる事実と思う。しかるに今主人の彩色を見ると、黄でもなければ黒でもない、灰色でもなければ褐色(とびいろ)でもない、さればとてこれらを交ぜた色でもない。ただ一種の色であるというよりほかに評し方のない色である。その上不思議な事は眼がない。もっともこれは寝ているところを写生したのだから無理もないが眼らしい所さえ見えないから盲猫(めくら)だか寝ている猫だか判然しないのである。吾輩は心中ひそかにいくらアンドレア・デル・サルトでもこれではしようがないと思った。しかしその熱心には感服せざるを得ない。なるべくなら動かずにおってやりたいと思ったが、さっきから小便が催うしている。身内(みうち)の筋肉はむずむずする。最早(もはや)一分も猶予(ゆうよ)が出来ぬ仕儀(しぎ)となったから、やむをえず失敬して両足を前へ存分のして、首を低く押し出してあーあと大(だい)なる欠伸をした。さてこうなって見ると、もうおとなしくしていても仕方がない。どうせ主人の予定は打(ぶ)ち壊(こ)わしたのだから、ついでに裏へ行って用を足(た)そうと思ってのそのそ這い出した。すると主人は失望と怒りを掻(か)き交ぜたような声をして、座敷の中から「この馬鹿野郎」と怒鳴(どな)った。この主人は人を罵(ののし)るときは必ず馬鹿野郎というのが癖である。ほかに悪口の言いようを知らないのだから仕方がないが、今まで辛棒した人の気も知らないで、無暗(むやみ)に馬鹿野郎呼(よば)わりは失敬だと思う。それも平生吾輩が彼の背中(せなか)へ乗る時に少しは好い顔でもするならこの漫罵(まんば)も甘んじて受けるが、こっちの便利になる事は何一つ快くしてくれた事もないのに、小便に立ったのを馬鹿野郎とは酷(ひど)い。元来人間というものは自己の力量に慢じてみんな増長している。少し人間より強いものが出て来て窘(いじ)めてやらなくてはこの先どこまで増長するか分らない。
我儘(わがまま)もこのくらいなら我慢するが吾輩は人間の不徳についてこれよりも数倍悲しむべき報道を耳にした事がある。
吾輩の家の裏に十坪ばかりの茶園(ちゃえん)がある。広くはないが瀟洒(さっぱり)とした心持ち好く日の当(あた)る所だ。うちの小供があまり騒いで楽々昼寝の出来ない時や、あまり退屈で腹加減のよくない折などは、吾輩はいつでもここへ出て浩然(こうぜん)の気を養うのが例である。ある小春の穏かな日の二時頃であったが、吾輩は昼飯後(ちゅうはんご)快よく一睡した後(のち)、運動かたがたこの茶園へと歩(ほ)を運ばした。茶の木の根を一本一本嗅ぎながら、西側の杉垣のそばまでくると、枯菊を押し倒してその上に大きな猫が前後不覚に寝ている。彼は吾輩の近づくのも一向(いっこう)心付かざるごとく、また心付くも無頓着なるごとく、大きな鼾(いびき)をして長々と体を横(よこた)えて眠っている。他(ひと)の庭内に忍び入りたるものがかくまで平気に睡(ねむ)られるものかと、吾輩は窃(ひそ)かにその大胆なる度胸に驚かざるを得なかった。彼は純粋の黒猫である。わずかに午(ご)を過ぎたる太陽は、透明なる光線を彼の皮膚の上に抛(な)げかけて、きらきらする柔毛(にこげ)の間より眼に見えぬ炎でも燃(も)え出(い)ずるように思われた。彼は猫中の大王とも云うべきほどの偉大なる体格を有している。吾輩の倍はたしかにある。吾輩は嘆賞の念と、好奇の心に前後を忘れて彼の前に佇立(ちょりつ)して余念もなく眺(なが)めていると、静かなる小春の風が、杉垣の上から出たる梧桐(ごとう)の枝を軽(かろ)く誘ってばらばらと二三枚の葉が枯菊の茂みに落ちた。大王はかっとその真丸(まんまる)の眼を開いた。今でも記憶している。その眼は人間の珍重する琥珀(こはく)というものよりも遥(はる)かに美しく輝いていた。彼は身動きもしない。双眸(そうぼう)の奥から射るごとき光を吾輩の矮小(わいしょう)なる額(ひたい)の上にあつめて、\emph{御めえ}は一体何だと云った。大王にしては少々言葉が卑(いや)しいと思ったが何しろその声の底に犬をも挫(ひ)しぐべき力が籠(こも)っているので吾輩は少なからず恐れを抱(いだ)いた。しかし挨拶(あいさつ)をしないと険呑(けんのん)だと思ったから「吾輩は猫である。名前はまだない」となるべく平気を装(よそお)って冷然と答えた。しかしこの時吾輩の心臓はたしかに平時よりも烈しく鼓動しておった。彼は大(おおい)に軽蔑(けいべつ)せる調子で「何、猫だ? 猫が聞いてあきれらあ。全(ぜん)てえどこに住んでるんだ」随分傍若無人(ぼうじゃくぶじん)である。「吾輩はここの教師の家(うち)にいるのだ」「どうせそんな事だろうと思った。いやに瘠(や)せてるじゃねえか」と大王だけに気焔(きえん)を吹きかける。言葉付から察するとどうも良家の猫とも思われない。しかしその膏切(あぶらぎ)って肥満しているところを見ると御馳走を食ってるらしい、豊かに暮しているらしい。吾輩は「そう云う君は一体誰だい」と聞かざるを得なかった。「己(お)れあ車屋の黒(くろ)よ」昂然(こうぜん)たるものだ。車屋の黒はこの近辺で知らぬ者なき乱暴猫である。しかし車屋だけに強いばかりでちっとも教育がないからあまり誰も交際しない。同盟敬遠主義の的(まと)になっている奴だ。吾輩は彼の名を聞いて少々尻こそばゆき感じを起すと同時に、一方では少々軽侮(けいぶ)の念も生じたのである。吾輩はまず彼がどのくらい無学であるかを試(ため)してみようと思って左(さ)の問答をして見た。
「一体車屋と教師とはどっちがえらいだろう」
「車屋の方が強いに極(きま)っていらあな。\emph{御めえ}の\emph{うち}の主人を見ねえ、まるで骨と皮ばかりだぜ」
「君も車屋の猫だけに大分(だいぶ)強そうだ。車屋にいると御馳走(ごちそう)が食えると見えるね」
「何(なあ)に\emph{おれ}なんざ、どこの国へ行ったって食い物に不自由はしねえつもりだ。\emph{御めえ}なんかも茶畠(ちゃばたけ)ばかりぐるぐる廻っていねえで、ちっと己(おれ)の後(あと)へくっ付いて来て見ねえ。一と月とたたねえうちに見違えるように太れるぜ」
「追ってそう願う事にしよう。しかし家(うち)は教師の方が車屋より大きいのに住んでいるように思われる」
「箆棒(べらぼう)め、うちなんかいくら大きくたって腹の足(た)しになるもんか」
彼は大(おおい)に肝癪(かんしゃく)に障(さわ)った様子で、寒竹(かんちく)をそいだような耳をしきりとぴく付かせてあららかに立ち去った。吾輩が車屋の黒と知己(ちき)になったのはこれからである。
その後(ご)吾輩は度々(たびたび)黒と邂逅(かいこう)する。邂逅する毎(ごと)に彼は車屋相当の気焔(きえん)を吐く。先に吾輩が耳にしたという不徳事件も実は黒から聞いたのである。
或る日例のごとく吾輩と黒は暖かい茶畠(ちゃばたけ)の中で寝転(ねころ)びながらいろいろ雑談をしていると、彼はいつもの自慢話(じまんばな)しをさも新しそうに繰り返したあとで、吾輩に向って下(しも)のごとく質問した。「\emph{御めえ}は今までに鼠を何匹とった事がある」智識は黒よりも余程発達しているつもりだが腕力と勇気とに至っては到底(とうてい)黒の比較にはならないと覚悟はしていたものの、この問に接したる時は、さすがに極(きま)りが善(よ)くはなかった。けれども事実は事実で詐(いつわ)る訳には行かないから、吾輩は「実はとろうとろうと思ってまだ捕(と)らない」と答えた。黒は彼の鼻の先からぴんと突張(つっぱ)っている長い髭(ひげ)をびりびりと震(ふる)わせて非常に笑った。元来黒は自慢をする丈(だけ)にどこか足りないところがあって、彼の気焔(きえん)を感心したように咽喉(のど)をころころ鳴らして謹聴していればはなはだ御(ぎょ)しやすい猫である。吾輩は彼と近付になってから直(すぐ)にこの呼吸を飲み込んだからこの場合にもなまじい己(おの)れを弁護してますます形勢をわるくするのも愚(ぐ)である、いっその事彼に自分の手柄話をしゃべらして御茶を濁すに若(し)くはないと思案を定(さだ)めた。そこでおとなしく「君などは年が年であるから大分(だいぶん)とったろう」とそそのかして見た。果然彼は墻壁(しょうへき)の欠所(けっしょ)に吶喊(とっかん)して来た。「たんとでもねえが三四十はとったろう」とは得意気なる彼の答であった。彼はなお語をつづけて「鼠の百や二百は一人でいつでも引き受けるが\emph{いたち}ってえ奴は手に合わねえ。一度\emph{いたち}に向って酷(ひど)い目に逢(あ)った」「へえなるほど」と相槌(あいづち)を打つ。黒は大きな眼をぱちつかせて云う。「去年の大掃除の時だ。うちの亭主が石灰(いしばい)の袋を持って椽(えん)の下へ這(は)い込んだら\emph{御めえ}大きな\emph{いたち}の野郎が面喰(めんくら)って飛び出したと思いねえ」「ふん」と感心して見せる。「\emph{いたち}ってけども何鼠の少し大きいぐれえのものだ。こん畜生(ちきしょう)って気で追っかけてとうとう泥溝(どぶ)の中へ追い込んだと思いねえ」「うまくやったね」と喝采(かっさい)してやる。「ところが\emph{御めえ}いざってえ段になると奴め最後(さいご)っ屁(ぺ)をこきゃがった。臭(くせ)えの臭くねえのってそれからってえものは\emph{いたち}を見ると胸が悪くならあ」彼はここに至ってあたかも去年の臭気を今(いま)なお感ずるごとく前足を揚げて鼻の頭を二三遍なで廻わした。吾輩も少々気の毒な感じがする。ちっと景気を付けてやろうと思って「しかし鼠なら君に睨(にら)まれては百年目だろう。君はあまり鼠を捕(と)るのが名人で鼠ばかり食うものだからそんなに肥って色つやが善いのだろう」黒の御機嫌をとるためのこの質問は不思議にも反対の結果を呈出(ていしゅつ)した。彼は喟然(きぜん)として大息(たいそく)していう。「考(かん)げえるとつまらねえ。いくら稼いで鼠をとったって――一てえ人間ほどふてえ奴は世の中にいねえぜ。人のとった鼠をみんな取り上げやがって交番へ持って行きゃあがる。交番じゃ誰が捕(と)ったか分らねえからその\emph{たんび}に五銭ずつくれるじゃねえか。うちの亭主なんか己(おれ)の御蔭でもう壱円五十銭くらい儲(もう)けていやがる癖に、碌(ろく)なものを食わせた事もありゃしねえ。おい人間てものあ体(てい)の善(い)い泥棒だぜ」さすが無学の黒もこのくらいの理窟(りくつ)はわかると見えてすこぶる怒(おこ)った容子(ようす)で背中の毛を逆立(さかだ)てている。吾輩は少々気味が悪くなったから善い加減にその場を胡魔化(ごまか)して家(うち)へ帰った。この時から吾輩は決して鼠をとるまいと決心した。しかし黒の子分になって鼠以外の御馳走を猟(あさ)ってあるく事もしなかった。御馳走を食うよりも寝ていた方が気楽でいい。教師の家(うち)にいると猫も教師のような性質になると見える。要心しないと今に胃弱になるかも知れない。
教師といえば吾輩の主人も近頃に至っては到底(とうてい)水彩画において望(のぞみ)のない事を悟ったものと見えて十二月一日の日記にこんな事をかきつけた。
\blockquote{○○と云う人に今日の会で始めて出逢(であ)った。あの人は大分(だいぶ)放蕩(ほうとう)をした人だと云うがなるほど通人(つうじん)らしい風采(ふうさい)をしている。こう云う質(たち)の人は女に好かれるものだから○○が放蕩をしたと云うよりも放蕩をするべく余儀なくせられたと云うのが適当であろう。あの人の妻君は芸者だそうだ、羨(うらや)ましい事である。元来放蕩家を悪くいう人の大部分は放蕩をする資格のないものが多い。また放蕩家をもって自任する連中のうちにも、放蕩する資格のないものが多い。これらは余儀なくされないのに無理に進んでやるのである。あたかも吾輩の水彩画に於けるがごときもので到底卒業する気づかいはない。しかるにも関せず、自分だけは通人だと思って済(すま)している。料理屋の酒を飲んだり待合へ這入(はい)るから通人となり得るという論が立つなら、吾輩も一廉(ひとかど)の水彩画家になり得る理窟(りくつ)だ。吾輩の水彩画のごときはかかない方がましであると同じように、愚昧(ぐまい)なる通人よりも山出しの大野暮(おおやぼ)の方が遥(はる)かに上等だ。}
通人論(つうじんろん)はちょっと首肯(しゅこう)しかねる。また芸者の妻君を羨しいなどというところは教師としては口にすべからざる愚劣の考であるが、自己の水彩画における批評眼だけはたしかなものだ。主人はかくのごとく自知(じち)の明(めい)あるにも関せずその自惚心(うぬぼれしん)はなかなか抜けない。中二日(なかふつか)置いて十二月四日の日記にこんな事を書いている。
\blockquote{昨夜(ゆうべ)は僕が水彩画をかいて到底物にならんと思って、そこらに抛(ほう)って置いたのを誰かが立派な額にして欄間(らんま)に懸(か)けてくれた夢を見た。さて額になったところを見ると我ながら急に上手になった。非常に嬉しい。これなら立派なものだと独(ひと)りで眺め暮らしていると、夜が明けて眼が覚(さ)めてやはり元の通り下手である事が朝日と共に明瞭になってしまった。}
主人は夢の裡(うち)まで水彩画の未練を背負(しょ)ってあるいていると見える。これでは水彩画家は無論夫子(ふうし)の所謂(いわゆる)通人にもなれない質(たち)だ。
主人が水彩画を夢に見た翌日例の金縁眼鏡(めがね)の美学者が久し振りで主人を訪問した。彼は座につくと劈頭(へきとう)第一に「画(え)はどうかね」と口を切った。主人は平気な顔をして「君の忠告に従って写生を力(つと)めているが、なるほど写生をすると今まで気のつかなかった物の形や、色の精細な変化などがよく分るようだ。西洋では昔(むか)しから写生を主張した結果今日(こんにち)のように発達したものと思われる。さすがアンドレア・デル・サルトだ」と日記の事は\emph{おくび}にも出さないで、またアンドレア・デル・サルトに感心する。美学者は笑いながら「実は君、あれは出鱈目(でたらめ)だよ」と頭を掻(か)く。「何が」と主人はまだ譃(いつ)わられた事に気がつかない。「何がって君のしきりに感服しているアンドレア・デル・サルトさ。あれは僕のちょっと捏造(ねつぞう)した話だ。君がそんなに真面目(まじめ)に信じようとは思わなかったハハハハ」と大喜悦の体(てい)である。吾輩は椽側でこの対話を聞いて彼の今日の日記にはいかなる事が記(しる)さるるであろうかと予(あらかじ)め想像せざるを得なかった。この美学者はこんな好(いい)加減な事を吹き散らして人を担(かつ)ぐのを唯一の楽(たのしみ)にしている男である。彼はアンドレア・デル・サルト事件が主人の情線(じょうせん)にいかなる響を伝えたかを毫(ごう)も顧慮せざるもののごとく得意になって下(しも)のような事を饒舌(しゃべ)った。「いや時々冗談(じょうだん)を言うと人が真(ま)に受けるので大(おおい)に滑稽的(こっけいてき)美感を挑撥(ちょうはつ)するのは面白い。せんだってある学生にニコラス・ニックルベーがギボンに忠告して彼の一世の大著述なる仏国革命史を仏語で書くのをやめにして英文で出版させたと言ったら、その学生がまた馬鹿に記憶の善い男で、日本文学会の演説会で真面目に僕の話した通りを繰り返したのは滑稽であった。ところがその時の傍聴者は約百名ばかりであったが、皆熱心にそれを傾聴しておった。それからまだ面白い話がある。せんだって或る文学者のいる席でハリソンの歴史小説セオファーノの話(はな)しが出たから僕はあれは歴史小説の中(うち)で白眉(はくび)である。ことに女主人公が死ぬところは鬼気(きき)人を襲うようだと評したら、僕の向うに坐っている知らんと云った事のない先生が、そうそうあすこは実に名文だといった。それで僕はこの男もやはり僕同様この小説を読んでおらないという事を知った」神経胃弱性の主人は眼を丸くして問いかけた。「そんな出鱈目(でたらめ)をいってもし相手が読んでいたらどうするつもりだ」あたかも人を欺(あざむ)くのは差支(さしつかえ)ない、ただ化(ばけ)の皮(かわ)があらわれた時は困るじゃないかと感じたもののごとくである。美学者は少しも動じない。「なにその時(とき)ゃ別の本と間違えたとか何とか云うばかりさ」と云ってけらけら笑っている。この美学者は金縁の眼鏡は掛けているがその性質が車屋の黒に似たところがある。主人は黙って日の出を輪に吹いて吾輩にはそんな勇気はないと云わんばかりの顔をしている。美学者はそれだから画(え)をかいても駄目だという目付で「しかし冗談(じょうだん)は冗談だが画というものは実際むずかしいものだよ、レオナルド・ダ・ヴィンチは門下生に寺院の壁の\emph{しみ}を写せと教えた事があるそうだ。なるほど雪隠(せついん)などに這入(はい)って雨の漏る壁を余念なく眺めていると、なかなかうまい模様画が自然に出来ているぜ。君注意して写生して見給えきっと面白いものが出来るから」「また欺(だま)すのだろう」「いえこれだけはたしかだよ。実際奇警な語じゃないか、ダ・ヴィンチでもいいそうな事だあね」「なるほど奇警には相違ないな」と主人は半分降参をした。しかし彼はまだ雪隠で写生はせぬようだ。
車屋の黒はその後(ご)跛(びっこ)になった。彼の光沢ある毛は漸々(だんだん)色が褪(さ)めて抜けて来る。吾輩が琥珀(こはく)よりも美しいと評した彼の眼には眼脂(めやに)が一杯たまっている。ことに著るしく吾輩の注意を惹(ひ)いたのは彼の元気の消沈とその体格の悪くなった事である。吾輩が例の茶園(ちゃえん)で彼に逢った最後の日、どうだと云って尋ねたら「\emph{いたち}の最後屁(さいごっぺ)と肴屋(さかなや)の天秤棒(てんびんぼう)には懲々(こりごり)だ」といった。
赤松の間に二三段の紅(こう)を綴った紅葉(こうよう)は昔(むか)しの夢のごとく散って\emph{つくばい}に近く代る代る花弁(はなびら)をこぼした紅白(こうはく)の山茶花(さざんか)も残りなく落ち尽した。三間半の南向の椽側に冬の日脚が早く傾いて木枯(こがらし)の吹かない日はほとんど稀(まれ)になってから吾輩の昼寝の時間も狭(せば)められたような気がする。
主人は毎日学校へ行く。帰ると書斎へ立て籠(こも)る。人が来ると、教師が厭(いや)だ厭だという。水彩画も滅多にかかない。タカジヤスターゼも功能がないといってやめてしまった。小供は感心に休まないで幼稚園へかよう。帰ると唱歌を歌って、毬(まり)をついて、時々吾輩を尻尾(しっぽ)でぶら下げる。
吾輩は御馳走(ごちそう)も食わないから別段肥(ふと)りもしないが、まずまず健康で跛(びっこ)にもならずにその日その日を暮している。鼠は決して取らない。おさんは未(いま)だに嫌(きら)いである。名前はまだつけてくれないが、欲をいっても際限がないから生涯(しょうがい)この教師の家(うち)で無名の猫で終るつもりだ。
\chapter*{二}
吾輩は新年来多少有名になったので、猫ながらちょっと鼻が高く感ぜらるるのはありがたい。
元朝早々主人の許(もと)へ一枚の絵端書(えはがき)が来た。これは彼の交友某画家からの年始状であるが、上部を赤、下部を深緑(ふかみど)りで塗って、その真中に一の動物が蹲踞(うずくま)っているところをパステルで書いてある。主人は例の書斎でこの絵を、横から見たり、竪(たて)から眺めたりして、うまい色だなという。すでに一応感服したものだから、もうやめにするかと思うとやはり横から見たり、竪から見たりしている。からだを拗(ね)じ向けたり、手を延ばして年寄が三世相(さんぜそう)を見るようにしたり、または窓の方へむいて鼻の先まで持って来たりして見ている。早くやめてくれないと膝(ひざ)が揺れて険呑(けんのん)でたまらない。ようやくの事で動揺があまり劇(はげ)しくなくなったと思ったら、小さな声で一体何をかいたのだろうと云(い)う。主人は絵端書の色には感服したが、かいてある動物の正体が分らぬので、さっきから苦心をしたものと見える。そんな分らぬ絵端書かと思いながら、寝ていた眼を上品に半(なか)ば開いて、落ちつき払って見ると紛(まぎ)れもない、自分の肖像だ。主人のようにアンドレア・デル・サルトを極(き)め込んだものでもあるまいが、画家だけに形体も色彩もちゃんと整って出来ている。誰が見たって猫に相違ない。少し眼識のあるものなら、猫の中(うち)でも他(ほか)の猫じゃない吾輩である事が判然とわかるように立派に描(か)いてある。このくらい明瞭な事を分らずにかくまで苦心するかと思うと、少し人間が気の毒になる。出来る事ならその絵が吾輩であると云う事を知らしてやりたい。吾輩であると云う事はよし分らないにしても、せめて猫であるという事だけは分らしてやりたい。しかし人間というものは到底(とうてい)吾輩猫属(ねこぞく)の言語を解し得るくらいに天の恵(めぐみ)に浴しておらん動物であるから、残念ながらそのままにしておいた。
ちょっと読者に断っておきたいが、元来人間が何ぞというと猫々と、事もなげに軽侮の口調をもって吾輩を評価する癖があるははなはだよくない。人間の糟(かす)から牛と馬が出来て、牛と馬の糞から猫が製造されたごとく考えるのは、自分の無智に心付かんで高慢な顔をする教師などにはありがちの事でもあろうが、はたから見てあまり見っともいい者じゃない。いくら猫だって、そう粗末簡便には出来ぬ。よそ目には一列一体、平等無差別、どの猫も自家固有の特色などはないようであるが、猫の社会に這入(はい)って見るとなかなか複雑なもので十人十色(といろ)という人間界の語(ことば)はそのままここにも応用が出来るのである。目付でも、鼻付でも、毛並でも、足並でも、みんな違う。髯(ひげ)の張り具合から耳の立ち按排(あんばい)、尻尾(しっぽ)の垂れ加減に至るまで同じものは一つもない。器量、不器量、好き嫌い、粋無粋(すいぶすい)の数(かず)を悉(つ)くして千差万別と云っても差支えないくらいである。そのように判然たる区別が存しているにもかかわらず、人間の眼はただ向上とか何とかいって、空ばかり見ているものだから、吾輩の性質は無論相貌(そうぼう)の末を識別する事すら到底出来ぬのは気の毒だ。同類相求むとは昔(むか)しからある語(ことば)だそうだがその通り、餅屋(もちや)は餅屋、猫は猫で、猫の事ならやはり猫でなくては分らぬ。いくら人間が発達したってこればかりは駄目である。いわんや実際をいうと彼等が自(みずか)ら信じているごとくえらくも何ともないのだからなおさらむずかしい。またいわんや同情に乏しい吾輩の主人のごときは、相互を残りなく解するというが愛の第一義であるということすら分らない男なのだから仕方がない。彼は性の悪い牡蠣(かき)のごとく書斎に吸い付いて、かつて外界に向って口を開(ひら)いた事がない。それで自分だけはすこぶる達観したような面構(つらがまえ)をしているのはちょっとおかしい。達観しない証拠には現に吾輩の肖像が眼の前にあるのに少しも悟った様子もなく今年は征露の第二年目だから大方熊の画(え)だろうなどと気の知れぬことをいってすましているのでもわかる。
吾輩が主人の膝(ひざ)の上で眼をねむりながらかく考えていると、やがて下女が第二の絵端書(えはがき)を持って来た。見ると活版で舶来の猫が四五疋(ひき)ずらりと行列してペンを握ったり書物を開いたり勉強をしている。その内の一疋は席を離れて机の角で西洋の猫じゃ猫じゃを躍(おど)っている。その上に日本の墨で「吾輩は猫である」と黒々とかいて、右の側(わき)に書を読むや躍(おど)るや猫の春一日(はるひとひ)という俳句さえ認(したた)められてある。これは主人の旧門下生より来たので誰が見たって一見して意味がわかるはずであるのに、迂濶(うかつ)な主人はまだ悟らないと見えて不思議そうに首を捻(ひね)って、はてな今年は猫の年かなと独言(ひとりごと)を言った。吾輩がこれほど有名になったのを未(ま)だ気が着かずにいると見える。
ところへ下女がまた第三の端書を持ってくる。今度は絵端書ではない。恭賀新年とかいて、傍(かたわ)らに乍恐縮(きょうしゅくながら)かの猫へも宜(よろ)しく御伝声(ごでんせい)奉願上候(ねがいあげたてまつりそろ)とある。いかに迂遠(うえん)な主人でもこう明らさまに書いてあれば分るものと見えてようやく気が付いたようにフンと言いながら吾輩の顔を見た。その眼付が今までとは違って多少尊敬の意を含んでいるように思われた。今まで世間から存在を認められなかった主人が急に一個の新面目(しんめんぼく)を施こしたのも、全く吾輩の御蔭だと思えばこのくらいの眼付は至当だろうと考える。
おりから門の格子(こうし)がチリン、チリン、チリリリリンと鳴る。大方来客であろう、来客なら下女が取次に出る。吾輩は肴屋(さかなや)の梅公がくる時のほかは出ない事に極(き)めているのだから、平気で、もとのごとく主人の膝に坐っておった。すると主人は高利貸にでも飛び込まれたように不安な顔付をして玄関の方を見る。何でも年賀の客を受けて酒の相手をするのが厭らしい。人間もこのくらい偏屈(へんくつ)になれば申し分はない。そんなら早くから外出でもすればよいのにそれほどの勇気も無い。いよいよ牡蠣の根性(こんじょう)をあらわしている。しばらくすると下女が来て寒月(かんげつ)さんがおいでになりましたという。この寒月という男はやはり主人の旧門下生であったそうだが、今では学校を卒業して、何でも主人より立派になっているという話(はな)しである。この男がどういう訳か、よく主人の所へ遊びに来る。来ると自分を恋(おも)っている女が有りそうな、無さそうな、世の中が面白そうな、つまらなそうな、凄(すご)いような艶(つや)っぽいような文句ばかり並べては帰る。主人のようなしなびかけた人間を求めて、わざわざこんな話しをしに来るのからして合点(がてん)が行かぬが、あの牡蠣的(かきてき)主人がそんな談話を聞いて時々相槌(あいづち)を打つのはなお面白い。
「しばらく御無沙汰をしました。実は去年の暮から大(おおい)に活動しているものですから、出(で)よう出ようと思っても、ついこの方角へ足が向かないので」と羽織の紐(ひも)をひねくりながら謎(なぞ)見たような事をいう。「どっちの方角へ足が向くかね」と主人は真面目な顔をして、黒木綿(くろもめん)の紋付羽織の袖口(そでぐち)を引張る。この羽織は木綿で\emph{ゆき}が短かい、下からべんべら者が左右へ五分くらいずつはみ出している。「エヘヘヘ少し違った方角で」と寒月君が笑う。見ると今日は前歯が一枚欠けている。「君歯をどうかしたかね」と主人は問題を転じた。「ええ実はある所で椎茸(しいたけ)を食いましてね」「何を食ったって?」「その、少し椎茸を食ったんで。椎茸の傘(かさ)を前歯で噛み切ろうとしたらぼろりと歯が欠けましたよ」「椎茸で前歯がかけるなんざ、何だか爺々臭(じじいくさ)いね。俳句にはなるかも知れないが、恋にはならんようだな」と平手で吾輩の頭を軽(かろ)く叩く。「ああその猫が例のですか、なかなか肥ってるじゃありませんか、それなら車屋の黒にだって負けそうもありませんね、立派なものだ」と寒月君は大(おおい)に吾輩を賞(ほ)める。「近頃大分(だいぶ)大きくなったのさ」と自慢そうに頭をぽかぽかなぐる。賞められたのは得意であるが頭が少々痛い。「一昨夜もちょいと合奏会をやりましてね」と寒月君はまた話しをもとへ戻す。「どこで」「どこでもそりゃ御聞きにならんでもよいでしょう。ヴァイオリンが三挺(ちょう)とピヤノの伴奏でなかなか面白かったです。ヴァイオリンも三挺くらいになると下手でも聞かれるものですね。二人は女で私(わたし)がその中へまじりましたが、自分でも善く弾(ひ)けたと思いました」「ふん、そしてその女というのは何者かね」と主人は羨(うらや)ましそうに問いかける。元来主人は平常枯木寒巌(こぼくかんがん)のような顔付はしているものの実のところは決して婦人に冷淡な方ではない、かつて西洋の或る小説を読んだら、その中にある一人物が出て来て、それが大抵の婦人には必ずちょっと惚(ほ)れる。勘定をして見ると往来を通る婦人の\emph{七割弱}には恋着(れんちゃく)するという事が諷刺的(ふうしてき)に書いてあったのを見て、これは真理だと感心したくらいな男である。そんな浮気な男が何故(なぜ)牡蠣的生涯を送っているかと云うのは吾輩猫などには到底(とうてい)分らない。或人は失恋のためだとも云うし、或人は胃弱のせいだとも云うし、また或人は金がなくて臆病な性質(たち)だからだとも云う。どっちにしたって明治の歴史に関係するほどな人物でもないのだから構わない。しかし寒月君の女連(おんなづ)れを羨まし気(げ)に尋ねた事だけは事実である。寒月君は面白そうに口取(くちとり)の蒲鉾(かまぼこ)を箸で挟んで半分前歯で食い切った。吾輩はまた欠けはせぬかと心配したが今度は大丈夫であった。「なに二人とも去(さ)る所の令嬢ですよ、御存じの方(かた)じゃありません」と余所余所(よそよそ)しい返事をする。「ナール」と主人は引張ったが「ほど」を略して考えている。寒月君はもう善(い)い加減な時分だと思ったものか「どうも好い天気ですな、御閑(おひま)ならごいっしょに散歩でもしましょうか、旅順が落ちたので市中は大変な景気ですよ」と促(うな)がして見る。主人は旅順の陥落より女連(おんなづれ)の身元を聞きたいと云う顔で、しばらく考え込んでいたがようやく決心をしたものと見えて「それじゃ出るとしよう」と思い切って立つ。やはり黒木綿の紋付羽織に、兄の紀念(かたみ)とかいう二十年来着古(きふ)るした結城紬(ゆうきつむぎ)の綿入を着たままである。いくら結城紬が丈夫だって、こう着つづけではたまらない。所々が薄くなって日に透かして見ると裏から\emph{つぎ}を当てた針の目が見える。主人の服装には師走(しわす)も正月もない。ふだん着も余所(よそ)ゆきもない。出るときは懐手(ふところで)をしてぶらりと出る。ほかに着る物がないからか、有っても面倒だから着換えないのか、吾輩には分らぬ。ただしこれだけは失恋のためとも思われない。
両人(ふたり)が出て行ったあとで、吾輩はちょっと失敬して寒月君の食い切った蒲鉾(かまぼこ)の残りを頂戴(ちょうだい)した。吾輩もこの頃では普通一般の猫ではない。まず桃川如燕(ももかわじょえん)以後の猫か、グレーの金魚を偸(ぬす)んだ猫くらいの資格は充分あると思う。車屋の黒などは固(もと)より眼中にない。蒲鉾の一切(ひときれ)くらい頂戴したって人からかれこれ云われる事もなかろう。それにこの人目を忍んで間食(かんしょく)をするという癖は、何も吾等猫族に限った事ではない。うちの御三(おさん)などはよく細君の留守中に餅菓子などを失敬しては頂戴し、頂戴しては失敬している。御三ばかりじゃない現に上品な仕付(しつけ)を受けつつあると細君から吹聴(ふいちょう)せられている小児(こども)ですらこの傾向がある。四五日前のことであったが、二人の小供が馬鹿に早くから眼を覚まして、まだ主人夫婦の寝ている間に対(むか)い合うて食卓に着いた。彼等は毎朝主人の食う麺麭(パン)の幾分に、砂糖をつけて食うのが例であるが、この日はちょうど砂糖壺(さとうつぼ)が卓(たく)の上に置かれて匙(さじ)さえ添えてあった。いつものように砂糖を分配してくれるものがないので、大きい方がやがて壺の中から一匙(ひとさじ)の砂糖をすくい出して自分の皿の上へあけた。すると小さいのが姉のした通り同分量の砂糖を同方法で自分の皿の上にあけた。少(しば)らく両人(りょうにん)は睨(にら)み合っていたが、大きいのがまた匙をとって一杯をわが皿の上に加えた。小さいのもすぐ匙をとってわが分量を姉と同一にした。すると姉がまた一杯すくった。妹も負けずに一杯を附加した。姉がまた壺へ手を懸ける、妹がまた匙をとる。見ている間(ま)に一杯一杯一杯と重なって、ついには両人(ふたり)の皿には山盛の砂糖が堆(うずたか)くなって、壺の中には一匙の砂糖も余っておらんようになったとき、主人が寝ぼけ眼(まなこ)を擦(こす)りながら寝室を出て来てせっかくしゃくい出した砂糖を元のごとく壺の中へ入れてしまった。こんなところを見ると、人間は利己主義から割り出した公平という念は猫より優(まさ)っているかも知れぬが、智慧(ちえ)はかえって猫より劣っているようだ。そんなに山盛にしないうちに早く甞(な)めてしまえばいいにと思ったが、例のごとく、吾輩の言う事などは通じないのだから、気の毒ながら御櫃(おはち)の上から黙って見物していた。
寒月君と出掛けた主人はどこをどう歩行(ある)いたものか、その晩遅く帰って来て、翌日食卓に就(つ)いたのは九時頃であった。例の御櫃の上から拝見していると、主人はだまって雑煮(ぞうに)を食っている。代えては食い、代えては食う。餅の切れは小さいが、何でも六切(むきれ)か七切(ななきれ)食って、最後の一切れを椀の中へ残して、もうよそうと箸(はし)を置いた。他人がそんな我儘(わがまま)をすると、なかなか承知しないのであるが、主人の威光を振り廻わして得意なる彼は、濁った汁の中に焦(こ)げ爛(ただ)れた餅の死骸を見て平気ですましている。妻君が袋戸(ふくろど)の奥からタカジヤスターゼを出して卓の上に置くと、主人は「それは利(き)かないから飲まん」という。「でもあなた澱粉質(でんぷんしつ)のものには大変功能があるそうですから、召し上ったらいいでしょう」と飲ませたがる。「澱粉だろうが何だろうが駄目だよ」と頑固(がんこ)に出る。「あなたはほんとに厭(あ)きっぽい」と細君が独言(ひとりごと)のようにいう。「厭きっぽいのじゃない薬が利かんのだ」「それだってせんだってじゅうは大変によく利くよく利くとおっしゃって毎日毎日上ったじゃありませんか」「こないだうちは利いたのだよ、この頃は利かないのだよ」と対句(ついく)のような返事をする。「そんなに飲んだり止(や)めたりしちゃ、いくら功能のある薬でも利く気遣(きづか)いはありません、もう少し辛防(しんぼう)がよくなくっちゃあ胃弱なんぞはほかの病気たあ違って直らないわねえ」とお盆を持って控えた御三(おさん)を顧みる。「それは本当のところでございます。もう少し召し上ってご覧にならないと、とても善(よ)い薬か悪い薬かわかりますまい」と御三は一も二もなく細君の肩を持つ。「何でもいい、飲まんのだから飲まんのだ、女なんかに何がわかるものか、黙っていろ」「どうせ女ですわ」と細君がタカジヤスターゼを主人の前へ突き付けて是非詰腹(つめばら)を切らせようとする。主人は何にも云わず立って書斎へ這入(はい)る。細君と御三は顔を見合せてにやにやと笑う。こんなときに後(あと)からくっ付いて行って膝(ひざ)の上へ乗ると、大変な目に逢(あ)わされるから、そっと庭から廻って書斎の椽側へ上(あが)って障子の隙(すき)から覗(のぞ)いて見ると、主人はエピクテタスとか云う人の本を披(ひら)いて見ておった。もしそれが平常(いつも)の通りわかるならちょっとえらいところがある。五六分するとその本を叩(たた)き付けるように机の上へ抛(ほう)り出す。大方そんな事だろうと思いながらなお注意していると、今度は日記帳を出して下(しも)のような事を書きつけた。
\blockquote{寒月と、根津、上野、池(いけ)の端(はた)、神田辺(へん)を散歩。池の端の待合の前で芸者が裾模様の春着(はるぎ)をきて羽根をついていた。衣装(いしょう)は美しいが顔はすこぶるまずい。何となくうちの猫に似ていた。}
何も顔のまずい例に特に吾輩を出さなくっても、よさそうなものだ。吾輩だって喜多床(きたどこ)へ行って顔さえ剃(す)って貰(もら)やあ、そんなに人間と異(ちが)ったところはありゃしない。人間はこう自惚(うぬぼ)れているから困る。
\blockquote{宝丹(ほうたん)の角(かど)を曲るとまた一人芸者が来た。これは背(せい)のすらりとした撫肩(なでがた)の恰好(かっこう)よく出来上った女で、着ている薄紫の衣服(きもの)も素直に着こなされて上品に見えた。白い歯を出して笑いながら「源ちゃん昨夕(ゆうべ)は――つい忙がしかったもんだから」と云った。ただしその声は旅鴉(たびがらす)のごとく皺枯(しゃが)れておったので、せっかくの風采(ふうさい)も大(おおい)に下落したように感ぜられたから、いわゆる源ちゃんなるもののいかなる人なるかを振り向いて見るも面倒になって、懐手(ふところで)のまま御成道(おなりみち)へ出た。寒月は何となくそわそわしているごとく見えた。}
人間の心理ほど解(げ)し難いものはない。この主人の今の心は怒(おこ)っているのだか、浮かれているのだか、または哲人の遺書に一道(いちどう)の慰安を求めつつあるのか、ちっとも分らない。世の中を冷笑しているのか、世の中へ交(まじ)りたいのだか、くだらぬ事に肝癪(かんしゃく)を起しているのか、物外(ぶつがい)に超然(ちょうぜん)としているのだかさっぱり見当(けんとう)が付かぬ。猫などはそこへ行くと単純なものだ。食いたければ食い、寝たければ寝る、怒(おこ)るときは一生懸命に怒り、泣くときは絶体絶命に泣く。第一日記などという無用のものは決してつけない。つける必要がないからである。主人のように裏表のある人間は日記でも書いて世間に出されない自己の面目を暗室内に発揮する必要があるかも知れないが、我等猫属(ねこぞく)に至ると行住坐臥(ぎょうじゅうざが)、行屎送尿(こうしそうにょう)ことごとく真正の日記であるから、別段そんな面倒な手数(てかず)をして、己(おの)れの真面目(しんめんもく)を保存するには及ばぬと思う。日記をつけるひまがあるなら椽側に寝ているまでの事さ。
\blockquote{神田の某亭で晩餐(ばんさん)を食う。久し振りで正宗を二三杯飲んだら、今朝は胃の具合が大変いい。胃弱には晩酌が一番だと思う。タカジヤスターゼは無論いかん。誰が何と云っても駄目だ。どうしたって利(き)かないものは利かないのだ。}
無暗(むやみ)にタカジヤスターゼを攻撃する。独りで喧嘩をしているようだ。今朝の肝癪がちょっとここへ尾を出す。人間の日記の本色はこう云う辺(へん)に存するのかも知れない。
\blockquote{せんだって○○は朝飯(あさめし)を廃すると胃がよくなると云うたから二三日(にさんち)朝飯をやめて見たが腹がぐうぐう鳴るばかりで功能はない。△△は是非香(こう)の物(もの)を断(た)てと忠告した。彼の説によるとすべて胃病の源因は漬物にある。漬物さえ断てば胃病の源を涸(か)らす訳だから本復は疑なしという論法であった。それから一週間ばかり香の物に箸(はし)を触れなかったが別段の験(げん)も見えなかったから近頃はまた食い出した。××に聞くとそれは按腹(あんぷく)揉療治(もみりょうじ)に限る。ただし普通のではゆかぬ。皆川流(みながわりゅう)という古流な揉(も)み方で一二度やらせれば大抵の胃病は根治出来る。安井息軒(やすいそっけん)も大変この按摩術(あんまじゅつ)を愛していた。坂本竜馬(さかもとりょうま)のような豪傑でも時々は治療をうけたと云うから、早速上根岸(かみねぎし)まで出掛けて揉(も)まして見た。ところが骨を揉(も)まなければ癒(なお)らぬとか、臓腑の位置を一度顛倒(てんとう)しなければ根治がしにくいとかいって、それはそれは残酷な揉(も)み方をやる。後で身体が綿のようになって昏睡病(こんすいびょう)にかかったような心持ちがしたので、一度で閉口してやめにした。A君は是非固形体を食うなという。それから、一日牛乳ばかり飲んで暮して見たが、この時は腸の中でどぼりどぼりと音がして大水でも出たように思われて終夜眠れなかった。B氏は横膈膜(おうかくまく)で呼吸して内臓を運動させれば自然と胃の働きが健全になる訳だから試しにやって御覧という。これも多少やったが何となく腹中(ふくちゅう)が不安で困る。それに時々思い出したように一心不乱にかかりはするものの五六分立つと忘れてしまう。忘れまいとすると横膈膜が気になって本を読む事も文章をかく事も出来ぬ。美学者の迷亭(めいてい)がこの体(てい)を見て、産気(さんけ)のついた男じゃあるまいし止(よ)すがいいと冷かしたからこの頃は廃(よ)してしまった。C先生は蕎麦(そば)を食ったらよかろうと云うから、早速\emph{かけ}と\emph{もり}をかわるがわる食ったが、これは腹が下(くだ)るばかりで何等の功能もなかった。余は年来の胃弱を直すために出来得る限りの方法を講じて見たがすべて駄目である。ただ昨夜(ゆうべ)寒月と傾けた三杯の正宗はたしかに利目(ききめ)がある。これからは毎晩二三杯ずつ飲む事にしよう。}
これも決して長く続く事はあるまい。主人の心は吾輩の眼球(めだま)のように間断なく変化している。何をやっても永持(ながもち)のしない男である。その上日記の上で胃病をこんなに心配している癖に、表向は大(おおい)に痩我慢をするからおかしい。せんだってその友人で某(なにがし)という学者が尋ねて来て、一種の見地から、すべての病気は父祖の罪悪と自己の罪悪の結果にほかならないと云う議論をした。大分(だいぶ)研究したものと見えて、条理が明晰(めいせき)で秩序が整然として立派な説であった。気の毒ながらうちの主人などは到底これを反駁(はんばく)するほどの頭脳も学問もないのである。しかし自分が胃病で苦しんでいる際(さい)だから、何とかかんとか弁解をして自己の面目を保とうと思った者と見えて、「君の説は面白いが、あのカーライルは胃弱だったぜ」とあたかもカーライルが胃弱だから自分の胃弱も名誉であると云ったような、見当違いの挨拶をした。すると友人は「カーライルが胃弱だって、胃弱の病人が必ずカーライルにはなれないさ」と極(き)め付けたので主人は黙然(もくねん)としていた。かくのごとく虚栄心に富んでいるものの実際はやはり胃弱でない方がいいと見えて、今夜から晩酌を始めるなどというのはちょっと滑稽だ。考えて見ると今朝雑煮(ぞうに)をあんなにたくさん食ったのも昨夜(ゆうべ)寒月君と正宗をひっくり返した影響かも知れない。吾輩もちょっと雑煮が食って見たくなった。
吾輩は猫ではあるが大抵のものは食う。車屋の黒のように横丁の肴屋(さかなや)まで遠征をする気力はないし、新道(しんみち)の二絃琴(にげんきん)の師匠の所(とこ)の三毛(みけ)のように贅沢(ぜいたく)は無論云える身分でない。従って存外嫌(きらい)は少ない方だ。小供の食いこぼした麺麭(パン)も食うし、餅菓子の\begin{comment}\includegraphics{../../../gaiji/2-92/2-92-68.png}\end{comment}(あん)もなめる。香(こう)の物(もの)はすこぶるまずいが経験のため沢庵(たくあん)を二切ばかりやった事がある。食って見ると妙なもので、大抵のものは食える。あれは嫌(いや)だ、これは嫌だと云うのは贅沢(ぜいたく)な我儘で到底教師の家(うち)にいる猫などの口にすべきところでない。主人の話しによると仏蘭西(フランス)にバルザックという小説家があったそうだ。この男が大の贅沢(ぜいたく)屋で――もっともこれは口の贅沢屋ではない、小説家だけに文章の贅沢を尽したという事である。バルザックが或る日自分の書いている小説中の人間の名をつけようと思っていろいろつけて見たが、どうしても気に入らない。ところへ友人が遊びに来たのでいっしょに散歩に出掛けた。友人は固(もと)より何(なんに)も知らずに連れ出されたのであるが、バルザックは兼(か)ねて自分の苦心している名を目付(めつけ)ようという考えだから往来へ出ると何もしないで店先の看板ばかり見て歩行(ある)いている。ところがやはり気に入った名がない。友人を連れて無暗(むやみ)にあるく。友人は訳がわからずにくっ付いて行く。彼等はついに朝から晩まで巴理(パリ)を探険した。その帰りがけにバルザックはふとある裁縫屋の看板が目についた。見るとその看板にマーカスという名がかいてある。バルザックは手を拍(う)って「これだこれだこれに限る。マーカスは好い名じゃないか。マーカスの上へZという頭文字をつける、すると申し分(ぶん)のない名が出来る。Zでなくてはいかん。Z. Marcus は実にうまい。どうも自分で作った名はうまくつけたつもりでも何となく故意(わざ)とらしいところがあって面白くない。ようやくの事で気に入った名が出来た」と友人の迷惑はまるで忘れて、一人嬉しがったというが、小説中の人間の名前をつけるに一日(いちんち)巴理(パリ)を探険しなくてはならぬようでは随分手数(てすう)のかかる話だ。贅沢もこのくらい出来れば結構なものだが吾輩のように牡蠣的(かきてき)主人を持つ身の上ではとてもそんな気は出ない。何でもいい、食えさえすれば、という気になるのも境遇のしからしむるところであろう。だから今雑煮(ぞうに)が食いたくなったのも決して贅沢の結果ではない、何でも食える時に食っておこうという考から、主人の食い剰(あま)した雑煮がもしや台所に残っていはすまいかと思い出したからである。\ldots{}\ldots{}台所へ廻って見る。
今朝見た通りの餅が、今朝見た通りの色で椀の底に膠着(こうちゃく)している。白状するが餅というものは今まで一辺(ぺん)も口に入れた事がない。見るとうまそうにもあるし、また少しは気味(きび)がわるくもある。前足で上にかかっている菜っ葉を掻(か)き寄せる。爪を見ると餅の上皮(うわかわ)が引き掛ってねばねばする。嗅(か)いで見ると釜の底の飯を御櫃(おはち)へ移す時のような香(におい)がする。食おうかな、やめようかな、とあたりを見廻す。幸か不幸か誰もいない。御三(おさん)は暮も春も同じような顔をして羽根をついている。小供は奥座敷で「何とおっしゃる兎さん」を歌っている。食うとすれば今だ。もしこの機をはずすと来年までは餅というものの味を知らずに暮してしまわねばならぬ。吾輩はこの刹那(せつな)に猫ながら一の真理を感得した。「得難き機会はすべての動物をして、好まざる事をも敢てせしむ」吾輩は実を云うとそんなに雑煮を食いたくはないのである。否椀底(わんてい)の様子を熟視すればするほど気味(きび)が悪くなって、食うのが厭になったのである。この時もし御三でも勝手口を開けたなら、奥の小供の足音がこちらへ近付くのを聞き得たなら、吾輩は惜気(おしげ)もなく椀を見棄てたろう、しかも雑煮の事は来年まで念頭に浮ばなかったろう。ところが誰も来ない、いくら\begin{comment}\includegraphics{../../../gaiji/1-92/1-92-39.png}\end{comment}躇(ちゅうちょ)していても誰も来ない。早く食わぬか食わぬかと催促されるような心持がする。吾輩は椀の中を覗(のぞ)き込みながら、早く誰か来てくれればいいと念じた。やはり誰も来てくれない。吾輩はとうとう雑煮を食わなければならぬ。最後にからだ全体の重量を椀の底へ落すようにして、あぐりと餅の角を一寸(いっすん)ばかり食い込んだ。このくらい力を込めて食い付いたのだから、大抵なものなら噛(か)み切れる訳だが、驚いた! もうよかろうと思って歯を引こうとすると引けない。もう一辺(ぺん)噛み直そうとすると動きがとれない。餅は魔物だなと疳(かん)づいた時はすでに遅かった。沼へでも落ちた人が足を抜こうと焦慮(あせ)るたびにぶくぶく深く沈むように、噛めば噛むほど口が重くなる、歯が動かなくなる。歯答えはあるが、歯答えがあるだけでどうしても始末をつける事が出来ない。美学者迷亭先生がかつて吾輩の主人を評して君は割り切れない男だといった事があるが、なるほどうまい事をいったものだ。この餅も主人と同じようにどうしても割り切れない。噛んでも噛んでも、三で十を割るごとく尽未来際方(じんみらいざいかた)のつく期(ご)はあるまいと思われた。この煩悶(はんもん)の際吾輩は覚えず第二の真理に逢着(ほうちゃく)した。「すべての動物は直覚的に事物の適不適を予知す」真理はすでに二つまで発明したが、餅がくっ付いているので毫(ごう)も愉快を感じない。歯が餅の肉に吸収されて、抜けるように痛い。早く食い切って逃げないと御三(おさん)が来る。小供の唱歌もやんだようだ、きっと台所へ馳(か)け出して来るに相違ない。煩悶の極(きょく)尻尾(しっぽ)をぐるぐる振って見たが何等の功能もない、耳を立てたり寝かしたりしたが駄目である。考えて見ると耳と尻尾(しっぽ)は餅と何等の関係もない。要するに振り損の、立て損の、寝かし損であると気が付いたからやめにした。ようやくの事これは前足の助けを借りて餅を払い落すに限ると考え付いた。まず右の方をあげて口の周囲を撫(な)で廻す。撫(な)でたくらいで割り切れる訳のものではない。今度は左(ひだ)りの方を伸(のば)して口を中心として急劇に円を劃(かく)して見る。そんな呪(まじな)いで魔は落ちない。辛防(しんぼう)が肝心(かんじん)だと思って左右交(かわ)る交(がわ)るに動かしたがやはり依然として歯は餅の中にぶら下っている。ええ面倒だと両足を一度に使う。すると不思議な事にこの時だけは後足(あとあし)二本で立つ事が出来た。何だか猫でないような感じがする。猫であろうが、あるまいがこうなった日にゃあ構うものか、何でも餅の魔が落ちるまでやるべしという意気込みで無茶苦茶に顔中引っ掻(か)き廻す。前足の運動が猛烈なのでややともすると中心を失って倒れかかる。倒れかかるたびに後足で調子をとらなくてはならぬから、一つ所にいる訳にも行かんので、台所中あちら、こちらと飛んで廻る。我ながらよくこんなに器用に起(た)っていられたものだと思う。第三の真理が驀地(ばくち)に現前(げんぜん)する。「危きに臨(のぞ)めば平常なし能(あた)わざるところのものを為(な)し能う。之(これ)を天祐(てんゆう)という」幸(さいわい)に天祐を享(う)けたる吾輩が一生懸命餅の魔と戦っていると、何だか足音がして奥より人が来るような気合(けわい)である。ここで人に来られては大変だと思って、いよいよ躍起(やっき)となって台所をかけ廻る。足音はだんだん近付いてくる。ああ残念だが天祐が少し足りない。とうとう小供に見付けられた。「あら猫が御雑煮を食べて踊を踊っている」と大きな声をする。この声を第一に聞きつけたのが御三である。羽根も羽子板も打ち遣(や)って勝手から「あらまあ」と飛込んで来る。細君は縮緬(ちりめん)の紋付で「いやな猫ねえ」と仰せられる。主人さえ書斎から出て来て「この馬鹿野郎」といった。面白い面白いと云うのは小供ばかりである。そうしてみんな申し合せたようにげらげら笑っている。腹は立つ、苦しくはある、踊はやめる訳にゆかぬ、弱った。ようやく笑いがやみそうになったら、五つになる女の子が「御かあ様、猫も随分ね」といったので狂瀾(きょうらん)を既倒(きとう)に何とかするという勢でまた大変笑われた。人間の同情に乏しい実行も大分(だいぶ)見聞(けんもん)したが、この時ほど恨(うら)めしく感じた事はなかった。ついに天祐もどっかへ消え失(う)せて、在来の通り四(よ)つ這(ばい)になって、眼を白黒するの醜態を演ずるまでに閉口した。さすが見殺しにするのも気の毒と見えて「まあ餅をとってやれ」と主人が御三に命ずる。御三はもっと踊らせようじゃありませんかという眼付で細君を見る。細君は踊は見たいが、殺してまで見る気はないのでだまっている。「取ってやらんと死んでしまう、早くとってやれ」と主人は再び下女を顧(かえり)みる。御三(おさん)は御馳走を半分食べかけて夢から起された時のように、気のない顔をして餅をつかんでぐいと引く。寒月(かんげつ)君じゃないが前歯がみんな折れるかと思った。どうも痛いの痛くないのって、餅の中へ堅く食い込んでいる歯を情(なさ)け容赦もなく引張るのだからたまらない。吾輩が「すべての安楽は困苦を通過せざるべからず」と云う第四の真理を経験して、けろけろとあたりを見廻した時には、家人はすでに奥座敷へ這入(はい)ってしまっておった。
こんな失敗をした時には内にいて御三なんぞに顔を見られるのも何となくばつが悪い。いっその事気を易(か)えて新道の二絃琴(にげんきん)の御師匠さんの所(とこ)の三毛子(みけこ)でも訪問しようと台所から裏へ出た。三毛子はこの近辺で有名な美貌家(びぼうか)である。吾輩は猫には相違ないが物の情(なさ)けは一通り心得ている。うちで主人の苦(にが)い顔を見たり、御三の険突(けんつく)を食って気分が勝(すぐ)れん時は必ずこの異性の朋友(ほうゆう)の許(もと)を訪問していろいろな話をする。すると、いつの間(ま)にか心が晴々(せいせい)して今までの心配も苦労も何もかも忘れて、生れ変ったような心持になる。女性の影響というものは実に莫大(ばくだい)なものだ。杉垣の隙から、いるかなと思って見渡すと、三毛子は正月だから首輪の新しいのをして行儀よく椽側(えんがわ)に坐っている。その背中の丸さ加減が言うに言われんほど美しい。曲線の美を尽している。尻尾(しっぽ)の曲がり加減、足の折り具合、物憂(ものう)げに耳をちょいちょい振る景色(けしき)なども到底(とうてい)形容が出来ん。ことによく日の当る所に暖かそうに、品(ひん)よく控(ひか)えているものだから、身体は静粛端正の態度を有するにも関らず、天鵞毛(びろうど)を欺(あざむ)くほどの滑(なめ)らかな満身の毛は春の光りを反射して風なきにむらむらと微動するごとくに思われる。吾輩はしばらく恍惚(こうこつ)として眺(なが)めていたが、やがて我に帰ると同時に、低い声で「三毛子さん三毛子さん」といいながら前足で招いた。三毛子は「あら先生」と椽を下りる。赤い首輪につけた鈴がちゃらちゃらと鳴る。おや正月になったら鈴までつけたな、どうもいい音(ね)だと感心している間(ま)に、吾輩の傍(そば)に来て「あら先生、おめでとう」と尾を左(ひだ)りへ振る。吾等猫属(ねこぞく)間で御互に挨拶をするときには尾を棒のごとく立てて、それを左りへぐるりと廻すのである。町内で吾輩を先生と呼んでくれるのはこの三毛子ばかりである。吾輩は前回断わった通りまだ名はないのであるが、教師の家(うち)にいるものだから三毛子だけは尊敬して先生先生といってくれる。吾輩も先生と云われて満更(まんざら)悪い心持ちもしないから、はいはいと返事をしている。「やあおめでとう、大層立派に御化粧が出来ましたね」「ええ去年の暮御師匠(おししょう)さんに買って頂いたの、宜(い)いでしょう」とちゃらちゃら鳴らして見せる。「なるほど善い音(ね)ですな、吾輩などは生れてから、そんな立派なものは見た事がないですよ」「あらいやだ、みんなぶら下げるのよ」とまたちゃらちゃら鳴らす。「いい音(ね)でしょう、あたし嬉しいわ」とちゃらちゃらちゃらちゃら続け様に鳴らす。「あなたのうちの御師匠さんは大変あなたを可愛がっていると見えますね」と吾身に引きくらべて暗(あん)に欣羨(きんせん)の意を洩(も)らす。三毛子は無邪気なものである「ほんとよ、まるで自分の小供のようよ」とあどけなく笑う。猫だって笑わないとは限らない。人間は自分よりほかに笑えるものが無いように思っているのは間違いである。吾輩が笑うのは鼻の孔(あな)を三角にして咽喉仏(のどぼとけ)を震動させて笑うのだから人間にはわからぬはずである。「一体あなたの所(とこ)の御主人は何ですか」「あら御主人だって、妙なのね。御師匠(おししょう)さんだわ。二絃琴(にげんきん)の御師匠さんよ」「それは吾輩も知っていますがね。その御身分は何なんです。いずれ昔(むか)しは立派な方なんでしょうな」「ええ」
\blockquote{CHECK HERE君を待つ間(ま)の姫小松\ldots{}\ldots{}\ldots{}\ldots{}\ldots{}}
障子の内で御師匠さんが二絃琴を弾(ひ)き出す。「宜(い)い声でしょう」と三毛子は自慢する。「宜(い)いようだが、吾輩にはよくわからん。全体何というものですか」「あれ? あれは何とかってものよ。御師匠さんはあれが大好きなの。\ldots{}\ldots{}御師匠さんはあれで六十二よ。随分丈夫だわね」六十二で生きているくらいだから丈夫と云わねばなるまい。吾輩は「はあ」と返事をした。少し間(ま)が抜けたようだが別に名答も出て来なかったから仕方がない。「あれでも、もとは身分が大変好かったんだって。いつでもそうおっしゃるの」「へえ元は何だったんです」「何でも天璋院(てんしょういん)様の御祐筆(ごゆうひつ)の妹の御嫁に行った先(さ)きの御(お)っかさんの甥(おい)の娘なんだって」「何ですって?」「あの天璋院様の御祐筆の妹の御嫁にいった\ldots{}\ldots{}」「なるほど。少し待って下さい。天璋院様の妹の御祐筆の\ldots{}\ldots{}」「あらそうじゃないの、天璋院様の御祐筆の妹の\ldots{}\ldots{}」「よろしい分りました天璋院様のでしょう」「ええ」「御祐筆のでしょう」「そうよ」「御嫁に行った」「妹の御嫁に行ったですよ」「そうそう間違った。妹の御嫁に入(い)った先きの」「御っかさんの甥の娘なんですとさ」「御っかさんの甥の娘なんですか」「ええ。分ったでしょう」「いいえ。何だか混雑して要領を得ないですよ。詰(つま)るところ天璋院様の何になるんですか」「あなたもよっぽど分らないのね。だから天璋院様の御祐筆の妹の御嫁に行った先きの御っかさんの甥の娘なんだって、先(さ)っきっから言ってるんじゃありませんか」「それはすっかり分っているんですがね」「それが分りさえすればいいんでしょう」「ええ」と仕方がないから降参をした。吾々は時とすると理詰の虚言(うそ)を吐(つ)かねばならぬ事がある。
障子の中(うち)で二絃琴の音(ね)がぱったりやむと、御師匠さんの声で「三毛や三毛や御飯だよ」と呼ぶ。三毛子は嬉しそうに「あら御師匠さんが呼んでいらっしゃるから、私(あた)し帰るわ、よくって?」わるいと云ったって仕方がない。「それじゃまた遊びにいらっしゃい」と鈴をちゃらちゃら鳴らして庭先までかけて行ったが急に戻って来て「あなた大変色が悪くってよ。どうかしやしなくって」と心配そうに問いかける。まさか雑煮(ぞうに)を食って踊りを踊ったとも云われないから「何別段の事もありませんが、少し考え事をしたら頭痛がしてね。あなたと話しでもしたら直るだろうと思って実は出掛けて来たのですよ」「そう。御大事になさいまし。さようなら」少しは名残(なご)り惜し気に見えた。これで雑煮の元気もさっぱりと回復した。いい心持になった。帰りに例の茶園(ちゃえん)を通り抜けようと思って霜柱(しもばしら)の融(と)けかかったのを踏みつけながら建仁寺(けんにんじ)の崩(くず)れから顔を出すとまた車屋の黒が枯菊の上に背(せ)を山にして欠伸(あくび)をしている。近頃は黒を見て恐怖するような吾輩ではないが、話しをされると面倒だから知らぬ顔をして行き過ぎようとした。黒の性質として他(ひと)が己(おの)れを軽侮(けいぶ)したと認むるや否や決して黙っていない。「おい、名なしの権兵衛(ごんべえ)、近頃じゃ乙(おつ)う高く留ってるじゃあねえか。いくら教師の飯を食ったって、そんな高慢ちきな面(つ)らあするねえ。人(ひと)つけ面白くもねえ」黒は吾輩の有名になったのを、まだ知らんと見える。説明してやりたいが到底(とうてい)分る奴ではないから、まず一応の挨拶をして出来得る限り早く御免蒙(ごめんこうむ)るに若(し)くはないと決心した。「いや黒君おめでとう。不相変(あいかわらず)元気がいいね」と尻尾(しっぽ)を立てて左へくるりと廻わす。黒は尻尾を立てたぎり挨拶もしない。「何おめでてえ? 正月でおめでたけりゃ、御めえなんざあ年が年中おめでてえ方だろう。気をつけろい、この吹(ふ)い子(ご)の向(むこ)う面(づら)め」吹い子の向うづらという句は罵詈(ばり)の言語であるようだが、吾輩には了解が出来なかった。「ちょっと伺(うか)がうが吹い子の向うづらと云うのはどう云う意味かね」「へん、手めえが悪体(あくたい)をつかれてる癖に、その訳(わけ)を聞きゃ世話あねえ、だから正月野郎だって事よ」正月野郎は詩的であるが、その意味に至ると吹い子の何とかよりも一層不明瞭な文句である。参考のためちょっと聞いておきたいが、聞いたって明瞭な答弁は得られぬに極(き)まっているから、面(めん)と対(むか)ったまま無言で立っておった。いささか手持無沙汰の体(てい)である。すると突然黒のうちの神(かみ)さんが大きな声を張り揚げて「おや棚へ上げて置いた鮭(しゃけ)がない。大変だ。またあの黒の畜生(ちきしょう)が取ったんだよ。ほんとに憎らしい猫だっちゃありゃあしない。今に帰って来たら、どうするか見ていやがれ」と怒鳴(どな)る。初春(はつはる)の長閑(のどか)な空気を無遠慮に震動させて、枝を鳴らさぬ君が御代(みよ)を大(おおい)に俗了(ぞくりょう)してしまう。黒は怒鳴るなら、怒鳴りたいだけ怒鳴っていろと云わぬばかりに横着な顔をして、四角な顋(あご)を前へ出しながら、あれを聞いたかと合図をする。今までは黒との応対で気がつかなかったが、見ると彼の足の下には一切れ二銭三厘に相当する鮭の骨が泥だらけになって転がっている。「君不相変(あいかわらず)やってるな」と今までの行き掛りは忘れて、つい感投詞を奉呈した。黒はそのくらいな事ではなかなか機嫌を直さない。「何がやってるでえ、この野郎。\emph{しゃけ}の一切や二切で相変らずたあ何だ。人を見縊(みく)びった事をいうねえ。憚(はばか)りながら車屋の黒だあ」と腕まくりの代りに右の前足を逆(さ)かに肩の辺(へん)まで掻(か)き上げた。「君が黒君だと云う事は、始めから知ってるさ」「知ってるのに、相変らずやってるたあ何だ。何だてえ事よ」と熱いのを頻(しき)りに吹き懸ける。人間なら胸倉(むなぐら)をとられて小突き廻されるところである。少々辟易(へきえき)して内心困った事になったなと思っていると、再び例の神さんの大声が聞える。「ちょいと西川さん、おい西川さんてば、用があるんだよこの人あ。牛肉を一斤(きん)すぐ持って来るんだよ。いいかい、分ったかい、牛肉の堅くないところを一斤だよ」と牛肉注文の声が四隣(しりん)の寂寞(せきばく)を破る。「へん年に一遍牛肉を誂(あつら)えると思って、いやに大きな声を出しゃあがらあ。牛肉一斤が隣り近所へ自慢なんだから始末に終えねえ阿魔(あま)だ」と黒は嘲(あざけ)りながら四つ足を踏張(ふんば)る。吾輩は挨拶のしようもないから黙って見ている。「一斤くらいじゃあ、承知が出来ねえんだが、仕方がねえ、いいから取っときゃ、今に食ってやらあ」と自分のために誂(あつら)えたもののごとくいう。「今度は本当の御馳走だ。結構結構」と吾輩はなるべく彼を帰そうとする。「御めっちの知った事じゃねえ。黙っていろ。うるせえや」と云いながら突然後足(あとあし)で霜柱(しもばしら)の崩(くず)れた奴を吾輩の頭へばさりと浴(あ)びせ掛ける。吾輩が驚ろいて、からだの泥を払っている間(ま)に黒は垣根を潜(くぐ)って、どこかへ姿を隠した。大方西川の牛(ぎゅう)を覘(ねらい)に行ったものであろう。
家(うち)へ帰ると座敷の中が、いつになく春めいて主人の笑い声さえ陽気に聞える。はてなと明け放した椽側から上(あが)って主人の傍(そば)へ寄って見ると見馴れぬ客が来ている。頭を奇麗に分けて、木綿(もめん)の紋付の羽織に小倉(こくら)の袴(はかま)を着けて至極(しごく)真面目そうな書生体(しょせいてい)の男である。主人の手あぶりの角を見ると春慶塗(しゅんけいぬ)りの巻煙草(まきたばこ)入れと並んで越智東風君(おちとうふうくん)を紹介致候(そろ)水島寒月という名刺があるので、この客の名前も、寒月君の友人であるという事も知れた。主客(しゅかく)の対話は途中からであるから前後がよく分らんが、何でも吾輩が前回に紹介した美学者迷亭君の事に関しているらしい。
「それで面白い趣向があるから是非いっしょに来いとおっしゃるので」と客は落ちついて云う。「何ですか、その西洋料理へ行って午飯(ひるめし)を食うのについて趣向があるというのですか」と主人は茶を続(つ)ぎ足して客の前へ押しやる。「さあ、その趣向というのが、その時は私にも分らなかったんですが、いずれあの方(かた)の事ですから、何か面白い種があるのだろうと思いまして\ldots{}\ldots{}」「いっしょに行きましたか、なるほど」「ところが驚いたのです」主人はそれ見たかと云わぬばかりに、膝(ひざ)の上に乗った吾輩の頭をぽかと叩(たた)く。少し痛い。「また馬鹿な茶番見たような事なんでしょう。あの男はあれが癖でね」と急にアンドレア・デル・サルト事件を思い出す。「へへー。君何か変ったものを食おうじゃないかとおっしゃるので」「何を食いました」「まず献立(こんだて)を見ながらいろいろ料理についての御話しがありました」「誂(あつ)らえない前にですか」「ええ」「それから」「それから首を捻(ひね)ってボイの方を御覧になって、どうも変ったものもないようだなとおっしゃるとボイは負けぬ気で鴨(かも)のロースか小牛のチャップなどは如何(いかが)ですと云うと、先生は、そんな月並(つきなみ)を食いにわざわざここまで来やしないとおっしゃるんで、ボイは月並という意味が分らんものですから妙な顔をして黙っていましたよ」「そうでしょう」「それから私の方を御向きになって、君仏蘭西(フランス)や英吉利(イギリス)へ行くと随分天明調(てんめいちょう)や万葉調(まんようちょう)が食えるんだが、日本じゃどこへ行ったって版で圧(お)したようで、どうも西洋料理へ這入(はい)る気がしないと云うような大気\begin{comment}\includegraphics{../../../gaiji/1-87/1-87-64.png}\end{comment}(だいきえん)で――全体あの方(かた)は洋行なすった事があるのですかな」「何迷亭が洋行なんかするもんですか、そりゃ金もあり、時もあり、行こうと思えばいつでも行かれるんですがね。大方これから行くつもりのところを、過去に見立てた洒落(しゃれ)なんでしょう」と主人は自分ながらうまい事を言ったつもりで誘い出し笑をする。客はさまで感服した様子もない。「そうですか、私はまたいつの間(ま)に洋行なさったかと思って、つい真面目に拝聴していました。それに見て来たように\emph{なめくじ}のソップの御話や蛙(かえる)のシチュの形容をなさるものですから」「そりゃ誰かに聞いたんでしょう、うそをつく事はなかなか名人ですからね」「どうもそうのようで」と花瓶(かびん)の水仙を眺める。少しく残念の気色(けしき)にも取られる。「じゃ趣向というのは、それなんですね」と主人が念を押す。「いえそれはほんの冒頭なので、本論はこれからなのです」「ふーん」と主人は好奇的な感投詞を挟(はさ)む。「それから、とても\emph{なめくじ}や蛙は食おうっても食えやしないから、まあ\emph{トチメンボー}くらいなところで負けとく事にしようじゃないか君と御相談なさるものですから、私はつい何の気なしに、それがいいでしょう、といってしまったので」「へー、とちめんぼうは妙ですな」「ええ全く妙なのですが、先生があまり真面目だものですから、つい気がつきませんでした」とあたかも主人に向って麁忽(そこつ)を詫(わ)びているように見える。「それからどうしました」と主人は無頓着に聞く。客の謝罪には一向同情を表しておらん。「それからボイにおい\emph{トチメンボー}を二人前(ににんまえ)持って来いというと、ボイが\emph{メンチボー}ですかと聞き直しましたが、先生はますます真面目(まじめ)な貌(かお)で\emph{メンチボー}じゃない\emph{トチメンボー}だと訂正されました」「なある。その\emph{トチメンボー}という料理は一体あるんですか」「さあ私も少しおかしいとは思いましたがいかにも先生が沈着であるし、その上あの通りの西洋通でいらっしゃるし、ことにその時は洋行なすったものと信じ切っていたものですから、私も口を添えて\emph{トチメンボー}だ\emph{トチメンボー}だとボイに教えてやりました」「ボイはどうしました」「ボイがね、今考えると実に滑稽(こっけい)なんですがね、しばらく思案していましてね、はなはだ御気の毒様ですが今日は\emph{トチメンボー}は御生憎様(おあいにくさま)で\emph{メンチボー}なら御二人前(おふたりまえ)すぐに出来ますと云うと、先生は非常に残念な様子で、それじゃせっかくここまで来た甲斐(かい)がない。どうか\emph{トチメンボー}を都合(つごう)して食わせてもらう訳(わけ)には行くまいかと、ボイに二十銭銀貨をやられると、ボイはそれではともかくも料理番と相談して参りましょうと奥へ行きましたよ」「大変\emph{トチメンボー}が食いたかったと見えますね」「しばらくしてボイが出て来て真(まこと)に御生憎で、御誂(おあつらえ)ならこしらえますが少々時間がかかります、と云うと迷亭先生は落ちついたもので、どうせ我々は正月でひまなんだから、少し待って食って行こうじゃないかと云いながらポッケットから葉巻を出してぷかりぷかり吹かし始められたので、私(わたく)しも仕方がないから、懐(ふところ)から日本新聞を出して読み出しました、するとボイはまた奥へ相談に行きましたよ」「いやに手数(てすう)が掛りますな」と主人は戦争の通信を読むくらいの意気込で席を前(すす)める。「するとボイがまた出て来て、近頃は\emph{トチメンボー}の材料が払底で亀屋へ行っても横浜の十五番へ行っても買われませんから当分の間は御生憎様でと気の毒そうに云うと、先生はそりゃ困ったな、せっかく来たのになあと私の方を御覧になってしきりに繰り返さるるので、私も黙っている訳にも参りませんから、どうも遺憾(いかん)ですな、遺憾極(きわま)るですなと調子を合せたのです」「ごもっともで」と主人が賛成する。何がごもっともだか吾輩にはわからん。「するとボイも気の毒だと見えて、その内材料が参りましたら、どうか願いますってんでしょう。先生が材料は何を使うかねと問われるとボイはへへへへと笑って返事をしないんです。材料は日本派の俳人だろうと先生が押し返して聞くとボイはへえさようで、それだものだから近頃は横浜へ行っても買われませんので、まことにお気の毒様と云いましたよ」「アハハハそれが落ちなんですか、こりゃ面白い」と主人はいつになく大きな声で笑う。膝(ひざ)が揺れて吾輩は落ちかかる。主人はそれにも頓着(とんじゃく)なく笑う。アンドレア・デル・サルトに罹(かか)ったのは自分一人でないと云う事を知ったので急に愉快になったものと見える。「それから二人で表へ出ると、どうだ君うまく行ったろう、橡面坊(とちめんぼう)を種に使ったところが面白かろうと大得意なんです。敬服の至りですと云って御別れしたようなものの実は午飯(ひるめし)の時刻が延びたので大変空腹になって弱りましたよ」「それは御迷惑でしたろう」と主人は始めて同情を表する。これには吾輩も異存はない。しばらく話しが途切れて吾輩の咽喉(のど)を鳴らす音が主客(しゅかく)の耳に入る。
東風君は冷めたくなった茶をぐっと飲み干して「実は今日参りましたのは、少々先生に御願があって参ったので」と改まる。「はあ、何か御用で」と主人も負けずに済(す)ます。「御承知の通り、文学美術が好きなものですから\ldots{}\ldots{}」「結構で」と油を注(さ)す。「同志だけがよりましてせんだってから朗読会というのを組織しまして、毎月一回会合してこの方面の研究をこれから続けたいつもりで、すでに第一回は去年の暮に開いたくらいであります」「ちょっと伺っておきますが、朗読会と云うと何か節奏(ふし)でも附けて、詩歌(しいか)文章の類(るい)を読むように聞えますが、一体どんな風にやるんです」「まあ初めは古人の作からはじめて、追々(おいおい)は同人の創作なんかもやるつもりです」「古人の作というと白楽天(はくらくてん)の琵琶行(びわこう)のようなものででもあるんですか」「いいえ」「蕪村(ぶそん)の春風馬堤曲(しゅんぷうばていきょく)の種類ですか」「いいえ」「それじゃ、どんなものをやったんです」「せんだっては近松の心中物(しんじゅうもの)をやりました」「近松? あの浄瑠璃(じょうるり)の近松ですか」近松に二人はない。近松といえば戯曲家の近松に極(きま)っている。それを聞き直す主人はよほど愚(ぐ)だと思っていると、主人は何にも分らずに吾輩の頭を叮嚀(ていねい)に撫(な)でている。藪睨(やぶにら)みから惚(ほ)れられたと自認している人間もある世の中だからこのくらいの誤謬(ごびゅう)は決して驚くに足らんと撫でらるるがままにすましていた。「ええ」と答えて東風子(とうふうし)は主人の顔色を窺(うかが)う。「それじゃ一人で朗読するのですか、または役割を極(き)めてやるんですか」「役を極めて懸合(かけあい)でやって見ました。その主意はなるべく作中の人物に同情を持ってその性格を発揮するのを第一として、それに手真似や身振りを添えます。白(せりふ)はなるべくその時代の人を写し出すのが主で、御嬢さんでも丁稚(でっち)でも、その人物が出てきたようにやるんです」「じゃ、まあ芝居見たようなものじゃありませんか」「ええ衣装(いしょう)と書割(かきわり)がないくらいなものですな」「失礼ながらうまく行きますか」「まあ第一回としては成功した方だと思います」「それでこの前やったとおっしゃる心中物というと」「その、船頭が御客を乗せて芳原(よしわら)へ行く所(とこ)なんで」「大変な幕をやりましたな」と教師だけにちょっと首を傾(かたむ)ける。鼻から吹き出した\emph{日の出}の煙りが耳を掠(かす)めて顔の横手へ廻る。「なあに、そんなに大変な事もないんです。登場の人物は御客と、船頭と、花魁(おいらん)と仲居(なかい)と遣手(やりて)と見番(けんばん)だけですから」と東風子は平気なものである。主人は花魁という名をきいてちょっと苦(にが)い顔をしたが、仲居、遣手、見番という術語について明瞭の智識がなかったと見えてまず質問を呈出した。「仲居というのは娼家(しょうか)の下婢(かひ)にあたるものですかな」「まだよく研究はして見ませんが仲居は茶屋の下女で、遣手というのが女部屋(おんなべや)の助役(じょやく)見たようなものだろうと思います」東風子はさっき、その人物が出て来るように仮色(こわいろ)を使うと云った癖に遣手や仲居の性格をよく解しておらんらしい。「なるほど仲居は茶屋に隷属(れいぞく)するもので、遣手は娼家に起臥(きが)する者ですね。次に\emph{見番}と云うのは人間ですかまたは一定の場所を指(さ)すのですか、もし人間とすれば男ですか女ですか」「見番は何でも男の人間だと思います」「何を司(つかさ)どっているんですかな」「さあそこまではまだ調べが届いておりません。その内調べて見ましょう」これで懸合をやった日には頓珍漢(とんちんかん)なものが出来るだろうと吾輩は主人の顔をちょっと見上げた。主人は存外真面目である。「それで朗読家は君のほかにどんな人が加わったんですか」「いろいろおりました。花魁が法学士のK君でしたが、口髯(くちひげ)を生やして、女の甘ったるいせりふを使(つ)かうのですからちょっと妙でした。それにその花魁が癪(しゃく)を起すところがあるので\ldots{}\ldots{}」「朗読でも癪を起さなくっちゃ、いけないんですか」と主人は心配そうに尋ねる。「ええとにかく表情が大事ですから」と東風子はどこまでも文芸家の気でいる。「うまく癪が起りましたか」と主人は警句を吐く。「癪だけは第一回には、ちと無理でした」と東風子も警句を吐く。「ところで君は何の役割でした」と主人が聞く。「私(わたく)しは船頭」「へー、君が船頭」君にして船頭が務(つと)まるものなら僕にも見番くらいはやれると云ったような語気を洩(も)らす。やがて「船頭は無理でしたか」と御世辞のないところを打ち明ける。東風子は別段癪に障った様子もない。やはり沈着な口調で「その船頭でせっかくの催しも竜頭蛇尾(りゅうとうだび)に終りました。実は会場の隣りに女学生が四五人下宿していましてね、それがどうして聞いたものか、その日は朗読会があるという事を、どこかで探知して会場の窓下へ来て傍聴していたものと見えます。私(わたく)しが船頭の仮色(こわいろ)を使って、ようやく調子づいてこれなら大丈夫と思って得意にやっていると、\ldots{}\ldots{}つまり身振りがあまり過ぎたのでしょう、今まで耐(こ)らえていた女学生が一度にわっと笑いだしたものですから、驚ろいた事も驚ろいたし、極(きま)りが悪(わ)るい事も悪るいし、それで腰を折られてから、どうしても後(あと)がつづけられないので、とうとうそれ限(ぎ)りで散会しました」第一回としては成功だと称する朗読会がこれでは、失敗はどんなものだろうと想像すると笑わずにはいられない。覚えず咽喉仏(のどぼとけ)がごろごろ鳴る。主人はいよいよ柔かに頭を撫(な)でてくれる。人を笑って可愛がられるのはありがたいが、いささか無気味なところもある。「それは飛んだ事で」と主人は正月早々弔詞(ちょうじ)を述べている。「第二回からは、もっと奮発して盛大にやるつもりなので、今日出ましたのも全くそのためで、実は先生にも一つ御入会の上御尽力を仰ぎたいので」「僕にはとても癪なんか起せませんよ」と消極的の主人はすぐに断わりかける。「いえ、癪などは起していただかんでもよろしいので、ここに賛助員の名簿が」と云いながら紫の風呂敷から大事そうに小菊版(こぎくばん)の帳面を出す。「これへどうか御署名の上御捺印(ごなついん)を願いたいので」と帳面を主人の膝(ひざ)の前へ開いたまま置く。見ると現今知名な文学博士、文学士連中の名が行儀よく勢揃(せいぞろい)をしている。「はあ賛成員にならん事もありませんが、どんな義務があるのですか」と牡蠣先生(かきせんせい)は掛念(けねん)の体(てい)に見える。「義務と申して別段是非願う事もないくらいで、ただ御名前だけを御記入下さって賛成の意さえ御表(おひょう)し被下(くださ)ればそれで結構です」「そんなら這入(はい)ります」と義務のかからぬ事を知るや否や主人は急に気軽になる。責任さえないと云う事が分っておれば謀叛(むほん)の連判状へでも名を書き入れますと云う顔付をする。加之(のみならず)こう知名の学者が名前を列(つら)ねている中に姓名だけでも入籍させるのは、今までこんな事に出合った事のない主人にとっては無上の光栄であるから返事の勢のあるのも無理はない。「ちょっと失敬」と主人は書斎へ印をとりに這入る。吾輩はぼたりと畳の上へ落ちる。東風子は菓子皿の中の\emph{カステラ}をつまんで一口に頬張(ほおば)る。モゴモゴしばらくは苦しそうである。吾輩は今朝の雑煮(ぞうに)事件をちょっと思い出す。主人が書斎から印形(いんぎょう)を持って出て来た時は、東風子の胃の中にカステラが落ちついた時であった。主人は菓子皿のカステラが一切(ひときれ)足りなくなった事には気が着かぬらしい。もし気がつくとすれば第一に疑われるものは吾輩であろう。
東風子が帰ってから、主人が書斎に入って机の上を見ると、いつの間(ま)にか迷亭先生の手紙が来ている。
\blockquote{「新年の御慶(ぎょけい)目出度(めでたく)申納候(もうしおさめそろ)。\ldots{}\ldots{}」}
いつになく出が真面目だと主人が思う。迷亭先生の手紙に真面目なのはほとんどないので、この間などは「其後(そのご)別に恋着(れんちゃく)せる婦人も無之(これなく)、いず方(かた)より艶書(えんしょ)も参らず、先(ま)ず先(ま)ず無事に消光罷(まか)り在り候(そろ)間、乍憚(はばかりながら)御休心可被下候(くださるべくそろ)」と云うのが来たくらいである。それに較(くら)べるとこの年始状は例外にも世間的である。
\blockquote{「一寸参堂仕り度(たく)候えども、大兄の消極主義に反して、出来得る限り積極的方針を以(もっ)て、此千古未曾有(みぞう)の新年を迎うる計画故、毎日毎日目の廻る程の多忙、御推察願上候(そろ)\ldots{}\ldots{}」}
なるほどあの男の事だから正月は遊び廻るのに忙がしいに違いないと、主人は腹の中で迷亭君に同意する。
\blockquote{「昨日は一刻のひまを偸(ぬす)み、東風子に\emph{トチメンボー}の御馳走(ごちそう)を致さんと存じ候処(そろところ)、生憎(あいにく)材料払底の為(た)め其意を果さず、遺憾(いかん)千万に存候(ぞんじそろ)。\ldots{}\ldots{}」}
そろそろ例の通りになって来たと主人は無言で微笑する。
\blockquote{「明日は某男爵の歌留多会(かるたかい)、明後日は審美学協会の新年宴会、其明日は鳥部教授歓迎会、其又明日は\ldots{}\ldots{}」}
うるさいなと、主人は読みとばす。
\blockquote{「右の如く謡曲会、俳句会、短歌会、新体詩会等、会の連発にて当分の間は、のべつ幕無しに出勤致し候(そろ)為め、不得已(やむをえず)賀状を以て拝趨(はいすう)の礼に易(か)え候段(そろだん)不悪(あしからず)御宥恕(ごゆうじょ)被下度候(くだされたくそろ)。\ldots{}\ldots{}」}
別段くるにも及ばんさと、主人は手紙に返事をする。
\blockquote{「今度御光来の節は久し振りにて晩餐でも供し度(たき)心得に御座候(そろ)。寒厨(かんちゅう)何の珍味も無之候(これなくそうら)えども、せめては\emph{トチメンボー}でもと只今より心掛居候(おりそろ)。\ldots{}\ldots{}」}
まだ\emph{トチメンボー}を振り廻している。失敬なと主人はちょっとむっとする。
\blockquote{「然(しか)し\emph{トチメンボー}は近頃材料払底の為め、ことに依ると間に合い兼候(かねそろ)も計りがたきにつき、其節は孔雀(くじゃく)の舌(した)でも御風味に入れ可申候(もうすべくそろ)。\ldots{}\ldots{}」}
両天秤(りょうてんびん)をかけたなと主人は、あとが読みたくなる。
\blockquote{「御承知の通り孔雀一羽につき、舌肉の分量は小指の半(なか)ばにも足らぬ程故健啖(けんたん)なる大兄の胃嚢(いぶくろ)を充(み)たす為には\ldots{}\ldots{}」}
うそをつけと主人は打ち遣(や)ったようにいう。
\blockquote{「是非共二三十羽の孔雀を捕獲致さざる可(べか)らずと存候(ぞんじそろ)。然る所孔雀は動物園、浅草花屋敷等には、ちらほら見受け候えども、普通の鳥屋抔(など)には一向(いっこう)見当り不申(もうさず)、苦心(くしん)此事(このこと)に御座候(そろ)。\ldots{}\ldots{}」}
独りで勝手に苦心しているのじゃないかと主人は毫(ごう)も感謝の意を表しない。
\blockquote{「此孔雀の舌の料理は往昔(おうせき)羅馬(ローマ)全盛の砌(みぎ)り、一時非常に流行致し候(そろ)ものにて、豪奢(ごうしゃ)風流の極度と平生よりひそかに食指(しょくし)を動かし居候(おりそろ)次第御諒察(ごりょうさつ)可被下候(くださるべくそろ)。\ldots{}\ldots{}」}
何が御諒察だ、馬鹿なと主人はすこぶる冷淡である。
\blockquote{「降(くだ)って十六七世紀の頃迄は全欧を通じて孔雀は宴席に欠くべからざる好味と相成居候(あいなりおりそろ)。レスター伯がエリザベス女皇(じょこう)をケニルウォースに招待致し候節(そろせつ)も慥(たし)か孔雀を使用致し候様(そろよう)記憶致候(いたしそろ)。有名なるレンブラントが画(えが)き候(そろ)饗宴の図にも孔雀が尾を広げたる儘(まま)卓上に横(よこた)わり居り候(そろ)\ldots{}\ldots{}」}
孔雀の料理史をかくくらいなら、そんなに多忙でもなさそうだと不平をこぼす。
\blockquote{「とにかく近頃の如く御馳走の食べ続けにては、さすがの小生も遠からぬうちに大兄の如く胃弱と相成(あいな)るは必定(ひつじょう)\ldots{}\ldots{}」}
大兄のごとくは余計だ。何も僕を胃弱の標準にしなくても済むと主人はつぶやいた。
\blockquote{「歴史家の説によれば羅馬人(ローマじん)は日に二度三度も宴会を開き候由(そろよし)。日に二度も三度も方丈(ほうじょう)の食饌(しょくせん)に就き候えば如何なる健胃の人にても消化機能に不調を醸(かも)すべく、従って自然は大兄の如く\ldots{}\ldots{}」}
また大兄のごとくか、失敬な。
\blockquote{「然(しか)るに贅沢(ぜいたく)と衛生とを両立せしめんと研究を尽したる彼等は不相当に多量の滋味を貪(むさぼ)ると同時に胃腸を常態に保持するの必要を認め、ここに一の秘法を案出致し候(そろ)\ldots{}\ldots{}」}
はてねと主人は急に熱心になる。
\blockquote{「彼等は食後必ず入浴致候(いたしそろ)。入浴後一種の方法によりて浴前(よくぜん)に嚥下(えんか)せるものを悉(ことごと)く嘔吐(おうと)し、胃内を掃除致し候(そろ)。胃内廓清(いないかくせい)の功を奏したる後(のち)又食卓に就(つ)き、飽(あ)く迄珍味を風好(ふうこう)し、風好し了(おわ)れば又湯に入りて之(これ)を吐出(としゅつ)致候(いたしそろ)。かくの如くすれば好物は貪(むさ)ぼり次第貪り候(そうろう)も毫(ごう)も内臓の諸機関に障害を生ぜず、一挙両得とは此等の事を可申(もうすべき)かと愚考致候(いたしそろ)\ldots{}\ldots{}」}
なるほど一挙両得に相違ない。主人は羨(うらや)ましそうな顔をする。
\blockquote{「廿世紀の今日(こんにち)交通の頻繁(ひんぱん)、宴会の増加は申す迄もなく、軍国多事征露の第二年とも相成候折柄(そろおりから)、吾人戦勝国の国民は、是非共羅馬(ローマ)人に傚(なら)って此入浴嘔吐の術を研究せざるべからざる機会に到着致し候(そろ)事と自信致候(いたしそろ)。左(さ)もなくば切角(せっかく)の大国民も近き将来に於て悉(ことごと)く大兄の如く胃病患者と相成る事と窃(ひそ)かに心痛罷(まか)りあり候(そろ)\ldots{}\ldots{}」}
また大兄のごとくか、癪(しゃく)に障(さわ)る男だと主人が思う。
\blockquote{「此際吾人西洋の事情に通ずる者が古史伝説を考究し、既に廃絶せる秘法を発見し、之を明治の社会に応用致し候わば所謂(いわば)禍(わざわい)を未萌(みほう)に防ぐの功徳(くどく)にも相成り平素逸楽(いつらく)を擅(ほしいまま)に致し候(そろ)御恩返も相立ち可申(もうすべく)と存候(ぞんじそろ)\ldots{}\ldots{}」}
何だか妙だなと首を捻(ひね)る。
「依(よっ)て此間中(じゅう)よりギボン、モンセン、スミス等諸家の著述を渉猟(しょうりょう)致し居候(おりそうら)えども未(いま)だに発見の端緒(たんしょ)をも見出(みいだ)し得ざるは残念の至に存候(ぞんじそろ)。然し御存じの如く小生は一度思い立ち候事(そろこと)は成功するまでは決して中絶仕(つかまつ)らざる性質に候えば嘔吐方(おうとほう)を再興致し候(そろ)も遠からぬうちと信じ居り候(そろ)次第。右は発見次第御報道可仕候(つかまつるべくそろ)につき、左様御承知可被下候(くださるべくそろ)。就(つい)てはさきに申上候(そろ)\emph{トチメンボー}及び孔雀の舌の御馳走も可相成(あいなるべく)は右発見後に致し度(たく)、左(さ)すれば小生の都合は勿論(もちろん)、既に胃弱に悩み居らるる大兄の為にも御便宜(ごべんぎ)かと存候(ぞんじそろ)草々不備」
何だとうとう担(かつ)がれたのか、あまり書き方が真面目だものだからつい仕舞(しまい)まで本気にして読んでいた。新年匆々(そうそう)こんな悪戯(いたずら)をやる迷亭はよっぽどひま人だなあと主人は笑いながら云った。
それから四五日は別段の事もなく過ぎ去った。白磁(はくじ)の水仙がだんだん凋(しぼ)んで、青軸(あおじく)の梅が瓶(びん)ながらだんだん開きかかるのを眺め暮らしてばかりいてもつまらんと思って、一両度(いちりょうど)三毛子を訪問して見たが逢(あ)われない。最初は留守だと思ったが、二返目(へんめ)には病気で寝ているという事が知れた。障子の中で例の御師匠さんと下女が話しをしているのを手水鉢(ちょうずばち)の葉蘭の影に隠れて聞いているとこうであった。
「三毛は御飯をたべるかい」「いいえ今朝からまだ何(なん)にも食べません、あったかにして御火燵(おこた)に寝かしておきました」何だか猫らしくない。まるで人間の取扱を受けている。
一方では自分の境遇と比べて見て羨(うらや)ましくもあるが、一方では己(おの)が愛している猫がかくまで厚遇を受けていると思えば嬉しくもある。
「どうも困るね、御飯をたべないと、身体(からだ)が疲れるばかりだからね」「そうでございますとも、私共でさえ一日御\begin{comment}\includegraphics{../../../gaiji/2-92/2-92-71.png}\end{comment}(ごぜん)をいただかないと、明くる日はとても働けませんもの」
下女は自分より猫の方が上等な動物であるような返事をする。実際この家(うち)では下女より猫の方が大切かも知れない。
「御医者様へ連れて行ったのかい」「ええ、あの御医者はよっぽど妙でございますよ。私が三毛をだいて診察場へ行くと、風邪(かぜ)でも引いたのかって私の脈(みゃく)をとろうとするんでしょう。いえ病人は私ではございません。これですって三毛を膝の上へ直したら、にやにや笑いながら、猫の病気はわしにも分らん、抛(ほう)っておいたら今に癒(なお)るだろうってんですもの、あんまり苛(ひど)いじゃございませんか。腹が立ったから、それじゃ見ていただかなくってもようございますこれでも大事の猫なんですって、三毛を懐(ふところ)へ入れてさっさと帰って参りました」「ほんにねえ」
「ほんにねえ」は到底(とうてい)吾輩のうちなどで聞かれる言葉ではない。やはり天璋院(てんしょういん)様の何とかの何とかでなくては使えない、はなはだ雅(が)であると感心した。
「何だかしくしく云うようだが\ldots{}\ldots{}」「ええきっと風邪を引いて咽喉(のど)が痛むんでございますよ。風邪を引くと、どなたでも御咳(おせき)が出ますからね\ldots{}\ldots{}」
天璋院様の何とかの何とかの下女だけに馬鹿叮嚀(ていねい)な言葉を使う。
「それに近頃は肺病とか云うものが出来てのう」「ほんとにこの頃のように肺病だのペストだのって新しい病気ばかり殖(ふ)えた日にゃ油断も隙もなりゃしませんのでございますよ」「旧幕時代に無い者に碌(ろく)な者はないから御前も気をつけないといかんよ」「そうでございましょうかねえ」
下女は大(おおい)に感動している。
「風邪(かぜ)を引くといってもあまり出あるきもしないようだったに\ldots{}\ldots{}」「いえね、あなた、それが近頃は悪い友達が出来ましてね」
下女は国事の秘密でも語る時のように大得意である。
「悪い友達?」「ええあの表通りの教師の所(とこ)にいる薄ぎたない雄猫(おねこ)でございますよ」「教師と云うのは、あの毎朝無作法な声を出す人かえ」「ええ顔を洗うたんびに鵝鳥(がちょう)が絞(し)め殺されるような声を出す人でござんす」
鵝鳥が絞め殺されるような声はうまい形容である。吾輩の主人は毎朝風呂場で含嗽(うがい)をやる時、楊枝(ようじ)で咽喉(のど)をつっ突いて妙な声を無遠慮に出す癖がある。機嫌の悪い時はやけにがあがあやる、機嫌の好い時は元気づいてなおがあがあやる。つまり機嫌のいい時も悪い時も休みなく勢よくがあがあやる。細君の話しではここへ引越す前まではこんな癖はなかったそうだが、ある時ふとやり出してから今日(きょう)まで一日もやめた事がないという。ちょっと厄介な癖であるが、なぜこんな事を根気よく続けているのか吾等猫などには到底(とうてい)想像もつかん。それもまず善いとして「薄ぎたない猫」とは随分酷評をやるものだとなお耳を立ててあとを聞く。
「あんな声を出して何の呪(まじな)いになるか知らん。御維新前(ごいっしんまえ)は中間(ちゅうげん)でも草履(ぞうり)取りでも相応の作法は心得たもので、屋敷町などで、あんな顔の洗い方をするものは一人もおらなかったよ」「そうでございましょうともねえ」
下女は無暗(むやみ)に感服しては、無暗に\emph{ねえ}を使用する。
「あんな主人を持っている猫だから、どうせ野良猫(のらねこ)さ、今度来たら少し叩(たた)いておやり」「叩いてやりますとも、三毛の病気になったのも全くあいつの御蔭に相違ございませんもの、きっと讐(かたき)をとってやります」
飛んだ冤罪(えんざい)を蒙(こうむ)ったものだ。こいつは滅多(めった)に近(ち)か寄(よ)れないと三毛子にはとうとう逢わずに帰った。
帰って見ると主人は書斎の中(うち)で何か沈吟(ちんぎん)の体(てい)で筆を執(と)っている。二絃琴(にげんきん)の御師匠さんの所(とこ)で聞いた評判を話したら、さぞ怒(おこ)るだろうが、知らぬが仏とやらで、うんうん云いながら神聖な詩人になりすましている。
ところへ当分多忙で行かれないと云って、わざわざ年始状をよこした迷亭君が飄然(ひょうぜん)とやって来る。「何か新体詩でも作っているのかね。面白いのが出来たら見せたまえ」と云う。「うん、ちょっとうまい文章だと思ったから今翻訳して見ようと思ってね」と主人は重たそうに口を開く。「文章? 誰(だ)れの文章だい」「誰れのか分らんよ」「無名氏か、無名氏の作にも随分善いのがあるからなかなか馬鹿に出来ない。全体どこにあったのか」と問う。「第二読本」と主人は落ちつきはらって答える。「第二読本? 第二読本がどうしたんだ」「僕の翻訳している名文と云うのは第二読本の中(うち)にあると云う事さ」「冗談(じょうだん)じゃない。孔雀の舌の讐(かたき)を際(きわ)どいところで討とうと云う寸法なんだろう」「僕は君のような法螺吹(ほらふ)きとは違うさ」と口髯(くちひげ)を捻(ひね)る。泰然たるものだ。「昔(むか)しある人が山陽に、先生近頃名文はござらぬかといったら、山陽が馬子(まご)の書いた借金の催促状を示して近来の名文はまずこれでしょうと云ったという話があるから、君の審美眼も存外たしかかも知れん。どれ読んで見給え、僕が批評してやるから」と迷亭先生は審美眼の本家(ほんけ)のような事を云う。主人は禅坊主が大燈国師(だいとうこくし)の遺誡(ゆいかい)を読むような声を出して読み始める。「巨人(きょじん)、引力(いんりょく)」「何だいその巨人引力と云うのは」「巨人引力と云う題さ」「妙な題だな、僕には意味がわからんね」「引力と云う名を持っている巨人というつもりさ」「少し無理な\emph{つもり}だが表題だからまず負けておくとしよう。それから早々(そうそう)本文を読むさ、君は声が善いからなかなか面白い」「雑(ま)ぜかえしてはいかんよ」と予(あらか)じめ念を押してまた読み始める。
\blockquote{ケートは窓から外面(そと)を眺(なが)める。小児(しょうに)が球(たま)を投げて遊んでいる。彼等は高く球を空中に擲(なげう)つ。球は上へ上へとのぼる。しばらくすると落ちて来る。彼等はまた球を高く擲つ。再び三度。擲つたびに球は落ちてくる。なぜ落ちるのか、なぜ上へ上へとのみのぼらぬかとケートが聞く。「巨人が地中に住む故に」と母が答える。「彼は巨人引力である。彼は強い。彼は万物を己(おの)れの方へと引く。彼は家屋を地上に引く。引かねば飛んでしまう。小児も飛んでしまう。葉が落ちるのを見たろう。あれは巨人引力が呼ぶのである。本を落す事があろう。巨人引力が来いというからである。球が空にあがる。巨人引力は呼ぶ。呼ぶと落ちてくる」}
「それぎりかい」「むむ、甘(うま)いじゃないか」「いやこれは恐れ入った。飛んだところで\emph{トチメンボー}の御返礼に預(あずか)った」「御返礼でもなんでもないさ、実際うまいから訳して見たのさ、君はそう思わんかね」と金縁の眼鏡の奥を見る。「どうも驚ろいたね。君にしてこの伎倆(ぎりょう)あらんとは、全く此度(こんど)という今度(こんど)は担(かつ)がれたよ、降参降参」と一人で承知して一人で喋舌(しゃべ)る。主人には一向(いっこう)通じない。「何も君を降参させる考えはないさ。ただ面白い文章だと思ったから訳して見たばかりさ」「いや実に面白い。そう来なくっちゃ本ものでない。凄(すご)いものだ。恐縮だ」「そんなに恐縮するには及ばん。僕も近頃は水彩画をやめたから、その代りに文章でもやろうと思ってね」「どうして遠近(えんきん)無差別(むさべつ)黒白(こくびゃく)平等(びょうどう)の水彩画の比じゃない。感服の至りだよ」「そうほめてくれると僕も乗り気になる」と主人はあくまでも疳違(かんちが)いをしている。
ところへ寒月(かんげつ)君が先日は失礼しましたと這入(はい)って来る。「いや失敬。今大変な名文を拝聴して\emph{トチメンボー}の亡魂を退治(たいじ)られたところで」と迷亭先生は訳のわからぬ事をほのめかす。「はあ、そうですか」とこれも訳の分らぬ挨拶をする。主人だけは左(さ)のみ浮かれた気色(けしき)もない。「先日は君の紹介で越智東風(おちとうふう)と云う人が来たよ」「ああ上(あが)りましたか、あの越智東風(おちこち)と云う男は至って正直な男ですが少し変っているところがあるので、あるいは御迷惑かと思いましたが、是非紹介してくれというものですから\ldots{}\ldots{}」「別に迷惑の事もないがね\ldots{}\ldots{}」「こちらへ上(あが)っても自分の姓名のことについて何か弁じて行きゃしませんか」「いいえ、そんな話もなかったようだ」「そうですか、どこへ行っても初対面の人には自分の名前の講釈(こうしゃく)をするのが癖でしてね」「どんな講釈をするんだい」と事あれかしと待ち構えた迷亭君は口を入れる。「あの東風(こち)と云うのを音(おん)で読まれると大変気にするので」「はてね」と迷亭先生は金唐皮(きんからかわ)の煙草入(たばこいれ)から煙草をつまみ出す。「私(わたく)しの名は越智東風(おちとうふう)ではありません、越智(おち)\emph{こち}ですと必ず断りますよ」「妙だね」と雲井(くもい)を腹の底まで呑(の)み込む。「それが全く文学熱から来たので、こちと読むと\emph{遠近}と云う成語(せいご)になる、のみならずその姓名が韻(いん)を踏んでいると云うのが得意なんです。それだから東風(こち)を音(おん)で読むと僕がせっかくの苦心を人が買ってくれないといって不平を云うのです」「こりゃなるほど変ってる」と迷亭先生は図に乗って腹の底から雲井を鼻の孔(あな)まで吐き返す。途中で煙が戸迷(とまど)いをして咽喉(のど)の出口へ引きかかる。先生は煙管(きせる)を握ってごほんごほんと咽(むせ)び返る。「先日来た時は朗読会で船頭になって女学生に笑われたといっていたよ」と主人は笑いながら云う。「うむそれそれ」と迷亭先生が煙管(きせる)で膝頭(ひざがしら)を叩(たた)く。吾輩は険呑(けんのん)になったから少し傍(そば)を離れる。「その朗読会さ。せんだって\emph{トチメンボー}を御馳走した時にね。その話しが出たよ。何でも第二回には知名の文士を招待して大会をやるつもりだから、先生にも是非御臨席を願いたいって。それから僕が今度も近松の世話物をやるつもりかいと聞くと、いえこの次はずっと新しい者を撰(えら)んで金色夜叉(こんじきやしゃ)にしましたと云うから、君にゃ何の役が当ってるかと聞いたら私は御宮(おみや)ですといったのさ。東風(とうふう)の御宮は面白かろう。僕は是非出席して喝采(かっさい)しようと思ってるよ」「面白いでしょう」と寒月君が妙な笑い方をする。「しかしあの男はどこまでも誠実で軽薄なところがないから好い。迷亭などとは大違いだ」と主人はアンドレア・デル・サルトと孔雀(くじゃく)の舌と\emph{トチメンボー}の復讐(かたき)を一度にとる。迷亭君は気にも留めない様子で「どうせ僕などは行徳(ぎょうとく)の俎(まないた)と云う格だからなあ」と笑う。「まずそんなところだろう」と主人が云う。実は行徳の俎と云う語を主人は解(かい)さないのであるが、さすが永年教師をして胡魔化(ごまか)しつけているものだから、こんな時には教場の経験を社交上にも応用するのである。「行徳の俎というのは何の事ですか」と寒月が真率(しんそつ)に聞く。主人は床の方を見て「あの水仙は暮に僕が風呂の帰りがけに買って来て挿(さ)したのだが、よく持つじゃないか」と行徳の俎を無理にねじ伏せる。「暮といえば、去年の暮に僕は実に不思議な経験をしたよ」と迷亭が煙管(きせる)を大神楽(だいかぐら)のごとく指の尖(さき)で廻わす。「どんな経験か、聞かし玉(たま)え」と主人は行徳の俎を遠く後(うしろ)に見捨てた気で、ほっと息をつく。迷亭先生の不思議な経験というのを聞くと左(さ)のごとくである。
「たしか暮の二十七日と記憶しているがね。例の東風(とうふう)から参堂の上是非文芸上の御高話を伺いたいから御在宿を願うと云う先(さ)き触(ぶ)れがあったので、朝から心待ちに待っていると先生なかなか来ないやね。昼飯を食ってストーブの前でバリー・ペーンの滑稽物(こっけいもの)を読んでいるところへ静岡の母から手紙が来たから見ると、年寄だけにいつまでも僕を小供のように思ってね。寒中は夜間外出をするなとか、冷水浴もいいがストーブを焚(た)いて室(へや)を煖(あたた)かにしてやらないと風邪(かぜ)を引くとかいろいろの注意があるのさ。なるほど親はありがたいものだ、他人ではとてもこうはいかないと、呑気(のんき)な僕もその時だけは大(おおい)に感動した。それにつけても、こんなにのらくらしていては勿体(もったい)ない。何か大著述でもして家名を揚げなくてはならん。母の生きているうちに天下をして明治の文壇に迷亭先生あるを知らしめたいと云う気になった。それからなお読んで行くと御前なんぞは実に仕合せ者だ。露西亜(ロシア)と戦争が始まって若い人達は大変な辛苦(しんく)をして御国(みくに)のために働らいているのに節季師走(せっきしわす)でもお正月のように気楽に遊んでいると書いてある。――僕はこれでも母の思ってるように遊んじゃいないやね――そのあとへ以(もっ)て来て、僕の小学校時代の朋友(ほうゆう)で今度の戦争に出て死んだり負傷したものの名前が列挙してあるのさ。その名前を一々読んだ時には何だか世の中が味気(あじき)なくなって人間もつまらないと云う気が起ったよ。一番仕舞(しまい)にね。私(わた)しも取る年に候えば初春(はつはる)の御雑煮(おぞうに)を祝い候も今度限りかと\ldots{}\ldots{}何だか心細い事が書いてあるんで、なおのこと気がくさくさしてしまって早く東風(とうふう)が来れば好いと思ったが、先生どうしても来ない。そのうちとうとう晩飯になったから、母へ返事でも書こうと思ってちょいと十二三行かいた。母の手紙は六尺以上もあるのだが僕にはとてもそんな芸は出来んから、いつでも十行内外で御免蒙(こうむ)る事に極(き)めてあるのさ。すると一日動かずにおったものだから、胃の具合が妙で苦しい。東風が来たら待たせておけと云う気になって、郵便を入れながら散歩に出掛けたと思い給え。いつになく富士見町の方へは足が向かないで土手(どて)三番町(さんばんちょう)の方へ我れ知らず出てしまった。ちょうどその晩は少し曇って、から風が御濠(おほり)の向(むこ)うから吹き付ける、非常に寒い。神楽坂(かぐらざか)の方から汽車がヒューと鳴って土手下を通り過ぎる。大変淋(さみ)しい感じがする。暮、戦死、老衰、無常迅速などと云う奴が頭の中をぐるぐる馳(か)け廻(めぐ)る。よく人が首を縊(くく)ると云うがこんな時にふと誘われて死ぬ気になるのじゃないかと思い出す。ちょいと首を上げて土手の上を見ると、いつの間(ま)にか例の松の真下(ました)に来ているのさ」
「例の松た、何だい」と主人が断句(だんく)を投げ入れる。
「首懸(くびかけ)の松さ」と迷亭は領(えり)を縮める。
「首懸の松は鴻(こう)の台(だい)でしょう」寒月が波紋(はもん)をひろげる。
「鴻(こう)の台(だい)のは鐘懸(かねかけ)の松で、土手三番町のは首懸(くびかけ)の松さ。なぜこう云う名が付いたかと云うと、昔(むか)しからの言い伝えで誰でもこの松の下へ来ると首が縊(くく)りたくなる。土手の上に松は何十本となくあるが、そら首縊(くびくく)りだと来て見ると必ずこの松へぶら下がっている。年に二三返(べん)はきっとぶら下がっている。どうしても他(ほか)の松では死ぬ気にならん。見ると、うまい具合に枝が往来の方へ横に出ている。ああ好い枝振りだ。あのままにしておくのは惜しいものだ。どうかしてあすこの所へ人間を下げて見たい、誰か来ないかしらと、四辺(あたり)を見渡すと生憎(あいにく)誰も来ない。仕方がない、自分で下がろうか知らん。いやいや自分が下がっては命がない、危(あぶ)ないからよそう。しかし昔の希臘人(ギリシャじん)は宴会の席で首縊(くびくく)りの真似をして余興を添えたと云う話しがある。一人が台の上へ登って縄の結び目へ首を入れる途端に他(ほか)のものが台を蹴返す。首を入れた当人は台を引かれると同時に縄をゆるめて飛び下りるという趣向(しゅこう)である。果してそれが事実なら別段恐るるにも及ばん、僕も一つ試みようと枝へ手を懸けて見ると好い具合に撓(しわ)る。撓り按排(あんばい)が実に美的である。首がかかってふわふわするところを想像して見ると嬉しくてたまらん。是非やる事にしようと思ったが、もし東風(とうふう)が来て待っていると気の毒だと考え出した。それではまず東風(とうふう)に逢(あ)って約束通り話しをして、それから出直そうと云う気になってついにうちへ帰ったのさ」
「それで市(いち)が栄えたのかい」と主人が聞く。
「面白いですな」と寒月がにやにやしながら云う。
「うちへ帰って見ると東風は来ていない。しかし今日(こんにち)は無拠処(よんどころなき)差支(さしつか)えがあって出られぬ、いずれ永日(えいじつ)御面晤(ごめんご)を期すという端書(はがき)があったので、やっと安心して、これなら心置きなく首が縊(くく)れる嬉しいと思った。で早速下駄を引き懸けて、急ぎ足で元の所へ引き返して見る\ldots{}\ldots{}」と云って主人と寒月の顔を見てすましている。
「見るとどうしたんだい」と主人は少し焦(じ)れる。
「いよいよ佳境に入りますね」と寒月は羽織の紐(ひも)をひねくる。
「見ると、もう誰か来て先へぶら下がっている。たった一足違いでねえ君、残念な事をしたよ。考えると何でもその時は死神(しにがみ)に取り着かれたんだね。ゼームスなどに云わせると副意識下の幽冥界(ゆうめいかい)と僕が存在している現実界が一種の因果法によって互に感応(かんのう)したんだろう。実に不思議な事があるものじゃないか」迷亭はすまし返っている。
主人はまたやられたと思いながら何も云わずに空也餅(くうやもち)を頬張(ほおば)って口をもごもご云わしている。
寒月は火鉢の灰を丁寧に掻(か)き馴(な)らして、俯向(うつむ)いてにやにや笑っていたが、やがて口を開く。極めて静かな調子である。
「なるほど伺って見ると不思議な事でちょっと有りそうにも思われませんが、私などは自分でやはり似たような経験をつい近頃したものですから、少しも疑がう気になりません」
「おや君も首を縊(くく)りたくなったのかい」
「いえ私のは首じゃないんで。これもちょうど明ければ昨年の暮の事でしかも先生と同日同刻くらいに起った出来事ですからなおさら不思議に思われます」
「こりゃ面白い」と迷亭も空也餅を頬張る。
「その日は向島の知人の家(うち)で忘年会兼(けん)合奏会がありまして、私もそれへヴァイオリンを携(たずさ)えて行きました。十五六人令嬢やら令夫人が集ってなかなか盛会で、近来の快事と思うくらいに万事が整っていました。晩餐(ばんさん)もすみ合奏もすんで四方(よも)の話しが出て時刻も大分(だいぶ)遅くなったから、もう暇乞(いとまご)いをして帰ろうかと思っていますと、某博士の夫人が私のそばへ来てあなたは○○子さんの御病気を御承知ですかと小声で聞きますので、実はその両三日前(りょうさんにちまえ)に逢った時は平常の通りどこも悪いようには見受けませんでしたから、私も驚ろいて精(くわ)しく様子を聞いて見ますと、私(わたく)しの逢ったその晩から急に発熱して、いろいろな譫語(うわごと)を絶間なく口走(くちばし)るそうで、それだけなら宜(い)いですがその譫語のうちに私の名が時々出て来るというのです」
主人は無論、迷亭先生も「御安(おやす)くないね」などという月並(つきなみ)は云わず、静粛に謹聴している。
「医者を呼んで見てもらうと、何だか病名はわからんが、何しろ熱が劇(はげ)しいので脳を犯しているから、もし睡眠剤(すいみんざい)が思うように功を奏しないと危険であると云う診断だそうで私はそれを聞くや否や一種いやな感じが起ったのです。ちょうど夢でうなされる時のような重くるしい感じで周囲の空気が急に固形体になって四方から吾が身をしめつけるごとく思われました。帰り道にもその事ばかりが頭の中にあって苦しくてたまらない。あの奇麗な、あの快活なあの健康な○○子さんが\ldots{}\ldots{}」
「ちょっと失敬だが待ってくれ給え。さっきから伺っていると○○子さんと云うのが二返(へん)ばかり聞えるようだが、もし差支(さしつか)えがなければ承(うけたま)わりたいね、君」と主人を顧(かえり)みると、主人も「うむ」と生返事(なまへんじ)をする。
「いやそれだけは当人の迷惑になるかも知れませんから廃(よ)しましょう」
「すべて曖々然(あいあいぜん)として昧々然(まいまいぜん)たるかたで行くつもりかね」
「冷笑なさってはいけません、極真面目(ごくまじめ)な話しなんですから\ldots{}\ldots{}とにかくあの婦人が急にそんな病気になった事を考えると、実に飛花落葉(ひからくよう)の感慨で胸が一杯になって、総身(そうしん)の活気が一度にストライキを起したように元気がにわかに滅入(めい)ってしまいまして、ただ蹌々(そうそう)として踉々(ろうろう)という形(かた)ちで吾妻橋(あずまばし)へきかかったのです。欄干に倚(よ)って下を見ると満潮(まんちょう)か干潮(かんちょう)か分りませんが、黒い水がかたまってただ動いているように見えます。花川戸(はなかわど)の方から人力車が一台馳(か)けて来て橋の上を通りました。その提灯(ちょうちん)の火を見送っていると、だんだん小くなって札幌(さっぽろ)ビールの処で消えました。私はまた水を見る。すると遥(はる)かの川上の方で私の名を呼ぶ声が聞えるのです。はてな今時分人に呼ばれる訳はないが誰だろうと水の面(おもて)をすかして見ましたが暗くて何(なん)にも分りません。気のせいに違いない早々(そうそう)帰ろうと思って一足二足あるき出すと、また微(かす)かな声で遠くから私の名を呼ぶのです。私はまた立ち留って耳を立てて聞きました。三度目に呼ばれた時には欄干に捕(つか)まっていながら膝頭(ひざがしら)ががくがく悸(ふる)え出したのです。その声は遠くの方か、川の底から出るようですが紛(まぎ)れもない○○子の声なんでしょう。私は覚えず「はーい」と返事をしたのです。その返事が大きかったものですから静かな水に響いて、自分で自分の声に驚かされて、はっと周囲を見渡しました。人も犬も月も何(なん)にも見えません。その時に私はこの「夜(よる)」の中に巻き込まれて、あの声の出る所へ行きたいと云う気がむらむらと起ったのです。○○子の声がまた苦しそうに、訴えるように、救を求めるように私の耳を刺し通したので、今度は「今直(すぐ)に行きます」と答えて欄干から半身を出して黒い水を眺めました。どうも私を呼ぶ声が浪(なみ)の下から無理に洩(も)れて来るように思われましてね。この水の下だなと思いながら私はとうとう欄干の上に乗りましたよ。今度呼んだら飛び込もうと決心して流を見つめているとまた憐れな声が糸のように浮いて来る。ここだと思って力を込めて一反(いったん)飛び上がっておいて、そして小石か何ぞのように未練なく落ちてしまいました」
「とうとう飛び込んだのかい」と主人が眼をぱちつかせて問う。
「そこまで行こうとは思わなかった」と迷亭が自分の鼻の頭をちょいとつまむ。
「飛び込んだ後(あと)は気が遠くなって、しばらくは夢中でした。やがて眼がさめて見ると寒くはあるが、どこも濡(ぬ)れた所(とこ)も何もない、水を飲んだような感じもしない。たしかに飛び込んだはずだが実に不思議だ。こりゃ変だと気が付いてそこいらを見渡すと驚きましたね。水の中へ飛び込んだつもりでいたところが、つい間違って橋の真中へ飛び下りたので、その時は実に残念でした。前と後(うし)ろの間違だけであの声の出る所へ行く事が出来なかったのです」寒月はにやにや笑いながら例のごとく羽織の紐(ひも)を荷厄介(にやっかい)にしている。
「ハハハハこれは面白い。僕の経験と善く似ているところが奇だ。やはりゼームス教授の材料になるね。人間の感応と云う題で写生文にしたらきっと文壇を驚かすよ。\ldots{}\ldots{}そしてその○○子さんの病気はどうなったかね」と迷亭先生が追窮する。
「二三日前(にさんちまえ)年始に行きましたら、門の内で下女と羽根を突いていましたから病気は全快したものと見えます」
主人は最前から沈思の体(てい)であったが、この時ようやく口を開いて、「僕にもある」と負けぬ気を出す。
「あるって、何があるんだい」迷亭の眼中に主人などは無論ない。
「僕のも去年の暮の事だ」
「みんな去年の暮は暗合(あんごう)で妙ですな」と寒月が笑う。欠けた前歯のうちに空也餅(くうやもち)が着いている。
「やはり同日同刻じゃないか」と迷亭がまぜ返す。
「いや日は違うようだ。何でも二十日(はつか)頃だよ。細君が御歳暮の代りに摂津大掾(せっつだいじょう)を聞かしてくれろと云うから、連れて行ってやらん事もないが今日の語り物は何だと聞いたら、細君が新聞を参考して鰻谷(うなぎだに)だと云うのさ。鰻谷は嫌いだから今日はよそうとその日はやめにした。翌日になると細君がまた新聞を持って来て今日は堀川(ほりかわ)だからいいでしょうと云う。堀川は三味線もので賑やかなばかりで実(み)がないからよそうと云うと、細君は不平な顔をして引き下がった。その翌日になると細君が云うには今日は三十三間堂です、私は是非摂津(せっつ)の三十三間堂が聞きたい。あなたは三十三間堂も御嫌いか知らないが、私に聞かせるのだからいっしょに行って下すっても宜(い)いでしょうと手詰(てづめ)の談判をする。御前がそんなに行きたいなら行っても宜(よ)ろしい、しかし一世一代と云うので大変な大入だから到底(とうてい)突懸(つっか)けに行ったって這入(はい)れる気遣(きづか)いはない。元来ああ云う場所へ行くには茶屋と云うものが在(あ)ってそれと交渉して相当の席を予約するのが正当の手続きだから、それを踏まないで常規を脱した事をするのはよくない、残念だが今日はやめようと云うと、細君は凄(すご)い眼付をして、私は女ですからそんなむずかしい手続きなんか知りませんが、大原のお母あさんも、鈴木の君代さんも正当の手続きを踏まないで立派に聞いて来たんですから、いくらあなたが教師だからって、そう手数(てすう)のかかる見物をしないでもすみましょう、あなたはあんまりだと泣くような声を出す。それじゃ駄目でもまあ行く事にしよう。晩飯をくって電車で行こうと降参をすると、行くなら四時までに向うへ着くようにしなくっちゃいけません、そんなぐずぐずしてはいられませんと急に勢がいい。なぜ四時までに行かなくては駄目なんだと聞き返すと、そのくらい早く行って場所をとらなくちゃ這入れないからですと鈴木の君代さんから教えられた通りを述べる。それじゃ四時を過ぎればもう駄目なんだねと念を押して見たら、ええ駄目ですともと答える。すると君不思議な事にはその時から急に悪寒(おかん)がし出してね」
「奥さんがですか」と寒月が聞く。
「なに細君はぴんぴんしていらあね。僕がさ。何だか穴の明いた風船玉のように一度に萎縮(いしゅく)する感じが起ると思うと、もう眼がぐらぐらして動けなくなった」
「急病だね」と迷亭が註釈を加える。
「ああ困った事になった。細君が年に一度の願だから是非叶(かな)えてやりたい。平生(いつも)叱りつけたり、口を聞かなかったり、身上(しんしょう)の苦労をさせたり、小供の世話をさせたりするばかりで何一つ洒掃薪水(さいそうしんすい)の労に酬(むく)いた事はない。今日は幸い時間もある、嚢中(のうちゅう)には四五枚の堵物(とぶつ)もある。連れて行けば行かれる。細君も行きたいだろう、僕も連れて行ってやりたい。是非連れて行ってやりたいがこう悪寒がして眼がくらんでは電車へ乗るどころか、靴脱(くつぬぎ)へ降りる事も出来ない。ああ気の毒だ気の毒だと思うとなお悪寒がしてなお眼がくらんでくる。早く医者に見てもらって服薬でもしたら四時前には全快するだろうと、それから細君と相談をして甘木(あまき)医学士を迎いにやると生憎(あいにく)昨夜(ゆうべ)が当番でまだ大学から帰らない。二時頃には御帰りになりますから、帰り次第すぐ上げますと云う返事である。困ったなあ、今杏仁水(きょうにんすい)でも飲めば四時前にはきっと癒(なお)るに極(きま)っているんだが、運の悪い時には何事も思うように行かんもので、たまさか妻君の喜ぶ笑顔を見て楽もうと云う予算も、がらりと外(はず)れそうになって来る。細君は恨(うら)めしい顔付をして、到底(とうてい)いらっしゃれませんかと聞く。行くよ必ず行くよ。四時までにはきっと直って見せるから安心しているがいい。早く顔でも洗って着物でも着換えて待っているがいい、と口では云ったようなものの胸中は無限の感慨である。悪寒はますます劇(はげ)しくなる、眼はいよいよぐらぐらする。もしや四時までに全快して約束を履行(りこう)する事が出来なかったら、気の狭い女の事だから何をするかも知れない。情(なさ)けない仕儀になって来た。どうしたら善かろう。万一の事を考えると今の内に有為転変(ういてんぺん)の理、生者必滅(しょうじゃひつめつ)の道を説き聞かして、もしもの変が起った時取り乱さないくらいの覚悟をさせるのも、夫(おっと)の妻(つま)に対する義務ではあるまいかと考え出した。僕は速(すみや)かに細君を書斎へ呼んだよ。呼んで御前は女だけれども many a slip 'twixt the cup and the lip と云う西洋の諺(ことわざ)くらいは心得ているだろうと聞くと、そんな横文字なんか誰が知るもんですか、あなたは人が英語を知らないのを御存じの癖にわざと英語を使って人にからかうのだから、宜(よろ)しゅうございます、どうせ英語なんかは出来ないんですから、そんなに英語が御好きなら、なぜ耶蘇学校(ヤソがっこう)の卒業生かなんかをお貰いなさらなかったんです。あなたくらい冷酷な人はありはしないと非常な権幕(けんまく)なんで、僕もせっかくの計画の腰を折られてしまった。君等にも弁解するが僕の英語は決して悪意で使った訳じゃない。全く妻(さい)を愛する至情から出たので、それを妻のように解釈されては僕も立つ瀬がない。それにさっきからの悪寒(おかん)と眩暈(めまい)で少し脳が乱れていたところへもって来て、早く有為転変、生者必滅の理を呑み込ませようと少し急(せ)き込んだものだから、つい細君の英語を知らないと云う事を忘れて、何の気も付かずに使ってしまった訳さ。考えるとこれは僕が悪(わ)るい、全く手落ちであった。この失敗で悪寒はますます強くなる。眼はいよいよぐらぐらする。妻君は命ぜられた通り風呂場へ行って両肌(もろはだ)を脱いで御化粧をして、箪笥(たんす)から着物を出して着換える。もういつでも出掛けられますと云う風情(ふぜい)で待ち構えている。僕は気が気でない。早く甘木君が来てくれれば善いがと思って時計を見るともう三時だ。四時にはもう一時間しかない。「そろそろ出掛けましょうか」と妻君が書斎の開き戸を明けて顔を出す。自分の妻(さい)を褒(ほ)めるのはおかしいようであるが、僕はこの時ほど細君を美しいと思った事はなかった。もろ肌を脱いで石鹸で磨(みが)き上げた皮膚がぴかついて黒縮緬(くろちりめん)の羽織と反映している。その顔が石鹸と摂津大掾(せっつだいじょう)を聞こうと云う希望との二つで、有形無形の両方面から輝やいて見える。どうしてもその希望を満足させて出掛けてやろうと云う気になる。それじゃ奮発して行こうかな、と一ぷくふかしているとようやく甘木先生が来た。うまい注文通りに行った。が容体をはなすと、甘木先生は僕の舌を眺(なが)めて、手を握って、胸を敲(たた)いて背を撫(な)でて、目縁(まぶち)を引っ繰り返して、頭蓋骨(ずがいこつ)をさすって、しばらく考え込んでいる。「どうも少し険呑(けんのん)のような気がしまして」と僕が云うと、先生は落ちついて、「いえ格別の事もございますまい」と云う。「あのちょっとくらい外出致しても差支(さしつか)えはございますまいね」と細君が聞く。「さよう」と先生はまた考え込む。「御気分さえ御悪くなければ\ldots{}\ldots{}」「気分は悪いですよ」と僕がいう。「じゃともかくも頓服(とんぷく)と水薬(すいやく)を上げますから」「へえどうか、何だかちと、危(あぶ)ないようになりそうですな」「いや決して御心配になるほどの事じゃございません、神経を御起しになるといけませんよ」と先生が帰る。三時は三十分過ぎた。下女を薬取りにやる。細君の厳命で馳(か)け出して行って、馳(か)け出して返ってくる。四時十五分前である。四時にはまだ十五分ある。すると四時十五分前頃から、今まで何とも無かったのに、急に嘔気(はきけ)を催(もよ)おして来た。細君は水薬(すいやく)を茶碗へ注(つ)いで僕の前へ置いてくれたから、茶碗を取り上げて飲もうとすると、胃の中からげーと云う者が吶喊(とっかん)して出てくる。やむをえず茶碗を下へ置く。細君は「早く御飲(おの)みになったら宜(い)いでしょう」と逼(せま)る。早く飲んで早く出掛けなくては義理が悪い。思い切って飲んでしまおうとまた茶碗を唇へつけるとまたゲーが執念深(しゅうねんぶか)く妨害をする。飲もうとしては茶碗を置き、飲もうとしては茶碗を置いていると茶の間の柱時計がチンチンチンチンと四時を打った。さあ四時だ愚図愚図してはおられんと茶碗をまた取り上げると、不思議だねえ君、実に不思議とはこの事だろう、四時の音と共に吐(は)き気(け)がすっかり留まって水薬が何の苦なしに飲めたよ。それから四時十分頃になると、甘木先生の名医という事も始めて理解する事が出来たんだが、背中がぞくぞくするのも、眼がぐらぐらするのも夢のように消えて、当分立つ事も出来まいと思った病気がたちまち全快したのは嬉しかった」
「それから歌舞伎座へいっしょに行ったのかい」と迷亭が要領を得んと云う顔付をして聞く。
「行きたかったが四時を過ぎちゃ、這入(はい)れないと云う細君の意見なんだから仕方がない、やめにしたさ。もう十五分ばかり早く甘木先生が来てくれたら僕の義理も立つし、妻(さい)も満足したろうに、わずか十五分の差でね、実に残念な事をした。考え出すとあぶないところだったと今でも思うのさ」
語り了(おわ)った主人はようやく自分の義務をすましたような風をする。これで両人に対して顔が立つと云う気かも知れん。
寒月は例のごとく欠けた歯を出して笑いながら「それは残念でしたな」と云う。
迷亭はとぼけた顔をして「君のような親切な夫(おっと)を持った妻君は実に仕合せだな」と独(ひと)り言(ごと)のようにいう。障子の蔭でエヘンと云う細君の咳払(せきばら)いが聞える。
吾輩はおとなしく三人の話しを順番に聞いていたがおかしくも悲しくもなかった。人間というものは時間を潰(つぶ)すために強(し)いて口を運動させて、おかしくもない事を笑ったり、面白くもない事を嬉しがったりするほかに能もない者だと思った。吾輩の主人の我儘(わがまま)で偏狭(へんきょう)な事は前から承知していたが、平常(ふだん)は言葉数を使わないので何だか了解しかねる点があるように思われていた。その了解しかねる点に少しは恐しいと云う感じもあったが、今の話を聞いてから急に軽蔑(けいべつ)したくなった。かれはなぜ両人の話しを沈黙して聞いていられないのだろう。負けぬ気になって愚(ぐ)にもつかぬ駄弁を弄(ろう)すれば何の所得があるだろう。エピクテタスにそんな事をしろと書いてあるのか知らん。要するに主人も寒月も迷亭も太平(たいへい)の逸民(いつみん)で、彼等は糸瓜(へちま)のごとく風に吹かれて超然と澄(すま)し切っているようなものの、その実はやはり娑婆気(しゃばけ)もあり慾気(よくけ)もある。競争の念、勝とう勝とうの心は彼等が日常の談笑中にもちらちらとほのめいて、一歩進めば彼等が平常罵倒(ばとう)している俗骨共(ぞっこつども)と一つ穴の動物になるのは猫より見て気の毒の至りである。ただその言語動作が普通の半可通(はんかつう)のごとく、文切(もんき)り形(がた)の厭味を帯びてないのはいささかの取(と)り得(え)でもあろう。
こう考えると急に三人の談話が面白くなくなったので、三毛子の様子でも見て来(き)ようかと二絃琴(にげんきん)の御師匠さんの庭口へ廻る。門松(かどまつ)注目飾(しめかざ)りはすでに取り払われて正月も早(は)や十日となったが、うららかな春日(はるび)は一流れの雲も見えぬ深き空より四海天下を一度に照らして、十坪に足らぬ庭の面(おも)も元日の曙光(しょこう)を受けた時より鮮(あざや)かな活気を呈している。椽側に座蒲団(ざぶとん)が一つあって人影も見えず、障子も立て切ってあるのは御師匠さんは湯にでも行ったのか知らん。御師匠さんは留守でも構わんが、三毛子は少しは宜(い)い方か、それが気掛りである。ひっそりして人の気合(けわい)もしないから、泥足のまま椽側(えんがわ)へ上(あが)って座蒲団の真中へ寝転(ねこ)ろんで見るといい心持ちだ。ついうとうととして、三毛子の事も忘れてうたた寝をしていると、急に障子のうちで人声がする。
「御苦労だった。出来たかえ」御師匠さんはやはり留守ではなかったのだ。
「はい遅くなりまして、仏師屋(ぶっしや)へ参りましたらちょうど出来上ったところだと申しまして」「どれお見せなさい。ああ奇麗に出来た、これで三毛も浮かばれましょう。金(きん)は剥(は)げる事はあるまいね」「ええ念を押しましたら上等を使ったからこれなら人間の位牌(いはい)よりも持つと申しておりました。\ldots{}\ldots{}それから猫誉信女(みょうよしんにょ)の誉の字は崩(くず)した方が恰好(かっこう)がいいから少し劃(かく)を易(か)えたと申しました」「どれどれ早速御仏壇へ上げて御線香でもあげましょう」
三毛子は、どうかしたのかな、何だか様子が変だと蒲団の上へ立ち上る。チーン南無猫誉信女(なむみょうよしんにょ)、南無阿弥陀仏(なむあみだぶつ)南無阿弥陀仏と御師匠さんの声がする。
「御前も回向(えこう)をしておやりなさい」
チーン南無猫誉信女南無阿弥陀仏南無阿弥陀仏と今度は下女の声がする。吾輩は急に動悸(どうき)がして来た。座蒲団の上に立ったまま、木彫(きぼり)の猫のように眼も動かさない。
「ほんとに残念な事を致しましたね。始めはちょいと風邪(かぜ)を引いたんでございましょうがねえ」「甘木さんが薬でも下さると、よかったかも知れないよ」「一体あの甘木さんが悪うございますよ、あんまり三毛を馬鹿にし過ぎまさあね」「そう人様(ひとさま)の事を悪く云うものではない。これも寿命(じゅみょう)だから」
三毛子も甘木先生に診察して貰ったものと見える。
「つまるところ表通りの教師のうちの野良猫(のらねこ)が無暗(むやみ)に誘い出したからだと、わたしは思うよ」「ええあの畜生(ちきしょう)が三毛のかたきでございますよ」
少し弁解したかったが、ここが我慢のしどころと唾(つば)を呑んで聞いている。話しはしばし途切(とぎ)れる。
「世の中は自由にならん者でのう。三毛のような器量よしは早死(はやじに)をするし。不器量な野良猫は達者でいたずらをしているし\ldots{}\ldots{}」「その通りでございますよ。三毛のような可愛らしい猫は鐘と太鼓で探してあるいたって、二人(ふたり)とはおりませんからね」
二匹と云う代りに二(ふ)たりといった。下女の考えでは猫と人間とは同種族ものと思っているらしい。そう云えばこの下女の顔は吾等猫属(ねこぞく)とはなはだ類似している。
「出来るものなら三毛の代りに\ldots{}\ldots{}」「あの教師の所の野良(のら)が死ぬと御誂(おあつら)え通りに参ったんでございますがねえ」
御誂え通りになっては、ちと困る。死ぬと云う事はどんなものか、まだ経験した事がないから好きとも嫌いとも云えないが、先日あまり寒いので火消壺(ひけしつぼ)の中へもぐり込んでいたら、下女が吾輩がいるのも知らんで上から蓋(ふた)をした事があった。その時の苦しさは考えても恐しくなるほどであった。白君の説明によるとあの苦しみが今少し続くと死ぬのであるそうだ。三毛子の身代(みがわ)りになるのなら苦情もないが、あの苦しみを受けなくては死ぬ事が出来ないのなら、誰のためでも死にたくはない。
「しかし猫でも坊さんの御経を読んでもらったり、戒名(かいみょう)をこしらえてもらったのだから心残りはあるまい」「そうでございますとも、全く果報者(かほうもの)でございますよ。ただ慾を云うとあの坊さんの御経があまり軽少だったようでございますね」「少し短か過ぎたようだったから、大変御早うございますねと御尋ねをしたら、月桂寺(げっけいじ)さんは、ええ利目(ききめ)のあるところをちょいとやっておきました、なに猫だからあのくらいで充分浄土へ行かれますとおっしゃったよ」「あらまあ\ldots{}\ldots{}しかしあの野良なんかは\ldots{}\ldots{}」
吾輩は名前はないとしばしば断っておくのに、この下女は野良野良と吾輩を呼ぶ。失敬な奴だ。
「罪が深いんですから、いくらありがたい御経だって浮かばれる事はございませんよ」
吾輩はその後(ご)野良が何百遍繰り返されたかを知らぬ。吾輩はこの際限なき談話を中途で聞き棄てて、布団(ふとん)をすべり落ちて椽側から飛び下りた時、八万八千八百八十本の毛髪を一度にたてて身震(みぶる)いをした。その後(ご)二絃琴(にげんきん)の御師匠さんの近所へは寄りついた事がない。今頃は御師匠さん自身が月桂寺さんから軽少な御回向(ごえこう)を受けているだろう。
近頃は外出する勇気もない。何だか世間が慵(もの)うく感ぜらるる。主人に劣らぬほどの無性猫(ぶしょうねこ)となった。主人が書斎にのみ閉じ籠(こも)っているのを人が失恋だ失恋だと評するのも無理はないと思うようになった。
鼠(ねずみ)はまだ取った事がないので、一時は御三(おさん)から放逐論(ほうちくろん)さえ呈出(ていしゅつ)された事もあったが、主人は吾輩の普通一般の猫でないと云う事を知っているものだから吾輩はやはりのらくらしてこの家(や)に起臥(きが)している。この点については深く主人の恩を感謝すると同時にその活眼(かつがん)に対して敬服の意を表するに躊躇(ちゅうちょ)しないつもりである。御三が吾輩を知らずして虐待をするのは別に腹も立たない。今に左甚五郎(ひだりじんごろう)が出て来て、吾輩の肖像を楼門(ろうもん)の柱に刻(きざ)み、日本のスタンランが好んで吾輩の似顔をカンヴァスの上に描(えが)くようになったら、彼等鈍瞎漢(どんかつかん)は始めて自己の不明を恥(は)ずるであろう。
\chapter*{三}
三毛子は死ぬ。黒は相手にならず、いささか寂寞(せきばく)の感はあるが、幸い人間に知己(ちき)が出来たのでさほど退屈とも思わぬ。せんだっては主人の許(もと)へ吾輩の写真を送ってくれと手紙で依頼した男がある。この間は岡山の名産吉備団子(きびだんご)をわざわざ吾輩の名宛で届けてくれた人がある。だんだん人間から同情を寄せらるるに従って、己(おのれ)が猫である事はようやく忘却してくる。猫よりはいつの間(ま)にか人間の方へ接近して来たような心持になって、同族を糾合(きゅうごう)して二本足の先生と雌雄(しゆう)を決しようなどと云(い)う量見は昨今のところ毛頭(もうとう)ない。それのみか折々は吾輩もまた人間世界の一人だと思う折さえあるくらいに進化したのはたのもしい。あえて同族を軽蔑(けいべつ)する次第ではない。ただ性情の近きところに向って一身の安きを置くは勢(いきおい)のしからしむるところで、これを変心とか、軽薄とか、裏切りとか評せられてはちと迷惑する。かような言語を弄(ろう)して人を罵詈(ばり)するものに限って融通の利(き)かぬ貧乏性の男が多いようだ。こう猫の習癖を脱化して見ると\emph{三毛子}や\emph{黒}の事ばかり荷厄介にしている訳には行かん。やはり人間同等の気位(きぐらい)で彼等の思想、言行を評隲(ひょうしつ)したくなる。これも無理はあるまい。ただそのくらいな見識を有している吾輩をやはり一般猫児(びょうじ)の毛の生(は)えたものくらいに思って、主人が吾輩に一言(いちごん)の挨拶もなく、吉備団子(きびだんご)をわが物顔に喰い尽したのは残念の次第である。写真もまだ撮(と)って送らぬ容子(ようす)だ。これも不平と云えば不平だが、主人は主人、吾輩は吾輩で、相互の見解が自然異(こと)なるのは致し方もあるまい。吾輩はどこまでも人間になりすましているのだから、交際をせぬ猫の動作は、どうしてもちょいと筆に上(のぼ)りにくい。迷亭、寒月諸先生の評判だけで御免蒙(こうむ)る事に致そう。
今日は上天気の日曜なので、主人はのそのそ書斎から出て来て、吾輩の傍(そば)へ筆硯(ふですずり)と原稿用紙を並べて腹這(はらばい)になって、しきりに何か唸(うな)っている。大方草稿を書き卸(おろ)す序開(じょびら)きとして妙な声を発するのだろうと注目していると、ややしばらくして筆太(ふでぶと)に「香一\begin{comment}\includegraphics{../../../gaiji/1-87/1-87-40.png}\end{comment}(こういっしゅ)」とかいた。はてな詩になるか、俳句になるか、香一\begin{comment}\includegraphics{../../../gaiji/1-87/1-87-40.png}\end{comment}とは、主人にしては少し洒落(しゃれ)過ぎているがと思う間もなく、彼は香一\begin{comment}\includegraphics{../../../gaiji/1-87/1-87-40.png}\end{comment}を書き放しにして、新たに行(ぎょう)を改めて「さっきから天然居士(てんねんこじ)の事をかこうと考えている」と筆を走らせた。筆はそれだけではたと留ったぎり動かない。主人は筆を持って首を捻(ひね)ったが別段名案もないものと見えて筆の穂を甞(な)めだした。唇が真黒になったと見ていると、今度はその下へちょいと丸をかいた。丸の中へ点を二つうって眼をつける。真中へ小鼻の開いた鼻をかいて、真一文字に口を横へ引張った、これでは文章でも俳句でもない。主人も自分で愛想(あいそ)が尽きたと見えて、そこそこに顔を塗り消してしまった。主人はまた行(ぎょう)を改める。彼の考によると行さえ改めれば詩か賛か語か録か何(なん)かになるだろうとただ宛(あて)もなく考えているらしい。やがて「天然居士は空間を研究し、論語を読み、焼芋(やきいも)を食い、鼻汁(はな)を垂らす人である」と言文一致体で一気呵成(いっきかせい)に書き流した、何となくごたごたした文章である。それから主人はこれを遠慮なく朗読して、いつになく「ハハハハ面白い」と笑ったが「鼻汁(はな)を垂らすのは、ちと酷(こく)だから消そう」とその句だけへ棒を引く。一本ですむところを二本引き三本引き、奇麗な併行線(へいこうせん)を描(か)く、線がほかの行(ぎょう)まで食(は)み出しても構わず引いている。線が八本並んでもあとの句が出来ないと見えて、今度は筆を捨てて髭(ひげ)を捻(ひね)って見る。文章を髭から捻り出して御覧に入れますと云う見幕(けんまく)で猛烈に捻ってはねじ上げ、ねじ下ろしているところへ、茶の間から妻君(さいくん)が出て来てぴたりと主人の鼻の先へ坐(す)わる。「あなたちょっと」と呼ぶ。「なんだ」と主人は水中で銅鑼(どら)を叩(たた)くような声を出す。返事が気に入らないと見えて妻君はまた「あなたちょっと」と出直す。「なんだよ」と今度は鼻の穴へ親指と人さし指を入れて鼻毛をぐっと抜く。「今月はちっと足りませんが\ldots{}\ldots{}」「足りんはずはない、医者へも薬礼はすましたし、本屋へも先月払ったじゃないか。今月は余らなければならん」とすまして抜き取った鼻毛を天下の奇観のごとく眺(なが)めている。「それでもあなたが御飯を召し上らんで麺麭(パン)を御食(おた)べになったり、ジャムを御舐(おな)めになるものですから」「元来ジャムは幾缶(いくかん)舐めたのかい」「今月は八つ入(い)りましたよ」「八つ? そんなに舐めた覚えはない」「あなたばかりじゃありません、子供も舐めます」「いくら舐めたって五六円くらいなものだ」と主人は平気な顔で鼻毛を一本一本丁寧に原稿紙の上へ植付ける。肉が付いているのでぴんと針を立てたごとくに立つ。主人は思わぬ発見をして感じ入った体(てい)で、ふっと吹いて見る。粘着力(ねんちゃくりょく)が強いので決して飛ばない。「いやに頑固(がんこ)だな」と主人は一生懸命に吹く。「ジャムばかりじゃないんです、ほかに買わなけりゃ、ならない物もあります」と妻君は大(おおい)に不平な気色(けしき)を両頬に漲(みなぎ)らす。「あるかも知れないさ」と主人はまた指を突っ込んでぐいと鼻毛を抜く。赤いのや、黒いのや、種々の色が交(まじ)る中に一本真白なのがある。大に驚いた様子で穴の開(あ)くほど眺めていた主人は指の股へ挟んだまま、その鼻毛を妻君の顔の前へ出す。「あら、いやだ」と妻君は顔をしかめて、主人の手を突き戻す。「ちょっと見ろ、鼻毛の白髪(しらが)だ」と主人は大に感動した様子である。さすがの妻君も笑いながら茶の間へ這入(はい)る。経済問題は断念したらしい。主人はまた天然居士(てんねんこじ)に取り懸(かか)る。
鼻毛で妻君を追払った主人は、まずこれで安心と云わぬばかりに鼻毛を抜いては原稿をかこうと焦(あせ)る体(てい)であるがなかなか筆は動かない。「\emph{焼芋を食う}も蛇足(だそく)だ、割愛(かつあい)しよう」とついにこの句も抹殺(まっさつ)する。「\emph{香一\begin{comment}\includegraphics{../../../gaiji/1-87/1-87-40.png}\end{comment}}もあまり唐突(とうとつ)だから已(や)めろ」と惜気もなく筆誅(ひっちゅう)する。余す所は「天然居士は空間を研究し論語を読む人である」と云う一句になってしまった。主人はこれでは何だか簡単過ぎるようだなと考えていたが、ええ面倒臭い、文章は御廃(おはい)しにして、銘だけにしろと、筆を十文字に揮(ふる)って原稿紙の上へ下手な文人画の蘭を勢よくかく。せっかくの苦心も一字残らず落第となった。それから裏を返して「空間に生れ、空間を究(きわ)め、空間に死す。空たり間たり天然居士(てんねんこじ)噫(ああ)」と意味不明な語を連(つら)ねているところへ例のごとく迷亭が這入(はい)って来る。迷亭は人の家(うち)も自分の家も同じものと心得ているのか案内も乞わず、ずかずか上ってくる、のみならず時には勝手口から飄然(ひょうぜん)と舞い込む事もある、心配、遠慮、気兼(きがね)、苦労、を生れる時どこかへ振り落した男である。
「また\emph{巨人引力}かね」と立ったまま主人に聞く。「そう、いつでも\emph{巨人引力}ばかり書いてはおらんさ。\emph{天然居士}の墓銘を撰(せん)しているところなんだ」と大袈裟(おおげさ)な事を云う。「\emph{天然居士}と云うなあやはり\emph{偶然童子}のような戒名かね」と迷亭は不相変(あいかわらず)出鱈目(でたらめ)を云う。「\emph{偶然童子}と云うのもあるのかい」「なに有りゃしないがまずその見当(けんとう)だろうと思っていらあね」「\emph{偶然童子}と云うのは僕の知ったものじゃないようだが\emph{天然居士}と云うのは、君の知ってる男だぜ」「一体だれが\emph{天然居士}なんて名を付けてすましているんだい」「例の曾呂崎(そろさき)の事だ。卒業して大学院へ這入って\emph{空間論}と云う題目で研究していたが、あまり勉強し過ぎて腹膜炎で死んでしまった。曾呂崎はあれでも僕の親友なんだからな」「親友でもいいさ、決して悪いと云やしない。しかしその曾呂崎を天然居士に変化させたのは一体誰の所作(しょさ)だい」「僕さ、僕がつけてやったんだ。元来坊主のつける戒名ほど俗なものは無いからな」と天然居士はよほど雅(が)な名のように自慢する。迷亭は笑いながら「まあその墓碑銘(ぼひめい)と云う奴を見せ給え」と原稿を取り上げて「何だ\ldots{}\ldots{}空間に生れ、空間を究(きわ)め、空間に死す。空たり間たり天然居士噫(ああ)」と大きな声で読み上(あげ)る。「なるほどこりゃあ善(い)い、天然居士相当のところだ」主人は嬉しそうに「善いだろう」と云う。「この墓銘(ぼめい)を沢庵石(たくあんいし)へ彫(ほ)り付けて本堂の裏手へ力石(ちからいし)のように抛(ほう)り出して置くんだね。雅(が)でいいや、天然居士も浮かばれる訳だ」「僕もそうしようと思っているのさ」と主人は至極(しごく)真面目に答えたが「僕あちょっと失敬するよ、じき帰るから猫にでもからかっていてくれ給え」と迷亭の返事も待たず風然(ふうぜん)と出て行く。
計らずも迷亭先生の接待掛りを命ぜられて無愛想(ぶあいそ)な顔もしていられないから、ニャーニャーと愛嬌(あいきょう)を振り蒔(ま)いて膝(ひざ)の上へ這(は)い上(あが)って見た。すると迷亭は「イヨー大分(だいぶ)肥(ふと)ったな、どれ」と無作法(ぶさほう)にも吾輩の襟髪(えりがみ)を攫(つか)んで宙へ釣るす。「あと足をこうぶら下げては、鼠(ねずみ)は取れそうもない、\ldots{}\ldots{}どうです奥さんこの猫は鼠を捕りますかね」と吾輩ばかりでは不足だと見えて、隣りの室(へや)の妻君に話しかける。「鼠どころじゃございません。御雑煮(おぞうに)を食べて踊りをおどるんですもの」と妻君は飛んだところで旧悪を暴(あば)く。吾輩は宙乗(ちゅうの)りをしながらも少々極りが悪かった。迷亭はまだ吾輩を卸(おろ)してくれない。「なるほど踊りでもおどりそうな顔だ。奥さんこの猫は油断のならない相好(そうごう)ですぜ。昔(むか)しの草双紙(くさぞうし)にある猫又(ねこまた)に似ていますよ」と勝手な事を言いながら、しきりに細君(さいくん)に話しかける。細君は迷惑そうに針仕事の手をやめて座敷へ出てくる。
「どうも御退屈様、もう帰りましょう」と茶を注(つ)ぎ易(か)えて迷亭の前へ出す。「どこへ行ったんですかね」「どこへ参るにも断わって行った事の無い男ですから分りかねますが、大方御医者へでも行ったんでしょう」「甘木さんですか、甘木さんもあんな病人に捕(つら)まっちゃ災難ですな」「へえ」と細君は挨拶のしようもないと見えて簡単な答えをする。迷亭は一向(いっこう)頓着しない。「近頃はどうです、少しは胃の加減が能(い)いんですか」「能(い)いか悪いか頓(とん)と分りません、いくら甘木さんにかかったって、あんなにジャムばかり甞(な)めては胃病の直る訳がないと思います」と細君は先刻(せんこく)の不平を暗(あん)に迷亭に洩(も)らす。「そんなにジャムを甞めるんですかまるで小供のようですね」「ジャムばかりじゃないんで、この頃は胃病の薬だとか云って大根卸(だいこおろ)しを無暗(むやみ)に甞めますので\ldots{}\ldots{}」「驚ろいたな」と迷亭は感嘆する。「何でも大根卸(だいこおろし)の中にはジヤスターゼが有るとか云う話しを新聞で読んでからです」「なるほどそれでジャムの損害を償(つぐな)おうと云う趣向ですな。なかなか考えていらあハハハハ」と迷亭は細君の訴(うったえ)を聞いて大(おおい)に愉快な気色(けしき)である。「この間などは赤ん坊にまで甞めさせまして\ldots{}\ldots{}」「ジャムをですか」「いいえ大根卸(だいこおろし)を\ldots{}\ldots{}あなた。坊や御父様がうまいものをやるからおいでてって、――たまに小供を可愛がってくれるかと思うとそんな馬鹿な事ばかりするんです。二三日前(にさんちまえ)には中の娘を抱いて箪笥(たんす)の上へあげましてね\ldots{}\ldots{}」「どう云う趣向がありました」と迷亭は何を聞いても趣向ずくめに解釈する。「なに趣向も何も有りゃしません、ただその上から飛び下りて見ろと云うんですわ、三つや四つの女の子ですもの、そんな御転婆(おてんば)な事が出来るはずがないです」「なるほどこりゃ趣向が無さ過ぎましたね。しかしあれで腹の中は毒のない善人ですよ」「あの上腹の中に毒があっちゃ、辛防(しんぼう)は出来ませんわ」と細君は大(おおい)に気焔(きえん)を揚げる。「まあそんなに不平を云わんでも善いでさあ。こうやって不足なくその日その日が暮らして行かれれば上(じょう)の分(ぶん)ですよ。苦沙弥君(くしゃみくん)などは道楽はせず、服装にも構わず、地味に世帯向(しょたいむ)きに出来上った人でさあ」と迷亭は柄(がら)にない説教を陽気な調子でやっている。「ところがあなた大違いで\ldots{}\ldots{}」「何か内々でやりますかね。油断のならない世の中だからね」と飄然(ひょうぜん)とふわふわした返事をする。「ほかの道楽はないですが、無暗(むやみ)に読みもしない本ばかり買いましてね。それも善い加減に見計(みはか)らって買ってくれると善いんですけれど、勝手に丸善へ行っちゃ何冊でも取って来て、月末になると知らん顔をしているんですもの、去年の暮なんか、月々のが溜(たま)って大変困りました」「なあに書物なんか取って来るだけ取って来て構わんですよ。払いをとりに来たら今にやる今にやると云っていりゃ帰ってしまいまさあ」「それでも、そういつまでも引張る訳にも参りませんから」と妻君は憮然(ぶぜん)としている。「それじゃ、訳を話して書籍費(しょじゃくひ)を削減させるさ」「どうして、そんな言(こと)を云ったって、なかなか聞くものですか、この間などは貴様は学者の妻(さい)にも似合わん、毫(ごう)も書籍(しょじゃく)の価値を解しておらん、昔(むか)し羅馬(ローマ)にこう云う話しがある。後学のため聞いておけと云うんです」「そりゃ面白い、どんな話しですか」迷亭は乗気になる。細君に同情を表しているというよりむしろ好奇心に駆(か)られている。「何んでも昔し羅馬(ローマ)に樽金(たるきん)とか云う王様があって\ldots{}\ldots{}」「樽金(たるきん)? 樽金はちと妙ですぜ」「私は唐人(とうじん)の名なんかむずかしくて覚えられませんわ。何でも七代目なんだそうです」「なるほど七代目樽金は妙ですな。ふんその七代目樽金がどうかしましたかい」「あら、あなたまで冷かしては立つ瀬がありませんわ。知っていらっしゃるなら教えて下さればいいじゃありませんか、人の悪い」と、細君は迷亭へ食って掛る。「何冷かすなんて、そんな人の悪い事をする僕じゃない。ただ七代目樽金は振(ふる)ってると思ってね\ldots{}\ldots{}ええお待ちなさいよ羅馬(ローマ)の七代目の王様ですね、こうっとたしかには覚えていないがタークイン・ゼ・プラウドの事でしょう。まあ誰でもいい、その王様がどうしました」「その王様の所へ一人の女が本を九冊持って来て買ってくれないかと云ったんだそうです」「なるほど」「王様がいくらなら売るといって聞いたら大変な高い事を云うんですって、あまり高いもんだから少し負けないかと云うとその女がいきなり九冊の内の三冊を火にくべて焚(や)いてしまったそうです」「惜しい事をしましたな」「その本の内には予言か何かほかで見られない事が書いてあるんですって」「へえー」「王様は九冊が六冊になったから少しは価(ね)も減ったろうと思って六冊でいくらだと聞くと、やはり元の通り一文も引かないそうです、それは乱暴だと云うと、その女はまた三冊をとって火にくべたそうです。王様はまだ未練があったと見えて、余った三冊をいくらで売ると聞くと、やはり九冊分のねだんをくれと云うそうです。九冊が六冊になり、六冊が三冊になっても代価は、元の通り一厘も引かない、それを引かせようとすると、残ってる三冊も火にくべるかも知れないので、王様はとうとう高い御金を出して焚(や)け余(あま)りの三冊を買ったんですって\ldots{}\ldots{}どうだこの話しで少しは書物のありがた味(み)が分ったろう、どうだと力味(りき)むのですけれど、私にゃ何がありがたいんだか、まあ分りませんね」と細君は一家の見識を立てて迷亭の返答を促(うな)がす。さすがの迷亭も少々窮したと見えて、袂(たもと)からハンケチを出して吾輩をじゃらしていたが「しかし奥さん」と急に何か考えついたように大きな声を出す。「あんなに本を買って矢鱈(やたら)に詰め込むものだから人から少しは学者だとか何とか云われるんですよ。この間ある文学雑誌を見たら苦沙弥君(くしゃみくん)の評が出ていましたよ」「ほんとに?」と細君は向き直る。主人の評判が気にかかるのは、やはり夫婦と見える。「何とかいてあったんです」「なあに二三行ばかりですがね。苦沙弥君の文は行雲流水(こううんりゅうすい)のごとしとありましたよ」細君は少しにこにこして「それぎりですか」「その次にね――出ずるかと思えば忽(たちま)ち消え、逝(ゆ)いては長(とこしな)えに帰るを忘るとありましたよ」細君は妙な顔をして「賞(ほ)めたんでしょうか」と心元ない調子である。「まあ賞めた方でしょうな」と迷亭は済ましてハンケチを吾輩の眼の前にぶら下げる。「書物は商買道具で仕方もござんすまいが、よっぽど偏屈(へんくつ)でしてねえ」迷亭はまた別途の方面から来たなと思って「偏屈は少々偏屈ですね、学問をするものはどうせあんなですよ」と調子を合わせるような弁護をするような不即不離の妙答をする。「せんだってなどは学校から帰ってすぐわきへ出るのに着物を着換えるのが面倒だものですから、あなた外套(がいとう)も脱がないで、机へ腰を掛けて御飯を食べるのです。御膳(おぜん)を火燵櫓(こたつやぐら)の上へ乗せまして――私は御櫃(おはち)を抱(かか)えて坐っておりましたがおかしくって\ldots{}\ldots{}」「何だかハイカラの首実検のようですな。しかしそんなところが苦沙弥君の苦沙弥君たるところで――とにかく月並(つきなみ)でない」と切(せつ)ない褒(ほ)め方をする。「月並か月並でないか女には分りませんが、なんぼ何でも、あまり乱暴ですわ」「しかし月並より好いですよ」と無暗に加勢すると細君は不満な様子で「一体、月並月並と皆さんが、よくおっしゃいますが、どんなのが月並なんです」と開き直って月並の定義を質問する、「月並ですか、月並と云うと――さようちと説明しにくいのですが\ldots{}\ldots{}」「そんな曖昧(あいまい)なものなら月並だって好さそうなものじゃありませんか」と細君は女人(にょにん)一流の論理法で詰め寄せる。「曖昧じゃありませんよ、ちゃんと分っています、ただ説明しにくいだけの事でさあ」「何でも自分の嫌いな事を月並と云うんでしょう」と細君は我(われ)知らず穿(うが)った事を云う。迷亭もこうなると何とか月並の処置を付けなければならぬ仕儀となる。「奥さん、月並と云うのはね、まず\emph{年は二八か二九からぬ}と\emph{言わず語らず物思い}の間(あいだ)に寝転んでいて、\emph{この日や天気晴朗}とくると必ず\emph{一瓢を携えて墨堤に遊ぶ}連中(れんじゅう)を云うんです」「そんな連中があるでしょうか」と細君は分らんものだから好(いい)加減な挨拶をする。「何だかごたごたして私には分りませんわ」とついに我(が)を折る。「それじゃ馬琴(ばきん)の胴へメジョオ・ペンデニスの首をつけて一二年欧州の空気で包んでおくんですね」「そうすると月並が出来るでしょうか」迷亭は返事をしないで笑っている。「何そんな手数(てすう)のかかる事をしないでも出来ます。中学校の生徒に白木屋の番頭を加えて二で割ると立派な月並が出来上ります」「そうでしょうか」と細君は首を捻(ひね)ったまま納得(なっとく)し兼ねたと云う風情(ふぜい)に見える。
「君まだいるのか」と主人はいつの間(ま)にやら帰って来て迷亭の傍(そば)へ坐(す)わる。「まだいるのかはちと酷(こく)だな、すぐ帰るから待ってい給えと言ったじゃないか」「万事あれなんですもの」と細君は迷亭を顧(かえり)みる。「今君の留守中に君の逸話を残らず聞いてしまったぜ」「女はとかく多弁でいかん、人間もこの猫くらい沈黙を守るといいがな」と主人は吾輩の頭を撫(な)でてくれる。「君は赤ん坊に大根卸(だいこおろ)しを甞(な)めさしたそうだな」「ふむ」と主人は笑ったが「赤ん坊でも近頃の赤ん坊はなかなか利口だぜ。それ以来、坊や辛(から)いのはどこと聞くときっと舌を出すから妙だ」「まるで犬に芸を仕込む気でいるから残酷だ。時に寒月(かんげつ)はもう来そうなものだな」「寒月が来るのかい」と主人は不審な顔をする。「来るんだ。午後一時までに苦沙弥(くしゃみ)の家(うち)へ来いと端書(はがき)を出しておいたから」「人の都合も聞かんで勝手な事をする男だ。寒月を呼んで何をするんだい」「なあに今日のはこっちの趣向じゃない寒月先生自身の要求さ。先生何でも理学協会で演説をするとか云うのでね。その稽古をやるから僕に聴いてくれと云うから、そりゃちょうどいい苦沙弥にも聞かしてやろうと云うのでね。そこで君の家(うち)へ呼ぶ事にしておいたのさ――なあに君はひま人だからちょうどいいやね――差支(さしつか)えなんぞある男じゃない、聞くがいいさ」と迷亭は独(ひと)りで呑み込んでいる。「物理学の演説なんか僕にゃ分らん」と主人は少々迷亭の専断(せんだん)を憤(いきどお)ったもののごとくに云う。「ところがその問題がマグネ付けられたノッズルについてなどと云う乾燥無味なものじゃないんだ。\emph{首縊りの力学}と云う脱俗超凡(だつぞくちょうぼん)な演題なのだから傾聴する価値があるさ」「君は首を縊(くく)り損(そ)くなった男だから傾聴するが好いが僕なんざあ\ldots{}\ldots{}」「歌舞伎座で悪寒(おかん)がするくらいの人間だから聞かれないと云う結論は出そうもないぜ」と例のごとく軽口を叩く。妻君はホホと笑って主人を顧(かえり)みながら次の間へ退く。主人は無言のまま吾輩の頭を撫(な)でる。この時のみは非常に丁寧な撫で方であった。
それから約七分くらいすると注文通り寒月君が来る。今日は晩に演舌(えんぜつ)をするというので例になく立派なフロックを着て、洗濯し立ての白襟(カラー)を聳(そび)やかして、男振りを二割方上げて、「少し後(おく)れまして」と落ちつき払って、挨拶をする。「さっきから二人で大待ちに待ったところなんだ。早速願おう、なあ君」と主人を見る。主人もやむを得ず「うむ」と生返事(なまへんじ)をする。寒月君はいそがない。「コップへ水を一杯頂戴しましょう」と云う。「いよー本式にやるのか次には拍手の請求とおいでなさるだろう」と迷亭は独りで騒ぎ立てる。寒月君は内隠(うちがく)しから草稿を取り出して徐(おもむ)ろに「稽古ですから、御遠慮なく御批評を願います」と前置をして、いよいよ演舌の御浚(おさら)いを始める。
「罪人を絞罪(こうざい)の刑に処すると云う事は重(おも)にアングロサクソン民族間に行われた方法でありまして、それより古代に溯(さかのぼ)って考えますと首縊(くびくく)りは重に自殺の方法として行われた者であります。猶太人中(ユダヤじんちゅう)に在(あ)っては罪人を石を抛(な)げつけて殺す習慣であったそうでございます。旧約全書を研究して見ますといわゆるハンギングなる語は罪人の死体を釣るして野獣または肉食鳥の餌食(えじき)とする意義と認められます。ヘロドタスの説に従って見ますと猶太人(ユダヤじん)はエジプトを去る以前から夜中(やちゅう)死骸を曝(さら)されることを痛く忌(い)み嫌ったように思われます。エジプト人は罪人の首を斬って胴だけを十字架に釘付(くぎづ)けにして夜中曝し物にしたそうで御座います。波斯人(ペルシャじん)は\ldots{}\ldots{}」「寒月君首縊りと縁がだんだん遠くなるようだが大丈夫かい」と迷亭が口を入れる。「これから本論に這入(はい)るところですから、少々御辛防(ごしんぼう)を願います。\ldots{}\ldots{}さて波斯人はどうかと申しますとこれもやはり処刑には磔(はりつけ)を用いたようでございます。但し生きているうちに張付(はりつ)けに致したものか、死んでから釘を打ったものかその辺(へん)はちと分りかねます\ldots{}\ldots{}」「そんな事は分らんでもいいさ」と主人は退屈そうに欠伸(あくび)をする。「まだいろいろ御話し致したい事もございますが、御迷惑であらっしゃいましょうから\ldots{}\ldots{}」「あらっしゃいましょうより、いらっしゃいましょうの方が聞きいいよ、ねえ苦沙弥君(くしゃみくん)」とまた迷亭が咎(とが)め立(だて)をすると主人は「どっちでも同じ事だ」と気のない返事をする。「さていよいよ本題に入りまして弁じます」「\emph{弁じます}なんか講釈師の云い草だ。演舌家はもっと上品な詞(ことば)を使って貰いたいね」と迷亭先生また交(ま)ぜ返す。「\emph{弁じます}が下品なら何と云ったらいいでしょう」と寒月君は少々むっとした調子で問いかける。「迷亭のは聴いているのか、交(ま)ぜ返しているのか判然しない。寒月君そんな弥次馬(やじうま)に構わず、さっさとやるが好い」と主人はなるべく早く難関を切り抜けようとする。「むっとして弁じましたる柳かな、かね」と迷亭はあいかわらず飄然(ひょうぜん)たる事を云う。寒月は思わず吹き出す。「真に処刑として絞殺を用いましたのは、私の調べました結果によりますると、オディセーの二十二巻目に出ております。即(すなわ)ち彼(か)のテレマカスがペネロピーの十二人の侍女を絞殺するという条(くだ)りでございます。希臘語(ギリシャご)で本文を朗読しても宜(よろ)しゅうございますが、ちと衒(てら)うような気味にもなりますからやめに致します。四百六十五行から、四百七十三行を御覧になると分ります」「希臘語云々(うんぬん)はよした方がいい、さも希臘語が出来ますと云わんばかりだ、ねえ苦沙弥君」「それは僕も賛成だ、そんな物欲しそうな事は言わん方が奥床(おくゆか)しくて好い」と主人はいつになく直ちに迷亭に加担する。両人(りょうにん)は毫(ごう)も希臘語が読めないのである。「それではこの両三句は今晩抜く事に致しまして次を弁じ――ええ申し上げます。
この絞殺を今から想像して見ますと、これを執行するに二つの方法があります。第一は、彼(か)のテレマカスがユーミアス及びフ\begin{comment}\includegraphics{../../../gaiji/1-06/1-06-84.png}\end{comment}リーシャスの援(たすけ)を藉(か)りて縄の一端を柱へ括(くく)りつけます。そしてその縄の所々へ結び目を穴に開けてこの穴へ女の頭を一つずつ入れておいて、片方の端(はじ)をぐいと引張って釣し上げたものと見るのです」「つまり西洋洗濯屋のシャツのように女がぶら下ったと見れば好いんだろう」「その通りで、それから第二は縄の一端を前のごとく柱へ括(くく)り付けて他の一端も始めから天井へ高く釣るのです。そしてその高い縄から何本か別の縄を下げて、それに結び目の輪になったのを付けて女の頸(くび)を入れておいて、いざと云う時に女の足台を取りはずすと云う趣向なのです」「たとえて云うと縄暖簾(なわのれん)の先へ提灯玉(ちょうちんだま)を釣したような景色(けしき)と思えば間違はあるまい」「提灯玉と云う玉は見た事がないから何とも申されませんが、もしあるとすればその辺(へん)のところかと思います。――それでこれから力学的に第一の場合は到底成立すべきものでないと云う事を証拠立てて御覧に入れます」「面白いな」と迷亭が云うと「うん面白い」と主人も一致する。
「まず女が同距離に釣られると仮定します。また一番地面に近い二人の女の首と首を繋(つな)いでいる縄はホリゾンタルと仮定します。そこでα\textsubscript{1}α\textsubscript{2}\ldots{}\ldots{}α\textsubscript{6}を縄が地平線と形づくる角度とし、T\textsubscript{1}T\textsubscript{2}\ldots{}\ldots{}T\textsubscript{6}を縄の各部が受ける力と見做(みな)し、T\textsubscript{7}=Xは縄のもっとも低い部分の受ける力とします。Wは勿論(もちろん)女の体重と御承知下さい。どうです御分りになりましたか」
迷亭と主人は顔を見合せて「大抵分った」と云う。但しこの大抵と云う度合は両人(りょうにん)が勝手に作ったのだから他人の場合には応用が出来ないかも知れない。「さて多角形に関する御存じの平均性理論によりますと、下(しも)のごとく十二の方程式が立ちます。T\textsubscript{1}cosα\textsubscript{1}=T\textsubscript{2}cosα\textsubscript{2}\ldots{}\ldots{} (1) T\textsubscript{2}cosα\textsubscript{2}=T\textsubscript{3}cosα\textsubscript{3}\ldots{}\ldots{} (2) \ldots{}\ldots{}」「方程式はそのくらいで沢山だろう」と主人は乱暴な事を云う。「実はこの式が演説の首脳なんですが」と寒月君ははなはだ残り惜し気に見える。「それじゃ首脳だけは逐(お)って伺う事にしようじゃないか」と迷亭も少々恐縮の体(てい)に見受けられる。「この式を略してしまうとせっかくの力学的研究がまるで駄目になるのですが\ldots{}\ldots{}」「何そんな遠慮はいらんから、ずんずん略すさ\ldots{}\ldots{}」と主人は平気で云う。「それでは仰せに従って、無理ですが略しましょう」「それがよかろう」と迷亭が妙なところで手をぱちぱちと叩く。
「それから英国へ移って論じますと、ベオウルフの中に絞首架(こうしゅか)即(すなわ)ちガルガと申す字が見えますから絞罪の刑はこの時代から行われたものに違ないと思われます。ブラクストーンの説によるともし絞罪に処せられる罪人が、万一縄の具合で死に切れぬ時は再度(ふたたび)同様の刑罰を受くべきものだとしてありますが、妙な事にはピヤース・プローマンの中には仮令(たとい)兇漢でも二度絞(し)める法はないと云う句があるのです。まあどっちが本当か知りませんが、悪くすると一度で死ねない事が往々実例にあるので。千七百八十六年に有名なフ\begin{comment}\includegraphics{../../../gaiji/1-06/1-06-84.png}\end{comment}ツ・ゼラルドと云う悪漢を絞めた事がありました。ところが妙なはずみで一度目には台から飛び降りるときに縄が切れてしまったのです。またやり直すと今度は縄が長過ぎて足が地面へ着いたのでやはり死ねなかったのです。とうとう三返目に見物人が手伝って往生(おうじょう)さしたと云う話しです」「やれやれ」と迷亭はこんなところへくると急に元気が出る。「本当に死に損(ぞこな)いだな」と主人まで浮かれ出す。「まだ面白い事があります首を縊(くく)ると背(せい)が一寸(いっすん)ばかり延びるそうです。これはたしかに医者が計って見たのだから間違はありません」「それは新工夫だね、どうだい苦沙弥(くしゃみ)などはちと釣って貰っちゃあ、一寸延びたら人間並になるかも知れないぜ」と迷亭が主人の方を向くと、主人は案外真面目で「寒月君、一寸くらい背(せい)が延びて生き返る事があるだろうか」と聞く。「それは駄目に極(きま)っています。釣られて脊髄(せきずい)が延びるからなんで、早く云うと背が延びると云うより壊(こわ)れるんですからね」「それじゃ、まあ止(や)めよう」と主人は断念する。
演説の続きは、まだなかなか長くあって寒月君は首縊りの生理作用にまで論及するはずでいたが、迷亭が無暗に風来坊(ふうらいぼう)のような珍語を挟(はさ)むのと、主人が時々遠慮なく欠伸(あくび)をするので、ついに中途でやめて帰ってしまった。その晩は寒月君がいかなる態度で、いかなる雄弁を振(ふる)ったか遠方で起った出来事の事だから吾輩には知れよう訳がない。
二三日(にさんち)は事もなく過ぎたが、或る日の午後二時頃また迷亭先生は例のごとく空々(くうくう)として偶然童子のごとく舞い込んで来た。座に着くと、いきなり「君、越智東風(おちとうふう)の高輪事件(たかなわじけん)を聞いたかい」と旅順陥落の号外を知らせに来たほどの勢を示す。「知らん、近頃は合(あ)わんから」と主人は平生(いつも)の通り陰気である。「きょうはその東風子(とうふうし)の失策物語を御報道に及ぼうと思って忙しいところをわざわざ来たんだよ」「またそんな仰山(ぎょうさん)な事を云う、君は全体不埒(ふらち)な男だ」「ハハハハハ不埒と云わんよりむしろ無埒(むらち)の方だろう。それだけはちょっと区別しておいて貰わんと名誉に関係するからな」「おんなし事だ」と主人は嘯(うそぶ)いている。純然たる天然居士の再来だ。「この前の日曜に東風子(とうふうし)が高輪泉岳寺(たかなわせんがくじ)に行ったんだそうだ。この寒いのによせばいいのに――第一今時(いまどき)泉岳寺などへ参るのはさも東京を知らない、田舎者(いなかもの)のようじゃないか」「それは東風の勝手さ。君がそれを留める権利はない」「なるほど権利は正(まさ)にない。権利はどうでもいいが、あの寺内に義士遺物保存会と云う見世物があるだろう。君知ってるか」「うんにゃ」「知らない? だって泉岳寺へ行った事はあるだろう」「いいや」「ない? こりゃ驚ろいた。道理で大変東風を弁護すると思った。江戸っ子が泉岳寺を知らないのは情(なさ)けない」「知らなくても教師は務(つと)まるからな」と主人はいよいよ天然居士になる。「そりゃ好いが、その展覧場へ東風が這入(はい)って見物していると、そこへ独逸人(ドイツじん)が夫婦連(づれ)で来たんだって。それが最初は日本語で東風に何か質問したそうだ。ところが先生例の通り独逸語が使って見たくてたまらん男だろう。そら二口三口べらべらやって見たとさ。すると存外うまく出来たんだ――あとで考えるとそれが災(わざわい)の本(もと)さね」「それからどうした」と主人はついに釣り込まれる。「独逸人が大鷹源吾(おおたかげんご)の蒔絵(まきえ)の印籠(いんろう)を見て、これを買いたいが売ってくれるだろうかと聞くんだそうだ。その時東風の返事が面白いじゃないか、日本人は清廉の君子(くんし)ばかりだから到底(とうてい)駄目だと云ったんだとさ。その辺は大分(だいぶ)景気がよかったが、それから独逸人の方では恰好(かっこう)な通弁を得たつもりでしきりに聞くそうだ」「何を?」「それがさ、何だか分るくらいなら心配はないんだが、早口で無暗(むやみ)に問い掛けるものだから少しも要領を得ないのさ。たまに分るかと思うと鳶口(とびぐち)や\emph{掛矢}の事を聞かれる。西洋の鳶口や\emph{掛矢}は先生何と翻訳して善いのか習った事が無いんだから弱(よ)わらあね」「もっともだ」と主人は教師の身の上に引き較(くら)べて同情を表する。「ところへ閑人(ひまじん)が物珍しそうにぽつぽつ集ってくる。仕舞(しまい)には東風と独逸人を四方から取り巻いて見物する。東風は顔を赤くしてへどもどする。初めの勢に引き易(か)えて先生大弱りの体(てい)さ」「結局どうなったんだい」「仕舞に東風が我慢出来なくなったと見えて\emph{さいなら}と日本語で云ってぐんぐん帰って来たそうだ、\emph{さいなら}は少し変だ君の国では\emph{さよなら}を\emph{さいなら}と云うかって聞いて見たら何やっぱり\emph{さよなら}ですが相手が西洋人だから調和を計るために、\emph{さいなら}にしたんだって、東風子は苦しい時でも調和を忘れない男だと感心した」「さいならはいいが西洋人はどうした」「西洋人はあっけに取られて茫然(ぼうぜん)と見ていたそうだハハハハ面白いじゃないか」「別段面白い事もないようだ。それをわざわざ報知(しらせ)に来る君の方がよっぽど面白いぜ」と主人は巻煙草(まきたばこ)の灰を火桶(ひおけ)の中へはたき落す。折柄(おりから)格子戸のベルが飛び上るほど鳴って「御免なさい」と鋭どい女の声がする。迷亭と主人は思わず顔を見合わせて沈黙する。
主人のうちへ女客は稀有(けう)だなと見ていると、かの鋭どい声の所有主は縮緬(ちりめん)の二枚重ねを畳へ擦(す)り付けながら這入(はい)って来る。年は四十の上を少し超(こ)したくらいだろう。抜け上った生(は)え際(ぎわ)から前髪が堤防工事のように高く聳(そび)えて、少なくとも顔の長さの二分の一だけ天に向ってせり出している。眼が切り通しの坂くらいな勾配(こうばい)で、直線に釣るし上げられて左右に対立する。直線とは鯨(くじら)より細いという形容である。鼻だけは無暗に大きい。人の鼻を盗んで来て顔の真中へ据(す)え付けたように見える。三坪ほどの小庭へ招魂社(しょうこんしゃ)の石灯籠(いしどうろう)を移した時のごとく、独(ひと)りで幅を利かしているが、何となく落ちつかない。その鼻はいわゆる鍵鼻(かぎばな)で、ひと度(たび)は精一杯高くなって見たが、これではあんまりだと中途から謙遜(けんそん)して、先の方へ行くと、初めの勢に似ず垂れかかって、下にある唇を覗(のぞ)き込んでいる。かく著(いちじ)るしい鼻だから、この女が物を言うときは口が物を言うと云わんより、鼻が口をきいているとしか思われない。吾輩はこの偉大なる鼻に敬意を表するため、以来はこの女を称して鼻子(はなこ)鼻子と呼ぶつもりである。鼻子は先ず初対面の挨拶を終って「どうも結構な御住居(おすまい)ですこと」と座敷中を睨(ね)め廻わす。主人は「嘘をつけ」と腹の中で言ったまま、ぷかぷか煙草(たばこ)をふかす。迷亭は天井を見ながら「君、ありゃ雨洩(あまも)りか、板の木目(もくめ)か、妙な模様が出ているぜ」と暗に主人を促(うな)がす。「無論雨の洩りさ」と主人が答えると「結構だなあ」と迷亭がすまして云う。鼻子は社交を知らぬ人達だと腹の中で憤(いきどお)る。しばらくは三人鼎坐(ていざ)のまま無言である。
「ちと伺いたい事があって、参ったんですが」と鼻子は再び話の口を切る。「はあ」と主人が極めて冷淡に受ける。これではならぬと鼻子は、「実は私はつい御近所で――あの向う横丁の角屋敷(かどやしき)なんですが」「あの大きな西洋館の倉のあるうちですか、道理であすこには金田(かねだ)と云う標札(ひょうさつ)が出ていますな」と主人はようやく金田の西洋館と、金田の倉を認識したようだが金田夫人に対する尊敬の度合(どあい)は前と同様である。「実は宿(やど)が出まして、御話を伺うんですが会社の方が大変忙がしいもんですから」と今度は少し利(き)いたろうという眼付をする。主人は一向(いっこう)動じない。鼻子の先刻(さっき)からの言葉遣いが初対面の女としてはあまり存在(ぞんざい)過ぎるのですでに不平なのである。「会社でも一つじゃ無いんです、二つも三つも兼ねているんです。それにどの会社でも重役なんで――多分御存知でしょうが」これでも恐れ入らぬかと云う顔付をする。元来ここの主人は\emph{博士}とか\emph{大学教授}とかいうと非常に恐縮する男であるが、妙な事には実業家に対する尊敬の度は極めて低い。実業家よりも中学校の先生の方がえらいと信じている。よし信じておらんでも、融通の利かぬ性質として、到底実業家、金満家の恩顧を蒙(こうむ)る事は覚束(おぼつか)ないと諦(あき)らめている。いくら先方が勢力家でも、財産家でも、自分が世話になる見込のないと思い切った人の利害には極めて無頓着である。それだから学者社会を除いて他の方面の事には極めて迂濶(うかつ)で、ことに実業界などでは、どこに、だれが何をしているか一向知らん。知っても尊敬畏服の念は毫(ごう)も起らんのである。鼻子の方では天(あめ)が下(した)の一隅にこんな変人がやはり日光に照らされて生活していようとは夢にも知らない。今まで世の中の人間にも大分(だいぶ)接して見たが、金田の妻(さい)ですと名乗って、急に取扱いの変らない場合はない、どこの会へ出ても、どんな身分の高い人の前でも立派に金田夫人で通して行かれる、いわんやこんな燻(くすぶ)り返った老書生においてをやで、私(わたし)の家(うち)は向う横丁の角屋敷(かどやしき)ですとさえ云えば職業などは聞かぬ先から驚くだろうと予期していたのである。
「金田って人を知ってるか」と主人は無雑作(むぞうさ)に迷亭に聞く。「知ってるとも、金田さんは僕の伯父の友達だ。この間なんざ園遊会へおいでになった」と迷亭は真面目な返事をする。「へえ、君の伯父さんてえな誰だい」「牧山男爵(まきやまだんしゃく)さ」と迷亭はいよいよ真面目である。主人が何か云おうとして云わぬ先に、鼻子は急に向き直って迷亭の方を見る。迷亭は大島紬(おおしまつむぎ)に古渡更紗(こわたりさらさ)か何か重ねてすましている。「おや、あなたが牧山様の――何でいらっしゃいますか、ちっとも存じませんで、はなはだ失礼を致しました。牧山様には始終御世話になると、宿(やど)で毎々御噂(おうわさ)を致しております」と急に叮嚀(ていねい)な言葉使をして、おまけに御辞儀までする、迷亭は「へええ何、ハハハハ」と笑っている。主人はあっ気(け)に取られて無言で二人を見ている。「たしか娘の縁辺(えんぺん)の事につきましてもいろいろ牧山さまへ御心配を願いましたそうで\ldots{}\ldots{}」「へえー、そうですか」とこればかりは迷亭にもちと唐突(とうとつ)過ぎたと見えてちょっと魂消(たまげ)たような声を出す。「実は方々からくれくれと申し込はございますが、こちらの身分もあるものでございますから、滅多(めった)な所(とこ)へも片付けられませんので\ldots{}\ldots{}」「ごもっともで」と迷亭はようやく安心する。「それについて、あなたに伺おうと思って上がったんですがね」と鼻子は主人の方を見て急に存在(ぞんざい)な言葉に返る。「あなたの所へ水島寒月(みずしまかんげつ)という男が度々(たびたび)上がるそうですが、あの人は全体どんな風な人でしょう」「寒月の事を聞いて、何(なん)にするんです」と主人は苦々(にがにが)しく云う。「やはり御令嬢の御婚儀上の関係で、寒月君の性行(せいこう)の一斑(いっぱん)を御承知になりたいという訳でしょう」と迷亭が気転を利(き)かす。「それが伺えれば大変都合が宜(よろ)しいのでございますが\ldots{}\ldots{}」「それじゃ、御令嬢を寒月におやりになりたいとおっしゃるんで」「やりたいなんてえんじゃ無いんです」と鼻子は急に主人を参らせる。「ほかにもだんだん口が有るんですから、無理に貰っていただかないだって困りゃしません」「それじゃ寒月の事なんか聞かんでも好いでしょう」と主人も躍起(やっき)となる。「しかし御隠しなさる訳もないでしょう」と鼻子も少々喧嘩腰になる。迷亭は双方の間に坐って、銀煙管(ぎんぎせる)を軍配団扇(ぐんばいうちわ)のように持って、心の裡(うち)で八卦(はっけ)よいやよいやと怒鳴っている。「じゃあ寒月の方で是非貰いたいとでも云ったのですか」と主人が正面から鉄砲を喰(くら)わせる。「貰いたいと云ったんじゃないんですけれども\ldots{}\ldots{}」「貰いたいだろうと思っていらっしゃるんですか」と主人はこの婦人鉄砲に限ると覚(さと)ったらしい。「話しはそんなに運んでるんじゃありませんが――寒月さんだって満更(まんざら)嬉しくない事もないでしょう」と土俵際で持ち直す。「寒月が何かその御令嬢に恋着(れんちゃく)したというような事でもありますか」あるなら云って見ろと云う権幕(けんまく)で主人は反(そ)り返る。「まあ、そんな見当(けんとう)でしょうね」今度は主人の鉄砲が少しも功を奏しない。今まで面白気(おもしろげ)に行司(ぎょうじ)気取りで見物していた迷亭も鼻子の一言(いちごん)に好奇心を挑撥(ちょうはつ)されたものと見えて、煙管(きせる)を置いて前へ乗り出す。「寒月が御嬢さんに付(つ)け文(ぶみ)でもしたんですか、こりゃ愉快だ、新年になって逸話がまた一つ殖(ふ)えて話しの好材料になる」と一人で喜んでいる。「付け文じゃないんです、もっと烈しいんでさあ、御二人とも御承知じゃありませんか」と鼻子は乙(おつ)にからまって来る。「君知ってるか」と主人は狐付きのような顔をして迷亭に聞く。迷亭も馬鹿気(ばかげ)た調子で「僕は知らん、知っていりゃ君だ」とつまらんところで謙遜(けんそん)する。「いえ御両人共(おふたりとも)御存じの事ですよ」と鼻子だけ大得意である。「へえー」と御両人は一度に感じ入る。「御忘れになったら私(わた)しから御話をしましょう。去年の暮向島の阿部さんの御屋敷で演奏会があって寒月さんも出掛けたじゃありませんか、その晩帰りに吾妻橋(あずまばし)で何かあったでしょう――詳しい事は言いますまい、当人の御迷惑になるかも知れませんから――あれだけの証拠がありゃ充分だと思いますが、どんなものでしょう」と金剛石(ダイヤ)入りの指環の嵌(はま)った指を、膝の上へ併(なら)べて、つんと居ずまいを直す。偉大なる鼻がますます異彩を放って、迷亭も主人も有れども無きがごとき有様である。
主人は無論、さすがの迷亭もこの不意撃(ふいうち)には胆(きも)を抜かれたものと見えて、しばらくは呆然(ぼうぜん)として瘧(おこり)の落ちた病人のように坐っていたが、驚愕(きょうがく)の箍(たが)がゆるんでだんだん持前の本態に復すると共に、滑稽と云う感じが一度に吶喊(とっかん)してくる。両人(ふたり)は申し合せたごとく「ハハハハハ」と笑い崩れる。鼻子ばかりは少し当てがはずれて、この際笑うのははなはだ失礼だと両人を睨(にら)みつける。「あれが御嬢さんですか、なるほどこりゃいい、おっしゃる通りだ、ねえ苦沙弥(くしゃみ)君、全く寒月はお嬢さんを恋(おも)ってるに相違ないね\ldots{}\ldots{}もう隠したってしようがないから白状しようじゃないか」「ウフン」と主人は云ったままである。「本当に御隠しなさってもいけませんよ、ちゃんと種は上ってるんですからね」と鼻子はまた得意になる。「こうなりゃ仕方がない。何でも寒月君に関する事実は御参考のために陳述するさ、おい苦沙弥君、君が主人だのに、そう、にやにや笑っていては埒(らち)があかんじゃないか、実に秘密というものは恐ろしいものだねえ。いくら隠しても、どこからか露見(ろけん)するからな。――しかし不思議と云えば不思議ですねえ、金田の奥さん、どうしてこの秘密を御探知になったんです、実に驚ろきますな」と迷亭は一人で喋舌(しゃべ)る。「私(わた)しの方だって、ぬかりはありませんやね」と鼻子はしたり顔をする。「あんまり、ぬかりが無さ過ぎるようですぜ。一体誰に御聞きになったんです」「じきこの裏にいる車屋の神(かみ)さんからです」「あの黒猫のいる車屋ですか」と主人は眼を丸くする。「ええ、寒月さんの事じゃ、よっぽど使いましたよ。寒月さんが、ここへ来る度に、どんな話しをするかと思って車屋の神さんを頼んで一々知らせて貰うんです」「そりゃ苛(ひど)い」と主人は大きな声を出す。「なあに、あなたが何をなさろうとおっしゃろうと、それに構ってるんじゃないんです。寒月さんの事だけですよ」「寒月の事だって、誰の事だって――全体あの車屋の神さんは気に食わん奴だ」と主人は一人怒(おこ)り出す。「しかしあなたの垣根のそとへ来て立っているのは向うの勝手じゃありませんか、話しが聞えてわるけりゃもっと小さい声でなさるか、もっと大きなうちへ御這入(おはい)んなさるがいいでしょう」と鼻子は少しも赤面した様子がない。「車屋ばかりじゃありません。新道(しんみち)の二絃琴(にげんきん)の師匠からも大分(だいぶ)いろいろな事を聞いています」「寒月の事をですか」「寒月さんばかりの事じゃありません」と少し凄(すご)い事を云う。主人は恐れ入るかと思うと「あの師匠はいやに上品ぶって自分だけ人間らしい顔をしている、馬鹿野郎です」「憚(はばか)り様(さま)、女ですよ。野郎は御門違(おかどちが)いです」と鼻子の言葉使いはますます御里(おさと)をあらわして来る。これではまるで喧嘩をしに来たようなものであるが、そこへ行くと迷亭はやはり迷亭でこの談判を面白そうに聞いている。鉄枴仙人(てっかいせんにん)が軍鶏(しゃも)の蹴合(けあ)いを見るような顔をして平気で聞いている。
悪口(あっこう)の交換では到底鼻子の敵でないと自覚した主人は、しばらく沈黙を守るのやむを得ざるに至らしめられていたが、ようやく思い付いたか「あなたは寒月の方から御嬢さんに恋着したようにばかりおっしゃるが、私(わたし)の聞いたんじゃ、少し違いますぜ、ねえ迷亭君」と迷亭の救いを求める。「うん、あの時の話しじゃ御嬢さんの方が、始め病気になって――何だか譫語(うわごと)をいったように聞いたね」「なにそんな事はありません」と金田夫人は判然たる直線流の言葉使いをする。「それでも寒月はたしかに○○博士の夫人から聞いたと云っていましたぜ」「それがこっちの手なんでさあ、○○博士の奥さんを頼んで寒月さんの気を引いて見たんでさあね」「○○の奥さんは、それを承知で引き受けたんですか」「ええ。引き受けて貰うたって、ただじゃ出来ませんやね、それやこれやでいろいろ物を使っているんですから」「是非寒月君の事を根堀り葉堀り御聞きにならなくっちゃ御帰りにならないと云う決心ですかね」と迷亭も少し気持を悪くしたと見えて、いつになく手障(てざわ)りのあらい言葉を使う。「いいや君、話したって損の行く事じゃなし、話そうじゃないか苦沙弥君――奥さん、私(わたし)でも苦沙弥でも寒月君に関する事実で差支(さしつか)えのない事は、みんな話しますからね、――そう、順を立ててだんだん聞いて下さると都合がいいですね」
鼻子はようやく納得(なっとく)してそろそろ質問を呈出する。一時荒立てた言葉使いも迷亭に対してはまたもとのごとく叮嚀になる。「寒月さんも理学士だそうですが、全体どんな事を専門にしているのでございます」「大学院では\emph{地球の磁気の研究}をやっています」と主人が真面目に答える。不幸にしてその意味が鼻子には分らんものだから「へえー」とは云ったが怪訝(けげん)な顔をしている。「それを勉強すると博士になれましょうか」と聞く。「博士にならなければやれないとおっしゃるんですか」と主人は不愉快そうに尋ねる。「ええ。ただの学士じゃね、いくらでもありますからね」と鼻子は平気で答える。主人は迷亭を見ていよいよいやな顔をする。「博士になるかならんかは僕等も保証する事が出来んから、ほかの事を聞いていただく事にしよう」と迷亭もあまり好い機嫌ではない。「近頃でもその地球の――何かを勉強しているんでございましょうか」「二三日前(にさんちまえ)は\emph{首縊りの力学}と云う研究の結果を理学協会で演説しました」と主人は何の気も付かずに云う。「おやいやだ、\emph{首縊り}だなんて、よっぽど変人ですねえ。そんな\emph{首縊り}や何かやってたんじゃ、とても博士にはなれますまいね」「本人が首を縊(くく)っちゃあむずかしいですが、\emph{首縊りの力学}なら成れないとも限らんです」「そうでしょうか」と今度は主人の方を見て顔色を窺(うかが)う。悲しい事に\emph{力学}と云う意味がわからんので落ちつきかねている。しかしこれしきの事を尋ねては金田夫人の面目に関すると思ってか、ただ相手の顔色で八卦(はっけ)を立てて見る。主人の顔は渋い。「そのほかになにか、分り易(やす)いものを勉強しておりますまいか」「そうですな、せんだって\emph{団栗のスタビリチーを論じて併せて天体の運行に及ぶ}と云う論文を書いた事があります」「団栗(どんぐり)なんぞでも大学校で勉強するものでしょうか」「さあ僕も素人(しろうと)だからよく分らんが、何しろ、寒月君がやるくらいなんだから、研究する価値があると見えますな」と迷亭はすまして冷かす。鼻子は学問上の質問は手に合わんと断念したものと見えて、今度は話題を転ずる。「御話は違いますが――この御正月に椎茸(しいたけ)を食べて前歯を二枚折ったそうじゃございませんか」「ええその欠けたところに空也餅(くうやもち)がくっ付いていましてね」と迷亭はこの質問こそ吾縄張内(なわばりうち)だと急に浮かれ出す。「色気のない人じゃございませんか、何だって楊子(ようじ)を使わないんでしょう」「今度逢(あ)ったら注意しておきましょう」と主人がくすくす笑う。「椎茸で歯がかけるくらいじゃ、よほど歯の性(しょう)が悪いと思われますが、如何(いかが)なものでしょう」「善いとは言われますまいな――ねえ迷亭」「善い事はないがちょっと愛嬌(あいきょう)があるよ。あれぎり、まだ填(つ)めないところが妙だ。今だに空也餅引掛所(ひっかけどころ)になってるなあ奇観だぜ」「歯を填める小遣(こづかい)がないので欠けなりにしておくんですか、または物好きで欠けなりにしておくんでしょうか」「何も永く前歯欠成(まえばかけなり)を名乗る訳でもないでしょうから御安心なさいよ」と迷亭の機嫌はだんだん回復してくる。鼻子はまた問題を改める。「何か御宅に手紙かなんぞ当人の書いたものでもございますならちょっと拝見したいもんでございますが」「端書(はがき)なら沢山あります、御覧なさい」と主人は書斎から三四十枚持って来る。「そんなに沢山拝見しないでも――その内の二三枚だけ\ldots{}\ldots{}」「どれどれ僕が好いのを撰(よ)ってやろう」と迷亭先生は「これなざあ面白いでしょう」と一枚の絵葉書を出す。「おや絵もかくんでございますか、なかなか器用ですね、どれ拝見しましょう」と眺めていたが「あらいやだ、狸(たぬき)だよ。何だって撰りに撰って狸なんぞかくんでしょうね――それでも狸と見えるから不思議だよ」と少し感心する。「その文句を読んで御覧なさい」と主人が笑いながら云う。鼻子は下女が新聞を読むように読み出す。「旧暦の歳(とし)の夜(よ)、山の狸が園遊会をやって盛(さかん)に舞踏します。その歌に曰(いわ)く、来(こ)いさ、としの夜(よ)で、御山婦美(おやまふみ)も来(く)まいぞ。スッポコポンノポン」「何ですこりゃ、人を馬鹿にしているじゃございませんか」と鼻子は不平の体(てい)である。「この天女(てんにょ)は御気に入りませんか」と迷亭がまた一枚出す。見ると天女が羽衣(はごろも)を着て琵琶(びわ)を弾(ひ)いている。「この天女の鼻が少し小さ過ぎるようですが」「何、それが人並ですよ、鼻より文句を読んで御覧なさい」文句にはこうある。「昔(むか)しある所に一人の天文学者がありました。ある夜(よ)いつものように高い台に登って、一心に星を見ていますと、空に美しい天女が現われ、この世では聞かれぬほどの微妙な音楽を奏し出したので、天文学者は身に沁(し)む寒さも忘れて聞き惚(ほ)れてしまいました。朝見るとその天文学者の死骸(しがい)に霜(しも)が真白に降っていました。これは本当の噺(はなし)だと、あのうそつきの爺(じい)やが申しました」「何の事ですこりゃ、意味も何もないじゃありませんか、これでも理学士で通るんですかね。ちっと文芸倶楽部でも読んだらよさそうなものですがねえ」と寒月君さんざんにやられる。迷亭は面白半分に「こりゃどうです」と三枚目を出す。今度は活版で帆懸舟(ほかけぶね)が印刷してあって、例のごとくその下に何か書き散らしてある。「よべの泊(とま)りの十六小女郎(じゅうろくこじょろ)、親がないとて、荒磯(ありそ)の千鳥、さよの寝覚(ねざめ)の千鳥に泣いた、親は船乗り波の底」「うまいのねえ、感心だ事、話せるじゃありませんか」「話せますかな」「ええこれなら三味線に乗りますよ」「三味線に乗りゃ本物だ。こりゃ如何(いかが)です」と迷亭は無暗(むやみ)に出す。「いえ、もうこれだけ拝見すれば、ほかのは沢山で、そんなに野暮(やぼ)でないんだと云う事は分りましたから」と一人で合点している。鼻子はこれで寒月に関する大抵の質問を卒(お)えたものと見えて、「これははなはだ失礼を致しました。どうか私の参った事は寒月さんへは内々に願います」と得手勝手(えてかって)な要求をする。寒月の事は何でも聞かなければならないが、自分の方の事は一切寒月へ知らしてはならないと云う方針と見える。迷亭も主人も「はあ」と気のない返事をすると「いずれその内御礼は致しますから」と念を入れて言いながら立つ。見送りに出た両人(ふたり)が席へ返るや否や迷亭が「ありゃ何だい」と云うと主人も「ありゃ何だい」と双方から同じ問をかける。奥の部屋で細君が怺(こら)え切れなかったと見えてクツクツ笑う声が聞える。迷亭は大きな声を出して「奥さん奥さん、月並の標本が来ましたぜ。月並もあのくらいになるとなかなか振(ふる)っていますなあ。さあ遠慮はいらんから、存分御笑いなさい」
主人は不満な口気(こうき)で「第一気に喰わん顔だ」と悪(にく)らしそうに云うと、迷亭はすぐ引きうけて「鼻が顔の中央に陣取って乙(おつ)に構えているなあ」とあとを付ける。「しかも曲っていらあ」「少し猫背(ねこぜ)だね。猫背の鼻は、ちと奇抜(きばつ)過ぎる」と面白そうに笑う。「夫(おっと)を剋(こく)する顔だ」と主人はなお口惜(くや)しそうである。「十九世紀で売れ残って、二十世紀で店曝(たなざら)しに逢うと云う相(そう)だ」と迷亭は妙な事ばかり云う。ところへ妻君が奥の間(ま)から出て来て、女だけに「あんまり悪口をおっしゃると、また車屋の神(かみ)さんに\emph{いつけ}られますよ」と注意する。「少し\emph{いつけ}る方が薬ですよ、奥さん」「しかし顔の讒訴(ざんそ)などをなさるのは、あまり下等ですわ、誰だって好んであんな鼻を持ってる訳でもありませんから――それに相手が婦人ですからね、あんまり苛(ひど)いわ」と鼻子の鼻を弁護すると、同時に自分の容貌(ようぼう)も間接に弁護しておく。「何ひどいものか、あんなのは婦人じゃない、愚人だ、ねえ迷亭君」「愚人かも知れんが、なかなかえら者だ、大分(だいぶ)引き掻(か)かれたじゃないか」「全体教師を何と心得ているんだろう」「裏の車屋くらいに心得ているのさ。ああ云う人物に尊敬されるには博士になるに限るよ、一体博士になっておかんのが君の不了見(ふりょうけん)さ、ねえ奥さん、そうでしょう」と迷亭は笑いながら細君を顧(かえり)みる。「博士なんて到底駄目ですよ」と主人は細君にまで見離される。「これでも今になるかも知れん、軽蔑(けいべつ)するな。貴様なぞは知るまいが昔(むか)しアイソクラチスと云う人は九十四歳で大著述をした。ソフォクリスが傑作を出して天下を驚かしたのは、ほとんど百歳の高齢だった。シモニジスは八十で妙詩を作った。おれだって\ldots{}\ldots{}」「馬鹿馬鹿しいわ、あなたのような胃病でそんなに永く生きられるものですか」と細君はちゃんと主人の寿命を予算している。「失敬な、――甘木さんへ行って聞いて見ろ――元来御前がこんな皺苦茶(しわくちゃ)な黒木綿(くろもめん)の羽織や、つぎだらけの着物を着せておくから、あんな女に馬鹿にされるんだ。あしたから迷亭の着ているような奴を着るから出しておけ」「出しておけって、あんな立派な御召(おめし)はござんせんわ。金田の奥さんが迷亭さんに叮嚀になったのは、伯父さんの名前を聞いてからですよ。着物の咎(とが)じゃございません」と細君うまく責任を逃(の)がれる。
主人は\emph{伯父さん}と云う言葉を聞いて急に思い出したように「君に伯父があると云う事は、今日始めて聞いた。今までついに噂(うわさ)をした事がないじゃないか、本当にあるのかい」と迷亭に聞く。迷亭は待ってたと云わぬばかりに「うんその伯父さ、その伯父が馬鹿に頑物(がんぶつ)でねえ――やはりその十九世紀から連綿と今日(こんにち)まで生き延びているんだがね」と主人夫婦を半々に見る。「オホホホホホ面白い事ばかりおっしゃって、どこに生きていらっしゃるんです」「静岡に生きてますがね、それがただ生きてるんじゃ無いです。頭にちょん髷(まげ)を頂いて生きてるんだから恐縮しまさあ。帽子を被(かぶ)れってえと、おれはこの年になるが、まだ帽子を被るほど寒さを感じた事はないと威張ってるんです――寒いから、もっと寝(ね)ていらっしゃいと云うと、人間は四時間寝れば充分だ。四時間以上寝るのは贅沢(ぜいたく)の沙汰だって朝暗いうちから起きてくるんです。それでね、おれも睡眠時間を四時間に縮めるには、永年修業をしたもんだ、若いうちはどうしても眠(ねむ)たくていかなんだが、近頃に至って始めて随処任意の庶境(しょきょう)に入(い)ってはなはだ嬉しいと自慢するんです。六十七になって寝られなくなるなあ当り前でさあ。修業も糸瓜(へちま)も入(い)ったものじゃないのに当人は全く克己(こっき)の力で成功したと思ってるんですからね。それで外出する時には、きっと鉄扇(てっせん)をもって出るんですがね」「なににするんだい」「何にするんだか分らない、ただ持って出るんだね。まあステッキの代りくらいに考えてるかも知れんよ。ところがせんだって妙な事がありましてね」と今度は細君の方へ話しかける。「へえー」と細君が差(さ)し合(あい)のない返事をする。「此年(ことし)の春突然手紙を寄こして山高帽子とフロックコートを至急送れと云うんです。ちょっと驚ろいたから、郵便で問い返したところが老人自身が着ると云う返事が来ました。二十三日に静岡で祝捷会(しゅくしょうかい)があるからそれまでに間(ま)に合うように、至急調達しろと云う命令なんです。ところがおかしいのは命令中にこうあるんです。帽子は好い加減な大きさのを買ってくれ、洋服も寸法を見計らって大丸(だいまる)へ注文してくれ\ldots{}\ldots{}」「近頃は大丸でも洋服を仕立てるのかい」「なあに、先生、白木屋(しろきや)と間違えたんだあね」「寸法を見計ってくれたって無理じゃないか」「そこが伯父の伯父たるところさ」「どうした?」「仕方がないから見計らって送ってやった」「君も乱暴だな。それで間に合ったのかい」「まあ、どうにか、こうにかおっついたんだろう。国の新聞を見たら、当日牧山翁は珍らしくフロックコートにて、例の鉄扇(てっせん)を持ち\ldots{}\ldots{}」「鉄扇だけは離さなかったと見えるね」「うん死んだら棺の中へ鉄扇だけは入れてやろうと思っているよ」「それでも帽子も洋服も、うまい具合に着られて善かった」「ところが大間違さ。僕も無事に行ってありがたいと思ってると、しばらくして国から小包が届いたから、何か礼でもくれた事と思って開けて見たら例の山高帽子さ、手紙が添えてあってね、せっかく御求め被下候(くだされそうら)えども少々大きく候間(そろあいだ)、帽子屋へ御遣(おつか)わしの上、御縮め被下度候(くだされたくそろ)。縮め賃は小為替(こがわせ)にて此方(こなた)より御送(おんおくり)可申上候(もうしあぐべきそろ)とあるのさ」「なるほど迂濶(うかつ)だな」と主人は己(おの)れより迂濶なものの天下にある事を発見して大(おおい)に満足の体(てい)に見える。やがて「それから、どうした」と聞く。「どうするったって仕方がないから僕が頂戴して被(かぶ)っていらあ」「あの帽子かあ」と主人がにやにや笑う。「その方(かた)が男爵でいらっしゃるんですか」と細君が不思議そうに尋ねる。「誰がです」「その鉄扇の伯父さまが」「なあに漢学者でさあ、若い時聖堂(せいどう)で朱子学(しゅしがく)か、何かにこり固まったものだから、電気灯の下で恭(うやうや)しく\emph{ちょん}髷(まげ)を頂いているんです。仕方がありません」とやたらに顋(あご)を撫(な)で廻す。「それでも君は、さっきの女に牧山男爵と云ったようだぜ」「そうおっしゃいましたよ、私も茶の間で聞いておりました」と細君もこれだけは主人の意見に同意する。「そうでしたかなアハハハハハ」と迷亭は訳(わけ)もなく笑う。「そりゃ嘘(うそ)ですよ。僕に男爵の伯父がありゃ、今頃は局長くらいになっていまさあ」と平気なものである。「何だか変だと思った」と主人は嬉しそうな、心配そうな顔付をする。「あらまあ、よく真面目であんな嘘が付けますねえ。あなたもよっぽど法螺(ほら)が御上手でいらっしゃる事」と細君は非常に感心する。「僕より、あの女の方が上(う)わ手(て)でさあ」「あなただって御負けなさる気遣(きづか)いはありません」「しかし奥さん、僕の法螺は単なる法螺ですよ。あの女のは、みんな魂胆があって、曰(いわ)く付きの嘘ですぜ。たちが悪いです。猿智慧(さるぢえ)から割り出した術数と、天来の滑稽趣味と混同されちゃ、コメディーの神様も活眼の士なきを嘆ぜざるを得ざる訳に立ち至りますからな」主人は俯目(ふしめ)になって「どうだか」と云う。妻君は笑いながら「同じ事ですわ」と云う。
吾輩は今まで向う横丁へ足を踏み込んだ事はない。角屋敷(かどやしき)の金田とは、どんな構えか見た事は無論ない。聞いた事さえ今が始めてである。主人の家(うち)で実業家が話頭に上(のぼ)った事は一返もないので、主人の飯を食う吾輩までがこの方面には単に無関係なるのみならず、はなはだ冷淡であった。しかるに先刻図(はか)らずも鼻子の訪問を受けて、余所(よそ)ながらその談話を拝聴し、その令嬢の艶美(えんび)を想像し、またその富貴(ふうき)、権勢を思い浮べて見ると、猫ながら安閑として椽側(えんがわ)に寝転んでいられなくなった。しかのみならず吾輩は寒月君に対してはなはだ同情の至りに堪えん。先方では博士の奥さんやら、車屋の神(かみ)さんやら、二絃琴(にげんきん)の天璋院(てんしょういん)まで買収して知らぬ間(ま)に、前歯の欠けたのさえ探偵しているのに、寒月君の方ではただニヤニヤして羽織の紐ばかり気にしているのは、いかに卒業したての理学士にせよ、あまり能がなさ過ぎる。と言って、ああ云う偉大な鼻を顔の中(うち)に安置している女の事だから、滅多(めった)な者では寄り付ける訳の者ではない。こう云う事件に関しては主人はむしろ無頓着でかつあまりに銭(ぜに)がなさ過ぎる。迷亭は銭に不自由はしないが、あんな偶然童子だから、寒月に援(たす)けを与える便宜(べんぎ)は尠(すくな)かろう。して見ると可哀相(かわいそう)なのは\emph{首縊りの力学}を演説する先生ばかりとなる。吾輩でも奮発して、敵城へ乗り込んでその動静を偵察してやらなくては、あまり不公平である。吾輩は猫だけれど、エピクテタスを読んで机の上へ叩きつけるくらいな学者の家(うち)に寄寓(きぐう)する猫で、世間一般の痴猫(ちびょう)、愚猫(ぐびょう)とは少しく撰(せん)を殊(こと)にしている。この冒険をあえてするくらいの義侠心は固(もと)より尻尾(しっぽ)の先に畳み込んである。何も寒月君に恩になったと云う訳もないが、これはただに個人のためにする血気躁狂(けっきそうきょう)の沙汰ではない。大きく云えば公平を好み中庸を愛する天意を現実にする天晴(あっぱれ)な美挙だ。人の許諾を経(へ)ずして吾妻橋(あずまばし)事件などを至る処に振り廻わす以上は、人の軒下に犬を忍ばして、その報道を得々として逢う人に吹聴(ふいちょう)する以上は、車夫、馬丁(ばてい)、無頼漢(ぶらいかん)、ごろつき書生、日雇婆(ひやといばばあ)、産婆、妖婆(ようば)、按摩(あんま)、頓馬(とんま)に至るまでを使用して国家有用の材に煩(はん)を及ぼして顧(かえり)みざる以上は――猫にも覚悟がある。幸い天気も好い、霜解(しもどけ)は少々閉口するが道のためには一命もすてる。足の裏へ泥が着いて、椽側(えんがわ)へ梅の花の印を押すくらいな事は、ただ御三(おさん)の迷惑にはなるか知れんが、吾輩の苦痛とは申されない。翌日(あす)とも云わずこれから出掛けようと勇猛精進(ゆうもうしょうじん)の大決心を起して台所まで飛んで出たが「待てよ」と考えた。吾輩は猫として進化の極度に達しているのみならず、脳力の発達においてはあえて中学の三年生に劣らざるつもりであるが、悲しいかな咽喉(のど)の構造だけはどこまでも猫なので人間の言語が饒舌(しゃべ)れない。よし首尾よく金田邸へ忍び込んで、充分敵の情勢を見届けたところで、肝心(かんじん)の寒月君に教えてやる訳に行かない。主人にも迷亭先生にも話せない。話せないとすれば土中にある金剛石(ダイヤモンド)の日を受けて光らぬと同じ事で、せっかくの智識も無用の長物となる。これは愚(ぐ)だ、やめようかしらんと上り口で佇(たたず)んで見た。
しかし一度思い立った事を中途でやめるのは、白雨(ゆうだち)が来るかと待っている時黒雲共(とも)隣国へ通り過ぎたように、何となく残り惜しい。それも非がこっちにあれば格別だが、いわゆる正義のため、人道のためなら、たとい無駄死(むだじに)をやるまでも進むのが、義務を知る男児の本懐であろう。無駄骨を折り、無駄足を汚(よご)すくらいは猫として適当のところである。猫と生れた因果(いんが)で寒月、迷亭、苦沙弥諸先生と三寸の舌頭(ぜっとう)に相互の思想を交換する技倆(ぎりょう)はないが、猫だけに忍びの術は諸先生より達者である。他人の出来ぬ事を成就(じょうじゅ)するのはそれ自身において愉快である。吾(われ)一箇でも、金田の内幕を知るのは、誰も知らぬより愉快である。人に告げられんでも人に知られているなと云う自覚を彼等に与うるだけが愉快である。こんなに愉快が続々出て来ては行かずにはいられない。やはり行く事に致そう。
向う横町へ来て見ると、聞いた通りの西洋館が角地面(かどじめん)を吾物顔(わがものがお)に占領している。この主人もこの西洋館のごとく傲慢(ごうまん)に構えているんだろうと、門を這入(はい)ってその建築を眺(なが)めて見たがただ人を威圧しようと、二階作りが無意味に突っ立っているほかに何等の能もない構造であった。迷亭のいわゆる月並(つきなみ)とはこれであろうか。玄関を右に見て、植込の中を通り抜けて、勝手口へ廻る。さすがに勝手は広い、苦沙弥先生の台所の十倍はたしかにある。せんだって日本新聞に詳しく書いてあった大隈伯(おおくまはく)の勝手にも劣るまいと思うくらい整然とぴかぴかしている。「模範勝手だな」と這入(はい)り込む。見ると漆喰(しっくい)で叩き上げた二坪ほどの土間に、例の車屋の神(かみ)さんが立ちながら、御飯焚(ごはんた)きと車夫を相手にしきりに何か弁じている。こいつは剣呑(けんのん)だと水桶(みずおけ)の裏へかくれる。「あの教師あ、うちの旦那の名を知らないのかね」と飯焚(めしたき)が云う。「知らねえ事があるもんか、この界隈(かいわい)で金田さんの御屋敷を知らなけりゃ眼も耳もねえ片輪(かたわ)だあな」これは抱え車夫の声である。「なんとも云えないよ。あの教師と来たら、本よりほかに何にも知らない変人なんだからねえ。旦那の事を少しでも知ってりゃ恐れるかも知れないが、駄目だよ、自分の小供の歳(とし)さえ知らないんだもの」と神さんが云う。「金田さんでも恐れねえかな、厄介な唐変木(とうへんぼく)だ。構(かま)あ事(こた)あねえ、みんなで威嚇(おど)かしてやろうじゃねえか」「それが好いよ。奥様の鼻が大き過ぎるの、顔が気に喰わないのって――そりゃあ酷(ひど)い事を云うんだよ。自分の面(つら)あ今戸焼(いまどやき)の狸(たぬき)見たような癖に――あれで一人前(いちにんまえ)だと思っているんだからやれ切れないじゃないか」「顔ばかりじゃない、手拭(てぬぐい)を提(さ)げて湯に行くところからして、いやに高慢ちきじゃないか。自分くらいえらい者は無いつもりでいるんだよ」と苦沙弥先生は飯焚にも大(おおい)に不人望である。「何でも大勢であいつの垣根の傍(そば)へ行って悪口をさんざんいってやるんだね」「そうしたらきっと恐れ入るよ」「しかしこっちの姿を見せちゃあ面白くねえから、声だけ聞かして、勉強の邪魔をした上に、出来るだけじらしてやれって、さっき奥様が言い付けておいでなすったぜ」「そりゃ分っているよ」と神さんは悪口の三分の一を引き受けると云う意味を示す。なるほどこの手合が苦沙弥先生を冷やかしに来るなと三人の横を、そっと通り抜けて奥へ這入る。
猫の足はあれども無きがごとし、どこを歩いても不器用な音のした試しがない。空を踏むがごとく、雲を行くがごとく、水中に磬(けい)を打つがごとく、洞裏(とうり)に瑟(しつ)を鼓(こ)するがごとく、醍醐(だいご)の妙味を甞(な)めて言詮(ごんせん)のほかに冷暖(れいだん)を自知(じち)するがごとし。月並な西洋館もなく、模範勝手もなく、車屋の神さんも、権助(ごんすけ)も、飯焚も、御嬢さまも、仲働(なかばたら)きも、鼻子夫人も、夫人の旦那様もない。行きたいところへ行って聞きたい話を聞いて、舌を出し尻尾(しっぽ)を掉(ふ)って、髭(ひげ)をぴんと立てて悠々(ゆうゆう)と帰るのみである。ことに吾輩はこの道に掛けては日本一の堪能(かんのう)である。草双紙(くさぞうし)にある猫又(ねこまた)の血脈を受けておりはせぬかと自(みずか)ら疑うくらいである。蟇(がま)の額(ひたい)には夜光(やこう)の明珠(めいしゅ)があると云うが、吾輩の尻尾には神祇釈教(しんぎしゃっきょう)恋無常(こいむじょう)は無論の事、満天下の人間を馬鹿にする一家相伝(いっかそうでん)の妙薬が詰め込んである。金田家の廊下を人の知らぬ間(ま)に横行するくらいは、仁王様が心太(ところてん)を踏み潰(つぶ)すよりも容易である。この時吾輩は我ながら、わが力量に感服して、これも普段大事にする尻尾の御蔭だなと気が付いて見るとただ置かれない。吾輩の尊敬する尻尾大明神を礼拝(らいはい)してニャン運長久を祈らばやと、ちょっと低頭して見たが、どうも少し見当(けんとう)が違うようである。なるべく尻尾の方を見て三拝しなければならん。尻尾の方を見ようと身体を廻すと尻尾も自然と廻る。追付こうと思って首をねじると、尻尾も同じ間隔をとって、先へ馳(か)け出す。なるほど天地玄黄(てんちげんこう)を三寸裏(り)に収めるほどの霊物だけあって、到底吾輩の手に合わない、尻尾を環(めぐ)る事七度(ななた)び半にして草臥(くたび)れたからやめにした。少々眼がくらむ。どこにいるのだかちょっと方角が分らなくなる。構うものかと滅茶苦茶にあるき廻る。障子の裏(うち)で鼻子の声がする。ここだと立ち留まって、左右の耳をはすに切って、息を凝(こ)らす。「貧乏教師の癖に生意気じゃありませんか」と例の金切(かなき)り声(ごえ)を振り立てる。「うん、生意気な奴だ、ちと懲(こ)らしめのためにいじめてやろう。あの学校にゃ国のものもいるからな」「誰がいるの?」「津木(つき)ピン助(すけ)や福地(ふくち)キシャゴがいるから、頼んでからかわしてやろう」吾輩は金田君の生国(しょうごく)は分らんが、妙な名前の人間ばかり揃(そろ)った所だと少々驚いた。金田君はなお語をついで、「あいつは英語の教師かい」と聞く。「はあ、車屋の神さんの話では英語のリードルか何か専門に教えるんだって云います」「どうせ碌(ろく)な教師じゃあるめえ」\emph{あるめえ}にも尠(すく)なからず感心した。「この間ピン助に遇(あ)ったら、私(わたし)の学校にゃ妙な奴がおります。生徒から先生\emph{番茶}は英語で何と云いますと聞かれて、\emph{番茶}は Savage tea であると真面目に答えたんで、教員間の物笑いとなっています、どうもあんな教員があるから、ほかのものの、迷惑になって困りますと云ったが、大方(おおかた)あいつの事だぜ」「あいつに極(きま)っていまさあ、そんな事を云いそうな面構(つらがま)えですよ、いやに髭(ひげ)なんか生(は)やして」「怪(け)しからん奴だ」髭を生やして怪しからなければ猫などは一疋だって怪しかりようがない。「それにあの迷亭とか、へべれけとか云う奴は、まあ何てえ、頓狂な跳返(はねっかえ)りなんでしょう、伯父の牧山男爵だなんて、あんな顔に男爵の伯父なんざ、有るはずがないと思ったんですもの」「御前がどこの馬の骨だか分らんものの言う事を真(ま)に受けるのも悪い」「悪いって、あんまり人を馬鹿にし過ぎるじゃありませんか」と大変残念そうである。不思議な事には寒月君の事は一言半句(いちごんはんく)も出ない。吾輩の忍んで来る前に評判記はすんだものか、またはすでに落第と事が極(きま)って念頭にないものか、その辺(へん)は懸念(けねん)もあるが仕方がない。しばらく佇(たたず)んでいると廊下を隔てて向うの座敷でベルの音がする。そらあすこにも何か事がある。後(おく)れぬ先に、とその方角へ歩を向ける。
来て見ると女が独(ひと)りで何か大声で話している。その声が鼻子とよく似ているところをもって推(お)すと、これが即ち当家の令嬢寒月君をして未遂入水(みすいじゅすい)をあえてせしめたる代物(しろもの)だろう。惜哉(おしいかな)障子越しで玉の御姿(おんすがた)を拝する事が出来ない。従って顔の真中に大きな鼻を祭り込んでいるか、どうだか受合えない。しかし談話の模様から鼻息の荒いところなどを綜合(そうごう)して考えて見ると、満更(まんざら)人の注意を惹(ひ)かぬ獅鼻(ししばな)とも思われない。女はしきりに喋舌(しゃべ)っているが相手の声が少しも聞えないのは、噂(うわさ)にきく電話というものであろう。「御前は大和(やまと)かい。明日(あした)ね、行くんだからね、鶉(うずら)の三を取っておいておくれ、いいかえ――分ったかい――なに分らない? おやいやだ。鶉の三を取るんだよ。――なんだって、――取れない? 取れないはずはない、とるんだよ――へへへへへ御冗談(ごじょうだん)をだって――何が御冗談なんだよ――いやに人をおひゃらかすよ。全体御前は誰だい。長吉(ちょうきち)だ? 長吉なんぞじゃ訳が分らない。お神さんに電話口へ出ろって御云いな――なに? 私(わたく)しで何でも弁じます?――お前は失敬だよ。妾(あた)しを誰だか知ってるのかい。金田だよ。――へへへへへ善く存じておりますだって。ほんとに馬鹿だよこの人あ。――金田だってえばさ。――なに?――毎度御贔屓(ごひいき)にあずかりましてありがとうございます?――何がありがたいんだね。御礼なんか聞きたかあないやね――おやまた笑ってるよ。お前はよっぽど愚物(ぐぶつ)だね。――仰せの通りだって?――あんまり人を馬鹿にすると電話を切ってしまうよ。いいのかい。困らないのかよ――黙ってちゃ分らないじゃないか、何とか御云いなさいな」電話は長吉の方から切ったものか何の返事もないらしい。令嬢は癇癪(かんしゃく)を起してやけに\emph{ベル}をジャラジャラと廻す。足元で狆(ちん)が驚ろいて急に吠え出す。これは迂濶(うかつ)に出来ないと、急に飛び下りて椽(えん)の下へもぐり込む。
折柄(おりから)廊下を近(ちかづ)く足音がして障子を開ける音がする。誰か来たなと一生懸命に聞いていると「御嬢様、旦那様と奥様が呼んでいらっしゃいます」と小間使らしい声がする。「知らないよ」と令嬢は剣突(けんつく)を食わせる。「ちょっと用があるから嬢(じょう)を呼んで来いとおっしゃいました」「うるさいね、知らないてば」と令嬢は第二の剣突を食わせる。「\ldots{}\ldots{}水島寒月さんの事で御用があるんだそうでございます」と小間使は気を利(き)かして機嫌を直そうとする。「寒月でも、水月でも知らないんだよ――大嫌いだわ、糸瓜(へちま)が戸迷(とまど)いをしたような顔をして」第三の剣突は、憐れなる寒月君が、留守中に頂戴する。「おや御前いつ束髪(そくはつ)に結(い)ったの」小間使はほっと一息ついて「今日(こんにち)」となるべく単簡(たんかん)な挨拶をする。「生意気だねえ、小間使の癖に」と第四の剣突を別方面から食わす。「そうして新しい半襟(はんえり)を掛けたじゃないか」「へえ、せんだって御嬢様からいただきましたので、結構過ぎて勿体(もったい)ないと思って行李(こうり)の中へしまっておきましたが、今までのがあまり汚(よご)れましたからかけ易(か)えました」「いつ、そんなものを上げた事があるの」「この御正月、白木屋へいらっしゃいまして、御求め遊ばしたので――鶯茶(うぐいすちゃ)へ相撲(すもう)の番附(ばんづけ)を染め出したのでございます。妾(あた)しには地味過ぎていやだから御前に上げようとおっしゃった、あれでございます」「あらいやだ。善く似合うのね。にくらしいわ」「恐れ入ります」「褒(ほ)めたんじゃない。にくらしいんだよ」「へえ」「そんなによく似合うものをなぜだまって貰ったんだい」「へえ」「御前にさえ、そのくらい似合うなら、妾(あた)しにだっておかしい事あないだろうじゃないか」「きっとよく御似合い遊ばします」「似あうのが分ってる癖になぜ黙っているんだい。そうしてすまして掛けているんだよ、人の悪い」剣突(けんつく)は留めどもなく連発される。このさき、事局はどう発展するかと謹聴している時、向うの座敷で「富子や、富子や」と大きな声で金田君が令嬢を呼ぶ。令嬢はやむを得ず「はい」と電話室を出て行く。吾輩より少し大きな狆(ちん)が顔の中心に眼と口を引き集めたような面(かお)をして付いて行く。吾輩は例の忍び足で再び勝手から往来へ出て、急いで主人の家に帰る。探険はまず十二分の成績(せいせき)である。
帰って見ると、奇麗な家(うち)から急に汚ない所へ移ったので、何だか日当りの善い山の上から薄黒い洞窟(どうくつ)の中へ入(はい)り込んだような心持ちがする。探険中は、ほかの事に気を奪われて部屋の装飾、襖(ふすま)、障子(しょうじ)の具合などには眼も留らなかったが、わが住居(すまい)の下等なるを感ずると同時に彼(か)のいわゆる月並(つきなみ)が恋しくなる。教師よりもやはり実業家がえらいように思われる。吾輩も少し変だと思って、例の尻尾(しっぽ)に伺いを立てて見たら、その通りその通りと尻尾の先から御託宣(ごたくせん)があった。座敷へ這入(はい)って見ると驚いたのは迷亭先生まだ帰らない、巻煙草(まきたばこ)の吸い殻を蜂の巣のごとく火鉢の中へ突き立てて、大胡坐(おおあぐら)で何か話し立てている。いつの間(ま)にか寒月君さえ来ている。主人は手枕をして天井の雨洩(あまもり)を余念もなく眺めている。あいかわらず太平の逸民の会合である。
「寒月君、君の事を譫語(うわごと)にまで言った婦人の名は、当時秘密であったようだが、もう話しても善かろう」と迷亭がからかい出す。「御話しをしても、私だけに関する事なら差支(さしつか)えないんですが、先方の迷惑になる事ですから」「まだ駄目かなあ」「それに○○博士夫人に約束をしてしまったもんですから」「他言をしないと云う約束かね」「ええ」と寒月君は例のごとく羽織の紐(ひも)をひねくる。その紐は売品にあるまじき紫色である。「その紐の色は、ちと天保調(てんぽうちょう)だな」と主人が寝ながら云う。主人は金田事件などには無頓着である。「そうさ、到底(とうてい)日露戦争時代のものではないな。陣笠(じんがさ)に立葵(たちあおい)の紋の付いたぶっ割(さ)き羽織でも着なくっちゃ納まりの付かない紐だ。織田信長が聟入(むこいり)をするとき頭の髪を茶筌(ちゃせん)に結(い)ったと云うがその節用いたのは、たしかそんな紐だよ」と迷亭の文句はあいかわらず長い。「実際これは爺(じじい)が長州征伐の時に用いたのです」と寒月君は真面目である。「もういい加減に博物館へでも献納してはどうだ。\emph{首縊りの力学}の演者、理学士水島寒月君ともあろうものが、売れ残りの旗本のような出(い)で立(たち)をするのはちと体面に関する訳だから」「御忠告の通りに致してもいいのですが、この紐が大変よく似合うと云ってくれる人もありますので――」「誰だい、そんな趣味のない事を云うのは」と主人は寝返りを打ちながら大きな声を出す。「それは御存じの方なんじゃないんで――」「御存じでなくてもいいや、一体誰だい」「去る女性(にょしょう)なんです」「ハハハハハよほど茶人だなあ、当てて見ようか、やはり隅田川の底から君の名を呼んだ女なんだろう、その羽織を着てもう一返御駄仏(おだぶつ)を極(き)め込んじゃどうだい」と迷亭が横合から飛び出す。「へへへへへもう水底から呼んではおりません。ここから乾(いぬい)の方角にあたる清浄(しょうじょう)な世界で\ldots{}\ldots{}」「あんまり清浄でもなさそうだ、毒々しい鼻だぜ」「へえ?」と寒月は不審な顔をする。「向う横丁の鼻がさっき押しかけて来たんだよ、ここへ、実に僕等二人は驚いたよ、ねえ苦沙弥君」「うむ」と主人は寝ながら茶を飲む。「鼻って誰の事です」「君の親愛なる久遠(くおん)の女性(にょしょう)の御母堂様だ」「へえー」「金田の妻(さい)という女が君の事を聞きに来たよ」と主人が真面目に説明してやる。驚くか、嬉しがるか、恥ずかしがるかと寒月君の様子を窺(うかが)って見ると別段の事もない。例の通り静かな調子で「どうか私に、あの娘を貰ってくれと云う依頼なんでしょう」と、また紫の紐をひねくる。「ところが大違さ。その御母堂なるものが偉大なる鼻の所有主(ぬし)でね\ldots{}\ldots{}」迷亭が半(なか)ば言い懸けると、主人が「おい君、僕はさっきから、あの鼻について俳体詩(はいたいし)を考えているんだがね」と木に竹を接(つ)いだような事を云う。隣の室(へや)で妻君がくすくす笑い出す。「随分君も呑気(のんき)だなあ出来たのかい」「少し出来た。第一句が\emph{この顔に鼻祭り}と云うのだ」「それから?」「次が\emph{この鼻に神酒供え}というのさ」「次の句は?」「まだそれぎりしか出来ておらん」「面白いですな」と寒月君がにやにや笑う。「次へ\emph{穴二つ幽かなり}と付けちゃどうだ」と迷亭はすぐ出来る。すると寒月が「\emph{奥深く毛も見えず}はいけますまいか」と各々(おのおの)出鱈目(でたらめ)を並べていると、垣根に近く、往来で「今戸焼(いまどやき)の狸(たぬき)今戸焼の狸」と四五人わいわい云う声がする。主人も迷亭もちょっと驚ろいて表の方を、垣の隙(すき)からすかして見ると「ワハハハハハ」と笑う声がして遠くへ散る足の音がする。「今戸焼の狸というな何だい」と迷亭が不思議そうに主人に聞く。「何だか分らん」と主人が答える。「なかなか振(ふる)っていますな」と寒月君が批評を加える。迷亭は何を思い出したか急に立ち上って「吾輩は年来美学上の見地からこの鼻について研究した事がございますから、その一斑(いっぱん)を披瀝(ひれき)して、御両君の清聴を煩(わずら)わしたいと思います」と演舌の真似をやる。主人はあまりの突然にぼんやりして無言のまま迷亭を見ている。寒月は「是非承(うけたまわ)りたいものです」と小声で云う。「いろいろ調べて見ましたが鼻の起源はどうも確(しか)と分りません。第一の不審は、もしこれを実用上の道具と仮定すれば穴が二つでたくさんである。何もこんなに横風(おうふう)に真中から突き出して見る必用がないのである。ところがどうしてだんだん御覧のごとく斯様(かよう)にせり出して参ったか」と自分の鼻を抓(つま)んで見せる。「あんまりせり出してもおらんじゃないか」と主人は御世辞のないところを云う。「とにかく引っ込んではおりませんからな。ただ二個の孔(あな)が併(なら)んでいる状体と混同なすっては、誤解を生ずるに至るかも計られませんから、予(あらかじ)め御注意をしておきます。――で愚見によりますと鼻の発達は吾々人間が鼻汁(はな)をかむと申す微細なる行為の結果が自然と蓄積してかく著明なる現象を呈出したものでございます」「佯(いつわ)りのない愚見だ」とまた主人が寸評を挿入(そうにゅう)する。「御承知の通り鼻汁(はな)をかむ時は、是非鼻を抓みます、鼻を抓んで、ことにこの局部だけに刺激を与えますと、進化論の大原則によって、この局部はこの刺激に応ずるがため他に比例して不相当な発達を致します。皮も自然堅くなります、肉も次第に硬(かた)くなります。ついに凝(こ)って骨となります」「それは少し――そう自由に肉が骨に一足飛に変化は出来ますまい」と理学士だけあって寒月君が抗議を申し込む。迷亭は何喰わぬ顔で陳(の)べ続ける。「いや御不審はごもっともですが論より証拠この通り骨があるから仕方がありません。すでに骨が出来る。骨は出来ても鼻汁(はな)は出ますな。出ればかまずにはいられません。この作用で骨の左右が削(けず)り取られて細い高い隆起と変化して参ります――実に恐ろしい作用です。点滴(てんてき)の石を穿(うが)つがごとく、賓頭顱(びんずる)の頭が自(おのず)から光明を放つがごとく、不思議薫(ふしぎくん)不思議臭(ふしぎしゅう)の喩(たとえ)のごとく、斯様(かよう)に鼻筋が通って堅くなります」{[#「なります」」は底本では「なります。」]}「それでも君のなんぞ、ぶくぶくだぜ」「演者自身の局部は回護(かいご)の恐れがありますから、わざと論じません。かの金田の御母堂の持たせらるる鼻のごときは、もっとも発達せるもっとも偉大なる天下の珍品として御両君に紹介しておきたいと思います」寒月君は思わずヒヤヤヤと云う。「しかし物も極度に達しますと偉観には相違ございませんが何となく怖(おそろ)しくて近づき難いものであります。あの鼻梁(びりょう)などは素晴しいには違いございませんが、少々峻嶮(しゅんけん)過ぎるかと思われます。古人のうちにてもソクラチス、ゴールドスミスもしくはサッカレーの鼻などは構造の上から云うと随分申し分はございましょうがその申し分のあるところに愛嬌(あいきょう)がございます。鼻高きが故に貴(たっと)からず、奇(き)なるがために貴しとはこの故でもございましょうか。下世話(げせわ)にも鼻より団子と申しますれば美的価値から申しますとまず迷亭くらいのところが適当かと存じます」寒月と主人は「フフフフ」と笑い出す。迷亭自身も愉快そうに笑う。「さてただ今(いま)まで弁じましたのは――」「先生\emph{弁じました}は少し講釈師のようで下品ですから、よしていただきましょう」と寒月君は先日の復讐(ふくしゅう)をやる。「さようしからば顔を洗って出直しましょうかな。――ええ――これから鼻と顔の権衡(けんこう)に一言(いちごん)論及したいと思います。他に関係なく単独に鼻論をやりますと、かの御母堂などはどこへ出しても恥ずかしからぬ鼻――鞍馬山(くらまやま)で展覧会があっても恐らく一等賞だろうと思われるくらいな鼻を所有していらせられますが、悲しいかなあれは眼、口、その他の諸先生と何等の相談もなく出来上った鼻であります。ジュリアス・シーザーの鼻は大したものに相違ございません。しかしシーザーの鼻を鋏(はさみ)でちょん切って、当家の猫の顔へ安置したらどんな者でございましょうか。喩(たと)えにも猫の額(ひたい)と云うくらいな地面へ、英雄の鼻柱が突兀(とっこつ)として聳(そび)えたら、碁盤の上へ奈良の大仏を据(す)え付けたようなもので、少しく比例を失するの極、その美的価値を落す事だろうと思います。御母堂の鼻はシーザーのそれのごとく、正(まさ)しく英姿颯爽(えいしさっそう)たる隆起に相違ございません。しかしその周囲を囲繞(いにょう)する顔面的条件は如何(いかが)な者でありましょう。無論当家の猫のごとく劣等ではない。しかし癲癇病(てんかんや)みの\emph{御かめ}のごとく眉(まゆ)の根に八字を刻んで、細い眼を釣るし上げらるるのは事実であります。諸君、この顔にしてこの鼻ありと嘆ぜざるを得んではありませんか」迷亭の言葉が少し途切れる途端(とたん)、裏の方で「まだ鼻の話しをしているんだよ。何てえ剛突(ごうつ)く張(ばり)だろう」と云う声が聞える。「車屋の神さんだ」と主人が迷亭に教えてやる。迷亭はまたやり初める。「計らざる裏手にあたって、新たに異性の傍聴者のある事を発見したのは演者の深く名誉と思うところであります。ことに宛転(えんてん)たる嬌音(きょうおん)をもって、乾燥なる講筵(こうえん)に一点の艶味(えんみ)を添えられたのは実に望外の幸福であります。なるべく通俗的に引き直して佳人淑女(かじんしゅくじょ)の眷顧(けんこ)に背(そむ)かざらん事を期する訳でありますが、これからは少々力学上の問題に立ち入りますので、勢(いきおい)御婦人方には御分りにくいかも知れません、どうか御辛防(ごしんぼう)を願います」寒月君は力学と云う語を聞いてまたにやにやする。「私の証拠立てようとするのは、この鼻とこの顔は到底調和しない。ツァイシングの\emph{黄金律}を失していると云う事なんで、それを厳格に力学上の公式から演繹(えんえき)して御覧に入れようと云うのであります。まずHを鼻の高さとします。αは鼻と顔の平面の交叉より生ずる角度であります。Wは無論鼻の重量と御承知下さい。どうです大抵お分りになりましたか。\ldots{}\ldots{}」「分るものか」と主人が云う。「寒月君はどうだい」「私にもちと分りかねますな」「そりゃ困ったな。苦沙弥(くしゃみ)はとにかく、君は理学士だから分るだろうと思ったのに。この式が演説の首脳なんだからこれを略しては今までやった甲斐(かい)がないのだが――まあ仕方がない。公式は略して結論だけ話そう」「結論があるか」と主人が不思議そうに聞く。「当り前さ結論のない演舌は、デザートのない西洋料理のようなものだ、――いいか両君能(よ)く聞き給え、これからが結論だぜ。――さて以上の公式にウィルヒョウ、ワイスマン諸家の説を参酌して考えて見ますと、先天的形体の遺伝は無論の事許さねばなりません。またこの形体に追陪(ついばい)して起る心意的状況は、たとい後天性は遺伝するものにあらずとの有力なる説あるにも関せず、ある程度までは必然の結果と認めねばなりません。従ってかくのごとく身分に不似合なる鼻の持主の生んだ子には、その鼻にも何か異状がある事と察せられます。寒月君などは、まだ年が御若いから金田令嬢の鼻の構造において特別の異状を認められんかも知れませんが、かかる遺伝は潜伏期の長いものでありますから、いつ何時(なんどき)気候の劇変と共に、急に発達して御母堂のそれのごとく、咄嗟(とっさ)の間(かん)に膨脹(ぼうちょう)するかも知れません、それ故にこの御婚儀は、迷亭の学理的論証によりますと、今の中御断念になった方が安全かと思われます、これには当家の御主人は無論の事、そこに寝ておらるる猫又殿(ねこまたどの)にも御異存は無かろうと存じます」主人はようよう起き返って「そりゃ無論さ。あんなものの娘を誰が貰うものか。寒月君もらっちゃいかんよ」と大変熱心に主張する。吾輩もいささか賛成の意を表するためににゃーにゃーと二声ばかり鳴いて見せる。寒月君は別段騒いだ様子もなく「先生方の御意向がそうなら、私は断念してもいいんですが、もし当人がそれを気にして病気にでもなったら罪ですから――」「ハハハハハ艶罪(えんざい)と云う訳(わけ)だ」主人だけは大(おおい)にむきになって「そんな馬鹿があるものか、あいつの娘なら碌(ろく)な者でないに極(きま)ってらあ。初めて人のうちへ来ておれをやり込めに掛った奴だ。傲慢(ごうまん)な奴だ」と独(ひと)りでぷんぷんする。するとまた垣根のそばで三四人が「ワハハハハハ」と云う声がする。一人が「高慢ちきな唐変木(とうへんぼく)だ」と云うと一人が「もっと大きな家(うち)へ這入(はい)りてえだろう」と云う。また一人が「御気の毒だが、いくら威張ったって蔭弁慶(かげべんけい)だ」と大きな声をする。主人は椽側(えんがわ)へ出て負けないような声で「やかましい、何だわざわざそんな塀(へい)の下へ来て」と怒鳴(どな)る。「ワハハハハハサヴェジ・チーだ、サヴェジ・チーだ」と口々に罵(のの)しる。主人は大(おおい)に逆鱗(げきりん)の体(てい)で突然起(た)ってステッキを持って、往来へ飛び出す。迷亭は手を拍(う)って「面白い、やれやれ」と云う。寒月は羽織の紐を撚(ひね)ってにやにやする。吾輩は主人のあとを付けて垣の崩れから往来へ出て見たら、真中に主人が手持無沙汰にステッキを突いて立っている。人通りは一人もない、ちょっと狐(きつね)に抓(つま)まれた体(てい)である。
\end{document}

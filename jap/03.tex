\chapter*{三}
 三毛子は死ぬ。黒は相手にならず、いささか寂寞(せきばく)の感はあるが、幸い人間に知己(ちき)が出来たのでさほど退屈とも思わぬ。せんだっては主人の許(もと)へ吾輩の写真を送ってくれと手紙で依頼した男がある。この間は岡山の名産吉備団子(きびだんご)をわざわざ吾輩の名宛で届けてくれた人がある。だんだん人間から同情を寄せらるるに従って、己(おのれ)が猫である事はようやく忘却してくる。猫よりはいつの間(ま)にか人間の方へ接近して来たような心持になって、同族を糾合(きゅうごう)して二本足の先生と雌雄(しゆう)を決しようなどと云(い)う量見は昨今のところ毛頭(もうとう)ない。それのみか折々は吾輩もまた人間世界の一人だと思う折さえあるくらいに進化したのはたのもしい。あえて同族を軽蔑(けいべつ)する次第ではない。ただ性情の近きところに向って一身の安きを置くは勢(いきおい)のしからしむるところで、これを変心とか、軽薄とか、裏切りとか評せられてはちと迷惑する。かような言語を弄(ろう)して人を罵詈(ばり)するものに限って融通の利(き)かぬ貧乏性の男が多いようだ。こう猫の習癖を脱化して見ると\emph{三毛子}や\emph{黒}の事ばかり荷厄介にしている訳には行かん。やはり人間同等の気位(きぐらい)で彼等の思想、言行を評隲(ひょうしつ)したくなる。これも無理はあるまい。ただそのくらいな見識を有している吾輩をやはり一般猫児(びょうじ)の毛の生(は)えたものくらいに思って、主人が吾輩に一言(いちごん)の挨拶もなく、吉備団子(きびだんご)をわが物顔に喰い尽したのは残念の次第である。写真もまだ撮(と)って送らぬ容子(ようす)だ。これも不平と云えば不平だが、主人は主人、吾輩は吾輩で、相互の見解が自然異(こと)なるのは致し方もあるまい。吾輩はどこまでも人間になりすましているのだから、交際をせぬ猫の動作は、どうしてもちょいと筆に上(のぼ)りにくい。迷亭、寒月諸先生の評判だけで御免蒙(こうむ)る事に致そう。\\
 今日は上天気の日曜なので、主人はのそのそ書斎から出て来て、吾輩の傍(そば)へ筆硯(ふですずり)と原稿用紙を並べて腹這(はらばい)になって、しきりに何か唸(うな)っている。大方草稿を書き卸(おろ)す序開(じょびら)きとして妙な声を発するのだろうと注目していると、ややしばらくして筆太(ふでぶと)に「香一\includegraphics{../../../gaiji/1-87/1-87-40.png}(こういっしゅ)」とかいた。はてな詩になるか、俳句になるか、香一\includegraphics{../../../gaiji/1-87/1-87-40.png}とは、主人にしては少し洒落(しゃれ)過ぎているがと思う間もなく、彼は香一\includegraphics{../../../gaiji/1-87/1-87-40.png}を書き放しにして、新たに行(ぎょう)を改めて「さっきから天然居士(てんねんこじ)の事をかこうと考えている」と筆を走らせた。筆はそれだけではたと留ったぎり動かない。主人は筆を持って首を捻(ひね)ったが別段名案もないものと見えて筆の穂を甞(な)めだした。唇が真黒になったと見ていると、今度はその下へちょいと丸をかいた。丸の中へ点を二つうって眼をつける。真中へ小鼻の開いた鼻をかいて、真一文字に口を横へ引張った、これでは文章でも俳句でもない。主人も自分で愛想(あいそ)が尽きたと見えて、そこそこに顔を塗り消してしまった。主人はまた行(ぎょう)を改める。彼の考によると行さえ改めれば詩か賛か語か録か何(なん)かになるだろうとただ宛(あて)もなく考えているらしい。やがて「天然居士は空間を研究し、論語を読み、焼芋(やきいも)を食い、鼻汁(はな)を垂らす人である」と言文一致体で一気呵成(いっきかせい)に書き流した、何となくごたごたした文章である。それから主人はこれを遠慮なく朗読して、いつになく「ハハハハ面白い」と笑ったが「鼻汁(はな)を垂らすのは、ちと酷(こく)だから消そう」とその句だけへ棒を引く。一本ですむところを二本引き三本引き、奇麗な併行線(へいこうせん)を描(か)く、線がほかの行(ぎょう)まで食(は)み出しても構わず引いている。線が八本並んでもあとの句が出来ないと見えて、今度は筆を捨てて髭(ひげ)を捻(ひね)って見る。文章を髭から捻り出して御覧に入れますと云う見幕(けんまく)で猛烈に捻ってはねじ上げ、ねじ下ろしているところへ、茶の間から妻君(さいくん)が出て来てぴたりと主人の鼻の先へ坐(す)わる。「あなたちょっと」と呼ぶ。「なんだ」と主人は水中で銅鑼(どら)を叩(たた)くような声を出す。返事が気に入らないと見えて妻君はまた「あなたちょっと」と出直す。「なんだよ」と今度は鼻の穴へ親指と人さし指を入れて鼻毛をぐっと抜く。「今月はちっと足りませんが\ldots{}\ldots{}」「足りんはずはない、医者へも薬礼はすましたし、本屋へも先月払ったじゃないか。今月は余らなければならん」とすまして抜き取った鼻毛を天下の奇観のごとく眺(なが)めている。「それでもあなたが御飯を召し上らんで麺麭(パン)を御食(おた)べになったり、ジャムを御舐(おな)めになるものですから」「元来ジャムは幾缶(いくかん)舐めたのかい」「今月は八つ入(い)りましたよ」「八つ? そんなに舐めた覚えはない」「あなたばかりじゃありません、子供も舐めます」「いくら舐めたって五六円くらいなものだ」と主人は平気な顔で鼻毛を一本一本丁寧に原稿紙の上へ植付ける。肉が付いているのでぴんと針を立てたごとくに立つ。主人は思わぬ発見をして感じ入った体(てい)で、ふっと吹いて見る。粘着力(ねんちゃくりょく)が強いので決して飛ばない。「いやに頑固(がんこ)だな」と主人は一生懸命に吹く。「ジャムばかりじゃないんです、ほかに買わなけりゃ、ならない物もあります」と妻君は大(おおい)に不平な気色(けしき)を両頬に漲(みなぎ)らす。「あるかも知れないさ」と主人はまた指を突っ込んでぐいと鼻毛を抜く。赤いのや、黒いのや、種々の色が交(まじ)る中に一本真白なのがある。大に驚いた様子で穴の開(あ)くほど眺めていた主人は指の股へ挟んだまま、その鼻毛を妻君の顔の前へ出す。「あら、いやだ」と妻君は顔をしかめて、主人の手を突き戻す。「ちょっと見ろ、鼻毛の白髪(しらが)だ」と主人は大に感動した様子である。さすがの妻君も笑いながら茶の間へ這入(はい)る。経済問題は断念したらしい。主人はまた天然居士(てんねんこじ)に取り懸(かか)る。\\
 鼻毛で妻君を追払った主人は、まずこれで安心と云わぬばかりに鼻毛を抜いては原稿をかこうと焦(あせ)る体(てい)であるがなかなか筆は動かない。「\emph{焼芋を食う}も蛇足(だそく)だ、割愛(かつあい)しよう」とついにこの句も抹殺(まっさつ)する。「\emph{香一\includegraphics{../../../gaiji/1-87/1-87-40.png}}もあまり唐突(とうとつ)だから已(や)めろ」と惜気もなく筆誅(ひっちゅう)する。余す所は「天然居士は空間を研究し論語を読む人である」と云う一句になってしまった。主人はこれでは何だか簡単過ぎるようだなと考えていたが、ええ面倒臭い、文章は御廃(おはい)しにして、銘だけにしろと、筆を十文字に揮(ふる)って原稿紙の上へ下手な文人画の蘭を勢よくかく。せっかくの苦心も一字残らず落第となった。それから裏を返して「空間に生れ、空間を究(きわ)め、空間に死す。空たり間たり天然居士(てんねんこじ)噫(ああ)」と意味不明な語を連(つら)ねているところへ例のごとく迷亭が這入(はい)って来る。迷亭は人の家(うち)も自分の家も同じものと心得ているのか案内も乞わず、ずかずか上ってくる、のみならず時には勝手口から飄然(ひょうぜん)と舞い込む事もある、心配、遠慮、気兼(きがね)、苦労、を生れる時どこかへ振り落した男である。\\
「また\emph{巨人引力}かね」と立ったまま主人に聞く。「そう、いつでも\emph{巨人引力}ばかり書いてはおらんさ。\emph{天然居士}の墓銘を撰(せん)しているところなんだ」と大袈裟(おおげさ)な事を云う。「\emph{天然居士}と云うなあやはり\emph{偶然童子}のような戒名かね」と迷亭は不相変(あいかわらず)出鱈目(でたらめ)を云う。「\emph{偶然童子}と云うのもあるのかい」「なに有りゃしないがまずその見当(けんとう)だろうと思っていらあね」「\emph{偶然童子}と云うのは僕の知ったものじゃないようだが\emph{天然居士}と云うのは、君の知ってる男だぜ」「一体だれが\emph{天然居士}なんて名を付けてすましているんだい」「例の曾呂崎(そろさき)の事だ。卒業して大学院へ這入って\emph{空間論}と云う題目で研究していたが、あまり勉強し過ぎて腹膜炎で死んでしまった。曾呂崎はあれでも僕の親友なんだからな」「親友でもいいさ、決して悪いと云やしない。しかしその曾呂崎を天然居士に変化させたのは一体誰の所作(しょさ)だい」「僕さ、僕がつけてやったんだ。元来坊主のつける戒名ほど俗なものは無いからな」と天然居士はよほど雅(が)な名のように自慢する。迷亭は笑いながら「まあその墓碑銘(ぼひめい)と云う奴を見せ給え」と原稿を取り上げて「何だ\ldots{}\ldots{}空間に生れ、空間を究(きわ)め、空間に死す。空たり間たり天然居士噫(ああ)」と大きな声で読み上(あげ)る。「なるほどこりゃあ善(い)い、天然居士相当のところだ」主人は嬉しそうに「善いだろう」と云う。「この墓銘(ぼめい)を沢庵石(たくあんいし)へ彫(ほ)り付けて本堂の裏手へ力石(ちからいし)のように抛(ほう)り出して置くんだね。雅(が)でいいや、天然居士も浮かばれる訳だ」「僕もそうしようと思っているのさ」と主人は至極(しごく)真面目に答えたが「僕あちょっと失敬するよ、じき帰るから猫にでもからかっていてくれ給え」と迷亭の返事も待たず風然(ふうぜん)と出て行く。\\
 計らずも迷亭先生の接待掛りを命ぜられて無愛想(ぶあいそ)な顔もしていられないから、ニャーニャーと愛嬌(あいきょう)を振り蒔(ま)いて膝(ひざ)の上へ這(は)い上(あが)って見た。すると迷亭は「イヨー大分(だいぶ)肥(ふと)ったな、どれ」と無作法(ぶさほう)にも吾輩の襟髪(えりがみ)を攫(つか)んで宙へ釣るす。「あと足をこうぶら下げては、鼠(ねずみ)は取れそうもない、\ldots{}\ldots{}どうです奥さんこの猫は鼠を捕りますかね」と吾輩ばかりでは不足だと見えて、隣りの室(へや)の妻君に話しかける。「鼠どころじゃございません。御雑煮(おぞうに)を食べて踊りをおどるんですもの」と妻君は飛んだところで旧悪を暴(あば)く。吾輩は宙乗(ちゅうの)りをしながらも少々極りが悪かった。迷亭はまだ吾輩を卸(おろ)してくれない。「なるほど踊りでもおどりそうな顔だ。奥さんこの猫は油断のならない相好(そうごう)ですぜ。昔(むか)しの草双紙(くさぞうし)にある猫又(ねこまた)に似ていますよ」と勝手な事を言いながら、しきりに細君(さいくん)に話しかける。細君は迷惑そうに針仕事の手をやめて座敷へ出てくる。\\
「どうも御退屈様、もう帰りましょう」と茶を注(つ)ぎ易(か)えて迷亭の前へ出す。「どこへ行ったんですかね」「どこへ参るにも断わって行った事の無い男ですから分りかねますが、大方御医者へでも行ったんでしょう」「甘木さんですか、甘木さんもあんな病人に捕(つら)まっちゃ災難ですな」「へえ」と細君は挨拶のしようもないと見えて簡単な答えをする。迷亭は一向(いっこう)頓着しない。「近頃はどうです、少しは胃の加減が能(い)いんですか」「能(い)いか悪いか頓(とん)と分りません、いくら甘木さんにかかったって、あんなにジャムばかり甞(な)めては胃病の直る訳がないと思います」と細君は先刻(せんこく)の不平を暗(あん)に迷亭に洩(も)らす。「そんなにジャムを甞めるんですかまるで小供のようですね」「ジャムばかりじゃないんで、この頃は胃病の薬だとか云って大根卸(だいこおろ)しを無暗(むやみ)に甞めますので\ldots{}\ldots{}」「驚ろいたな」と迷亭は感嘆する。「何でも大根卸(だいこおろし)の中にはジヤスターゼが有るとか云う話しを新聞で読んでからです」「なるほどそれでジャムの損害を償(つぐな)おうと云う趣向ですな。なかなか考えていらあハハハハ」と迷亭は細君の訴(うったえ)を聞いて大(おおい)に愉快な気色(けしき)である。「この間などは赤ん坊にまで甞めさせまして\ldots{}\ldots{}」「ジャムをですか」「いいえ大根卸(だいこおろし)を\ldots{}\ldots{}あなた。坊や御父様がうまいものをやるからおいでてって、――たまに小供を可愛がってくれるかと思うとそんな馬鹿な事ばかりするんです。二三日前(にさんちまえ)には中の娘を抱いて箪笥(たんす)の上へあげましてね\ldots{}\ldots{}」「どう云う趣向がありました」と迷亭は何を聞いても趣向ずくめに解釈する。「なに趣向も何も有りゃしません、ただその上から飛び下りて見ろと云うんですわ、三つや四つの女の子ですもの、そんな御転婆(おてんば)な事が出来るはずがないです」「なるほどこりゃ趣向が無さ過ぎましたね。しかしあれで腹の中は毒のない善人ですよ」「あの上腹の中に毒があっちゃ、辛防(しんぼう)は出来ませんわ」と細君は大(おおい)に気焔(きえん)を揚げる。「まあそんなに不平を云わんでも善いでさあ。こうやって不足なくその日その日が暮らして行かれれば上(じょう)の分(ぶん)ですよ。苦沙弥君(くしゃみくん)などは道楽はせず、服装にも構わず、地味に世帯向(しょたいむ)きに出来上った人でさあ」と迷亭は柄(がら)にない説教を陽気な調子でやっている。「ところがあなた大違いで\ldots{}\ldots{}」「何か内々でやりますかね。油断のならない世の中だからね」と飄然(ひょうぜん)とふわふわした返事をする。「ほかの道楽はないですが、無暗(むやみ)に読みもしない本ばかり買いましてね。それも善い加減に見計(みはか)らって買ってくれると善いんですけれど、勝手に丸善へ行っちゃ何冊でも取って来て、月末になると知らん顔をしているんですもの、去年の暮なんか、月々のが溜(たま)って大変困りました」「なあに書物なんか取って来るだけ取って来て構わんですよ。払いをとりに来たら今にやる今にやると云っていりゃ帰ってしまいまさあ」「それでも、そういつまでも引張る訳にも参りませんから」と妻君は憮然(ぶぜん)としている。「それじゃ、訳を話して書籍費(しょじゃくひ)を削減させるさ」「どうして、そんな言(こと)を云ったって、なかなか聞くものですか、この間などは貴様は学者の妻(さい)にも似合わん、毫(ごう)も書籍(しょじゃく)の価値を解しておらん、昔(むか)し羅馬(ローマ)にこう云う話しがある。後学のため聞いておけと云うんです」「そりゃ面白い、どんな話しですか」迷亭は乗気になる。細君に同情を表しているというよりむしろ好奇心に駆(か)られている。「何んでも昔し羅馬(ローマ)に樽金(たるきん)とか云う王様があって\ldots{}\ldots{}」「樽金(たるきん)? 樽金はちと妙ですぜ」「私は唐人(とうじん)の名なんかむずかしくて覚えられませんわ。何でも七代目なんだそうです」「なるほど七代目樽金は妙ですな。ふんその七代目樽金がどうかしましたかい」「あら、あなたまで冷かしては立つ瀬がありませんわ。知っていらっしゃるなら教えて下さればいいじゃありませんか、人の悪い」と、細君は迷亭へ食って掛る。「何冷かすなんて、そんな人の悪い事をする僕じゃない。ただ七代目樽金は振(ふる)ってると思ってね\ldots{}\ldots{}ええお待ちなさいよ羅馬(ローマ)の七代目の王様ですね、こうっとたしかには覚えていないがタークイン・ゼ・プラウドの事でしょう。まあ誰でもいい、その王様がどうしました」「その王様の所へ一人の女が本を九冊持って来て買ってくれないかと云ったんだそうです」「なるほど」「王様がいくらなら売るといって聞いたら大変な高い事を云うんですって、あまり高いもんだから少し負けないかと云うとその女がいきなり九冊の内の三冊を火にくべて焚(や)いてしまったそうです」「惜しい事をしましたな」「その本の内には予言か何かほかで見られない事が書いてあるんですって」「へえー」「王様は九冊が六冊になったから少しは価(ね)も減ったろうと思って六冊でいくらだと聞くと、やはり元の通り一文も引かないそうです、それは乱暴だと云うと、その女はまた三冊をとって火にくべたそうです。王様はまだ未練があったと見えて、余った三冊をいくらで売ると聞くと、やはり九冊分のねだんをくれと云うそうです。九冊が六冊になり、六冊が三冊になっても代価は、元の通り一厘も引かない、それを引かせようとすると、残ってる三冊も火にくべるかも知れないので、王様はとうとう高い御金を出して焚(や)け余(あま)りの三冊を買ったんですって\ldots{}\ldots{}どうだこの話しで少しは書物のありがた味(み)が分ったろう、どうだと力味(りき)むのですけれど、私にゃ何がありがたいんだか、まあ分りませんね」と細君は一家の見識を立てて迷亭の返答を促(うな)がす。さすがの迷亭も少々窮したと見えて、袂(たもと)からハンケチを出して吾輩をじゃらしていたが「しかし奥さん」と急に何か考えついたように大きな声を出す。「あんなに本を買って矢鱈(やたら)に詰め込むものだから人から少しは学者だとか何とか云われるんですよ。この間ある文学雑誌を見たら苦沙弥君(くしゃみくん)の評が出ていましたよ」「ほんとに?」と細君は向き直る。主人の評判が気にかかるのは、やはり夫婦と見える。「何とかいてあったんです」「なあに二三行ばかりですがね。苦沙弥君の文は行雲流水(こううんりゅうすい)のごとしとありましたよ」細君は少しにこにこして「それぎりですか」「その次にね――出ずるかと思えば忽(たちま)ち消え、逝(ゆ)いては長(とこしな)えに帰るを忘るとありましたよ」細君は妙な顔をして「賞(ほ)めたんでしょうか」と心元ない調子である。「まあ賞めた方でしょうな」と迷亭は済ましてハンケチを吾輩の眼の前にぶら下げる。「書物は商買道具で仕方もござんすまいが、よっぽど偏屈(へんくつ)でしてねえ」迷亭はまた別途の方面から来たなと思って「偏屈は少々偏屈ですね、学問をするものはどうせあんなですよ」と調子を合わせるような弁護をするような不即不離の妙答をする。「せんだってなどは学校から帰ってすぐわきへ出るのに着物を着換えるのが面倒だものですから、あなた外套(がいとう)も脱がないで、机へ腰を掛けて御飯を食べるのです。御膳(おぜん)を火燵櫓(こたつやぐら)の上へ乗せまして――私は御櫃(おはち)を抱(かか)えて坐っておりましたがおかしくって\ldots{}\ldots{}」「何だかハイカラの首実検のようですな。しかしそんなところが苦沙弥君の苦沙弥君たるところで――とにかく月並(つきなみ)でない」と切(せつ)ない褒(ほ)め方をする。「月並か月並でないか女には分りませんが、なんぼ何でも、あまり乱暴ですわ」「しかし月並より好いですよ」と無暗に加勢すると細君は不満な様子で「一体、月並月並と皆さんが、よくおっしゃいますが、どんなのが月並なんです」と開き直って月並の定義を質問する、「月並ですか、月並と云うと――さようちと説明しにくいのですが\ldots{}\ldots{}」「そんな曖昧(あいまい)なものなら月並だって好さそうなものじゃありませんか」と細君は女人(にょにん)一流の論理法で詰め寄せる。「曖昧じゃありませんよ、ちゃんと分っています、ただ説明しにくいだけの事でさあ」「何でも自分の嫌いな事を月並と云うんでしょう」と細君は我(われ)知らず穿(うが)った事を云う。迷亭もこうなると何とか月並の処置を付けなければならぬ仕儀となる。「奥さん、月並と云うのはね、まず\emph{年は二八か二九からぬ}と\emph{言わず語らず物思い}の間(あいだ)に寝転んでいて、\emph{この日や天気晴朗}とくると必ず\emph{一瓢を携えて墨堤に遊ぶ}連中(れんじゅう)を云うんです」「そんな連中があるでしょうか」と細君は分らんものだから好(いい)加減な挨拶をする。「何だかごたごたして私には分りませんわ」とついに我(が)を折る。「それじゃ馬琴(ばきん)の胴へメジョオ・ペンデニスの首をつけて一二年欧州の空気で包んでおくんですね」「そうすると月並が出来るでしょうか」迷亭は返事をしないで笑っている。「何そんな手数(てすう)のかかる事をしないでも出来ます。中学校の生徒に白木屋の番頭を加えて二で割ると立派な月並が出来上ります」「そうでしょうか」と細君は首を捻(ひね)ったまま納得(なっとく)し兼ねたと云う風情(ふぜい)に見える。\\
「君まだいるのか」と主人はいつの間(ま)にやら帰って来て迷亭の傍(そば)へ坐(す)わる。「まだいるのかはちと酷(こく)だな、すぐ帰るから待ってい給えと言ったじゃないか」「万事あれなんですもの」と細君は迷亭を顧(かえり)みる。「今君の留守中に君の逸話を残らず聞いてしまったぜ」「女はとかく多弁でいかん、人間もこの猫くらい沈黙を守るといいがな」と主人は吾輩の頭を撫(な)でてくれる。「君は赤ん坊に大根卸(だいこおろ)しを甞(な)めさしたそうだな」「ふむ」と主人は笑ったが「赤ん坊でも近頃の赤ん坊はなかなか利口だぜ。それ以来、坊や辛(から)いのはどこと聞くときっと舌を出すから妙だ」「まるで犬に芸を仕込む気でいるから残酷だ。時に寒月(かんげつ)はもう来そうなものだな」「寒月が来るのかい」と主人は不審な顔をする。「来るんだ。午後一時までに苦沙弥(くしゃみ)の家(うち)へ来いと端書(はがき)を出しておいたから」「人の都合も聞かんで勝手な事をする男だ。寒月を呼んで何をするんだい」「なあに今日のはこっちの趣向じゃない寒月先生自身の要求さ。先生何でも理学協会で演説をするとか云うのでね。その稽古をやるから僕に聴いてくれと云うから、そりゃちょうどいい苦沙弥にも聞かしてやろうと云うのでね。そこで君の家(うち)へ呼ぶ事にしておいたのさ――なあに君はひま人だからちょうどいいやね――差支(さしつか)えなんぞある男じゃない、聞くがいいさ」と迷亭は独(ひと)りで呑み込んでいる。「物理学の演説なんか僕にゃ分らん」と主人は少々迷亭の専断(せんだん)を憤(いきどお)ったもののごとくに云う。「ところがその問題がマグネ付けられたノッズルについてなどと云う乾燥無味なものじゃないんだ。\emph{首縊りの力学}と云う脱俗超凡(だつぞくちょうぼん)な演題なのだから傾聴する価値があるさ」「君は首を縊(くく)り損(そ)くなった男だから傾聴するが好いが僕なんざあ\ldots{}\ldots{}」「歌舞伎座で悪寒(おかん)がするくらいの人間だから聞かれないと云う結論は出そうもないぜ」と例のごとく軽口を叩く。妻君はホホと笑って主人を顧(かえり)みながら次の間へ退く。主人は無言のまま吾輩の頭を撫(な)でる。この時のみは非常に丁寧な撫で方であった。\\
 それから約七分くらいすると注文通り寒月君が来る。今日は晩に演舌(えんぜつ)をするというので例になく立派なフロックを着て、洗濯し立ての白襟(カラー)を聳(そび)やかして、男振りを二割方上げて、「少し後(おく)れまして」と落ちつき払って、挨拶をする。「さっきから二人で大待ちに待ったところなんだ。早速願おう、なあ君」と主人を見る。主人もやむを得ず「うむ」と生返事(なまへんじ)をする。寒月君はいそがない。「コップへ水を一杯頂戴しましょう」と云う。「いよー本式にやるのか次には拍手の請求とおいでなさるだろう」と迷亭は独りで騒ぎ立てる。寒月君は内隠(うちがく)しから草稿を取り出して徐(おもむ)ろに「稽古ですから、御遠慮なく御批評を願います」と前置をして、いよいよ演舌の御浚(おさら)いを始める。\\
「罪人を絞罪(こうざい)の刑に処すると云う事は重(おも)にアングロサクソン民族間に行われた方法でありまして、それより古代に溯(さかのぼ)って考えますと首縊(くびくく)りは重に自殺の方法として行われた者であります。猶太人中(ユダヤじんちゅう)に在(あ)っては罪人を石を抛(な)げつけて殺す習慣であったそうでございます。旧約全書を研究して見ますといわゆるハンギングなる語は罪人の死体を釣るして野獣または肉食鳥の餌食(えじき)とする意義と認められます。ヘロドタスの説に従って見ますと猶太人(ユダヤじん)はエジプトを去る以前から夜中(やちゅう)死骸を曝(さら)されることを痛く忌(い)み嫌ったように思われます。エジプト人は罪人の首を斬って胴だけを十字架に釘付(くぎづ)けにして夜中曝し物にしたそうで御座います。波斯人(ペルシャじん)は\ldots{}\ldots{}」「寒月君首縊りと縁がだんだん遠くなるようだが大丈夫かい」と迷亭が口を入れる。「これから本論に這入(はい)るところですから、少々御辛防(ごしんぼう)を願います。\ldots{}\ldots{}さて波斯人はどうかと申しますとこれもやはり処刑には磔(はりつけ)を用いたようでございます。但し生きているうちに張付(はりつ)けに致したものか、死んでから釘を打ったものかその辺(へん)はちと分りかねます\ldots{}\ldots{}」「そんな事は分らんでもいいさ」と主人は退屈そうに欠伸(あくび)をする。「まだいろいろ御話し致したい事もございますが、御迷惑であらっしゃいましょうから\ldots{}\ldots{}」「あらっしゃいましょうより、いらっしゃいましょうの方が聞きいいよ、ねえ苦沙弥君(くしゃみくん)」とまた迷亭が咎(とが)め立(だて)をすると主人は「どっちでも同じ事だ」と気のない返事をする。「さていよいよ本題に入りまして弁じます」「\emph{弁じます}なんか講釈師の云い草だ。演舌家はもっと上品な詞(ことば)を使って貰いたいね」と迷亭先生また交(ま)ぜ返す。「\emph{弁じます}が下品なら何と云ったらいいでしょう」と寒月君は少々むっとした調子で問いかける。「迷亭のは聴いているのか、交(ま)ぜ返しているのか判然しない。寒月君そんな弥次馬(やじうま)に構わず、さっさとやるが好い」と主人はなるべく早く難関を切り抜けようとする。「むっとして弁じましたる柳かな、かね」と迷亭はあいかわらず飄然(ひょうぜん)たる事を云う。寒月は思わず吹き出す。「真に処刑として絞殺を用いましたのは、私の調べました結果によりますると、オディセーの二十二巻目に出ております。即(すなわ)ち彼(か)のテレマカスがペネロピーの十二人の侍女を絞殺するという条(くだ)りでございます。希臘語(ギリシャご)で本文を朗読しても宜(よろ)しゅうございますが、ちと衒(てら)うような気味にもなりますからやめに致します。四百六十五行から、四百七十三行を御覧になると分ります」「希臘語云々(うんぬん)はよした方がいい、さも希臘語が出来ますと云わんばかりだ、ねえ苦沙弥君」「それは僕も賛成だ、そんな物欲しそうな事は言わん方が奥床(おくゆか)しくて好い」と主人はいつになく直ちに迷亭に加担する。両人(りょうにん)は毫(ごう)も希臘語が読めないのである。「それではこの両三句は今晩抜く事に致しまして次を弁じ――ええ申し上げます。\\
 この絞殺を今から想像して見ますと、これを執行するに二つの方法があります。第一は、彼(か)のテレマカスがユーミアス及びフ\includegraphics{../../../gaiji/1-06/1-06-84.png}リーシャスの援(たすけ)を藉(か)りて縄の一端を柱へ括(くく)りつけます。そしてその縄の所々へ結び目を穴に開けてこの穴へ女の頭を一つずつ入れておいて、片方の端(はじ)をぐいと引張って釣し上げたものと見るのです」「つまり西洋洗濯屋のシャツのように女がぶら下ったと見れば好いんだろう」「その通りで、それから第二は縄の一端を前のごとく柱へ括(くく)り付けて他の一端も始めから天井へ高く釣るのです。そしてその高い縄から何本か別の縄を下げて、それに結び目の輪になったのを付けて女の頸(くび)を入れておいて、いざと云う時に女の足台を取りはずすと云う趣向なのです」「たとえて云うと縄暖簾(なわのれん)の先へ提灯玉(ちょうちんだま)を釣したような景色(けしき)と思えば間違はあるまい」「提灯玉と云う玉は見た事がないから何とも申されませんが、もしあるとすればその辺(へん)のところかと思います。――それでこれから力学的に第一の場合は到底成立すべきものでないと云う事を証拠立てて御覧に入れます」「面白いな」と迷亭が云うと「うん面白い」と主人も一致する。\\
「まず女が同距離に釣られると仮定します。また一番地面に近い二人の女の首と首を繋(つな)いでいる縄はホリゾンタルと仮定します。そこでα\textsubscript{1}α\textsubscript{2}\ldots{}\ldots{}α\textsubscript{6}を縄が地平線と形づくる角度とし、T\textsubscript{1}T\textsubscript{2}\ldots{}\ldots{}T\textsubscript{6}を縄の各部が受ける力と見做(みな)し、T\textsubscript{7}=Xは縄のもっとも低い部分の受ける力とします。Wは勿論(もちろん)女の体重と御承知下さい。どうです御分りになりましたか」\\
 迷亭と主人は顔を見合せて「大抵分った」と云う。但しこの大抵と云う度合は両人(りょうにん)が勝手に作ったのだから他人の場合には応用が出来ないかも知れない。「さて多角形に関する御存じの平均性理論によりますと、下(しも)のごとく十二の方程式が立ちます。T\textsubscript{1}cosα\textsubscript{1}=T\textsubscript{2}cosα\textsubscript{2}\ldots{}\ldots{}
(1)
T\textsubscript{2}cosα\textsubscript{2}=T\textsubscript{3}cosα\textsubscript{3}\ldots{}\ldots{}
(2)
\ldots{}\ldots{}」「方程式はそのくらいで沢山だろう」と主人は乱暴な事を云う。「実はこの式が演説の首脳なんですが」と寒月君ははなはだ残り惜し気に見える。「それじゃ首脳だけは逐(お)って伺う事にしようじゃないか」と迷亭も少々恐縮の体(てい)に見受けられる。「この式を略してしまうとせっかくの力学的研究がまるで駄目になるのですが\ldots{}\ldots{}」「何そんな遠慮はいらんから、ずんずん略すさ\ldots{}\ldots{}」と主人は平気で云う。「それでは仰せに従って、無理ですが略しましょう」「それがよかろう」と迷亭が妙なところで手をぱちぱちと叩く。\\
「それから英国へ移って論じますと、ベオウルフの中に絞首架(こうしゅか)即(すなわ)ちガルガと申す字が見えますから絞罪の刑はこの時代から行われたものに違ないと思われます。ブラクストーンの説によるともし絞罪に処せられる罪人が、万一縄の具合で死に切れぬ時は再度(ふたたび)同様の刑罰を受くべきものだとしてありますが、妙な事にはピヤース・プローマンの中には仮令(たとい)兇漢でも二度絞(し)める法はないと云う句があるのです。まあどっちが本当か知りませんが、悪くすると一度で死ねない事が往々実例にあるので。千七百八十六年に有名なフ\includegraphics{../../../gaiji/1-06/1-06-84.png}ツ・ゼラルドと云う悪漢を絞めた事がありました。ところが妙なはずみで一度目には台から飛び降りるときに縄が切れてしまったのです。またやり直すと今度は縄が長過ぎて足が地面へ着いたのでやはり死ねなかったのです。とうとう三返目に見物人が手伝って往生(おうじょう)さしたと云う話しです」「やれやれ」と迷亭はこんなところへくると急に元気が出る。「本当に死に損(ぞこな)いだな」と主人まで浮かれ出す。「まだ面白い事があります首を縊(くく)ると背(せい)が一寸(いっすん)ばかり延びるそうです。これはたしかに医者が計って見たのだから間違はありません」「それは新工夫だね、どうだい苦沙弥(くしゃみ)などはちと釣って貰っちゃあ、一寸延びたら人間並になるかも知れないぜ」と迷亭が主人の方を向くと、主人は案外真面目で「寒月君、一寸くらい背(せい)が延びて生き返る事があるだろうか」と聞く。「それは駄目に極(きま)っています。釣られて脊髄(せきずい)が延びるからなんで、早く云うと背が延びると云うより壊(こわ)れるんですからね」「それじゃ、まあ止(や)めよう」と主人は断念する。\\
 演説の続きは、まだなかなか長くあって寒月君は首縊りの生理作用にまで論及するはずでいたが、迷亭が無暗に風来坊(ふうらいぼう)のような珍語を挟(はさ)むのと、主人が時々遠慮なく欠伸(あくび)をするので、ついに中途でやめて帰ってしまった。その晩は寒月君がいかなる態度で、いかなる雄弁を振(ふる)ったか遠方で起った出来事の事だから吾輩には知れよう訳がない。\\
 二三日(にさんち)は事もなく過ぎたが、或る日の午後二時頃また迷亭先生は例のごとく空々(くうくう)として偶然童子のごとく舞い込んで来た。座に着くと、いきなり「君、越智東風(おちとうふう)の高輪事件(たかなわじけん)を聞いたかい」と旅順陥落の号外を知らせに来たほどの勢を示す。「知らん、近頃は合(あ)わんから」と主人は平生(いつも)の通り陰気である。「きょうはその東風子(とうふうし)の失策物語を御報道に及ぼうと思って忙しいところをわざわざ来たんだよ」「またそんな仰山(ぎょうさん)な事を云う、君は全体不埒(ふらち)な男だ」「ハハハハハ不埒と云わんよりむしろ無埒(むらち)の方だろう。それだけはちょっと区別しておいて貰わんと名誉に関係するからな」「おんなし事だ」と主人は嘯(うそぶ)いている。純然たる天然居士の再来だ。「この前の日曜に東風子(とうふうし)が高輪泉岳寺(たかなわせんがくじ)に行ったんだそうだ。この寒いのによせばいいのに――第一今時(いまどき)泉岳寺などへ参るのはさも東京を知らない、田舎者(いなかもの)のようじゃないか」「それは東風の勝手さ。君がそれを留める権利はない」「なるほど権利は正(まさ)にない。権利はどうでもいいが、あの寺内に義士遺物保存会と云う見世物があるだろう。君知ってるか」「うんにゃ」「知らない? だって泉岳寺へ行った事はあるだろう」「いいや」「ない? こりゃ驚ろいた。道理で大変東風を弁護すると思った。江戸っ子が泉岳寺を知らないのは情(なさ)けない」「知らなくても教師は務(つと)まるからな」と主人はいよいよ天然居士になる。「そりゃ好いが、その展覧場へ東風が這入(はい)って見物していると、そこへ独逸人(ドイツじん)が夫婦連(づれ)で来たんだって。それが最初は日本語で東風に何か質問したそうだ。ところが先生例の通り独逸語が使って見たくてたまらん男だろう。そら二口三口べらべらやって見たとさ。すると存外うまく出来たんだ――あとで考えるとそれが災(わざわい)の本(もと)さね」「それからどうした」と主人はついに釣り込まれる。「独逸人が大鷹源吾(おおたかげんご)の蒔絵(まきえ)の印籠(いんろう)を見て、これを買いたいが売ってくれるだろうかと聞くんだそうだ。その時東風の返事が面白いじゃないか、日本人は清廉の君子(くんし)ばかりだから到底(とうてい)駄目だと云ったんだとさ。その辺は大分(だいぶ)景気がよかったが、それから独逸人の方では恰好(かっこう)な通弁を得たつもりでしきりに聞くそうだ」「何を?」「それがさ、何だか分るくらいなら心配はないんだが、早口で無暗(むやみ)に問い掛けるものだから少しも要領を得ないのさ。たまに分るかと思うと鳶口(とびぐち)や\emph{掛矢}の事を聞かれる。西洋の鳶口や\emph{掛矢}は先生何と翻訳して善いのか習った事が無いんだから弱(よ)わらあね」「もっともだ」と主人は教師の身の上に引き較(くら)べて同情を表する。「ところへ閑人(ひまじん)が物珍しそうにぽつぽつ集ってくる。仕舞(しまい)には東風と独逸人を四方から取り巻いて見物する。東風は顔を赤くしてへどもどする。初めの勢に引き易(か)えて先生大弱りの体(てい)さ」「結局どうなったんだい」「仕舞に東風が我慢出来なくなったと見えて\emph{さいなら}と日本語で云ってぐんぐん帰って来たそうだ、\emph{さいなら}は少し変だ君の国では\emph{さよなら}を\emph{さいなら}と云うかって聞いて見たら何やっぱり\emph{さよなら}ですが相手が西洋人だから調和を計るために、\emph{さいなら}にしたんだって、東風子は苦しい時でも調和を忘れない男だと感心した」「さいならはいいが西洋人はどうした」「西洋人はあっけに取られて茫然(ぼうぜん)と見ていたそうだハハハハ面白いじゃないか」「別段面白い事もないようだ。それをわざわざ報知(しらせ)に来る君の方がよっぽど面白いぜ」と主人は巻煙草(まきたばこ)の灰を火桶(ひおけ)の中へはたき落す。折柄(おりから)格子戸のベルが飛び上るほど鳴って「御免なさい」と鋭どい女の声がする。迷亭と主人は思わず顔を見合わせて沈黙する。\\
 主人のうちへ女客は稀有(けう)だなと見ていると、かの鋭どい声の所有主は縮緬(ちりめん)の二枚重ねを畳へ擦(す)り付けながら這入(はい)って来る。年は四十の上を少し超(こ)したくらいだろう。抜け上った生(は)え際(ぎわ)から前髪が堤防工事のように高く聳(そび)えて、少なくとも顔の長さの二分の一だけ天に向ってせり出している。眼が切り通しの坂くらいな勾配(こうばい)で、直線に釣るし上げられて左右に対立する。直線とは鯨(くじら)より細いという形容である。鼻だけは無暗に大きい。人の鼻を盗んで来て顔の真中へ据(す)え付けたように見える。三坪ほどの小庭へ招魂社(しょうこんしゃ)の石灯籠(いしどうろう)を移した時のごとく、独(ひと)りで幅を利かしているが、何となく落ちつかない。その鼻はいわゆる鍵鼻(かぎばな)で、ひと度(たび)は精一杯高くなって見たが、これではあんまりだと中途から謙遜(けんそん)して、先の方へ行くと、初めの勢に似ず垂れかかって、下にある唇を覗(のぞ)き込んでいる。かく著(いちじ)るしい鼻だから、この女が物を言うときは口が物を言うと云わんより、鼻が口をきいているとしか思われない。吾輩はこの偉大なる鼻に敬意を表するため、以来はこの女を称して鼻子(はなこ)鼻子と呼ぶつもりである。鼻子は先ず初対面の挨拶を終って「どうも結構な御住居(おすまい)ですこと」と座敷中を睨(ね)め廻わす。主人は「嘘をつけ」と腹の中で言ったまま、ぷかぷか煙草(たばこ)をふかす。迷亭は天井を見ながら「君、ありゃ雨洩(あまも)りか、板の木目(もくめ)か、妙な模様が出ているぜ」と暗に主人を促(うな)がす。「無論雨の洩りさ」と主人が答えると「結構だなあ」と迷亭がすまして云う。鼻子は社交を知らぬ人達だと腹の中で憤(いきどお)る。しばらくは三人鼎坐(ていざ)のまま無言である。\\
「ちと伺いたい事があって、参ったんですが」と鼻子は再び話の口を切る。「はあ」と主人が極めて冷淡に受ける。これではならぬと鼻子は、「実は私はつい御近所で――あの向う横丁の角屋敷(かどやしき)なんですが」「あの大きな西洋館の倉のあるうちですか、道理であすこには金田(かねだ)と云う標札(ひょうさつ)が出ていますな」と主人はようやく金田の西洋館と、金田の倉を認識したようだが金田夫人に対する尊敬の度合(どあい)は前と同様である。「実は宿(やど)が出まして、御話を伺うんですが会社の方が大変忙がしいもんですから」と今度は少し利(き)いたろうという眼付をする。主人は一向(いっこう)動じない。鼻子の先刻(さっき)からの言葉遣いが初対面の女としてはあまり存在(ぞんざい)過ぎるのですでに不平なのである。「会社でも一つじゃ無いんです、二つも三つも兼ねているんです。それにどの会社でも重役なんで――多分御存知でしょうが」これでも恐れ入らぬかと云う顔付をする。元来ここの主人は\emph{博士}とか\emph{大学教授}とかいうと非常に恐縮する男であるが、妙な事には実業家に対する尊敬の度は極めて低い。実業家よりも中学校の先生の方がえらいと信じている。よし信じておらんでも、融通の利かぬ性質として、到底実業家、金満家の恩顧を蒙(こうむ)る事は覚束(おぼつか)ないと諦(あき)らめている。いくら先方が勢力家でも、財産家でも、自分が世話になる見込のないと思い切った人の利害には極めて無頓着である。それだから学者社会を除いて他の方面の事には極めて迂濶(うかつ)で、ことに実業界などでは、どこに、だれが何をしているか一向知らん。知っても尊敬畏服の念は毫(ごう)も起らんのである。鼻子の方では天(あめ)が下(した)の一隅にこんな変人がやはり日光に照らされて生活していようとは夢にも知らない。今まで世の中の人間にも大分(だいぶ)接して見たが、金田の妻(さい)ですと名乗って、急に取扱いの変らない場合はない、どこの会へ出ても、どんな身分の高い人の前でも立派に金田夫人で通して行かれる、いわんやこんな燻(くすぶ)り返った老書生においてをやで、私(わたし)の家(うち)は向う横丁の角屋敷(かどやしき)ですとさえ云えば職業などは聞かぬ先から驚くだろうと予期していたのである。\\
「金田って人を知ってるか」と主人は無雑作(むぞうさ)に迷亭に聞く。「知ってるとも、金田さんは僕の伯父の友達だ。この間なんざ園遊会へおいでになった」と迷亭は真面目な返事をする。「へえ、君の伯父さんてえな誰だい」「牧山男爵(まきやまだんしゃく)さ」と迷亭はいよいよ真面目である。主人が何か云おうとして云わぬ先に、鼻子は急に向き直って迷亭の方を見る。迷亭は大島紬(おおしまつむぎ)に古渡更紗(こわたりさらさ)か何か重ねてすましている。「おや、あなたが牧山様の――何でいらっしゃいますか、ちっとも存じませんで、はなはだ失礼を致しました。牧山様には始終御世話になると、宿(やど)で毎々御噂(おうわさ)を致しております」と急に叮嚀(ていねい)な言葉使をして、おまけに御辞儀までする、迷亭は「へええ何、ハハハハ」と笑っている。主人はあっ気(け)に取られて無言で二人を見ている。「たしか娘の縁辺(えんぺん)の事につきましてもいろいろ牧山さまへ御心配を願いましたそうで\ldots{}\ldots{}」「へえー、そうですか」とこればかりは迷亭にもちと唐突(とうとつ)過ぎたと見えてちょっと魂消(たまげ)たような声を出す。「実は方々からくれくれと申し込はございますが、こちらの身分もあるものでございますから、滅多(めった)な所(とこ)へも片付けられませんので\ldots{}\ldots{}」「ごもっともで」と迷亭はようやく安心する。「それについて、あなたに伺おうと思って上がったんですがね」と鼻子は主人の方を見て急に存在(ぞんざい)な言葉に返る。「あなたの所へ水島寒月(みずしまかんげつ)という男が度々(たびたび)上がるそうですが、あの人は全体どんな風な人でしょう」「寒月の事を聞いて、何(なん)にするんです」と主人は苦々(にがにが)しく云う。「やはり御令嬢の御婚儀上の関係で、寒月君の性行(せいこう)の一斑(いっぱん)を御承知になりたいという訳でしょう」と迷亭が気転を利(き)かす。「それが伺えれば大変都合が宜(よろ)しいのでございますが\ldots{}\ldots{}」「それじゃ、御令嬢を寒月におやりになりたいとおっしゃるんで」「やりたいなんてえんじゃ無いんです」と鼻子は急に主人を参らせる。「ほかにもだんだん口が有るんですから、無理に貰っていただかないだって困りゃしません」「それじゃ寒月の事なんか聞かんでも好いでしょう」と主人も躍起(やっき)となる。「しかし御隠しなさる訳もないでしょう」と鼻子も少々喧嘩腰になる。迷亭は双方の間に坐って、銀煙管(ぎんぎせる)を軍配団扇(ぐんばいうちわ)のように持って、心の裡(うち)で八卦(はっけ)よいやよいやと怒鳴っている。「じゃあ寒月の方で是非貰いたいとでも云ったのですか」と主人が正面から鉄砲を喰(くら)わせる。「貰いたいと云ったんじゃないんですけれども\ldots{}\ldots{}」「貰いたいだろうと思っていらっしゃるんですか」と主人はこの婦人鉄砲に限ると覚(さと)ったらしい。「話しはそんなに運んでるんじゃありませんが――寒月さんだって満更(まんざら)嬉しくない事もないでしょう」と土俵際で持ち直す。「寒月が何かその御令嬢に恋着(れんちゃく)したというような事でもありますか」あるなら云って見ろと云う権幕(けんまく)で主人は反(そ)り返る。「まあ、そんな見当(けんとう)でしょうね」今度は主人の鉄砲が少しも功を奏しない。今まで面白気(おもしろげ)に行司(ぎょうじ)気取りで見物していた迷亭も鼻子の一言(いちごん)に好奇心を挑撥(ちょうはつ)されたものと見えて、煙管(きせる)を置いて前へ乗り出す。「寒月が御嬢さんに付(つ)け文(ぶみ)でもしたんですか、こりゃ愉快だ、新年になって逸話がまた一つ殖(ふ)えて話しの好材料になる」と一人で喜んでいる。「付け文じゃないんです、もっと烈しいんでさあ、御二人とも御承知じゃありませんか」と鼻子は乙(おつ)にからまって来る。「君知ってるか」と主人は狐付きのような顔をして迷亭に聞く。迷亭も馬鹿気(ばかげ)た調子で「僕は知らん、知っていりゃ君だ」とつまらんところで謙遜(けんそん)する。「いえ御両人共(おふたりとも)御存じの事ですよ」と鼻子だけ大得意である。「へえー」と御両人は一度に感じ入る。「御忘れになったら私(わた)しから御話をしましょう。去年の暮向島の阿部さんの御屋敷で演奏会があって寒月さんも出掛けたじゃありませんか、その晩帰りに吾妻橋(あずまばし)で何かあったでしょう――詳しい事は言いますまい、当人の御迷惑になるかも知れませんから――あれだけの証拠がありゃ充分だと思いますが、どんなものでしょう」と金剛石(ダイヤ)入りの指環の嵌(はま)った指を、膝の上へ併(なら)べて、つんと居ずまいを直す。偉大なる鼻がますます異彩を放って、迷亭も主人も有れども無きがごとき有様である。\\
 主人は無論、さすがの迷亭もこの不意撃(ふいうち)には胆(きも)を抜かれたものと見えて、しばらくは呆然(ぼうぜん)として瘧(おこり)の落ちた病人のように坐っていたが、驚愕(きょうがく)の箍(たが)がゆるんでだんだん持前の本態に復すると共に、滑稽と云う感じが一度に吶喊(とっかん)してくる。両人(ふたり)は申し合せたごとく「ハハハハハ」と笑い崩れる。鼻子ばかりは少し当てがはずれて、この際笑うのははなはだ失礼だと両人を睨(にら)みつける。「あれが御嬢さんですか、なるほどこりゃいい、おっしゃる通りだ、ねえ苦沙弥(くしゃみ)君、全く寒月はお嬢さんを恋(おも)ってるに相違ないね\ldots{}\ldots{}もう隠したってしようがないから白状しようじゃないか」「ウフン」と主人は云ったままである。「本当に御隠しなさってもいけませんよ、ちゃんと種は上ってるんですからね」と鼻子はまた得意になる。「こうなりゃ仕方がない。何でも寒月君に関する事実は御参考のために陳述するさ、おい苦沙弥君、君が主人だのに、そう、にやにや笑っていては埒(らち)があかんじゃないか、実に秘密というものは恐ろしいものだねえ。いくら隠しても、どこからか露見(ろけん)するからな。――しかし不思議と云えば不思議ですねえ、金田の奥さん、どうしてこの秘密を御探知になったんです、実に驚ろきますな」と迷亭は一人で喋舌(しゃべ)る。「私(わた)しの方だって、ぬかりはありませんやね」と鼻子はしたり顔をする。「あんまり、ぬかりが無さ過ぎるようですぜ。一体誰に御聞きになったんです」「じきこの裏にいる車屋の神(かみ)さんからです」「あの黒猫のいる車屋ですか」と主人は眼を丸くする。「ええ、寒月さんの事じゃ、よっぽど使いましたよ。寒月さんが、ここへ来る度に、どんな話しをするかと思って車屋の神さんを頼んで一々知らせて貰うんです」「そりゃ苛(ひど)い」と主人は大きな声を出す。「なあに、あなたが何をなさろうとおっしゃろうと、それに構ってるんじゃないんです。寒月さんの事だけですよ」「寒月の事だって、誰の事だって――全体あの車屋の神さんは気に食わん奴だ」と主人は一人怒(おこ)り出す。「しかしあなたの垣根のそとへ来て立っているのは向うの勝手じゃありませんか、話しが聞えてわるけりゃもっと小さい声でなさるか、もっと大きなうちへ御這入(おはい)んなさるがいいでしょう」と鼻子は少しも赤面した様子がない。「車屋ばかりじゃありません。新道(しんみち)の二絃琴(にげんきん)の師匠からも大分(だいぶ)いろいろな事を聞いています」「寒月の事をですか」「寒月さんばかりの事じゃありません」と少し凄(すご)い事を云う。主人は恐れ入るかと思うと「あの師匠はいやに上品ぶって自分だけ人間らしい顔をしている、馬鹿野郎です」「憚(はばか)り様(さま)、女ですよ。野郎は御門違(おかどちが)いです」と鼻子の言葉使いはますます御里(おさと)をあらわして来る。これではまるで喧嘩をしに来たようなものであるが、そこへ行くと迷亭はやはり迷亭でこの談判を面白そうに聞いている。鉄枴仙人(てっかいせんにん)が軍鶏(しゃも)の蹴合(けあ)いを見るような顔をして平気で聞いている。\\
 悪口(あっこう)の交換では到底鼻子の敵でないと自覚した主人は、しばらく沈黙を守るのやむを得ざるに至らしめられていたが、ようやく思い付いたか「あなたは寒月の方から御嬢さんに恋着したようにばかりおっしゃるが、私(わたし)の聞いたんじゃ、少し違いますぜ、ねえ迷亭君」と迷亭の救いを求める。「うん、あの時の話しじゃ御嬢さんの方が、始め病気になって――何だか譫語(うわごと)をいったように聞いたね」「なにそんな事はありません」と金田夫人は判然たる直線流の言葉使いをする。「それでも寒月はたしかに○○博士の夫人から聞いたと云っていましたぜ」「それがこっちの手なんでさあ、○○博士の奥さんを頼んで寒月さんの気を引いて見たんでさあね」「○○の奥さんは、それを承知で引き受けたんですか」「ええ。引き受けて貰うたって、ただじゃ出来ませんやね、それやこれやでいろいろ物を使っているんですから」「是非寒月君の事を根堀り葉堀り御聞きにならなくっちゃ御帰りにならないと云う決心ですかね」と迷亭も少し気持を悪くしたと見えて、いつになく手障(てざわ)りのあらい言葉を使う。「いいや君、話したって損の行く事じゃなし、話そうじゃないか苦沙弥君――奥さん、私(わたし)でも苦沙弥でも寒月君に関する事実で差支(さしつか)えのない事は、みんな話しますからね、――そう、順を立ててだんだん聞いて下さると都合がいいですね」\\
 鼻子はようやく納得(なっとく)してそろそろ質問を呈出する。一時荒立てた言葉使いも迷亭に対してはまたもとのごとく叮嚀になる。「寒月さんも理学士だそうですが、全体どんな事を専門にしているのでございます」「大学院では\emph{地球の磁気の研究}をやっています」と主人が真面目に答える。不幸にしてその意味が鼻子には分らんものだから「へえー」とは云ったが怪訝(けげん)な顔をしている。「それを勉強すると博士になれましょうか」と聞く。「博士にならなければやれないとおっしゃるんですか」と主人は不愉快そうに尋ねる。「ええ。ただの学士じゃね、いくらでもありますからね」と鼻子は平気で答える。主人は迷亭を見ていよいよいやな顔をする。「博士になるかならんかは僕等も保証する事が出来んから、ほかの事を聞いていただく事にしよう」と迷亭もあまり好い機嫌ではない。「近頃でもその地球の――何かを勉強しているんでございましょうか」「二三日前(にさんちまえ)は\emph{首縊りの力学}と云う研究の結果を理学協会で演説しました」と主人は何の気も付かずに云う。「おやいやだ、\emph{首縊り}だなんて、よっぽど変人ですねえ。そんな\emph{首縊り}や何かやってたんじゃ、とても博士にはなれますまいね」「本人が首を縊(くく)っちゃあむずかしいですが、\emph{首縊りの力学}なら成れないとも限らんです」「そうでしょうか」と今度は主人の方を見て顔色を窺(うかが)う。悲しい事に\emph{力学}と云う意味がわからんので落ちつきかねている。しかしこれしきの事を尋ねては金田夫人の面目に関すると思ってか、ただ相手の顔色で八卦(はっけ)を立てて見る。主人の顔は渋い。「そのほかになにか、分り易(やす)いものを勉強しておりますまいか」「そうですな、せんだって\emph{団栗のスタビリチーを論じて併せて天体の運行に及ぶ}と云う論文を書いた事があります」「団栗(どんぐり)なんぞでも大学校で勉強するものでしょうか」「さあ僕も素人(しろうと)だからよく分らんが、何しろ、寒月君がやるくらいなんだから、研究する価値があると見えますな」と迷亭はすまして冷かす。鼻子は学問上の質問は手に合わんと断念したものと見えて、今度は話題を転ずる。「御話は違いますが――この御正月に椎茸(しいたけ)を食べて前歯を二枚折ったそうじゃございませんか」「ええその欠けたところに空也餅(くうやもち)がくっ付いていましてね」と迷亭はこの質問こそ吾縄張内(なわばりうち)だと急に浮かれ出す。「色気のない人じゃございませんか、何だって楊子(ようじ)を使わないんでしょう」「今度逢(あ)ったら注意しておきましょう」と主人がくすくす笑う。「椎茸で歯がかけるくらいじゃ、よほど歯の性(しょう)が悪いと思われますが、如何(いかが)なものでしょう」「善いとは言われますまいな――ねえ迷亭」「善い事はないがちょっと愛嬌(あいきょう)があるよ。あれぎり、まだ填(つ)めないところが妙だ。今だに空也餅引掛所(ひっかけどころ)になってるなあ奇観だぜ」「歯を填める小遣(こづかい)がないので欠けなりにしておくんですか、または物好きで欠けなりにしておくんでしょうか」「何も永く前歯欠成(まえばかけなり)を名乗る訳でもないでしょうから御安心なさいよ」と迷亭の機嫌はだんだん回復してくる。鼻子はまた問題を改める。「何か御宅に手紙かなんぞ当人の書いたものでもございますならちょっと拝見したいもんでございますが」「端書(はがき)なら沢山あります、御覧なさい」と主人は書斎から三四十枚持って来る。「そんなに沢山拝見しないでも――その内の二三枚だけ\ldots{}\ldots{}」「どれどれ僕が好いのを撰(よ)ってやろう」と迷亭先生は「これなざあ面白いでしょう」と一枚の絵葉書を出す。「おや絵もかくんでございますか、なかなか器用ですね、どれ拝見しましょう」と眺めていたが「あらいやだ、狸(たぬき)だよ。何だって撰りに撰って狸なんぞかくんでしょうね――それでも狸と見えるから不思議だよ」と少し感心する。「その文句を読んで御覧なさい」と主人が笑いながら云う。鼻子は下女が新聞を読むように読み出す。「旧暦の歳(とし)の夜(よ)、山の狸が園遊会をやって盛(さかん)に舞踏します。その歌に曰(いわ)く、来(こ)いさ、としの夜(よ)で、御山婦美(おやまふみ)も来(く)まいぞ。スッポコポンノポン」「何ですこりゃ、人を馬鹿にしているじゃございませんか」と鼻子は不平の体(てい)である。「この天女(てんにょ)は御気に入りませんか」と迷亭がまた一枚出す。見ると天女が羽衣(はごろも)を着て琵琶(びわ)を弾(ひ)いている。「この天女の鼻が少し小さ過ぎるようですが」「何、それが人並ですよ、鼻より文句を読んで御覧なさい」文句にはこうある。「昔(むか)しある所に一人の天文学者がありました。ある夜(よ)いつものように高い台に登って、一心に星を見ていますと、空に美しい天女が現われ、この世では聞かれぬほどの微妙な音楽を奏し出したので、天文学者は身に沁(し)む寒さも忘れて聞き惚(ほ)れてしまいました。朝見るとその天文学者の死骸(しがい)に霜(しも)が真白に降っていました。これは本当の噺(はなし)だと、あのうそつきの爺(じい)やが申しました」「何の事ですこりゃ、意味も何もないじゃありませんか、これでも理学士で通るんですかね。ちっと文芸倶楽部でも読んだらよさそうなものですがねえ」と寒月君さんざんにやられる。迷亭は面白半分に「こりゃどうです」と三枚目を出す。今度は活版で帆懸舟(ほかけぶね)が印刷してあって、例のごとくその下に何か書き散らしてある。「よべの泊(とま)りの十六小女郎(じゅうろくこじょろ)、親がないとて、荒磯(ありそ)の千鳥、さよの寝覚(ねざめ)の千鳥に泣いた、親は船乗り波の底」「うまいのねえ、感心だ事、話せるじゃありませんか」「話せますかな」「ええこれなら三味線に乗りますよ」「三味線に乗りゃ本物だ。こりゃ如何(いかが)です」と迷亭は無暗(むやみ)に出す。「いえ、もうこれだけ拝見すれば、ほかのは沢山で、そんなに野暮(やぼ)でないんだと云う事は分りましたから」と一人で合点している。鼻子はこれで寒月に関する大抵の質問を卒(お)えたものと見えて、「これははなはだ失礼を致しました。どうか私の参った事は寒月さんへは内々に願います」と得手勝手(えてかって)な要求をする。寒月の事は何でも聞かなければならないが、自分の方の事は一切寒月へ知らしてはならないと云う方針と見える。迷亭も主人も「はあ」と気のない返事をすると「いずれその内御礼は致しますから」と念を入れて言いながら立つ。見送りに出た両人(ふたり)が席へ返るや否や迷亭が「ありゃ何だい」と云うと主人も「ありゃ何だい」と双方から同じ問をかける。奥の部屋で細君が怺(こら)え切れなかったと見えてクツクツ笑う声が聞える。迷亭は大きな声を出して「奥さん奥さん、月並の標本が来ましたぜ。月並もあのくらいになるとなかなか振(ふる)っていますなあ。さあ遠慮はいらんから、存分御笑いなさい」\\
 主人は不満な口気(こうき)で「第一気に喰わん顔だ」と悪(にく)らしそうに云うと、迷亭はすぐ引きうけて「鼻が顔の中央に陣取って乙(おつ)に構えているなあ」とあとを付ける。「しかも曲っていらあ」「少し猫背(ねこぜ)だね。猫背の鼻は、ちと奇抜(きばつ)過ぎる」と面白そうに笑う。「夫(おっと)を剋(こく)する顔だ」と主人はなお口惜(くや)しそうである。「十九世紀で売れ残って、二十世紀で店曝(たなざら)しに逢うと云う相(そう)だ」と迷亭は妙な事ばかり云う。ところへ妻君が奥の間(ま)から出て来て、女だけに「あんまり悪口をおっしゃると、また車屋の神(かみ)さんに\emph{いつけ}られますよ」と注意する。「少し\emph{いつけ}る方が薬ですよ、奥さん」「しかし顔の讒訴(ざんそ)などをなさるのは、あまり下等ですわ、誰だって好んであんな鼻を持ってる訳でもありませんから――それに相手が婦人ですからね、あんまり苛(ひど)いわ」と鼻子の鼻を弁護すると、同時に自分の容貌(ようぼう)も間接に弁護しておく。「何ひどいものか、あんなのは婦人じゃない、愚人だ、ねえ迷亭君」「愚人かも知れんが、なかなかえら者だ、大分(だいぶ)引き掻(か)かれたじゃないか」「全体教師を何と心得ているんだろう」「裏の車屋くらいに心得ているのさ。ああ云う人物に尊敬されるには博士になるに限るよ、一体博士になっておかんのが君の不了見(ふりょうけん)さ、ねえ奥さん、そうでしょう」と迷亭は笑いながら細君を顧(かえり)みる。「博士なんて到底駄目ですよ」と主人は細君にまで見離される。「これでも今になるかも知れん、軽蔑(けいべつ)するな。貴様なぞは知るまいが昔(むか)しアイソクラチスと云う人は九十四歳で大著述をした。ソフォクリスが傑作を出して天下を驚かしたのは、ほとんど百歳の高齢だった。シモニジスは八十で妙詩を作った。おれだって\ldots{}\ldots{}」「馬鹿馬鹿しいわ、あなたのような胃病でそんなに永く生きられるものですか」と細君はちゃんと主人の寿命を予算している。「失敬な、――甘木さんへ行って聞いて見ろ――元来御前がこんな皺苦茶(しわくちゃ)な黒木綿(くろもめん)の羽織や、つぎだらけの着物を着せておくから、あんな女に馬鹿にされるんだ。あしたから迷亭の着ているような奴を着るから出しておけ」「出しておけって、あんな立派な御召(おめし)はござんせんわ。金田の奥さんが迷亭さんに叮嚀になったのは、伯父さんの名前を聞いてからですよ。着物の咎(とが)じゃございません」と細君うまく責任を逃(の)がれる。\\
 主人は\emph{伯父さん}と云う言葉を聞いて急に思い出したように「君に伯父があると云う事は、今日始めて聞いた。今までついに噂(うわさ)をした事がないじゃないか、本当にあるのかい」と迷亭に聞く。迷亭は待ってたと云わぬばかりに「うんその伯父さ、その伯父が馬鹿に頑物(がんぶつ)でねえ――やはりその十九世紀から連綿と今日(こんにち)まで生き延びているんだがね」と主人夫婦を半々に見る。「オホホホホホ面白い事ばかりおっしゃって、どこに生きていらっしゃるんです」「静岡に生きてますがね、それがただ生きてるんじゃ無いです。頭にちょん髷(まげ)を頂いて生きてるんだから恐縮しまさあ。帽子を被(かぶ)れってえと、おれはこの年になるが、まだ帽子を被るほど寒さを感じた事はないと威張ってるんです――寒いから、もっと寝(ね)ていらっしゃいと云うと、人間は四時間寝れば充分だ。四時間以上寝るのは贅沢(ぜいたく)の沙汰だって朝暗いうちから起きてくるんです。それでね、おれも睡眠時間を四時間に縮めるには、永年修業をしたもんだ、若いうちはどうしても眠(ねむ)たくていかなんだが、近頃に至って始めて随処任意の庶境(しょきょう)に入(い)ってはなはだ嬉しいと自慢するんです。六十七になって寝られなくなるなあ当り前でさあ。修業も糸瓜(へちま)も入(い)ったものじゃないのに当人は全く克己(こっき)の力で成功したと思ってるんですからね。それで外出する時には、きっと鉄扇(てっせん)をもって出るんですがね」「なににするんだい」「何にするんだか分らない、ただ持って出るんだね。まあステッキの代りくらいに考えてるかも知れんよ。ところがせんだって妙な事がありましてね」と今度は細君の方へ話しかける。「へえー」と細君が差(さ)し合(あい)のない返事をする。「此年(ことし)の春突然手紙を寄こして山高帽子とフロックコートを至急送れと云うんです。ちょっと驚ろいたから、郵便で問い返したところが老人自身が着ると云う返事が来ました。二十三日に静岡で祝捷会(しゅくしょうかい)があるからそれまでに間(ま)に合うように、至急調達しろと云う命令なんです。ところがおかしいのは命令中にこうあるんです。帽子は好い加減な大きさのを買ってくれ、洋服も寸法を見計らって大丸(だいまる)へ注文してくれ\ldots{}\ldots{}」「近頃は大丸でも洋服を仕立てるのかい」「なあに、先生、白木屋(しろきや)と間違えたんだあね」「寸法を見計ってくれたって無理じゃないか」「そこが伯父の伯父たるところさ」「どうした?」「仕方がないから見計らって送ってやった」「君も乱暴だな。それで間に合ったのかい」「まあ、どうにか、こうにかおっついたんだろう。国の新聞を見たら、当日牧山翁は珍らしくフロックコートにて、例の鉄扇(てっせん)を持ち\ldots{}\ldots{}」「鉄扇だけは離さなかったと見えるね」「うん死んだら棺の中へ鉄扇だけは入れてやろうと思っているよ」「それでも帽子も洋服も、うまい具合に着られて善かった」「ところが大間違さ。僕も無事に行ってありがたいと思ってると、しばらくして国から小包が届いたから、何か礼でもくれた事と思って開けて見たら例の山高帽子さ、手紙が添えてあってね、せっかく御求め被下候(くだされそうら)えども少々大きく候間(そろあいだ)、帽子屋へ御遣(おつか)わしの上、御縮め被下度候(くだされたくそろ)。縮め賃は小為替(こがわせ)にて此方(こなた)より御送(おんおくり)可申上候(もうしあぐべきそろ)とあるのさ」「なるほど迂濶(うかつ)だな」と主人は己(おの)れより迂濶なものの天下にある事を発見して大(おおい)に満足の体(てい)に見える。やがて「それから、どうした」と聞く。「どうするったって仕方がないから僕が頂戴して被(かぶ)っていらあ」「あの帽子かあ」と主人がにやにや笑う。「その方(かた)が男爵でいらっしゃるんですか」と細君が不思議そうに尋ねる。「誰がです」「その鉄扇の伯父さまが」「なあに漢学者でさあ、若い時聖堂(せいどう)で朱子学(しゅしがく)か、何かにこり固まったものだから、電気灯の下で恭(うやうや)しく\emph{ちょん}髷(まげ)を頂いているんです。仕方がありません」とやたらに顋(あご)を撫(な)で廻す。「それでも君は、さっきの女に牧山男爵と云ったようだぜ」「そうおっしゃいましたよ、私も茶の間で聞いておりました」と細君もこれだけは主人の意見に同意する。「そうでしたかなアハハハハハ」と迷亭は訳(わけ)もなく笑う。「そりゃ嘘(うそ)ですよ。僕に男爵の伯父がありゃ、今頃は局長くらいになっていまさあ」と平気なものである。「何だか変だと思った」と主人は嬉しそうな、心配そうな顔付をする。「あらまあ、よく真面目であんな嘘が付けますねえ。あなたもよっぽど法螺(ほら)が御上手でいらっしゃる事」と細君は非常に感心する。「僕より、あの女の方が上(う)わ手(て)でさあ」「あなただって御負けなさる気遣(きづか)いはありません」「しかし奥さん、僕の法螺は単なる法螺ですよ。あの女のは、みんな魂胆があって、曰(いわ)く付きの嘘ですぜ。たちが悪いです。猿智慧(さるぢえ)から割り出した術数と、天来の滑稽趣味と混同されちゃ、コメディーの神様も活眼の士なきを嘆ぜざるを得ざる訳に立ち至りますからな」主人は俯目(ふしめ)になって「どうだか」と云う。妻君は笑いながら「同じ事ですわ」と云う。\\
 吾輩は今まで向う横丁へ足を踏み込んだ事はない。角屋敷(かどやしき)の金田とは、どんな構えか見た事は無論ない。聞いた事さえ今が始めてである。主人の家(うち)で実業家が話頭に上(のぼ)った事は一返もないので、主人の飯を食う吾輩までがこの方面には単に無関係なるのみならず、はなはだ冷淡であった。しかるに先刻図(はか)らずも鼻子の訪問を受けて、余所(よそ)ながらその談話を拝聴し、その令嬢の艶美(えんび)を想像し、またその富貴(ふうき)、権勢を思い浮べて見ると、猫ながら安閑として椽側(えんがわ)に寝転んでいられなくなった。しかのみならず吾輩は寒月君に対してはなはだ同情の至りに堪えん。先方では博士の奥さんやら、車屋の神(かみ)さんやら、二絃琴(にげんきん)の天璋院(てんしょういん)まで買収して知らぬ間(ま)に、前歯の欠けたのさえ探偵しているのに、寒月君の方ではただニヤニヤして羽織の紐ばかり気にしているのは、いかに卒業したての理学士にせよ、あまり能がなさ過ぎる。と言って、ああ云う偉大な鼻を顔の中(うち)に安置している女の事だから、滅多(めった)な者では寄り付ける訳の者ではない。こう云う事件に関しては主人はむしろ無頓着でかつあまりに銭(ぜに)がなさ過ぎる。迷亭は銭に不自由はしないが、あんな偶然童子だから、寒月に援(たす)けを与える便宜(べんぎ)は尠(すくな)かろう。して見ると可哀相(かわいそう)なのは\emph{首縊りの力学}を演説する先生ばかりとなる。吾輩でも奮発して、敵城へ乗り込んでその動静を偵察してやらなくては、あまり不公平である。吾輩は猫だけれど、エピクテタスを読んで机の上へ叩きつけるくらいな学者の家(うち)に寄寓(きぐう)する猫で、世間一般の痴猫(ちびょう)、愚猫(ぐびょう)とは少しく撰(せん)を殊(こと)にしている。この冒険をあえてするくらいの義侠心は固(もと)より尻尾(しっぽ)の先に畳み込んである。何も寒月君に恩になったと云う訳もないが、これはただに個人のためにする血気躁狂(けっきそうきょう)の沙汰ではない。大きく云えば公平を好み中庸を愛する天意を現実にする天晴(あっぱれ)な美挙だ。人の許諾を経(へ)ずして吾妻橋(あずまばし)事件などを至る処に振り廻わす以上は、人の軒下に犬を忍ばして、その報道を得々として逢う人に吹聴(ふいちょう)する以上は、車夫、馬丁(ばてい)、無頼漢(ぶらいかん)、ごろつき書生、日雇婆(ひやといばばあ)、産婆、妖婆(ようば)、按摩(あんま)、頓馬(とんま)に至るまでを使用して国家有用の材に煩(はん)を及ぼして顧(かえり)みざる以上は――猫にも覚悟がある。幸い天気も好い、霜解(しもどけ)は少々閉口するが道のためには一命もすてる。足の裏へ泥が着いて、椽側(えんがわ)へ梅の花の印を押すくらいな事は、ただ御三(おさん)の迷惑にはなるか知れんが、吾輩の苦痛とは申されない。翌日(あす)とも云わずこれから出掛けようと勇猛精進(ゆうもうしょうじん)の大決心を起して台所まで飛んで出たが「待てよ」と考えた。吾輩は猫として進化の極度に達しているのみならず、脳力の発達においてはあえて中学の三年生に劣らざるつもりであるが、悲しいかな咽喉(のど)の構造だけはどこまでも猫なので人間の言語が饒舌(しゃべ)れない。よし首尾よく金田邸へ忍び込んで、充分敵の情勢を見届けたところで、肝心(かんじん)の寒月君に教えてやる訳に行かない。主人にも迷亭先生にも話せない。話せないとすれば土中にある金剛石(ダイヤモンド)の日を受けて光らぬと同じ事で、せっかくの智識も無用の長物となる。これは愚(ぐ)だ、やめようかしらんと上り口で佇(たたず)んで見た。\\
 しかし一度思い立った事を中途でやめるのは、白雨(ゆうだち)が来るかと待っている時黒雲共(とも)隣国へ通り過ぎたように、何となく残り惜しい。それも非がこっちにあれば格別だが、いわゆる正義のため、人道のためなら、たとい無駄死(むだじに)をやるまでも進むのが、義務を知る男児の本懐であろう。無駄骨を折り、無駄足を汚(よご)すくらいは猫として適当のところである。猫と生れた因果(いんが)で寒月、迷亭、苦沙弥諸先生と三寸の舌頭(ぜっとう)に相互の思想を交換する技倆(ぎりょう)はないが、猫だけに忍びの術は諸先生より達者である。他人の出来ぬ事を成就(じょうじゅ)するのはそれ自身において愉快である。吾(われ)一箇でも、金田の内幕を知るのは、誰も知らぬより愉快である。人に告げられんでも人に知られているなと云う自覚を彼等に与うるだけが愉快である。こんなに愉快が続々出て来ては行かずにはいられない。やはり行く事に致そう。\\
 向う横町へ来て見ると、聞いた通りの西洋館が角地面(かどじめん)を吾物顔(わがものがお)に占領している。この主人もこの西洋館のごとく傲慢(ごうまん)に構えているんだろうと、門を這入(はい)ってその建築を眺(なが)めて見たがただ人を威圧しようと、二階作りが無意味に突っ立っているほかに何等の能もない構造であった。迷亭のいわゆる月並(つきなみ)とはこれであろうか。玄関を右に見て、植込の中を通り抜けて、勝手口へ廻る。さすがに勝手は広い、苦沙弥先生の台所の十倍はたしかにある。せんだって日本新聞に詳しく書いてあった大隈伯(おおくまはく)の勝手にも劣るまいと思うくらい整然とぴかぴかしている。「模範勝手だな」と這入(はい)り込む。見ると漆喰(しっくい)で叩き上げた二坪ほどの土間に、例の車屋の神(かみ)さんが立ちながら、御飯焚(ごはんた)きと車夫を相手にしきりに何か弁じている。こいつは剣呑(けんのん)だと水桶(みずおけ)の裏へかくれる。「あの教師あ、うちの旦那の名を知らないのかね」と飯焚(めしたき)が云う。「知らねえ事があるもんか、この界隈(かいわい)で金田さんの御屋敷を知らなけりゃ眼も耳もねえ片輪(かたわ)だあな」これは抱え車夫の声である。「なんとも云えないよ。あの教師と来たら、本よりほかに何にも知らない変人なんだからねえ。旦那の事を少しでも知ってりゃ恐れるかも知れないが、駄目だよ、自分の小供の歳(とし)さえ知らないんだもの」と神さんが云う。「金田さんでも恐れねえかな、厄介な唐変木(とうへんぼく)だ。構(かま)あ事(こた)あねえ、みんなで威嚇(おど)かしてやろうじゃねえか」「それが好いよ。奥様の鼻が大き過ぎるの、顔が気に喰わないのって――そりゃあ酷(ひど)い事を云うんだよ。自分の面(つら)あ今戸焼(いまどやき)の狸(たぬき)見たような癖に――あれで一人前(いちにんまえ)だと思っているんだからやれ切れないじゃないか」「顔ばかりじゃない、手拭(てぬぐい)を提(さ)げて湯に行くところからして、いやに高慢ちきじゃないか。自分くらいえらい者は無いつもりでいるんだよ」と苦沙弥先生は飯焚にも大(おおい)に不人望である。「何でも大勢であいつの垣根の傍(そば)へ行って悪口をさんざんいってやるんだね」「そうしたらきっと恐れ入るよ」「しかしこっちの姿を見せちゃあ面白くねえから、声だけ聞かして、勉強の邪魔をした上に、出来るだけじらしてやれって、さっき奥様が言い付けておいでなすったぜ」「そりゃ分っているよ」と神さんは悪口の三分の一を引き受けると云う意味を示す。なるほどこの手合が苦沙弥先生を冷やかしに来るなと三人の横を、そっと通り抜けて奥へ這入る。\\
 猫の足はあれども無きがごとし、どこを歩いても不器用な音のした試しがない。空を踏むがごとく、雲を行くがごとく、水中に磬(けい)を打つがごとく、洞裏(とうり)に瑟(しつ)を鼓(こ)するがごとく、醍醐(だいご)の妙味を甞(な)めて言詮(ごんせん)のほかに冷暖(れいだん)を自知(じち)するがごとし。月並な西洋館もなく、模範勝手もなく、車屋の神さんも、権助(ごんすけ)も、飯焚も、御嬢さまも、仲働(なかばたら)きも、鼻子夫人も、夫人の旦那様もない。行きたいところへ行って聞きたい話を聞いて、舌を出し尻尾(しっぽ)を掉(ふ)って、髭(ひげ)をぴんと立てて悠々(ゆうゆう)と帰るのみである。ことに吾輩はこの道に掛けては日本一の堪能(かんのう)である。草双紙(くさぞうし)にある猫又(ねこまた)の血脈を受けておりはせぬかと自(みずか)ら疑うくらいである。蟇(がま)の額(ひたい)には夜光(やこう)の明珠(めいしゅ)があると云うが、吾輩の尻尾には神祇釈教(しんぎしゃっきょう)恋無常(こいむじょう)は無論の事、満天下の人間を馬鹿にする一家相伝(いっかそうでん)の妙薬が詰め込んである。金田家の廊下を人の知らぬ間(ま)に横行するくらいは、仁王様が心太(ところてん)を踏み潰(つぶ)すよりも容易である。この時吾輩は我ながら、わが力量に感服して、これも普段大事にする尻尾の御蔭だなと気が付いて見るとただ置かれない。吾輩の尊敬する尻尾大明神を礼拝(らいはい)してニャン運長久を祈らばやと、ちょっと低頭して見たが、どうも少し見当(けんとう)が違うようである。なるべく尻尾の方を見て三拝しなければならん。尻尾の方を見ようと身体を廻すと尻尾も自然と廻る。追付こうと思って首をねじると、尻尾も同じ間隔をとって、先へ馳(か)け出す。なるほど天地玄黄(てんちげんこう)を三寸裏(り)に収めるほどの霊物だけあって、到底吾輩の手に合わない、尻尾を環(めぐ)る事七度(ななた)び半にして草臥(くたび)れたからやめにした。少々眼がくらむ。どこにいるのだかちょっと方角が分らなくなる。構うものかと滅茶苦茶にあるき廻る。障子の裏(うち)で鼻子の声がする。ここだと立ち留まって、左右の耳をはすに切って、息を凝(こ)らす。「貧乏教師の癖に生意気じゃありませんか」と例の金切(かなき)り声(ごえ)を振り立てる。「うん、生意気な奴だ、ちと懲(こ)らしめのためにいじめてやろう。あの学校にゃ国のものもいるからな」「誰がいるの?」「津木(つき)ピン助(すけ)や福地(ふくち)キシャゴがいるから、頼んでからかわしてやろう」吾輩は金田君の生国(しょうごく)は分らんが、妙な名前の人間ばかり揃(そろ)った所だと少々驚いた。金田君はなお語をついで、「あいつは英語の教師かい」と聞く。「はあ、車屋の神さんの話では英語のリードルか何か専門に教えるんだって云います」「どうせ碌(ろく)な教師じゃあるめえ」\emph{あるめえ}にも尠(すく)なからず感心した。「この間ピン助に遇(あ)ったら、私(わたし)の学校にゃ妙な奴がおります。生徒から先生\emph{番茶}は英語で何と云いますと聞かれて、\emph{番茶}は
Savage tea
であると真面目に答えたんで、教員間の物笑いとなっています、どうもあんな教員があるから、ほかのものの、迷惑になって困りますと云ったが、大方(おおかた)あいつの事だぜ」「あいつに極(きま)っていまさあ、そんな事を云いそうな面構(つらがま)えですよ、いやに髭(ひげ)なんか生(は)やして」「怪(け)しからん奴だ」髭を生やして怪しからなければ猫などは一疋だって怪しかりようがない。「それにあの迷亭とか、へべれけとか云う奴は、まあ何てえ、頓狂な跳返(はねっかえ)りなんでしょう、伯父の牧山男爵だなんて、あんな顔に男爵の伯父なんざ、有るはずがないと思ったんですもの」「御前がどこの馬の骨だか分らんものの言う事を真(ま)に受けるのも悪い」「悪いって、あんまり人を馬鹿にし過ぎるじゃありませんか」と大変残念そうである。不思議な事には寒月君の事は一言半句(いちごんはんく)も出ない。吾輩の忍んで来る前に評判記はすんだものか、またはすでに落第と事が極(きま)って念頭にないものか、その辺(へん)は懸念(けねん)もあるが仕方がない。しばらく佇(たたず)んでいると廊下を隔てて向うの座敷でベルの音がする。そらあすこにも何か事がある。後(おく)れぬ先に、とその方角へ歩を向ける。\\
 来て見ると女が独(ひと)りで何か大声で話している。その声が鼻子とよく似ているところをもって推(お)すと、これが即ち当家の令嬢寒月君をして未遂入水(みすいじゅすい)をあえてせしめたる代物(しろもの)だろう。惜哉(おしいかな)障子越しで玉の御姿(おんすがた)を拝する事が出来ない。従って顔の真中に大きな鼻を祭り込んでいるか、どうだか受合えない。しかし談話の模様から鼻息の荒いところなどを綜合(そうごう)して考えて見ると、満更(まんざら)人の注意を惹(ひ)かぬ獅鼻(ししばな)とも思われない。女はしきりに喋舌(しゃべ)っているが相手の声が少しも聞えないのは、噂(うわさ)にきく電話というものであろう。「御前は大和(やまと)かい。明日(あした)ね、行くんだからね、鶉(うずら)の三を取っておいておくれ、いいかえ――分ったかい――なに分らない? おやいやだ。鶉の三を取るんだよ。――なんだって、――取れない? 取れないはずはない、とるんだよ――へへへへへ御冗談(ごじょうだん)をだって――何が御冗談なんだよ――いやに人をおひゃらかすよ。全体御前は誰だい。長吉(ちょうきち)だ? 長吉なんぞじゃ訳が分らない。お神さんに電話口へ出ろって御云いな――なに? 私(わたく)しで何でも弁じます?――お前は失敬だよ。妾(あた)しを誰だか知ってるのかい。金田だよ。――へへへへへ善く存じておりますだって。ほんとに馬鹿だよこの人あ。――金田だってえばさ。――なに?――毎度御贔屓(ごひいき)にあずかりましてありがとうございます?――何がありがたいんだね。御礼なんか聞きたかあないやね――おやまた笑ってるよ。お前はよっぽど愚物(ぐぶつ)だね。――仰せの通りだって?――あんまり人を馬鹿にすると電話を切ってしまうよ。いいのかい。困らないのかよ――黙ってちゃ分らないじゃないか、何とか御云いなさいな」電話は長吉の方から切ったものか何の返事もないらしい。令嬢は癇癪(かんしゃく)を起してやけに\emph{ベル}をジャラジャラと廻す。足元で狆(ちん)が驚ろいて急に吠え出す。これは迂濶(うかつ)に出来ないと、急に飛び下りて椽(えん)の下へもぐり込む。\\
 折柄(おりから)廊下を近(ちかづ)く足音がして障子を開ける音がする。誰か来たなと一生懸命に聞いていると「御嬢様、旦那様と奥様が呼んでいらっしゃいます」と小間使らしい声がする。「知らないよ」と令嬢は剣突(けんつく)を食わせる。「ちょっと用があるから嬢(じょう)を呼んで来いとおっしゃいました」「うるさいね、知らないてば」と令嬢は第二の剣突を食わせる。「\ldots{}\ldots{}水島寒月さんの事で御用があるんだそうでございます」と小間使は気を利(き)かして機嫌を直そうとする。「寒月でも、水月でも知らないんだよ――大嫌いだわ、糸瓜(へちま)が戸迷(とまど)いをしたような顔をして」第三の剣突は、憐れなる寒月君が、留守中に頂戴する。「おや御前いつ束髪(そくはつ)に結(い)ったの」小間使はほっと一息ついて「今日(こんにち)」となるべく単簡(たんかん)な挨拶をする。「生意気だねえ、小間使の癖に」と第四の剣突を別方面から食わす。「そうして新しい半襟(はんえり)を掛けたじゃないか」「へえ、せんだって御嬢様からいただきましたので、結構過ぎて勿体(もったい)ないと思って行李(こうり)の中へしまっておきましたが、今までのがあまり汚(よご)れましたからかけ易(か)えました」「いつ、そんなものを上げた事があるの」「この御正月、白木屋へいらっしゃいまして、御求め遊ばしたので――鶯茶(うぐいすちゃ)へ相撲(すもう)の番附(ばんづけ)を染め出したのでございます。妾(あた)しには地味過ぎていやだから御前に上げようとおっしゃった、あれでございます」「あらいやだ。善く似合うのね。にくらしいわ」「恐れ入ります」「褒(ほ)めたんじゃない。にくらしいんだよ」「へえ」「そんなによく似合うものをなぜだまって貰ったんだい」「へえ」「御前にさえ、そのくらい似合うなら、妾(あた)しにだっておかしい事あないだろうじゃないか」「きっとよく御似合い遊ばします」「似あうのが分ってる癖になぜ黙っているんだい。そうしてすまして掛けているんだよ、人の悪い」剣突(けんつく)は留めどもなく連発される。このさき、事局はどう発展するかと謹聴している時、向うの座敷で「富子や、富子や」と大きな声で金田君が令嬢を呼ぶ。令嬢はやむを得ず「はい」と電話室を出て行く。吾輩より少し大きな狆(ちん)が顔の中心に眼と口を引き集めたような面(かお)をして付いて行く。吾輩は例の忍び足で再び勝手から往来へ出て、急いで主人の家に帰る。探険はまず十二分の成績(せいせき)である。\\
 帰って見ると、奇麗な家(うち)から急に汚ない所へ移ったので、何だか日当りの善い山の上から薄黒い洞窟(どうくつ)の中へ入(はい)り込んだような心持ちがする。探険中は、ほかの事に気を奪われて部屋の装飾、襖(ふすま)、障子(しょうじ)の具合などには眼も留らなかったが、わが住居(すまい)の下等なるを感ずると同時に彼(か)のいわゆる月並(つきなみ)が恋しくなる。教師よりもやはり実業家がえらいように思われる。吾輩も少し変だと思って、例の尻尾(しっぽ)に伺いを立てて見たら、その通りその通りと尻尾の先から御託宣(ごたくせん)があった。座敷へ這入(はい)って見ると驚いたのは迷亭先生まだ帰らない、巻煙草(まきたばこ)の吸い殻を蜂の巣のごとく火鉢の中へ突き立てて、大胡坐(おおあぐら)で何か話し立てている。いつの間(ま)にか寒月君さえ来ている。主人は手枕をして天井の雨洩(あまもり)を余念もなく眺めている。あいかわらず太平の逸民の会合である。\\
「寒月君、君の事を譫語(うわごと)にまで言った婦人の名は、当時秘密であったようだが、もう話しても善かろう」と迷亭がからかい出す。「御話しをしても、私だけに関する事なら差支(さしつか)えないんですが、先方の迷惑になる事ですから」「まだ駄目かなあ」「それに○○博士夫人に約束をしてしまったもんですから」「他言をしないと云う約束かね」「ええ」と寒月君は例のごとく羽織の紐(ひも)をひねくる。その紐は売品にあるまじき紫色である。「その紐の色は、ちと天保調(てんぽうちょう)だな」と主人が寝ながら云う。主人は金田事件などには無頓着である。「そうさ、到底(とうてい)日露戦争時代のものではないな。陣笠(じんがさ)に立葵(たちあおい)の紋の付いたぶっ割(さ)き羽織でも着なくっちゃ納まりの付かない紐だ。織田信長が聟入(むこいり)をするとき頭の髪を茶筌(ちゃせん)に結(い)ったと云うがその節用いたのは、たしかそんな紐だよ」と迷亭の文句はあいかわらず長い。「実際これは爺(じじい)が長州征伐の時に用いたのです」と寒月君は真面目である。「もういい加減に博物館へでも献納してはどうだ。\emph{首縊りの力学}の演者、理学士水島寒月君ともあろうものが、売れ残りの旗本のような出(い)で立(たち)をするのはちと体面に関する訳だから」「御忠告の通りに致してもいいのですが、この紐が大変よく似合うと云ってくれる人もありますので――」「誰だい、そんな趣味のない事を云うのは」と主人は寝返りを打ちながら大きな声を出す。「それは御存じの方なんじゃないんで――」「御存じでなくてもいいや、一体誰だい」「去る女性(にょしょう)なんです」「ハハハハハよほど茶人だなあ、当てて見ようか、やはり隅田川の底から君の名を呼んだ女なんだろう、その羽織を着てもう一返御駄仏(おだぶつ)を極(き)め込んじゃどうだい」と迷亭が横合から飛び出す。「へへへへへもう水底から呼んではおりません。ここから乾(いぬい)の方角にあたる清浄(しょうじょう)な世界で\ldots{}\ldots{}」「あんまり清浄でもなさそうだ、毒々しい鼻だぜ」「へえ?」と寒月は不審な顔をする。「向う横丁の鼻がさっき押しかけて来たんだよ、ここへ、実に僕等二人は驚いたよ、ねえ苦沙弥君」「うむ」と主人は寝ながら茶を飲む。「鼻って誰の事です」「君の親愛なる久遠(くおん)の女性(にょしょう)の御母堂様だ」「へえー」「金田の妻(さい)という女が君の事を聞きに来たよ」と主人が真面目に説明してやる。驚くか、嬉しがるか、恥ずかしがるかと寒月君の様子を窺(うかが)って見ると別段の事もない。例の通り静かな調子で「どうか私に、あの娘を貰ってくれと云う依頼なんでしょう」と、また紫の紐をひねくる。「ところが大違さ。その御母堂なるものが偉大なる鼻の所有主(ぬし)でね\ldots{}\ldots{}」迷亭が半(なか)ば言い懸けると、主人が「おい君、僕はさっきから、あの鼻について俳体詩(はいたいし)を考えているんだがね」と木に竹を接(つ)いだような事を云う。隣の室(へや)で妻君がくすくす笑い出す。「随分君も呑気(のんき)だなあ出来たのかい」「少し出来た。第一句が\emph{この顔に鼻祭り}と云うのだ」「それから?」「次が\emph{この鼻に神酒供え}というのさ」「次の句は?」「まだそれぎりしか出来ておらん」「面白いですな」と寒月君がにやにや笑う。「次へ\emph{穴二つ幽かなり}と付けちゃどうだ」と迷亭はすぐ出来る。すると寒月が「\emph{奥深く毛も見えず}はいけますまいか」と各々(おのおの)出鱈目(でたらめ)を並べていると、垣根に近く、往来で「今戸焼(いまどやき)の狸(たぬき)今戸焼の狸」と四五人わいわい云う声がする。主人も迷亭もちょっと驚ろいて表の方を、垣の隙(すき)からすかして見ると「ワハハハハハ」と笑う声がして遠くへ散る足の音がする。「今戸焼の狸というな何だい」と迷亭が不思議そうに主人に聞く。「何だか分らん」と主人が答える。「なかなか振(ふる)っていますな」と寒月君が批評を加える。迷亭は何を思い出したか急に立ち上って「吾輩は年来美学上の見地からこの鼻について研究した事がございますから、その一斑(いっぱん)を披瀝(ひれき)して、御両君の清聴を煩(わずら)わしたいと思います」と演舌の真似をやる。主人はあまりの突然にぼんやりして無言のまま迷亭を見ている。寒月は「是非承(うけたまわ)りたいものです」と小声で云う。「いろいろ調べて見ましたが鼻の起源はどうも確(しか)と分りません。第一の不審は、もしこれを実用上の道具と仮定すれば穴が二つでたくさんである。何もこんなに横風(おうふう)に真中から突き出して見る必用がないのである。ところがどうしてだんだん御覧のごとく斯様(かよう)にせり出して参ったか」と自分の鼻を抓(つま)んで見せる。「あんまりせり出してもおらんじゃないか」と主人は御世辞のないところを云う。「とにかく引っ込んではおりませんからな。ただ二個の孔(あな)が併(なら)んでいる状体と混同なすっては、誤解を生ずるに至るかも計られませんから、予(あらかじ)め御注意をしておきます。――で愚見によりますと鼻の発達は吾々人間が鼻汁(はな)をかむと申す微細なる行為の結果が自然と蓄積してかく著明なる現象を呈出したものでございます」「佯(いつわ)りのない愚見だ」とまた主人が寸評を挿入(そうにゅう)する。「御承知の通り鼻汁(はな)をかむ時は、是非鼻を抓みます、鼻を抓んで、ことにこの局部だけに刺激を与えますと、進化論の大原則によって、この局部はこの刺激に応ずるがため他に比例して不相当な発達を致します。皮も自然堅くなります、肉も次第に硬(かた)くなります。ついに凝(こ)って骨となります」「それは少し――そう自由に肉が骨に一足飛に変化は出来ますまい」と理学士だけあって寒月君が抗議を申し込む。迷亭は何喰わぬ顔で陳(の)べ続ける。「いや御不審はごもっともですが論より証拠この通り骨があるから仕方がありません。すでに骨が出来る。骨は出来ても鼻汁(はな)は出ますな。出ればかまずにはいられません。この作用で骨の左右が削(けず)り取られて細い高い隆起と変化して参ります――実に恐ろしい作用です。点滴(てんてき)の石を穿(うが)つがごとく、賓頭顱(びんずる)の頭が自(おのず)から光明を放つがごとく、不思議薫(ふしぎくん)不思議臭(ふしぎしゅう)の喩(たとえ)のごとく、斯様(かよう)に鼻筋が通って堅くなります」{[#「なります」」は底本では「なります。」]}「それでも君のなんぞ、ぶくぶくだぜ」「演者自身の局部は回護(かいご)の恐れがありますから、わざと論じません。かの金田の御母堂の持たせらるる鼻のごときは、もっとも発達せるもっとも偉大なる天下の珍品として御両君に紹介しておきたいと思います」寒月君は思わずヒヤヤヤと云う。「しかし物も極度に達しますと偉観には相違ございませんが何となく怖(おそろ)しくて近づき難いものであります。あの鼻梁(びりょう)などは素晴しいには違いございませんが、少々峻嶮(しゅんけん)過ぎるかと思われます。古人のうちにてもソクラチス、ゴールドスミスもしくはサッカレーの鼻などは構造の上から云うと随分申し分はございましょうがその申し分のあるところに愛嬌(あいきょう)がございます。鼻高きが故に貴(たっと)からず、奇(き)なるがために貴しとはこの故でもございましょうか。下世話(げせわ)にも鼻より団子と申しますれば美的価値から申しますとまず迷亭くらいのところが適当かと存じます」寒月と主人は「フフフフ」と笑い出す。迷亭自身も愉快そうに笑う。「さてただ今(いま)まで弁じましたのは――」「先生\emph{弁じました}は少し講釈師のようで下品ですから、よしていただきましょう」と寒月君は先日の復讐(ふくしゅう)をやる。「さようしからば顔を洗って出直しましょうかな。――ええ――これから鼻と顔の権衡(けんこう)に一言(いちごん)論及したいと思います。他に関係なく単独に鼻論をやりますと、かの御母堂などはどこへ出しても恥ずかしからぬ鼻――鞍馬山(くらまやま)で展覧会があっても恐らく一等賞だろうと思われるくらいな鼻を所有していらせられますが、悲しいかなあれは眼、口、その他の諸先生と何等の相談もなく出来上った鼻であります。ジュリアス・シーザーの鼻は大したものに相違ございません。しかしシーザーの鼻を鋏(はさみ)でちょん切って、当家の猫の顔へ安置したらどんな者でございましょうか。喩(たと)えにも猫の額(ひたい)と云うくらいな地面へ、英雄の鼻柱が突兀(とっこつ)として聳(そび)えたら、碁盤の上へ奈良の大仏を据(す)え付けたようなもので、少しく比例を失するの極、その美的価値を落す事だろうと思います。御母堂の鼻はシーザーのそれのごとく、正(まさ)しく英姿颯爽(えいしさっそう)たる隆起に相違ございません。しかしその周囲を囲繞(いにょう)する顔面的条件は如何(いかが)な者でありましょう。無論当家の猫のごとく劣等ではない。しかし癲癇病(てんかんや)みの\emph{御かめ}のごとく眉(まゆ)の根に八字を刻んで、細い眼を釣るし上げらるるのは事実であります。諸君、この顔にしてこの鼻ありと嘆ぜざるを得んではありませんか」迷亭の言葉が少し途切れる途端(とたん)、裏の方で「まだ鼻の話しをしているんだよ。何てえ剛突(ごうつ)く張(ばり)だろう」と云う声が聞える。「車屋の神さんだ」と主人が迷亭に教えてやる。迷亭はまたやり初める。「計らざる裏手にあたって、新たに異性の傍聴者のある事を発見したのは演者の深く名誉と思うところであります。ことに宛転(えんてん)たる嬌音(きょうおん)をもって、乾燥なる講筵(こうえん)に一点の艶味(えんみ)を添えられたのは実に望外の幸福であります。なるべく通俗的に引き直して佳人淑女(かじんしゅくじょ)の眷顧(けんこ)に背(そむ)かざらん事を期する訳でありますが、これからは少々力学上の問題に立ち入りますので、勢(いきおい)御婦人方には御分りにくいかも知れません、どうか御辛防(ごしんぼう)を願います」寒月君は力学と云う語を聞いてまたにやにやする。「私の証拠立てようとするのは、この鼻とこの顔は到底調和しない。ツァイシングの\emph{黄金律}を失していると云う事なんで、それを厳格に力学上の公式から演繹(えんえき)して御覧に入れようと云うのであります。まずHを鼻の高さとします。αは鼻と顔の平面の交叉より生ずる角度であります。Wは無論鼻の重量と御承知下さい。どうです大抵お分りになりましたか。\ldots{}\ldots{}」「分るものか」と主人が云う。「寒月君はどうだい」「私にもちと分りかねますな」「そりゃ困ったな。苦沙弥(くしゃみ)はとにかく、君は理学士だから分るだろうと思ったのに。この式が演説の首脳なんだからこれを略しては今までやった甲斐(かい)がないのだが――まあ仕方がない。公式は略して結論だけ話そう」「結論があるか」と主人が不思議そうに聞く。「当り前さ結論のない演舌は、デザートのない西洋料理のようなものだ、――いいか両君能(よ)く聞き給え、これからが結論だぜ。――さて以上の公式にウィルヒョウ、ワイスマン諸家の説を参酌して考えて見ますと、先天的形体の遺伝は無論の事許さねばなりません。またこの形体に追陪(ついばい)して起る心意的状況は、たとい後天性は遺伝するものにあらずとの有力なる説あるにも関せず、ある程度までは必然の結果と認めねばなりません。従ってかくのごとく身分に不似合なる鼻の持主の生んだ子には、その鼻にも何か異状がある事と察せられます。寒月君などは、まだ年が御若いから金田令嬢の鼻の構造において特別の異状を認められんかも知れませんが、かかる遺伝は潜伏期の長いものでありますから、いつ何時(なんどき)気候の劇変と共に、急に発達して御母堂のそれのごとく、咄嗟(とっさ)の間(かん)に膨脹(ぼうちょう)するかも知れません、それ故にこの御婚儀は、迷亭の学理的論証によりますと、今の中御断念になった方が安全かと思われます、これには当家の御主人は無論の事、そこに寝ておらるる猫又殿(ねこまたどの)にも御異存は無かろうと存じます」主人はようよう起き返って「そりゃ無論さ。あんなものの娘を誰が貰うものか。寒月君もらっちゃいかんよ」と大変熱心に主張する。吾輩もいささか賛成の意を表するためににゃーにゃーと二声ばかり鳴いて見せる。寒月君は別段騒いだ様子もなく「先生方の御意向がそうなら、私は断念してもいいんですが、もし当人がそれを気にして病気にでもなったら罪ですから――」「ハハハハハ艶罪(えんざい)と云う訳(わけ)だ」主人だけは大(おおい)にむきになって「そんな馬鹿があるものか、あいつの娘なら碌(ろく)な者でないに極(きま)ってらあ。初めて人のうちへ来ておれをやり込めに掛った奴だ。傲慢(ごうまん)な奴だ」と独(ひと)りでぷんぷんする。するとまた垣根のそばで三四人が「ワハハハハハ」と云う声がする。一人が「高慢ちきな唐変木(とうへんぼく)だ」と云うと一人が「もっと大きな家(うち)へ這入(はい)りてえだろう」と云う。また一人が「御気の毒だが、いくら威張ったって蔭弁慶(かげべんけい)だ」と大きな声をする。主人は椽側(えんがわ)へ出て負けないような声で「やかましい、何だわざわざそんな塀(へい)の下へ来て」と怒鳴(どな)る。「ワハハハハハサヴェジ・チーだ、サヴェジ・チーだ」と口々に罵(のの)しる。主人は大(おおい)に逆鱗(げきりん)の体(てい)で突然起(た)ってステッキを持って、往来へ飛び出す。迷亭は手を拍(う)って「面白い、やれやれ」と云う。寒月は羽織の紐を撚(ひね)ってにやにやする。吾輩は主人のあとを付けて垣の崩れから往来へ出て見たら、真中に主人が手持無沙汰にステッキを突いて立っている。人通りは一人もない、ちょっと狐(きつね)に抓(つま)まれた体(てい)である。\\

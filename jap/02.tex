\chapter*{二}
 吾輩は新年来多少有名になったので、猫ながらちょっと鼻が高く感ぜらるるのはありがたい。\\
 元朝早々主人の許(もと)へ一枚の絵端書(えはがき)が来た。これは彼の交友某画家からの年始状であるが、上部を赤、下部を深緑(ふかみど)りで塗って、その真中に一の動物が蹲踞(うずくま)っているところをパステルで書いてある。主人は例の書斎でこの絵を、横から見たり、竪(たて)から眺めたりして、うまい色だなという。すでに一応感服したものだから、もうやめにするかと思うとやはり横から見たり、竪から見たりしている。からだを拗(ね)じ向けたり、手を延ばして年寄が三世相(さんぜそう)を見るようにしたり、または窓の方へむいて鼻の先まで持って来たりして見ている。早くやめてくれないと膝(ひざ)が揺れて険呑(けんのん)でたまらない。ようやくの事で動揺があまり劇(はげ)しくなくなったと思ったら、小さな声で一体何をかいたのだろうと云(い)う。主人は絵端書の色には感服したが、かいてある動物の正体が分らぬので、さっきから苦心をしたものと見える。そんな分らぬ絵端書かと思いながら、寝ていた眼を上品に半(なか)ば開いて、落ちつき払って見ると紛(まぎ)れもない、自分の肖像だ。主人のようにアンドレア・デル・サルトを極(き)め込んだものでもあるまいが、画家だけに形体も色彩もちゃんと整って出来ている。誰が見たって猫に相違ない。少し眼識のあるものなら、猫の中(うち)でも他(ほか)の猫じゃない吾輩である事が判然とわかるように立派に描(か)いてある。このくらい明瞭な事を分らずにかくまで苦心するかと思うと、少し人間が気の毒になる。出来る事ならその絵が吾輩であると云う事を知らしてやりたい。吾輩であると云う事はよし分らないにしても、せめて猫であるという事だけは分らしてやりたい。しかし人間というものは到底(とうてい)吾輩猫属(ねこぞく)の言語を解し得るくらいに天の恵(めぐみ)に浴しておらん動物であるから、残念ながらそのままにしておいた。\\
 ちょっと読者に断っておきたいが、元来人間が何ぞというと猫々と、事もなげに軽侮の口調をもって吾輩を評価する癖があるははなはだよくない。人間の糟(かす)から牛と馬が出来て、牛と馬の糞から猫が製造されたごとく考えるのは、自分の無智に心付かんで高慢な顔をする教師などにはありがちの事でもあろうが、はたから見てあまり見っともいい者じゃない。いくら猫だって、そう粗末簡便には出来ぬ。よそ目には一列一体、平等無差別、どの猫も自家固有の特色などはないようであるが、猫の社会に這入(はい)って見るとなかなか複雑なもので十人十色(といろ)という人間界の語(ことば)はそのままここにも応用が出来るのである。目付でも、鼻付でも、毛並でも、足並でも、みんな違う。髯(ひげ)の張り具合から耳の立ち按排(あんばい)、尻尾(しっぽ)の垂れ加減に至るまで同じものは一つもない。器量、不器量、好き嫌い、粋無粋(すいぶすい)の数(かず)を悉(つ)くして千差万別と云っても差支えないくらいである。そのように判然たる区別が存しているにもかかわらず、人間の眼はただ向上とか何とかいって、空ばかり見ているものだから、吾輩の性質は無論相貌(そうぼう)の末を識別する事すら到底出来ぬのは気の毒だ。同類相求むとは昔(むか)しからある語(ことば)だそうだがその通り、餅屋(もちや)は餅屋、猫は猫で、猫の事ならやはり猫でなくては分らぬ。いくら人間が発達したってこればかりは駄目である。いわんや実際をいうと彼等が自(みずか)ら信じているごとくえらくも何ともないのだからなおさらむずかしい。またいわんや同情に乏しい吾輩の主人のごときは、相互を残りなく解するというが愛の第一義であるということすら分らない男なのだから仕方がない。彼は性の悪い牡蠣(かき)のごとく書斎に吸い付いて、かつて外界に向って口を開(ひら)いた事がない。それで自分だけはすこぶる達観したような面構(つらがまえ)をしているのはちょっとおかしい。達観しない証拠には現に吾輩の肖像が眼の前にあるのに少しも悟った様子もなく今年は征露の第二年目だから大方熊の画(え)だろうなどと気の知れぬことをいってすましているのでもわかる。\\
 吾輩が主人の膝(ひざ)の上で眼をねむりながらかく考えていると、やがて下女が第二の絵端書(えはがき)を持って来た。見ると活版で舶来の猫が四五疋(ひき)ずらりと行列してペンを握ったり書物を開いたり勉強をしている。その内の一疋は席を離れて机の角で西洋の猫じゃ猫じゃを躍(おど)っている。その上に日本の墨で「吾輩は猫である」と黒々とかいて、右の側(わき)に書を読むや躍(おど)るや猫の春一日(はるひとひ)という俳句さえ認(したた)められてある。これは主人の旧門下生より来たので誰が見たって一見して意味がわかるはずであるのに、迂濶(うかつ)な主人はまだ悟らないと見えて不思議そうに首を捻(ひね)って、はてな今年は猫の年かなと独言(ひとりごと)を言った。吾輩がこれほど有名になったのを未(ま)だ気が着かずにいると見える。\\
 ところへ下女がまた第三の端書を持ってくる。今度は絵端書ではない。恭賀新年とかいて、傍(かたわ)らに乍恐縮(きょうしゅくながら)かの猫へも宜(よろ)しく御伝声(ごでんせい)奉願上候(ねがいあげたてまつりそろ)とある。いかに迂遠(うえん)な主人でもこう明らさまに書いてあれば分るものと見えてようやく気が付いたようにフンと言いながら吾輩の顔を見た。その眼付が今までとは違って多少尊敬の意を含んでいるように思われた。今まで世間から存在を認められなかった主人が急に一個の新面目(しんめんぼく)を施こしたのも、全く吾輩の御蔭だと思えばこのくらいの眼付は至当だろうと考える。\\
 おりから門の格子(こうし)がチリン、チリン、チリリリリンと鳴る。大方来客であろう、来客なら下女が取次に出る。吾輩は肴屋(さかなや)の梅公がくる時のほかは出ない事に極(き)めているのだから、平気で、もとのごとく主人の膝に坐っておった。すると主人は高利貸にでも飛び込まれたように不安な顔付をして玄関の方を見る。何でも年賀の客を受けて酒の相手をするのが厭らしい。人間もこのくらい偏屈(へんくつ)になれば申し分はない。そんなら早くから外出でもすればよいのにそれほどの勇気も無い。いよいよ牡蠣の根性(こんじょう)をあらわしている。しばらくすると下女が来て寒月(かんげつ)さんがおいでになりましたという。この寒月という男はやはり主人の旧門下生であったそうだが、今では学校を卒業して、何でも主人より立派になっているという話(はな)しである。この男がどういう訳か、よく主人の所へ遊びに来る。来ると自分を恋(おも)っている女が有りそうな、無さそうな、世の中が面白そうな、つまらなそうな、凄(すご)いような艶(つや)っぽいような文句ばかり並べては帰る。主人のようなしなびかけた人間を求めて、わざわざこんな話しをしに来るのからして合点(がてん)が行かぬが、あの牡蠣的(かきてき)主人がそんな談話を聞いて時々相槌(あいづち)を打つのはなお面白い。\\
「しばらく御無沙汰をしました。実は去年の暮から大(おおい)に活動しているものですから、出(で)よう出ようと思っても、ついこの方角へ足が向かないので」と羽織の紐(ひも)をひねくりながら謎(なぞ)見たような事をいう。「どっちの方角へ足が向くかね」と主人は真面目な顔をして、黒木綿(くろもめん)の紋付羽織の袖口(そでぐち)を引張る。この羽織は木綿で\emph{ゆき}が短かい、下からべんべら者が左右へ五分くらいずつはみ出している。「エヘヘヘ少し違った方角で」と寒月君が笑う。見ると今日は前歯が一枚欠けている。「君歯をどうかしたかね」と主人は問題を転じた。「ええ実はある所で椎茸(しいたけ)を食いましてね」「何を食ったって?」「その、少し椎茸を食ったんで。椎茸の傘(かさ)を前歯で噛み切ろうとしたらぼろりと歯が欠けましたよ」「椎茸で前歯がかけるなんざ、何だか爺々臭(じじいくさ)いね。俳句にはなるかも知れないが、恋にはならんようだな」と平手で吾輩の頭を軽(かろ)く叩く。「ああその猫が例のですか、なかなか肥ってるじゃありませんか、それなら車屋の黒にだって負けそうもありませんね、立派なものだ」と寒月君は大(おおい)に吾輩を賞(ほ)める。「近頃大分(だいぶ)大きくなったのさ」と自慢そうに頭をぽかぽかなぐる。賞められたのは得意であるが頭が少々痛い。「一昨夜もちょいと合奏会をやりましてね」と寒月君はまた話しをもとへ戻す。「どこで」「どこでもそりゃ御聞きにならんでもよいでしょう。ヴァイオリンが三挺(ちょう)とピヤノの伴奏でなかなか面白かったです。ヴァイオリンも三挺くらいになると下手でも聞かれるものですね。二人は女で私(わたし)がその中へまじりましたが、自分でも善く弾(ひ)けたと思いました」「ふん、そしてその女というのは何者かね」と主人は羨(うらや)ましそうに問いかける。元来主人は平常枯木寒巌(こぼくかんがん)のような顔付はしているものの実のところは決して婦人に冷淡な方ではない、かつて西洋の或る小説を読んだら、その中にある一人物が出て来て、それが大抵の婦人には必ずちょっと惚(ほ)れる。勘定をして見ると往来を通る婦人の\emph{七割弱}には恋着(れんちゃく)するという事が諷刺的(ふうしてき)に書いてあったのを見て、これは真理だと感心したくらいな男である。そんな浮気な男が何故(なぜ)牡蠣的生涯を送っているかと云うのは吾輩猫などには到底(とうてい)分らない。或人は失恋のためだとも云うし、或人は胃弱のせいだとも云うし、また或人は金がなくて臆病な性質(たち)だからだとも云う。どっちにしたって明治の歴史に関係するほどな人物でもないのだから構わない。しかし寒月君の女連(おんなづ)れを羨まし気(げ)に尋ねた事だけは事実である。寒月君は面白そうに口取(くちとり)の蒲鉾(かまぼこ)を箸で挟んで半分前歯で食い切った。吾輩はまた欠けはせぬかと心配したが今度は大丈夫であった。「なに二人とも去(さ)る所の令嬢ですよ、御存じの方(かた)じゃありません」と余所余所(よそよそ)しい返事をする。「ナール」と主人は引張ったが「ほど」を略して考えている。寒月君はもう善(い)い加減な時分だと思ったものか「どうも好い天気ですな、御閑(おひま)ならごいっしょに散歩でもしましょうか、旅順が落ちたので市中は大変な景気ですよ」と促(うな)がして見る。主人は旅順の陥落より女連(おんなづれ)の身元を聞きたいと云う顔で、しばらく考え込んでいたがようやく決心をしたものと見えて「それじゃ出るとしよう」と思い切って立つ。やはり黒木綿の紋付羽織に、兄の紀念(かたみ)とかいう二十年来着古(きふ)るした結城紬(ゆうきつむぎ)の綿入を着たままである。いくら結城紬が丈夫だって、こう着つづけではたまらない。所々が薄くなって日に透かして見ると裏から\emph{つぎ}を当てた針の目が見える。主人の服装には師走(しわす)も正月もない。ふだん着も余所(よそ)ゆきもない。出るときは懐手(ふところで)をしてぶらりと出る。ほかに着る物がないからか、有っても面倒だから着換えないのか、吾輩には分らぬ。ただしこれだけは失恋のためとも思われない。\\
 両人(ふたり)が出て行ったあとで、吾輩はちょっと失敬して寒月君の食い切った蒲鉾(かまぼこ)の残りを頂戴(ちょうだい)した。吾輩もこの頃では普通一般の猫ではない。まず桃川如燕(ももかわじょえん)以後の猫か、グレーの金魚を偸(ぬす)んだ猫くらいの資格は充分あると思う。車屋の黒などは固(もと)より眼中にない。蒲鉾の一切(ひときれ)くらい頂戴したって人からかれこれ云われる事もなかろう。それにこの人目を忍んで間食(かんしょく)をするという癖は、何も吾等猫族に限った事ではない。うちの御三(おさん)などはよく細君の留守中に餅菓子などを失敬しては頂戴し、頂戴しては失敬している。御三ばかりじゃない現に上品な仕付(しつけ)を受けつつあると細君から吹聴(ふいちょう)せられている小児(こども)ですらこの傾向がある。四五日前のことであったが、二人の小供が馬鹿に早くから眼を覚まして、まだ主人夫婦の寝ている間に対(むか)い合うて食卓に着いた。彼等は毎朝主人の食う麺麭(パン)の幾分に、砂糖をつけて食うのが例であるが、この日はちょうど砂糖壺(さとうつぼ)が卓(たく)の上に置かれて匙(さじ)さえ添えてあった。いつものように砂糖を分配してくれるものがないので、大きい方がやがて壺の中から一匙(ひとさじ)の砂糖をすくい出して自分の皿の上へあけた。すると小さいのが姉のした通り同分量の砂糖を同方法で自分の皿の上にあけた。少(しば)らく両人(りょうにん)は睨(にら)み合っていたが、大きいのがまた匙をとって一杯をわが皿の上に加えた。小さいのもすぐ匙をとってわが分量を姉と同一にした。すると姉がまた一杯すくった。妹も負けずに一杯を附加した。姉がまた壺へ手を懸ける、妹がまた匙をとる。見ている間(ま)に一杯一杯一杯と重なって、ついには両人(ふたり)の皿には山盛の砂糖が堆(うずたか)くなって、壺の中には一匙の砂糖も余っておらんようになったとき、主人が寝ぼけ眼(まなこ)を擦(こす)りながら寝室を出て来てせっかくしゃくい出した砂糖を元のごとく壺の中へ入れてしまった。こんなところを見ると、人間は利己主義から割り出した公平という念は猫より優(まさ)っているかも知れぬが、智慧(ちえ)はかえって猫より劣っているようだ。そんなに山盛にしないうちに早く甞(な)めてしまえばいいにと思ったが、例のごとく、吾輩の言う事などは通じないのだから、気の毒ながら御櫃(おはち)の上から黙って見物していた。\\
 寒月君と出掛けた主人はどこをどう歩行(ある)いたものか、その晩遅く帰って来て、翌日食卓に就(つ)いたのは九時頃であった。例の御櫃の上から拝見していると、主人はだまって雑煮(ぞうに)を食っている。代えては食い、代えては食う。餅の切れは小さいが、何でも六切(むきれ)か七切(ななきれ)食って、最後の一切れを椀の中へ残して、もうよそうと箸(はし)を置いた。他人がそんな我儘(わがまま)をすると、なかなか承知しないのであるが、主人の威光を振り廻わして得意なる彼は、濁った汁の中に焦(こ)げ爛(ただ)れた餅の死骸を見て平気ですましている。妻君が袋戸(ふくろど)の奥からタカジヤスターゼを出して卓の上に置くと、主人は「それは利(き)かないから飲まん」という。「でもあなた澱粉質(でんぷんしつ)のものには大変功能があるそうですから、召し上ったらいいでしょう」と飲ませたがる。「澱粉だろうが何だろうが駄目だよ」と頑固(がんこ)に出る。「あなたはほんとに厭(あ)きっぽい」と細君が独言(ひとりごと)のようにいう。「厭きっぽいのじゃない薬が利かんのだ」「それだってせんだってじゅうは大変によく利くよく利くとおっしゃって毎日毎日上ったじゃありませんか」「こないだうちは利いたのだよ、この頃は利かないのだよ」と対句(ついく)のような返事をする。「そんなに飲んだり止(や)めたりしちゃ、いくら功能のある薬でも利く気遣(きづか)いはありません、もう少し辛防(しんぼう)がよくなくっちゃあ胃弱なんぞはほかの病気たあ違って直らないわねえ」とお盆を持って控えた御三(おさん)を顧みる。「それは本当のところでございます。もう少し召し上ってご覧にならないと、とても善(よ)い薬か悪い薬かわかりますまい」と御三は一も二もなく細君の肩を持つ。「何でもいい、飲まんのだから飲まんのだ、女なんかに何がわかるものか、黙っていろ」「どうせ女ですわ」と細君がタカジヤスターゼを主人の前へ突き付けて是非詰腹(つめばら)を切らせようとする。主人は何にも云わず立って書斎へ這入(はい)る。細君と御三は顔を見合せてにやにやと笑う。こんなときに後(あと)からくっ付いて行って膝(ひざ)の上へ乗ると、大変な目に逢(あ)わされるから、そっと庭から廻って書斎の椽側へ上(あが)って障子の隙(すき)から覗(のぞ)いて見ると、主人はエピクテタスとか云う人の本を披(ひら)いて見ておった。もしそれが平常(いつも)の通りわかるならちょっとえらいところがある。五六分するとその本を叩(たた)き付けるように机の上へ抛(ほう)り出す。大方そんな事だろうと思いながらなお注意していると、今度は日記帳を出して下(しも)のような事を書きつけた。\\

寒月と、根津、上野、池(いけ)の端(はた)、神田辺(へん)を散歩。池の端の待合の前で芸者が裾模様の春着(はるぎ)をきて羽根をついていた。衣装(いしょう)は美しいが顔はすこぶるまずい。何となくうちの猫に似ていた。\\

 何も顔のまずい例に特に吾輩を出さなくっても、よさそうなものだ。吾輩だって喜多床(きたどこ)へ行って顔さえ剃(す)って貰(もら)やあ、そんなに人間と異(ちが)ったところはありゃしない。人間はこう自惚(うぬぼ)れているから困る。\\

宝丹(ほうたん)の角(かど)を曲るとまた一人芸者が来た。これは背(せい)のすらりとした撫肩(なでがた)の恰好(かっこう)よく出来上った女で、着ている薄紫の衣服(きもの)も素直に着こなされて上品に見えた。白い歯を出して笑いながら「源ちゃん昨夕(ゆうべ)は――つい忙がしかったもんだから」と云った。ただしその声は旅鴉(たびがらす)のごとく皺枯(しゃが)れておったので、せっかくの風采(ふうさい)も大(おおい)に下落したように感ぜられたから、いわゆる源ちゃんなるもののいかなる人なるかを振り向いて見るも面倒になって、懐手(ふところで)のまま御成道(おなりみち)へ出た。寒月は何となくそわそわしているごとく見えた。\\

 人間の心理ほど解(げ)し難いものはない。この主人の今の心は怒(おこ)っているのだか、浮かれているのだか、または哲人の遺書に一道(いちどう)の慰安を求めつつあるのか、ちっとも分らない。世の中を冷笑しているのか、世の中へ交(まじ)りたいのだか、くだらぬ事に肝癪(かんしゃく)を起しているのか、物外(ぶつがい)に超然(ちょうぜん)としているのだかさっぱり見当(けんとう)が付かぬ。猫などはそこへ行くと単純なものだ。食いたければ食い、寝たければ寝る、怒(おこ)るときは一生懸命に怒り、泣くときは絶体絶命に泣く。第一日記などという無用のものは決してつけない。つける必要がないからである。主人のように裏表のある人間は日記でも書いて世間に出されない自己の面目を暗室内に発揮する必要があるかも知れないが、我等猫属(ねこぞく)に至ると行住坐臥(ぎょうじゅうざが)、行屎送尿(こうしそうにょう)ことごとく真正の日記であるから、別段そんな面倒な手数(てかず)をして、己(おの)れの真面目(しんめんもく)を保存するには及ばぬと思う。日記をつけるひまがあるなら椽側に寝ているまでの事さ。\\

神田の某亭で晩餐(ばんさん)を食う。久し振りで正宗を二三杯飲んだら、今朝は胃の具合が大変いい。胃弱には晩酌が一番だと思う。タカジヤスターゼは無論いかん。誰が何と云っても駄目だ。どうしたって利(き)かないものは利かないのだ。\\

 無暗(むやみ)にタカジヤスターゼを攻撃する。独りで喧嘩をしているようだ。今朝の肝癪がちょっとここへ尾を出す。人間の日記の本色はこう云う辺(へん)に存するのかも知れない。\\

せんだって○○は朝飯(あさめし)を廃すると胃がよくなると云うたから二三日(にさんち)朝飯をやめて見たが腹がぐうぐう鳴るばかりで功能はない。△△は是非香(こう)の物(もの)を断(た)てと忠告した。彼の説によるとすべて胃病の源因は漬物にある。漬物さえ断てば胃病の源を涸(か)らす訳だから本復は疑なしという論法であった。それから一週間ばかり香の物に箸(はし)を触れなかったが別段の験(げん)も見えなかったから近頃はまた食い出した。××に聞くとそれは按腹(あんぷく)揉療治(もみりょうじ)に限る。ただし普通のではゆかぬ。皆川流(みながわりゅう)という古流な揉(も)み方で一二度やらせれば大抵の胃病は根治出来る。安井息軒(やすいそっけん)も大変この按摩術(あんまじゅつ)を愛していた。坂本竜馬(さかもとりょうま)のような豪傑でも時々は治療をうけたと云うから、早速上根岸(かみねぎし)まで出掛けて揉(も)まして見た。ところが骨を揉(も)まなければ癒(なお)らぬとか、臓腑の位置を一度顛倒(てんとう)しなければ根治がしにくいとかいって、それはそれは残酷な揉(も)み方をやる。後で身体が綿のようになって昏睡病(こんすいびょう)にかかったような心持ちがしたので、一度で閉口してやめにした。A君は是非固形体を食うなという。それから、一日牛乳ばかり飲んで暮して見たが、この時は腸の中でどぼりどぼりと音がして大水でも出たように思われて終夜眠れなかった。B氏は横膈膜(おうかくまく)で呼吸して内臓を運動させれば自然と胃の働きが健全になる訳だから試しにやって御覧という。これも多少やったが何となく腹中(ふくちゅう)が不安で困る。それに時々思い出したように一心不乱にかかりはするものの五六分立つと忘れてしまう。忘れまいとすると横膈膜が気になって本を読む事も文章をかく事も出来ぬ。美学者の迷亭(めいてい)がこの体(てい)を見て、産気(さんけ)のついた男じゃあるまいし止(よ)すがいいと冷かしたからこの頃は廃(よ)してしまった。C先生は蕎麦(そば)を食ったらよかろうと云うから、早速\emph{かけ}と\emph{もり}をかわるがわる食ったが、これは腹が下(くだ)るばかりで何等の功能もなかった。余は年来の胃弱を直すために出来得る限りの方法を講じて見たがすべて駄目である。ただ昨夜(ゆうべ)寒月と傾けた三杯の正宗はたしかに利目(ききめ)がある。これからは毎晩二三杯ずつ飲む事にしよう。\\

 これも決して長く続く事はあるまい。主人の心は吾輩の眼球(めだま)のように間断なく変化している。何をやっても永持(ながもち)のしない男である。その上日記の上で胃病をこんなに心配している癖に、表向は大(おおい)に痩我慢をするからおかしい。せんだってその友人で某(なにがし)という学者が尋ねて来て、一種の見地から、すべての病気は父祖の罪悪と自己の罪悪の結果にほかならないと云う議論をした。大分(だいぶ)研究したものと見えて、条理が明晰(めいせき)で秩序が整然として立派な説であった。気の毒ながらうちの主人などは到底これを反駁(はんばく)するほどの頭脳も学問もないのである。しかし自分が胃病で苦しんでいる際(さい)だから、何とかかんとか弁解をして自己の面目を保とうと思った者と見えて、「君の説は面白いが、あのカーライルは胃弱だったぜ」とあたかもカーライルが胃弱だから自分の胃弱も名誉であると云ったような、見当違いの挨拶をした。すると友人は「カーライルが胃弱だって、胃弱の病人が必ずカーライルにはなれないさ」と極(き)め付けたので主人は黙然(もくねん)としていた。かくのごとく虚栄心に富んでいるものの実際はやはり胃弱でない方がいいと見えて、今夜から晩酌を始めるなどというのはちょっと滑稽だ。考えて見ると今朝雑煮(ぞうに)をあんなにたくさん食ったのも昨夜(ゆうべ)寒月君と正宗をひっくり返した影響かも知れない。吾輩もちょっと雑煮が食って見たくなった。\\
 吾輩は猫ではあるが大抵のものは食う。車屋の黒のように横丁の肴屋(さかなや)まで遠征をする気力はないし、新道(しんみち)の二絃琴(にげんきん)の師匠の所(とこ)の三毛(みけ)のように贅沢(ぜいたく)は無論云える身分でない。従って存外嫌(きらい)は少ない方だ。小供の食いこぼした麺麭(パン)も食うし、餅菓子の\includegraphics{../../../gaiji/2-92/2-92-68.png}(あん)もなめる。香(こう)の物(もの)はすこぶるまずいが経験のため沢庵(たくあん)を二切ばかりやった事がある。食って見ると妙なもので、大抵のものは食える。あれは嫌(いや)だ、これは嫌だと云うのは贅沢(ぜいたく)な我儘で到底教師の家(うち)にいる猫などの口にすべきところでない。主人の話しによると仏蘭西(フランス)にバルザックという小説家があったそうだ。この男が大の贅沢(ぜいたく)屋で――もっともこれは口の贅沢屋ではない、小説家だけに文章の贅沢を尽したという事である。バルザックが或る日自分の書いている小説中の人間の名をつけようと思っていろいろつけて見たが、どうしても気に入らない。ところへ友人が遊びに来たのでいっしょに散歩に出掛けた。友人は固(もと)より何(なんに)も知らずに連れ出されたのであるが、バルザックは兼(か)ねて自分の苦心している名を目付(めつけ)ようという考えだから往来へ出ると何もしないで店先の看板ばかり見て歩行(ある)いている。ところがやはり気に入った名がない。友人を連れて無暗(むやみ)にあるく。友人は訳がわからずにくっ付いて行く。彼等はついに朝から晩まで巴理(パリ)を探険した。その帰りがけにバルザックはふとある裁縫屋の看板が目についた。見るとその看板にマーカスという名がかいてある。バルザックは手を拍(う)って「これだこれだこれに限る。マーカスは好い名じゃないか。マーカスの上へZという頭文字をつける、すると申し分(ぶん)のない名が出来る。Zでなくてはいかん。Z.
Marcus
は実にうまい。どうも自分で作った名はうまくつけたつもりでも何となく故意(わざ)とらしいところがあって面白くない。ようやくの事で気に入った名が出来た」と友人の迷惑はまるで忘れて、一人嬉しがったというが、小説中の人間の名前をつけるに一日(いちんち)巴理(パリ)を探険しなくてはならぬようでは随分手数(てすう)のかかる話だ。贅沢もこのくらい出来れば結構なものだが吾輩のように牡蠣的(かきてき)主人を持つ身の上ではとてもそんな気は出ない。何でもいい、食えさえすれば、という気になるのも境遇のしからしむるところであろう。だから今雑煮(ぞうに)が食いたくなったのも決して贅沢の結果ではない、何でも食える時に食っておこうという考から、主人の食い剰(あま)した雑煮がもしや台所に残っていはすまいかと思い出したからである。\ldots{}\ldots{}台所へ廻って見る。\\
 今朝見た通りの餅が、今朝見た通りの色で椀の底に膠着(こうちゃく)している。白状するが餅というものは今まで一辺(ぺん)も口に入れた事がない。見るとうまそうにもあるし、また少しは気味(きび)がわるくもある。前足で上にかかっている菜っ葉を掻(か)き寄せる。爪を見ると餅の上皮(うわかわ)が引き掛ってねばねばする。嗅(か)いで見ると釜の底の飯を御櫃(おはち)へ移す時のような香(におい)がする。食おうかな、やめようかな、とあたりを見廻す。幸か不幸か誰もいない。御三(おさん)は暮も春も同じような顔をして羽根をついている。小供は奥座敷で「何とおっしゃる兎さん」を歌っている。食うとすれば今だ。もしこの機をはずすと来年までは餅というものの味を知らずに暮してしまわねばならぬ。吾輩はこの刹那(せつな)に猫ながら一の真理を感得した。「得難き機会はすべての動物をして、好まざる事をも敢てせしむ」吾輩は実を云うとそんなに雑煮を食いたくはないのである。否椀底(わんてい)の様子を熟視すればするほど気味(きび)が悪くなって、食うのが厭になったのである。この時もし御三でも勝手口を開けたなら、奥の小供の足音がこちらへ近付くのを聞き得たなら、吾輩は惜気(おしげ)もなく椀を見棄てたろう、しかも雑煮の事は来年まで念頭に浮ばなかったろう。ところが誰も来ない、いくら\includegraphics{../../../gaiji/1-92/1-92-39.png}躇(ちゅうちょ)していても誰も来ない。早く食わぬか食わぬかと催促されるような心持がする。吾輩は椀の中を覗(のぞ)き込みながら、早く誰か来てくれればいいと念じた。やはり誰も来てくれない。吾輩はとうとう雑煮を食わなければならぬ。最後にからだ全体の重量を椀の底へ落すようにして、あぐりと餅の角を一寸(いっすん)ばかり食い込んだ。このくらい力を込めて食い付いたのだから、大抵なものなら噛(か)み切れる訳だが、驚いた! もうよかろうと思って歯を引こうとすると引けない。もう一辺(ぺん)噛み直そうとすると動きがとれない。餅は魔物だなと疳(かん)づいた時はすでに遅かった。沼へでも落ちた人が足を抜こうと焦慮(あせ)るたびにぶくぶく深く沈むように、噛めば噛むほど口が重くなる、歯が動かなくなる。歯答えはあるが、歯答えがあるだけでどうしても始末をつける事が出来ない。美学者迷亭先生がかつて吾輩の主人を評して君は割り切れない男だといった事があるが、なるほどうまい事をいったものだ。この餅も主人と同じようにどうしても割り切れない。噛んでも噛んでも、三で十を割るごとく尽未来際方(じんみらいざいかた)のつく期(ご)はあるまいと思われた。この煩悶(はんもん)の際吾輩は覚えず第二の真理に逢着(ほうちゃく)した。「すべての動物は直覚的に事物の適不適を予知す」真理はすでに二つまで発明したが、餅がくっ付いているので毫(ごう)も愉快を感じない。歯が餅の肉に吸収されて、抜けるように痛い。早く食い切って逃げないと御三(おさん)が来る。小供の唱歌もやんだようだ、きっと台所へ馳(か)け出して来るに相違ない。煩悶の極(きょく)尻尾(しっぽ)をぐるぐる振って見たが何等の功能もない、耳を立てたり寝かしたりしたが駄目である。考えて見ると耳と尻尾(しっぽ)は餅と何等の関係もない。要するに振り損の、立て損の、寝かし損であると気が付いたからやめにした。ようやくの事これは前足の助けを借りて餅を払い落すに限ると考え付いた。まず右の方をあげて口の周囲を撫(な)で廻す。撫(な)でたくらいで割り切れる訳のものではない。今度は左(ひだ)りの方を伸(のば)して口を中心として急劇に円を劃(かく)して見る。そんな呪(まじな)いで魔は落ちない。辛防(しんぼう)が肝心(かんじん)だと思って左右交(かわ)る交(がわ)るに動かしたがやはり依然として歯は餅の中にぶら下っている。ええ面倒だと両足を一度に使う。すると不思議な事にこの時だけは後足(あとあし)二本で立つ事が出来た。何だか猫でないような感じがする。猫であろうが、あるまいがこうなった日にゃあ構うものか、何でも餅の魔が落ちるまでやるべしという意気込みで無茶苦茶に顔中引っ掻(か)き廻す。前足の運動が猛烈なのでややともすると中心を失って倒れかかる。倒れかかるたびに後足で調子をとらなくてはならぬから、一つ所にいる訳にも行かんので、台所中あちら、こちらと飛んで廻る。我ながらよくこんなに器用に起(た)っていられたものだと思う。第三の真理が驀地(ばくち)に現前(げんぜん)する。「危きに臨(のぞ)めば平常なし能(あた)わざるところのものを為(な)し能う。之(これ)を天祐(てんゆう)という」幸(さいわい)に天祐を享(う)けたる吾輩が一生懸命餅の魔と戦っていると、何だか足音がして奥より人が来るような気合(けわい)である。ここで人に来られては大変だと思って、いよいよ躍起(やっき)となって台所をかけ廻る。足音はだんだん近付いてくる。ああ残念だが天祐が少し足りない。とうとう小供に見付けられた。「あら猫が御雑煮を食べて踊を踊っている」と大きな声をする。この声を第一に聞きつけたのが御三である。羽根も羽子板も打ち遣(や)って勝手から「あらまあ」と飛込んで来る。細君は縮緬(ちりめん)の紋付で「いやな猫ねえ」と仰せられる。主人さえ書斎から出て来て「この馬鹿野郎」といった。面白い面白いと云うのは小供ばかりである。そうしてみんな申し合せたようにげらげら笑っている。腹は立つ、苦しくはある、踊はやめる訳にゆかぬ、弱った。ようやく笑いがやみそうになったら、五つになる女の子が「御かあ様、猫も随分ね」といったので狂瀾(きょうらん)を既倒(きとう)に何とかするという勢でまた大変笑われた。人間の同情に乏しい実行も大分(だいぶ)見聞(けんもん)したが、この時ほど恨(うら)めしく感じた事はなかった。ついに天祐もどっかへ消え失(う)せて、在来の通り四(よ)つ這(ばい)になって、眼を白黒するの醜態を演ずるまでに閉口した。さすが見殺しにするのも気の毒と見えて「まあ餅をとってやれ」と主人が御三に命ずる。御三はもっと踊らせようじゃありませんかという眼付で細君を見る。細君は踊は見たいが、殺してまで見る気はないのでだまっている。「取ってやらんと死んでしまう、早くとってやれ」と主人は再び下女を顧(かえり)みる。御三(おさん)は御馳走を半分食べかけて夢から起された時のように、気のない顔をして餅をつかんでぐいと引く。寒月(かんげつ)君じゃないが前歯がみんな折れるかと思った。どうも痛いの痛くないのって、餅の中へ堅く食い込んでいる歯を情(なさ)け容赦もなく引張るのだからたまらない。吾輩が「すべての安楽は困苦を通過せざるべからず」と云う第四の真理を経験して、けろけろとあたりを見廻した時には、家人はすでに奥座敷へ這入(はい)ってしまっておった。\\
 こんな失敗をした時には内にいて御三なんぞに顔を見られるのも何となくばつが悪い。いっその事気を易(か)えて新道の二絃琴(にげんきん)の御師匠さんの所(とこ)の三毛子(みけこ)でも訪問しようと台所から裏へ出た。三毛子はこの近辺で有名な美貌家(びぼうか)である。吾輩は猫には相違ないが物の情(なさ)けは一通り心得ている。うちで主人の苦(にが)い顔を見たり、御三の険突(けんつく)を食って気分が勝(すぐ)れん時は必ずこの異性の朋友(ほうゆう)の許(もと)を訪問していろいろな話をする。すると、いつの間(ま)にか心が晴々(せいせい)して今までの心配も苦労も何もかも忘れて、生れ変ったような心持になる。女性の影響というものは実に莫大(ばくだい)なものだ。杉垣の隙から、いるかなと思って見渡すと、三毛子は正月だから首輪の新しいのをして行儀よく椽側(えんがわ)に坐っている。その背中の丸さ加減が言うに言われんほど美しい。曲線の美を尽している。尻尾(しっぽ)の曲がり加減、足の折り具合、物憂(ものう)げに耳をちょいちょい振る景色(けしき)なども到底(とうてい)形容が出来ん。ことによく日の当る所に暖かそうに、品(ひん)よく控(ひか)えているものだから、身体は静粛端正の態度を有するにも関らず、天鵞毛(びろうど)を欺(あざむ)くほどの滑(なめ)らかな満身の毛は春の光りを反射して風なきにむらむらと微動するごとくに思われる。吾輩はしばらく恍惚(こうこつ)として眺(なが)めていたが、やがて我に帰ると同時に、低い声で「三毛子さん三毛子さん」といいながら前足で招いた。三毛子は「あら先生」と椽を下りる。赤い首輪につけた鈴がちゃらちゃらと鳴る。おや正月になったら鈴までつけたな、どうもいい音(ね)だと感心している間(ま)に、吾輩の傍(そば)に来て「あら先生、おめでとう」と尾を左(ひだ)りへ振る。吾等猫属(ねこぞく)間で御互に挨拶をするときには尾を棒のごとく立てて、それを左りへぐるりと廻すのである。町内で吾輩を先生と呼んでくれるのはこの三毛子ばかりである。吾輩は前回断わった通りまだ名はないのであるが、教師の家(うち)にいるものだから三毛子だけは尊敬して先生先生といってくれる。吾輩も先生と云われて満更(まんざら)悪い心持ちもしないから、はいはいと返事をしている。「やあおめでとう、大層立派に御化粧が出来ましたね」「ええ去年の暮御師匠(おししょう)さんに買って頂いたの、宜(い)いでしょう」とちゃらちゃら鳴らして見せる。「なるほど善い音(ね)ですな、吾輩などは生れてから、そんな立派なものは見た事がないですよ」「あらいやだ、みんなぶら下げるのよ」とまたちゃらちゃら鳴らす。「いい音(ね)でしょう、あたし嬉しいわ」とちゃらちゃらちゃらちゃら続け様に鳴らす。「あなたのうちの御師匠さんは大変あなたを可愛がっていると見えますね」と吾身に引きくらべて暗(あん)に欣羨(きんせん)の意を洩(も)らす。三毛子は無邪気なものである「ほんとよ、まるで自分の小供のようよ」とあどけなく笑う。猫だって笑わないとは限らない。人間は自分よりほかに笑えるものが無いように思っているのは間違いである。吾輩が笑うのは鼻の孔(あな)を三角にして咽喉仏(のどぼとけ)を震動させて笑うのだから人間にはわからぬはずである。「一体あなたの所(とこ)の御主人は何ですか」「あら御主人だって、妙なのね。御師匠(おししょう)さんだわ。二絃琴(にげんきん)の御師匠さんよ」「それは吾輩も知っていますがね。その御身分は何なんです。いずれ昔(むか)しは立派な方なんでしょうな」「ええ」\\
  君を待つ間(ま)の姫小松\ldots{}\ldots{}\ldots{}\ldots{}\ldots{}\\
 障子の内で御師匠さんが二絃琴を弾(ひ)き出す。「宜(い)い声でしょう」と三毛子は自慢する。「宜(い)いようだが、吾輩にはよくわからん。全体何というものですか」「あれ? あれは何とかってものよ。御師匠さんはあれが大好きなの。\ldots{}\ldots{}御師匠さんはあれで六十二よ。随分丈夫だわね」六十二で生きているくらいだから丈夫と云わねばなるまい。吾輩は「はあ」と返事をした。少し間(ま)が抜けたようだが別に名答も出て来なかったから仕方がない。「あれでも、もとは身分が大変好かったんだって。いつでもそうおっしゃるの」「へえ元は何だったんです」「何でも天璋院(てんしょういん)様の御祐筆(ごゆうひつ)の妹の御嫁に行った先(さ)きの御(お)っかさんの甥(おい)の娘なんだって」「何ですって?」「あの天璋院様の御祐筆の妹の御嫁にいった\ldots{}\ldots{}」「なるほど。少し待って下さい。天璋院様の妹の御祐筆の\ldots{}\ldots{}」「あらそうじゃないの、天璋院様の御祐筆の妹の\ldots{}\ldots{}」「よろしい分りました天璋院様のでしょう」「ええ」「御祐筆のでしょう」「そうよ」「御嫁に行った」「妹の御嫁に行ったですよ」「そうそう間違った。妹の御嫁に入(い)った先きの」「御っかさんの甥の娘なんですとさ」「御っかさんの甥の娘なんですか」「ええ。分ったでしょう」「いいえ。何だか混雑して要領を得ないですよ。詰(つま)るところ天璋院様の何になるんですか」「あなたもよっぽど分らないのね。だから天璋院様の御祐筆の妹の御嫁に行った先きの御っかさんの甥の娘なんだって、先(さ)っきっから言ってるんじゃありませんか」「それはすっかり分っているんですがね」「それが分りさえすればいいんでしょう」「ええ」と仕方がないから降参をした。吾々は時とすると理詰の虚言(うそ)を吐(つ)かねばならぬ事がある。\\
 障子の中(うち)で二絃琴の音(ね)がぱったりやむと、御師匠さんの声で「三毛や三毛や御飯だよ」と呼ぶ。三毛子は嬉しそうに「あら御師匠さんが呼んでいらっしゃるから、私(あた)し帰るわ、よくって?」わるいと云ったって仕方がない。「それじゃまた遊びにいらっしゃい」と鈴をちゃらちゃら鳴らして庭先までかけて行ったが急に戻って来て「あなた大変色が悪くってよ。どうかしやしなくって」と心配そうに問いかける。まさか雑煮(ぞうに)を食って踊りを踊ったとも云われないから「何別段の事もありませんが、少し考え事をしたら頭痛がしてね。あなたと話しでもしたら直るだろうと思って実は出掛けて来たのですよ」「そう。御大事になさいまし。さようなら」少しは名残(なご)り惜し気に見えた。これで雑煮の元気もさっぱりと回復した。いい心持になった。帰りに例の茶園(ちゃえん)を通り抜けようと思って霜柱(しもばしら)の融(と)けかかったのを踏みつけながら建仁寺(けんにんじ)の崩(くず)れから顔を出すとまた車屋の黒が枯菊の上に背(せ)を山にして欠伸(あくび)をしている。近頃は黒を見て恐怖するような吾輩ではないが、話しをされると面倒だから知らぬ顔をして行き過ぎようとした。黒の性質として他(ひと)が己(おの)れを軽侮(けいぶ)したと認むるや否や決して黙っていない。「おい、名なしの権兵衛(ごんべえ)、近頃じゃ乙(おつ)う高く留ってるじゃあねえか。いくら教師の飯を食ったって、そんな高慢ちきな面(つ)らあするねえ。人(ひと)つけ面白くもねえ」黒は吾輩の有名になったのを、まだ知らんと見える。説明してやりたいが到底(とうてい)分る奴ではないから、まず一応の挨拶をして出来得る限り早く御免蒙(ごめんこうむ)るに若(し)くはないと決心した。「いや黒君おめでとう。不相変(あいかわらず)元気がいいね」と尻尾(しっぽ)を立てて左へくるりと廻わす。黒は尻尾を立てたぎり挨拶もしない。「何おめでてえ? 正月でおめでたけりゃ、御めえなんざあ年が年中おめでてえ方だろう。気をつけろい、この吹(ふ)い子(ご)の向(むこ)う面(づら)め」吹い子の向うづらという句は罵詈(ばり)の言語であるようだが、吾輩には了解が出来なかった。「ちょっと伺(うか)がうが吹い子の向うづらと云うのはどう云う意味かね」「へん、手めえが悪体(あくたい)をつかれてる癖に、その訳(わけ)を聞きゃ世話あねえ、だから正月野郎だって事よ」正月野郎は詩的であるが、その意味に至ると吹い子の何とかよりも一層不明瞭な文句である。参考のためちょっと聞いておきたいが、聞いたって明瞭な答弁は得られぬに極(き)まっているから、面(めん)と対(むか)ったまま無言で立っておった。いささか手持無沙汰の体(てい)である。すると突然黒のうちの神(かみ)さんが大きな声を張り揚げて「おや棚へ上げて置いた鮭(しゃけ)がない。大変だ。またあの黒の畜生(ちきしょう)が取ったんだよ。ほんとに憎らしい猫だっちゃありゃあしない。今に帰って来たら、どうするか見ていやがれ」と怒鳴(どな)る。初春(はつはる)の長閑(のどか)な空気を無遠慮に震動させて、枝を鳴らさぬ君が御代(みよ)を大(おおい)に俗了(ぞくりょう)してしまう。黒は怒鳴るなら、怒鳴りたいだけ怒鳴っていろと云わぬばかりに横着な顔をして、四角な顋(あご)を前へ出しながら、あれを聞いたかと合図をする。今までは黒との応対で気がつかなかったが、見ると彼の足の下には一切れ二銭三厘に相当する鮭の骨が泥だらけになって転がっている。「君不相変(あいかわらず)やってるな」と今までの行き掛りは忘れて、つい感投詞を奉呈した。黒はそのくらいな事ではなかなか機嫌を直さない。「何がやってるでえ、この野郎。\emph{しゃけ}の一切や二切で相変らずたあ何だ。人を見縊(みく)びった事をいうねえ。憚(はばか)りながら車屋の黒だあ」と腕まくりの代りに右の前足を逆(さ)かに肩の辺(へん)まで掻(か)き上げた。「君が黒君だと云う事は、始めから知ってるさ」「知ってるのに、相変らずやってるたあ何だ。何だてえ事よ」と熱いのを頻(しき)りに吹き懸ける。人間なら胸倉(むなぐら)をとられて小突き廻されるところである。少々辟易(へきえき)して内心困った事になったなと思っていると、再び例の神さんの大声が聞える。「ちょいと西川さん、おい西川さんてば、用があるんだよこの人あ。牛肉を一斤(きん)すぐ持って来るんだよ。いいかい、分ったかい、牛肉の堅くないところを一斤だよ」と牛肉注文の声が四隣(しりん)の寂寞(せきばく)を破る。「へん年に一遍牛肉を誂(あつら)えると思って、いやに大きな声を出しゃあがらあ。牛肉一斤が隣り近所へ自慢なんだから始末に終えねえ阿魔(あま)だ」と黒は嘲(あざけ)りながら四つ足を踏張(ふんば)る。吾輩は挨拶のしようもないから黙って見ている。「一斤くらいじゃあ、承知が出来ねえんだが、仕方がねえ、いいから取っときゃ、今に食ってやらあ」と自分のために誂(あつら)えたもののごとくいう。「今度は本当の御馳走だ。結構結構」と吾輩はなるべく彼を帰そうとする。「御めっちの知った事じゃねえ。黙っていろ。うるせえや」と云いながら突然後足(あとあし)で霜柱(しもばしら)の崩(くず)れた奴を吾輩の頭へばさりと浴(あ)びせ掛ける。吾輩が驚ろいて、からだの泥を払っている間(ま)に黒は垣根を潜(くぐ)って、どこかへ姿を隠した。大方西川の牛(ぎゅう)を覘(ねらい)に行ったものであろう。\\
 家(うち)へ帰ると座敷の中が、いつになく春めいて主人の笑い声さえ陽気に聞える。はてなと明け放した椽側から上(あが)って主人の傍(そば)へ寄って見ると見馴れぬ客が来ている。頭を奇麗に分けて、木綿(もめん)の紋付の羽織に小倉(こくら)の袴(はかま)を着けて至極(しごく)真面目そうな書生体(しょせいてい)の男である。主人の手あぶりの角を見ると春慶塗(しゅんけいぬ)りの巻煙草(まきたばこ)入れと並んで越智東風君(おちとうふうくん)を紹介致候(そろ)水島寒月という名刺があるので、この客の名前も、寒月君の友人であるという事も知れた。主客(しゅかく)の対話は途中からであるから前後がよく分らんが、何でも吾輩が前回に紹介した美学者迷亭君の事に関しているらしい。\\
「それで面白い趣向があるから是非いっしょに来いとおっしゃるので」と客は落ちついて云う。「何ですか、その西洋料理へ行って午飯(ひるめし)を食うのについて趣向があるというのですか」と主人は茶を続(つ)ぎ足して客の前へ押しやる。「さあ、その趣向というのが、その時は私にも分らなかったんですが、いずれあの方(かた)の事ですから、何か面白い種があるのだろうと思いまして\ldots{}\ldots{}」「いっしょに行きましたか、なるほど」「ところが驚いたのです」主人はそれ見たかと云わぬばかりに、膝(ひざ)の上に乗った吾輩の頭をぽかと叩(たた)く。少し痛い。「また馬鹿な茶番見たような事なんでしょう。あの男はあれが癖でね」と急にアンドレア・デル・サルト事件を思い出す。「へへー。君何か変ったものを食おうじゃないかとおっしゃるので」「何を食いました」「まず献立(こんだて)を見ながらいろいろ料理についての御話しがありました」「誂(あつ)らえない前にですか」「ええ」「それから」「それから首を捻(ひね)ってボイの方を御覧になって、どうも変ったものもないようだなとおっしゃるとボイは負けぬ気で鴨(かも)のロースか小牛のチャップなどは如何(いかが)ですと云うと、先生は、そんな月並(つきなみ)を食いにわざわざここまで来やしないとおっしゃるんで、ボイは月並という意味が分らんものですから妙な顔をして黙っていましたよ」「そうでしょう」「それから私の方を御向きになって、君仏蘭西(フランス)や英吉利(イギリス)へ行くと随分天明調(てんめいちょう)や万葉調(まんようちょう)が食えるんだが、日本じゃどこへ行ったって版で圧(お)したようで、どうも西洋料理へ這入(はい)る気がしないと云うような大気\includegraphics{../../../gaiji/1-87/1-87-64.png}(だいきえん)で――全体あの方(かた)は洋行なすった事があるのですかな」「何迷亭が洋行なんかするもんですか、そりゃ金もあり、時もあり、行こうと思えばいつでも行かれるんですがね。大方これから行くつもりのところを、過去に見立てた洒落(しゃれ)なんでしょう」と主人は自分ながらうまい事を言ったつもりで誘い出し笑をする。客はさまで感服した様子もない。「そうですか、私はまたいつの間(ま)に洋行なさったかと思って、つい真面目に拝聴していました。それに見て来たように\emph{なめくじ}のソップの御話や蛙(かえる)のシチュの形容をなさるものですから」「そりゃ誰かに聞いたんでしょう、うそをつく事はなかなか名人ですからね」「どうもそうのようで」と花瓶(かびん)の水仙を眺める。少しく残念の気色(けしき)にも取られる。「じゃ趣向というのは、それなんですね」と主人が念を押す。「いえそれはほんの冒頭なので、本論はこれからなのです」「ふーん」と主人は好奇的な感投詞を挟(はさ)む。「それから、とても\emph{なめくじ}や蛙は食おうっても食えやしないから、まあ\emph{トチメンボー}くらいなところで負けとく事にしようじゃないか君と御相談なさるものですから、私はつい何の気なしに、それがいいでしょう、といってしまったので」「へー、とちめんぼうは妙ですな」「ええ全く妙なのですが、先生があまり真面目だものですから、つい気がつきませんでした」とあたかも主人に向って麁忽(そこつ)を詫(わ)びているように見える。「それからどうしました」と主人は無頓着に聞く。客の謝罪には一向同情を表しておらん。「それからボイにおい\emph{トチメンボー}を二人前(ににんまえ)持って来いというと、ボイが\emph{メンチボー}ですかと聞き直しましたが、先生はますます真面目(まじめ)な貌(かお)で\emph{メンチボー}じゃない\emph{トチメンボー}だと訂正されました」「なある。その\emph{トチメンボー}という料理は一体あるんですか」「さあ私も少しおかしいとは思いましたがいかにも先生が沈着であるし、その上あの通りの西洋通でいらっしゃるし、ことにその時は洋行なすったものと信じ切っていたものですから、私も口を添えて\emph{トチメンボー}だ\emph{トチメンボー}だとボイに教えてやりました」「ボイはどうしました」「ボイがね、今考えると実に滑稽(こっけい)なんですがね、しばらく思案していましてね、はなはだ御気の毒様ですが今日は\emph{トチメンボー}は御生憎様(おあいにくさま)で\emph{メンチボー}なら御二人前(おふたりまえ)すぐに出来ますと云うと、先生は非常に残念な様子で、それじゃせっかくここまで来た甲斐(かい)がない。どうか\emph{トチメンボー}を都合(つごう)して食わせてもらう訳(わけ)には行くまいかと、ボイに二十銭銀貨をやられると、ボイはそれではともかくも料理番と相談して参りましょうと奥へ行きましたよ」「大変\emph{トチメンボー}が食いたかったと見えますね」「しばらくしてボイが出て来て真(まこと)に御生憎で、御誂(おあつらえ)ならこしらえますが少々時間がかかります、と云うと迷亭先生は落ちついたもので、どうせ我々は正月でひまなんだから、少し待って食って行こうじゃないかと云いながらポッケットから葉巻を出してぷかりぷかり吹かし始められたので、私(わたく)しも仕方がないから、懐(ふところ)から日本新聞を出して読み出しました、するとボイはまた奥へ相談に行きましたよ」「いやに手数(てすう)が掛りますな」と主人は戦争の通信を読むくらいの意気込で席を前(すす)める。「するとボイがまた出て来て、近頃は\emph{トチメンボー}の材料が払底で亀屋へ行っても横浜の十五番へ行っても買われませんから当分の間は御生憎様でと気の毒そうに云うと、先生はそりゃ困ったな、せっかく来たのになあと私の方を御覧になってしきりに繰り返さるるので、私も黙っている訳にも参りませんから、どうも遺憾(いかん)ですな、遺憾極(きわま)るですなと調子を合せたのです」「ごもっともで」と主人が賛成する。何がごもっともだか吾輩にはわからん。「するとボイも気の毒だと見えて、その内材料が参りましたら、どうか願いますってんでしょう。先生が材料は何を使うかねと問われるとボイはへへへへと笑って返事をしないんです。材料は日本派の俳人だろうと先生が押し返して聞くとボイはへえさようで、それだものだから近頃は横浜へ行っても買われませんので、まことにお気の毒様と云いましたよ」「アハハハそれが落ちなんですか、こりゃ面白い」と主人はいつになく大きな声で笑う。膝(ひざ)が揺れて吾輩は落ちかかる。主人はそれにも頓着(とんじゃく)なく笑う。アンドレア・デル・サルトに罹(かか)ったのは自分一人でないと云う事を知ったので急に愉快になったものと見える。「それから二人で表へ出ると、どうだ君うまく行ったろう、橡面坊(とちめんぼう)を種に使ったところが面白かろうと大得意なんです。敬服の至りですと云って御別れしたようなものの実は午飯(ひるめし)の時刻が延びたので大変空腹になって弱りましたよ」「それは御迷惑でしたろう」と主人は始めて同情を表する。これには吾輩も異存はない。しばらく話しが途切れて吾輩の咽喉(のど)を鳴らす音が主客(しゅかく)の耳に入る。\\
 東風君は冷めたくなった茶をぐっと飲み干して「実は今日参りましたのは、少々先生に御願があって参ったので」と改まる。「はあ、何か御用で」と主人も負けずに済(す)ます。「御承知の通り、文学美術が好きなものですから\ldots{}\ldots{}」「結構で」と油を注(さ)す。「同志だけがよりましてせんだってから朗読会というのを組織しまして、毎月一回会合してこの方面の研究をこれから続けたいつもりで、すでに第一回は去年の暮に開いたくらいであります」「ちょっと伺っておきますが、朗読会と云うと何か節奏(ふし)でも附けて、詩歌(しいか)文章の類(るい)を読むように聞えますが、一体どんな風にやるんです」「まあ初めは古人の作からはじめて、追々(おいおい)は同人の創作なんかもやるつもりです」「古人の作というと白楽天(はくらくてん)の琵琶行(びわこう)のようなものででもあるんですか」「いいえ」「蕪村(ぶそん)の春風馬堤曲(しゅんぷうばていきょく)の種類ですか」「いいえ」「それじゃ、どんなものをやったんです」「せんだっては近松の心中物(しんじゅうもの)をやりました」「近松? あの浄瑠璃(じょうるり)の近松ですか」近松に二人はない。近松といえば戯曲家の近松に極(きま)っている。それを聞き直す主人はよほど愚(ぐ)だと思っていると、主人は何にも分らずに吾輩の頭を叮嚀(ていねい)に撫(な)でている。藪睨(やぶにら)みから惚(ほ)れられたと自認している人間もある世の中だからこのくらいの誤謬(ごびゅう)は決して驚くに足らんと撫でらるるがままにすましていた。「ええ」と答えて東風子(とうふうし)は主人の顔色を窺(うかが)う。「それじゃ一人で朗読するのですか、または役割を極(き)めてやるんですか」「役を極めて懸合(かけあい)でやって見ました。その主意はなるべく作中の人物に同情を持ってその性格を発揮するのを第一として、それに手真似や身振りを添えます。白(せりふ)はなるべくその時代の人を写し出すのが主で、御嬢さんでも丁稚(でっち)でも、その人物が出てきたようにやるんです」「じゃ、まあ芝居見たようなものじゃありませんか」「ええ衣装(いしょう)と書割(かきわり)がないくらいなものですな」「失礼ながらうまく行きますか」「まあ第一回としては成功した方だと思います」「それでこの前やったとおっしゃる心中物というと」「その、船頭が御客を乗せて芳原(よしわら)へ行く所(とこ)なんで」「大変な幕をやりましたな」と教師だけにちょっと首を傾(かたむ)ける。鼻から吹き出した\emph{日の出}の煙りが耳を掠(かす)めて顔の横手へ廻る。「なあに、そんなに大変な事もないんです。登場の人物は御客と、船頭と、花魁(おいらん)と仲居(なかい)と遣手(やりて)と見番(けんばん)だけですから」と東風子は平気なものである。主人は花魁という名をきいてちょっと苦(にが)い顔をしたが、仲居、遣手、見番という術語について明瞭の智識がなかったと見えてまず質問を呈出した。「仲居というのは娼家(しょうか)の下婢(かひ)にあたるものですかな」「まだよく研究はして見ませんが仲居は茶屋の下女で、遣手というのが女部屋(おんなべや)の助役(じょやく)見たようなものだろうと思います」東風子はさっき、その人物が出て来るように仮色(こわいろ)を使うと云った癖に遣手や仲居の性格をよく解しておらんらしい。「なるほど仲居は茶屋に隷属(れいぞく)するもので、遣手は娼家に起臥(きが)する者ですね。次に\emph{見番}と云うのは人間ですかまたは一定の場所を指(さ)すのですか、もし人間とすれば男ですか女ですか」「見番は何でも男の人間だと思います」「何を司(つかさ)どっているんですかな」「さあそこまではまだ調べが届いておりません。その内調べて見ましょう」これで懸合をやった日には頓珍漢(とんちんかん)なものが出来るだろうと吾輩は主人の顔をちょっと見上げた。主人は存外真面目である。「それで朗読家は君のほかにどんな人が加わったんですか」「いろいろおりました。花魁が法学士のK君でしたが、口髯(くちひげ)を生やして、女の甘ったるいせりふを使(つ)かうのですからちょっと妙でした。それにその花魁が癪(しゃく)を起すところがあるので\ldots{}\ldots{}」「朗読でも癪を起さなくっちゃ、いけないんですか」と主人は心配そうに尋ねる。「ええとにかく表情が大事ですから」と東風子はどこまでも文芸家の気でいる。「うまく癪が起りましたか」と主人は警句を吐く。「癪だけは第一回には、ちと無理でした」と東風子も警句を吐く。「ところで君は何の役割でした」と主人が聞く。「私(わたく)しは船頭」「へー、君が船頭」君にして船頭が務(つと)まるものなら僕にも見番くらいはやれると云ったような語気を洩(も)らす。やがて「船頭は無理でしたか」と御世辞のないところを打ち明ける。東風子は別段癪に障った様子もない。やはり沈着な口調で「その船頭でせっかくの催しも竜頭蛇尾(りゅうとうだび)に終りました。実は会場の隣りに女学生が四五人下宿していましてね、それがどうして聞いたものか、その日は朗読会があるという事を、どこかで探知して会場の窓下へ来て傍聴していたものと見えます。私(わたく)しが船頭の仮色(こわいろ)を使って、ようやく調子づいてこれなら大丈夫と思って得意にやっていると、\ldots{}\ldots{}つまり身振りがあまり過ぎたのでしょう、今まで耐(こ)らえていた女学生が一度にわっと笑いだしたものですから、驚ろいた事も驚ろいたし、極(きま)りが悪(わ)るい事も悪るいし、それで腰を折られてから、どうしても後(あと)がつづけられないので、とうとうそれ限(ぎ)りで散会しました」第一回としては成功だと称する朗読会がこれでは、失敗はどんなものだろうと想像すると笑わずにはいられない。覚えず咽喉仏(のどぼとけ)がごろごろ鳴る。主人はいよいよ柔かに頭を撫(な)でてくれる。人を笑って可愛がられるのはありがたいが、いささか無気味なところもある。「それは飛んだ事で」と主人は正月早々弔詞(ちょうじ)を述べている。「第二回からは、もっと奮発して盛大にやるつもりなので、今日出ましたのも全くそのためで、実は先生にも一つ御入会の上御尽力を仰ぎたいので」「僕にはとても癪なんか起せませんよ」と消極的の主人はすぐに断わりかける。「いえ、癪などは起していただかんでもよろしいので、ここに賛助員の名簿が」と云いながら紫の風呂敷から大事そうに小菊版(こぎくばん)の帳面を出す。「これへどうか御署名の上御捺印(ごなついん)を願いたいので」と帳面を主人の膝(ひざ)の前へ開いたまま置く。見ると現今知名な文学博士、文学士連中の名が行儀よく勢揃(せいぞろい)をしている。「はあ賛成員にならん事もありませんが、どんな義務があるのですか」と牡蠣先生(かきせんせい)は掛念(けねん)の体(てい)に見える。「義務と申して別段是非願う事もないくらいで、ただ御名前だけを御記入下さって賛成の意さえ御表(おひょう)し被下(くださ)ればそれで結構です」「そんなら這入(はい)ります」と義務のかからぬ事を知るや否や主人は急に気軽になる。責任さえないと云う事が分っておれば謀叛(むほん)の連判状へでも名を書き入れますと云う顔付をする。加之(のみならず)こう知名の学者が名前を列(つら)ねている中に姓名だけでも入籍させるのは、今までこんな事に出合った事のない主人にとっては無上の光栄であるから返事の勢のあるのも無理はない。「ちょっと失敬」と主人は書斎へ印をとりに這入る。吾輩はぼたりと畳の上へ落ちる。東風子は菓子皿の中の\emph{カステラ}をつまんで一口に頬張(ほおば)る。モゴモゴしばらくは苦しそうである。吾輩は今朝の雑煮(ぞうに)事件をちょっと思い出す。主人が書斎から印形(いんぎょう)を持って出て来た時は、東風子の胃の中にカステラが落ちついた時であった。主人は菓子皿のカステラが一切(ひときれ)足りなくなった事には気が着かぬらしい。もし気がつくとすれば第一に疑われるものは吾輩であろう。\\
 東風子が帰ってから、主人が書斎に入って机の上を見ると、いつの間(ま)にか迷亭先生の手紙が来ている。\\

「新年の御慶(ぎょけい)目出度(めでたく)申納候(もうしおさめそろ)。\ldots{}\ldots{}」\\

 いつになく出が真面目だと主人が思う。迷亭先生の手紙に真面目なのはほとんどないので、この間などは「其後(そのご)別に恋着(れんちゃく)せる婦人も無之(これなく)、いず方(かた)より艶書(えんしょ)も参らず、先(ま)ず先(ま)ず無事に消光罷(まか)り在り候(そろ)間、乍憚(はばかりながら)御休心可被下候(くださるべくそろ)」と云うのが来たくらいである。それに較(くら)べるとこの年始状は例外にも世間的である。\\

「一寸参堂仕り度(たく)候えども、大兄の消極主義に反して、出来得る限り積極的方針を以(もっ)て、此千古未曾有(みぞう)の新年を迎うる計画故、毎日毎日目の廻る程の多忙、御推察願上候(そろ)\ldots{}\ldots{}」\\

 なるほどあの男の事だから正月は遊び廻るのに忙がしいに違いないと、主人は腹の中で迷亭君に同意する。\\

「昨日は一刻のひまを偸(ぬす)み、東風子に\emph{トチメンボー}の御馳走(ごちそう)を致さんと存じ候処(そろところ)、生憎(あいにく)材料払底の為(た)め其意を果さず、遺憾(いかん)千万に存候(ぞんじそろ)。\ldots{}\ldots{}」\\

 そろそろ例の通りになって来たと主人は無言で微笑する。\\

「明日は某男爵の歌留多会(かるたかい)、明後日は審美学協会の新年宴会、其明日は鳥部教授歓迎会、其又明日は\ldots{}\ldots{}」\\

 うるさいなと、主人は読みとばす。\\

「右の如く謡曲会、俳句会、短歌会、新体詩会等、会の連発にて当分の間は、のべつ幕無しに出勤致し候(そろ)為め、不得已(やむをえず)賀状を以て拝趨(はいすう)の礼に易(か)え候段(そろだん)不悪(あしからず)御宥恕(ごゆうじょ)被下度候(くだされたくそろ)。\ldots{}\ldots{}」\\

 別段くるにも及ばんさと、主人は手紙に返事をする。\\

「今度御光来の節は久し振りにて晩餐でも供し度(たき)心得に御座候(そろ)。寒厨(かんちゅう)何の珍味も無之候(これなくそうら)えども、せめては\emph{トチメンボー}でもと只今より心掛居候(おりそろ)。\ldots{}\ldots{}」\\

 まだ\emph{トチメンボー}を振り廻している。失敬なと主人はちょっとむっとする。\\

「然(しか)し\emph{トチメンボー}は近頃材料払底の為め、ことに依ると間に合い兼候(かねそろ)も計りがたきにつき、其節は孔雀(くじゃく)の舌(した)でも御風味に入れ可申候(もうすべくそろ)。\ldots{}\ldots{}」\\

 両天秤(りょうてんびん)をかけたなと主人は、あとが読みたくなる。\\

「御承知の通り孔雀一羽につき、舌肉の分量は小指の半(なか)ばにも足らぬ程故健啖(けんたん)なる大兄の胃嚢(いぶくろ)を充(み)たす為には\ldots{}\ldots{}」\\

 うそをつけと主人は打ち遣(や)ったようにいう。\\

「是非共二三十羽の孔雀を捕獲致さざる可(べか)らずと存候(ぞんじそろ)。然る所孔雀は動物園、浅草花屋敷等には、ちらほら見受け候えども、普通の鳥屋抔(など)には一向(いっこう)見当り不申(もうさず)、苦心(くしん)此事(このこと)に御座候(そろ)。\ldots{}\ldots{}」\\

 独りで勝手に苦心しているのじゃないかと主人は毫(ごう)も感謝の意を表しない。\\

「此孔雀の舌の料理は往昔(おうせき)羅馬(ローマ)全盛の砌(みぎ)り、一時非常に流行致し候(そろ)ものにて、豪奢(ごうしゃ)風流の極度と平生よりひそかに食指(しょくし)を動かし居候(おりそろ)次第御諒察(ごりょうさつ)可被下候(くださるべくそろ)。\ldots{}\ldots{}」\\

 何が御諒察だ、馬鹿なと主人はすこぶる冷淡である。\\

「降(くだ)って十六七世紀の頃迄は全欧を通じて孔雀は宴席に欠くべからざる好味と相成居候(あいなりおりそろ)。レスター伯がエリザベス女皇(じょこう)をケニルウォースに招待致し候節(そろせつ)も慥(たし)か孔雀を使用致し候様(そろよう)記憶致候(いたしそろ)。有名なるレンブラントが画(えが)き候(そろ)饗宴の図にも孔雀が尾を広げたる儘(まま)卓上に横(よこた)わり居り候(そろ)\ldots{}\ldots{}」\\

 孔雀の料理史をかくくらいなら、そんなに多忙でもなさそうだと不平をこぼす。\\

「とにかく近頃の如く御馳走の食べ続けにては、さすがの小生も遠からぬうちに大兄の如く胃弱と相成(あいな)るは必定(ひつじょう)\ldots{}\ldots{}」\\

 大兄のごとくは余計だ。何も僕を胃弱の標準にしなくても済むと主人はつぶやいた。\\

「歴史家の説によれば羅馬人(ローマじん)は日に二度三度も宴会を開き候由(そろよし)。日に二度も三度も方丈(ほうじょう)の食饌(しょくせん)に就き候えば如何なる健胃の人にても消化機能に不調を醸(かも)すべく、従って自然は大兄の如く\ldots{}\ldots{}」\\

 また大兄のごとくか、失敬な。\\

「然(しか)るに贅沢(ぜいたく)と衛生とを両立せしめんと研究を尽したる彼等は不相当に多量の滋味を貪(むさぼ)ると同時に胃腸を常態に保持するの必要を認め、ここに一の秘法を案出致し候(そろ)\ldots{}\ldots{}」\\

 はてねと主人は急に熱心になる。\\

「彼等は食後必ず入浴致候(いたしそろ)。入浴後一種の方法によりて浴前(よくぜん)に嚥下(えんか)せるものを悉(ことごと)く嘔吐(おうと)し、胃内を掃除致し候(そろ)。胃内廓清(いないかくせい)の功を奏したる後(のち)又食卓に就(つ)き、飽(あ)く迄珍味を風好(ふうこう)し、風好し了(おわ)れば又湯に入りて之(これ)を吐出(としゅつ)致候(いたしそろ)。かくの如くすれば好物は貪(むさ)ぼり次第貪り候(そうろう)も毫(ごう)も内臓の諸機関に障害を生ぜず、一挙両得とは此等の事を可申(もうすべき)かと愚考致候(いたしそろ)\ldots{}\ldots{}」\\

 なるほど一挙両得に相違ない。主人は羨(うらや)ましそうな顔をする。\\

「廿世紀の今日(こんにち)交通の頻繁(ひんぱん)、宴会の増加は申す迄もなく、軍国多事征露の第二年とも相成候折柄(そろおりから)、吾人戦勝国の国民は、是非共羅馬(ローマ)人に傚(なら)って此入浴嘔吐の術を研究せざるべからざる機会に到着致し候(そろ)事と自信致候(いたしそろ)。左(さ)もなくば切角(せっかく)の大国民も近き将来に於て悉(ことごと)く大兄の如く胃病患者と相成る事と窃(ひそ)かに心痛罷(まか)りあり候(そろ)\ldots{}\ldots{}」\\

 また大兄のごとくか、癪(しゃく)に障(さわ)る男だと主人が思う。\\

「此際吾人西洋の事情に通ずる者が古史伝説を考究し、既に廃絶せる秘法を発見し、之を明治の社会に応用致し候わば所謂(いわば)禍(わざわい)を未萌(みほう)に防ぐの功徳(くどく)にも相成り平素逸楽(いつらく)を擅(ほしいまま)に致し候(そろ)御恩返も相立ち可申(もうすべく)と存候(ぞんじそろ)\ldots{}\ldots{}」\\

 何だか妙だなと首を捻(ひね)る。\\

「依(よっ)て此間中(じゅう)よりギボン、モンセン、スミス等諸家の著述を渉猟(しょうりょう)致し居候(おりそうら)えども未(いま)だに発見の端緒(たんしょ)をも見出(みいだ)し得ざるは残念の至に存候(ぞんじそろ)。然し御存じの如く小生は一度思い立ち候事(そろこと)は成功するまでは決して中絶仕(つかまつ)らざる性質に候えば嘔吐方(おうとほう)を再興致し候(そろ)も遠からぬうちと信じ居り候(そろ)次第。右は発見次第御報道可仕候(つかまつるべくそろ)につき、左様御承知可被下候(くださるべくそろ)。就(つい)てはさきに申上候(そろ)\emph{トチメンボー}及び孔雀の舌の御馳走も可相成(あいなるべく)は右発見後に致し度(たく)、左(さ)すれば小生の都合は勿論(もちろん)、既に胃弱に悩み居らるる大兄の為にも御便宜(ごべんぎ)かと存候(ぞんじそろ)草々不備」\\

 何だとうとう担(かつ)がれたのか、あまり書き方が真面目だものだからつい仕舞(しまい)まで本気にして読んでいた。新年匆々(そうそう)こんな悪戯(いたずら)をやる迷亭はよっぽどひま人だなあと主人は笑いながら云った。\\
 それから四五日は別段の事もなく過ぎ去った。白磁(はくじ)の水仙がだんだん凋(しぼ)んで、青軸(あおじく)の梅が瓶(びん)ながらだんだん開きかかるのを眺め暮らしてばかりいてもつまらんと思って、一両度(いちりょうど)三毛子を訪問して見たが逢(あ)われない。最初は留守だと思ったが、二返目(へんめ)には病気で寝ているという事が知れた。障子の中で例の御師匠さんと下女が話しをしているのを手水鉢(ちょうずばち)の葉蘭の影に隠れて聞いているとこうであった。\\
「三毛は御飯をたべるかい」「いいえ今朝からまだ何(なん)にも食べません、あったかにして御火燵(おこた)に寝かしておきました」何だか猫らしくない。まるで人間の取扱を受けている。\\
 一方では自分の境遇と比べて見て羨(うらや)ましくもあるが、一方では己(おの)が愛している猫がかくまで厚遇を受けていると思えば嬉しくもある。\\
「どうも困るね、御飯をたべないと、身体(からだ)が疲れるばかりだからね」「そうでございますとも、私共でさえ一日御\includegraphics{../../../gaiji/2-92/2-92-71.png}(ごぜん)をいただかないと、明くる日はとても働けませんもの」\\
 下女は自分より猫の方が上等な動物であるような返事をする。実際この家(うち)では下女より猫の方が大切かも知れない。\\
「御医者様へ連れて行ったのかい」「ええ、あの御医者はよっぽど妙でございますよ。私が三毛をだいて診察場へ行くと、風邪(かぜ)でも引いたのかって私の脈(みゃく)をとろうとするんでしょう。いえ病人は私ではございません。これですって三毛を膝の上へ直したら、にやにや笑いながら、猫の病気はわしにも分らん、抛(ほう)っておいたら今に癒(なお)るだろうってんですもの、あんまり苛(ひど)いじゃございませんか。腹が立ったから、それじゃ見ていただかなくってもようございますこれでも大事の猫なんですって、三毛を懐(ふところ)へ入れてさっさと帰って参りました」「ほんにねえ」\\
「ほんにねえ」は到底(とうてい)吾輩のうちなどで聞かれる言葉ではない。やはり天璋院(てんしょういん)様の何とかの何とかでなくては使えない、はなはだ雅(が)であると感心した。\\
「何だかしくしく云うようだが\ldots{}\ldots{}」「ええきっと風邪を引いて咽喉(のど)が痛むんでございますよ。風邪を引くと、どなたでも御咳(おせき)が出ますからね\ldots{}\ldots{}」\\
 天璋院様の何とかの何とかの下女だけに馬鹿叮嚀(ていねい)な言葉を使う。\\
「それに近頃は肺病とか云うものが出来てのう」「ほんとにこの頃のように肺病だのペストだのって新しい病気ばかり殖(ふ)えた日にゃ油断も隙もなりゃしませんのでございますよ」「旧幕時代に無い者に碌(ろく)な者はないから御前も気をつけないといかんよ」「そうでございましょうかねえ」\\
 下女は大(おおい)に感動している。\\
「風邪(かぜ)を引くといってもあまり出あるきもしないようだったに\ldots{}\ldots{}」「いえね、あなた、それが近頃は悪い友達が出来ましてね」\\
 下女は国事の秘密でも語る時のように大得意である。\\
「悪い友達?」「ええあの表通りの教師の所(とこ)にいる薄ぎたない雄猫(おねこ)でございますよ」「教師と云うのは、あの毎朝無作法な声を出す人かえ」「ええ顔を洗うたんびに鵝鳥(がちょう)が絞(し)め殺されるような声を出す人でござんす」\\
 鵝鳥が絞め殺されるような声はうまい形容である。吾輩の主人は毎朝風呂場で含嗽(うがい)をやる時、楊枝(ようじ)で咽喉(のど)をつっ突いて妙な声を無遠慮に出す癖がある。機嫌の悪い時はやけにがあがあやる、機嫌の好い時は元気づいてなおがあがあやる。つまり機嫌のいい時も悪い時も休みなく勢よくがあがあやる。細君の話しではここへ引越す前まではこんな癖はなかったそうだが、ある時ふとやり出してから今日(きょう)まで一日もやめた事がないという。ちょっと厄介な癖であるが、なぜこんな事を根気よく続けているのか吾等猫などには到底(とうてい)想像もつかん。それもまず善いとして「薄ぎたない猫」とは随分酷評をやるものだとなお耳を立ててあとを聞く。\\
「あんな声を出して何の呪(まじな)いになるか知らん。御維新前(ごいっしんまえ)は中間(ちゅうげん)でも草履(ぞうり)取りでも相応の作法は心得たもので、屋敷町などで、あんな顔の洗い方をするものは一人もおらなかったよ」「そうでございましょうともねえ」\\
 下女は無暗(むやみ)に感服しては、無暗に\emph{ねえ}を使用する。\\
「あんな主人を持っている猫だから、どうせ野良猫(のらねこ)さ、今度来たら少し叩(たた)いておやり」「叩いてやりますとも、三毛の病気になったのも全くあいつの御蔭に相違ございませんもの、きっと讐(かたき)をとってやります」\\
 飛んだ冤罪(えんざい)を蒙(こうむ)ったものだ。こいつは滅多(めった)に近(ち)か寄(よ)れないと三毛子にはとうとう逢わずに帰った。\\
 帰って見ると主人は書斎の中(うち)で何か沈吟(ちんぎん)の体(てい)で筆を執(と)っている。二絃琴(にげんきん)の御師匠さんの所(とこ)で聞いた評判を話したら、さぞ怒(おこ)るだろうが、知らぬが仏とやらで、うんうん云いながら神聖な詩人になりすましている。\\
 ところへ当分多忙で行かれないと云って、わざわざ年始状をよこした迷亭君が飄然(ひょうぜん)とやって来る。「何か新体詩でも作っているのかね。面白いのが出来たら見せたまえ」と云う。「うん、ちょっとうまい文章だと思ったから今翻訳して見ようと思ってね」と主人は重たそうに口を開く。「文章? 誰(だ)れの文章だい」「誰れのか分らんよ」「無名氏か、無名氏の作にも随分善いのがあるからなかなか馬鹿に出来ない。全体どこにあったのか」と問う。「第二読本」と主人は落ちつきはらって答える。「第二読本? 第二読本がどうしたんだ」「僕の翻訳している名文と云うのは第二読本の中(うち)にあると云う事さ」「冗談(じょうだん)じゃない。孔雀の舌の讐(かたき)を際(きわ)どいところで討とうと云う寸法なんだろう」「僕は君のような法螺吹(ほらふ)きとは違うさ」と口髯(くちひげ)を捻(ひね)る。泰然たるものだ。「昔(むか)しある人が山陽に、先生近頃名文はござらぬかといったら、山陽が馬子(まご)の書いた借金の催促状を示して近来の名文はまずこれでしょうと云ったという話があるから、君の審美眼も存外たしかかも知れん。どれ読んで見給え、僕が批評してやるから」と迷亭先生は審美眼の本家(ほんけ)のような事を云う。主人は禅坊主が大燈国師(だいとうこくし)の遺誡(ゆいかい)を読むような声を出して読み始める。「巨人(きょじん)、引力(いんりょく)」「何だいその巨人引力と云うのは」「巨人引力と云う題さ」「妙な題だな、僕には意味がわからんね」「引力と云う名を持っている巨人というつもりさ」「少し無理な\emph{つもり}だが表題だからまず負けておくとしよう。それから早々(そうそう)本文を読むさ、君は声が善いからなかなか面白い」「雑(ま)ぜかえしてはいかんよ」と予(あらか)じめ念を押してまた読み始める。\\

ケートは窓から外面(そと)を眺(なが)める。小児(しょうに)が球(たま)を投げて遊んでいる。彼等は高く球を空中に擲(なげう)つ。球は上へ上へとのぼる。しばらくすると落ちて来る。彼等はまた球を高く擲つ。再び三度。擲つたびに球は落ちてくる。なぜ落ちるのか、なぜ上へ上へとのみのぼらぬかとケートが聞く。「巨人が地中に住む故に」と母が答える。「彼は巨人引力である。彼は強い。彼は万物を己(おの)れの方へと引く。彼は家屋を地上に引く。引かねば飛んでしまう。小児も飛んでしまう。葉が落ちるのを見たろう。あれは巨人引力が呼ぶのである。本を落す事があろう。巨人引力が来いというからである。球が空にあがる。巨人引力は呼ぶ。呼ぶと落ちてくる」\\

「それぎりかい」「むむ、甘(うま)いじゃないか」「いやこれは恐れ入った。飛んだところで\emph{トチメンボー}の御返礼に預(あずか)った」「御返礼でもなんでもないさ、実際うまいから訳して見たのさ、君はそう思わんかね」と金縁の眼鏡の奥を見る。「どうも驚ろいたね。君にしてこの伎倆(ぎりょう)あらんとは、全く此度(こんど)という今度(こんど)は担(かつ)がれたよ、降参降参」と一人で承知して一人で喋舌(しゃべ)る。主人には一向(いっこう)通じない。「何も君を降参させる考えはないさ。ただ面白い文章だと思ったから訳して見たばかりさ」「いや実に面白い。そう来なくっちゃ本ものでない。凄(すご)いものだ。恐縮だ」「そんなに恐縮するには及ばん。僕も近頃は水彩画をやめたから、その代りに文章でもやろうと思ってね」「どうして遠近(えんきん)無差別(むさべつ)黒白(こくびゃく)平等(びょうどう)の水彩画の比じゃない。感服の至りだよ」「そうほめてくれると僕も乗り気になる」と主人はあくまでも疳違(かんちが)いをしている。\\
 ところへ寒月(かんげつ)君が先日は失礼しましたと這入(はい)って来る。「いや失敬。今大変な名文を拝聴して\emph{トチメンボー}の亡魂を退治(たいじ)られたところで」と迷亭先生は訳のわからぬ事をほのめかす。「はあ、そうですか」とこれも訳の分らぬ挨拶をする。主人だけは左(さ)のみ浮かれた気色(けしき)もない。「先日は君の紹介で越智東風(おちとうふう)と云う人が来たよ」「ああ上(あが)りましたか、あの越智東風(おちこち)と云う男は至って正直な男ですが少し変っているところがあるので、あるいは御迷惑かと思いましたが、是非紹介してくれというものですから\ldots{}\ldots{}」「別に迷惑の事もないがね\ldots{}\ldots{}」「こちらへ上(あが)っても自分の姓名のことについて何か弁じて行きゃしませんか」「いいえ、そんな話もなかったようだ」「そうですか、どこへ行っても初対面の人には自分の名前の講釈(こうしゃく)をするのが癖でしてね」「どんな講釈をするんだい」と事あれかしと待ち構えた迷亭君は口を入れる。「あの東風(こち)と云うのを音(おん)で読まれると大変気にするので」「はてね」と迷亭先生は金唐皮(きんからかわ)の煙草入(たばこいれ)から煙草をつまみ出す。「私(わたく)しの名は越智東風(おちとうふう)ではありません、越智(おち)\emph{こち}ですと必ず断りますよ」「妙だね」と雲井(くもい)を腹の底まで呑(の)み込む。「それが全く文学熱から来たので、こちと読むと\emph{遠近}と云う成語(せいご)になる、のみならずその姓名が韻(いん)を踏んでいると云うのが得意なんです。それだから東風(こち)を音(おん)で読むと僕がせっかくの苦心を人が買ってくれないといって不平を云うのです」「こりゃなるほど変ってる」と迷亭先生は図に乗って腹の底から雲井を鼻の孔(あな)まで吐き返す。途中で煙が戸迷(とまど)いをして咽喉(のど)の出口へ引きかかる。先生は煙管(きせる)を握ってごほんごほんと咽(むせ)び返る。「先日来た時は朗読会で船頭になって女学生に笑われたといっていたよ」と主人は笑いながら云う。「うむそれそれ」と迷亭先生が煙管(きせる)で膝頭(ひざがしら)を叩(たた)く。吾輩は険呑(けんのん)になったから少し傍(そば)を離れる。「その朗読会さ。せんだって\emph{トチメンボー}を御馳走した時にね。その話しが出たよ。何でも第二回には知名の文士を招待して大会をやるつもりだから、先生にも是非御臨席を願いたいって。それから僕が今度も近松の世話物をやるつもりかいと聞くと、いえこの次はずっと新しい者を撰(えら)んで金色夜叉(こんじきやしゃ)にしましたと云うから、君にゃ何の役が当ってるかと聞いたら私は御宮(おみや)ですといったのさ。東風(とうふう)の御宮は面白かろう。僕は是非出席して喝采(かっさい)しようと思ってるよ」「面白いでしょう」と寒月君が妙な笑い方をする。「しかしあの男はどこまでも誠実で軽薄なところがないから好い。迷亭などとは大違いだ」と主人はアンドレア・デル・サルトと孔雀(くじゃく)の舌と\emph{トチメンボー}の復讐(かたき)を一度にとる。迷亭君は気にも留めない様子で「どうせ僕などは行徳(ぎょうとく)の俎(まないた)と云う格だからなあ」と笑う。「まずそんなところだろう」と主人が云う。実は行徳の俎と云う語を主人は解(かい)さないのであるが、さすが永年教師をして胡魔化(ごまか)しつけているものだから、こんな時には教場の経験を社交上にも応用するのである。「行徳の俎というのは何の事ですか」と寒月が真率(しんそつ)に聞く。主人は床の方を見て「あの水仙は暮に僕が風呂の帰りがけに買って来て挿(さ)したのだが、よく持つじゃないか」と行徳の俎を無理にねじ伏せる。「暮といえば、去年の暮に僕は実に不思議な経験をしたよ」と迷亭が煙管(きせる)を大神楽(だいかぐら)のごとく指の尖(さき)で廻わす。「どんな経験か、聞かし玉(たま)え」と主人は行徳の俎を遠く後(うしろ)に見捨てた気で、ほっと息をつく。迷亭先生の不思議な経験というのを聞くと左(さ)のごとくである。\\
「たしか暮の二十七日と記憶しているがね。例の東風(とうふう)から参堂の上是非文芸上の御高話を伺いたいから御在宿を願うと云う先(さ)き触(ぶ)れがあったので、朝から心待ちに待っていると先生なかなか来ないやね。昼飯を食ってストーブの前でバリー・ペーンの滑稽物(こっけいもの)を読んでいるところへ静岡の母から手紙が来たから見ると、年寄だけにいつまでも僕を小供のように思ってね。寒中は夜間外出をするなとか、冷水浴もいいがストーブを焚(た)いて室(へや)を煖(あたた)かにしてやらないと風邪(かぜ)を引くとかいろいろの注意があるのさ。なるほど親はありがたいものだ、他人ではとてもこうはいかないと、呑気(のんき)な僕もその時だけは大(おおい)に感動した。それにつけても、こんなにのらくらしていては勿体(もったい)ない。何か大著述でもして家名を揚げなくてはならん。母の生きているうちに天下をして明治の文壇に迷亭先生あるを知らしめたいと云う気になった。それからなお読んで行くと御前なんぞは実に仕合せ者だ。露西亜(ロシア)と戦争が始まって若い人達は大変な辛苦(しんく)をして御国(みくに)のために働らいているのに節季師走(せっきしわす)でもお正月のように気楽に遊んでいると書いてある。――僕はこれでも母の思ってるように遊んじゃいないやね――そのあとへ以(もっ)て来て、僕の小学校時代の朋友(ほうゆう)で今度の戦争に出て死んだり負傷したものの名前が列挙してあるのさ。その名前を一々読んだ時には何だか世の中が味気(あじき)なくなって人間もつまらないと云う気が起ったよ。一番仕舞(しまい)にね。私(わた)しも取る年に候えば初春(はつはる)の御雑煮(おぞうに)を祝い候も今度限りかと\ldots{}\ldots{}何だか心細い事が書いてあるんで、なおのこと気がくさくさしてしまって早く東風(とうふう)が来れば好いと思ったが、先生どうしても来ない。そのうちとうとう晩飯になったから、母へ返事でも書こうと思ってちょいと十二三行かいた。母の手紙は六尺以上もあるのだが僕にはとてもそんな芸は出来んから、いつでも十行内外で御免蒙(こうむ)る事に極(き)めてあるのさ。すると一日動かずにおったものだから、胃の具合が妙で苦しい。東風が来たら待たせておけと云う気になって、郵便を入れながら散歩に出掛けたと思い給え。いつになく富士見町の方へは足が向かないで土手(どて)三番町(さんばんちょう)の方へ我れ知らず出てしまった。ちょうどその晩は少し曇って、から風が御濠(おほり)の向(むこ)うから吹き付ける、非常に寒い。神楽坂(かぐらざか)の方から汽車がヒューと鳴って土手下を通り過ぎる。大変淋(さみ)しい感じがする。暮、戦死、老衰、無常迅速などと云う奴が頭の中をぐるぐる馳(か)け廻(めぐ)る。よく人が首を縊(くく)ると云うがこんな時にふと誘われて死ぬ気になるのじゃないかと思い出す。ちょいと首を上げて土手の上を見ると、いつの間(ま)にか例の松の真下(ました)に来ているのさ」\\
「例の松た、何だい」と主人が断句(だんく)を投げ入れる。\\
「首懸(くびかけ)の松さ」と迷亭は領(えり)を縮める。\\
「首懸の松は鴻(こう)の台(だい)でしょう」寒月が波紋(はもん)をひろげる。\\
「鴻(こう)の台(だい)のは鐘懸(かねかけ)の松で、土手三番町のは首懸(くびかけ)の松さ。なぜこう云う名が付いたかと云うと、昔(むか)しからの言い伝えで誰でもこの松の下へ来ると首が縊(くく)りたくなる。土手の上に松は何十本となくあるが、そら首縊(くびくく)りだと来て見ると必ずこの松へぶら下がっている。年に二三返(べん)はきっとぶら下がっている。どうしても他(ほか)の松では死ぬ気にならん。見ると、うまい具合に枝が往来の方へ横に出ている。ああ好い枝振りだ。あのままにしておくのは惜しいものだ。どうかしてあすこの所へ人間を下げて見たい、誰か来ないかしらと、四辺(あたり)を見渡すと生憎(あいにく)誰も来ない。仕方がない、自分で下がろうか知らん。いやいや自分が下がっては命がない、危(あぶ)ないからよそう。しかし昔の希臘人(ギリシャじん)は宴会の席で首縊(くびくく)りの真似をして余興を添えたと云う話しがある。一人が台の上へ登って縄の結び目へ首を入れる途端に他(ほか)のものが台を蹴返す。首を入れた当人は台を引かれると同時に縄をゆるめて飛び下りるという趣向(しゅこう)である。果してそれが事実なら別段恐るるにも及ばん、僕も一つ試みようと枝へ手を懸けて見ると好い具合に撓(しわ)る。撓り按排(あんばい)が実に美的である。首がかかってふわふわするところを想像して見ると嬉しくてたまらん。是非やる事にしようと思ったが、もし東風(とうふう)が来て待っていると気の毒だと考え出した。それではまず東風(とうふう)に逢(あ)って約束通り話しをして、それから出直そうと云う気になってついにうちへ帰ったのさ」\\
「それで市(いち)が栄えたのかい」と主人が聞く。\\
「面白いですな」と寒月がにやにやしながら云う。\\
「うちへ帰って見ると東風は来ていない。しかし今日(こんにち)は無拠処(よんどころなき)差支(さしつか)えがあって出られぬ、いずれ永日(えいじつ)御面晤(ごめんご)を期すという端書(はがき)があったので、やっと安心して、これなら心置きなく首が縊(くく)れる嬉しいと思った。で早速下駄を引き懸けて、急ぎ足で元の所へ引き返して見る\ldots{}\ldots{}」と云って主人と寒月の顔を見てすましている。\\
「見るとどうしたんだい」と主人は少し焦(じ)れる。\\
「いよいよ佳境に入りますね」と寒月は羽織の紐(ひも)をひねくる。\\
「見ると、もう誰か来て先へぶら下がっている。たった一足違いでねえ君、残念な事をしたよ。考えると何でもその時は死神(しにがみ)に取り着かれたんだね。ゼームスなどに云わせると副意識下の幽冥界(ゆうめいかい)と僕が存在している現実界が一種の因果法によって互に感応(かんのう)したんだろう。実に不思議な事があるものじゃないか」迷亭はすまし返っている。\\
 主人はまたやられたと思いながら何も云わずに空也餅(くうやもち)を頬張(ほおば)って口をもごもご云わしている。\\
 寒月は火鉢の灰を丁寧に掻(か)き馴(な)らして、俯向(うつむ)いてにやにや笑っていたが、やがて口を開く。極めて静かな調子である。\\
「なるほど伺って見ると不思議な事でちょっと有りそうにも思われませんが、私などは自分でやはり似たような経験をつい近頃したものですから、少しも疑がう気になりません」\\
「おや君も首を縊(くく)りたくなったのかい」\\
「いえ私のは首じゃないんで。これもちょうど明ければ昨年の暮の事でしかも先生と同日同刻くらいに起った出来事ですからなおさら不思議に思われます」\\
「こりゃ面白い」と迷亭も空也餅を頬張る。\\
「その日は向島の知人の家(うち)で忘年会兼(けん)合奏会がありまして、私もそれへヴァイオリンを携(たずさ)えて行きました。十五六人令嬢やら令夫人が集ってなかなか盛会で、近来の快事と思うくらいに万事が整っていました。晩餐(ばんさん)もすみ合奏もすんで四方(よも)の話しが出て時刻も大分(だいぶ)遅くなったから、もう暇乞(いとまご)いをして帰ろうかと思っていますと、某博士の夫人が私のそばへ来てあなたは○○子さんの御病気を御承知ですかと小声で聞きますので、実はその両三日前(りょうさんにちまえ)に逢った時は平常の通りどこも悪いようには見受けませんでしたから、私も驚ろいて精(くわ)しく様子を聞いて見ますと、私(わたく)しの逢ったその晩から急に発熱して、いろいろな譫語(うわごと)を絶間なく口走(くちばし)るそうで、それだけなら宜(い)いですがその譫語のうちに私の名が時々出て来るというのです」\\
 主人は無論、迷亭先生も「御安(おやす)くないね」などという月並(つきなみ)は云わず、静粛に謹聴している。\\
「医者を呼んで見てもらうと、何だか病名はわからんが、何しろ熱が劇(はげ)しいので脳を犯しているから、もし睡眠剤(すいみんざい)が思うように功を奏しないと危険であると云う診断だそうで私はそれを聞くや否や一種いやな感じが起ったのです。ちょうど夢でうなされる時のような重くるしい感じで周囲の空気が急に固形体になって四方から吾が身をしめつけるごとく思われました。帰り道にもその事ばかりが頭の中にあって苦しくてたまらない。あの奇麗な、あの快活なあの健康な○○子さんが\ldots{}\ldots{}」\\
「ちょっと失敬だが待ってくれ給え。さっきから伺っていると○○子さんと云うのが二返(へん)ばかり聞えるようだが、もし差支(さしつか)えがなければ承(うけたま)わりたいね、君」と主人を顧(かえり)みると、主人も「うむ」と生返事(なまへんじ)をする。\\
「いやそれだけは当人の迷惑になるかも知れませんから廃(よ)しましょう」\\
「すべて曖々然(あいあいぜん)として昧々然(まいまいぜん)たるかたで行くつもりかね」\\
「冷笑なさってはいけません、極真面目(ごくまじめ)な話しなんですから\ldots{}\ldots{}とにかくあの婦人が急にそんな病気になった事を考えると、実に飛花落葉(ひからくよう)の感慨で胸が一杯になって、総身(そうしん)の活気が一度にストライキを起したように元気がにわかに滅入(めい)ってしまいまして、ただ蹌々(そうそう)として踉々(ろうろう)という形(かた)ちで吾妻橋(あずまばし)へきかかったのです。欄干に倚(よ)って下を見ると満潮(まんちょう)か干潮(かんちょう)か分りませんが、黒い水がかたまってただ動いているように見えます。花川戸(はなかわど)の方から人力車が一台馳(か)けて来て橋の上を通りました。その提灯(ちょうちん)の火を見送っていると、だんだん小くなって札幌(さっぽろ)ビールの処で消えました。私はまた水を見る。すると遥(はる)かの川上の方で私の名を呼ぶ声が聞えるのです。はてな今時分人に呼ばれる訳はないが誰だろうと水の面(おもて)をすかして見ましたが暗くて何(なん)にも分りません。気のせいに違いない早々(そうそう)帰ろうと思って一足二足あるき出すと、また微(かす)かな声で遠くから私の名を呼ぶのです。私はまた立ち留って耳を立てて聞きました。三度目に呼ばれた時には欄干に捕(つか)まっていながら膝頭(ひざがしら)ががくがく悸(ふる)え出したのです。その声は遠くの方か、川の底から出るようですが紛(まぎ)れもない○○子の声なんでしょう。私は覚えず「はーい」と返事をしたのです。その返事が大きかったものですから静かな水に響いて、自分で自分の声に驚かされて、はっと周囲を見渡しました。人も犬も月も何(なん)にも見えません。その時に私はこの「夜(よる)」の中に巻き込まれて、あの声の出る所へ行きたいと云う気がむらむらと起ったのです。○○子の声がまた苦しそうに、訴えるように、救を求めるように私の耳を刺し通したので、今度は「今直(すぐ)に行きます」と答えて欄干から半身を出して黒い水を眺めました。どうも私を呼ぶ声が浪(なみ)の下から無理に洩(も)れて来るように思われましてね。この水の下だなと思いながら私はとうとう欄干の上に乗りましたよ。今度呼んだら飛び込もうと決心して流を見つめているとまた憐れな声が糸のように浮いて来る。ここだと思って力を込めて一反(いったん)飛び上がっておいて、そして小石か何ぞのように未練なく落ちてしまいました」\\
「とうとう飛び込んだのかい」と主人が眼をぱちつかせて問う。\\
「そこまで行こうとは思わなかった」と迷亭が自分の鼻の頭をちょいとつまむ。\\
「飛び込んだ後(あと)は気が遠くなって、しばらくは夢中でした。やがて眼がさめて見ると寒くはあるが、どこも濡(ぬ)れた所(とこ)も何もない、水を飲んだような感じもしない。たしかに飛び込んだはずだが実に不思議だ。こりゃ変だと気が付いてそこいらを見渡すと驚きましたね。水の中へ飛び込んだつもりでいたところが、つい間違って橋の真中へ飛び下りたので、その時は実に残念でした。前と後(うし)ろの間違だけであの声の出る所へ行く事が出来なかったのです」寒月はにやにや笑いながら例のごとく羽織の紐(ひも)を荷厄介(にやっかい)にしている。\\
「ハハハハこれは面白い。僕の経験と善く似ているところが奇だ。やはりゼームス教授の材料になるね。人間の感応と云う題で写生文にしたらきっと文壇を驚かすよ。\ldots{}\ldots{}そしてその○○子さんの病気はどうなったかね」と迷亭先生が追窮する。\\
「二三日前(にさんちまえ)年始に行きましたら、門の内で下女と羽根を突いていましたから病気は全快したものと見えます」\\
 主人は最前から沈思の体(てい)であったが、この時ようやく口を開いて、「僕にもある」と負けぬ気を出す。\\
「あるって、何があるんだい」迷亭の眼中に主人などは無論ない。\\
「僕のも去年の暮の事だ」\\
「みんな去年の暮は暗合(あんごう)で妙ですな」と寒月が笑う。欠けた前歯のうちに空也餅(くうやもち)が着いている。\\
「やはり同日同刻じゃないか」と迷亭がまぜ返す。\\
「いや日は違うようだ。何でも二十日(はつか)頃だよ。細君が御歳暮の代りに摂津大掾(せっつだいじょう)を聞かしてくれろと云うから、連れて行ってやらん事もないが今日の語り物は何だと聞いたら、細君が新聞を参考して鰻谷(うなぎだに)だと云うのさ。鰻谷は嫌いだから今日はよそうとその日はやめにした。翌日になると細君がまた新聞を持って来て今日は堀川(ほりかわ)だからいいでしょうと云う。堀川は三味線もので賑やかなばかりで実(み)がないからよそうと云うと、細君は不平な顔をして引き下がった。その翌日になると細君が云うには今日は三十三間堂です、私は是非摂津(せっつ)の三十三間堂が聞きたい。あなたは三十三間堂も御嫌いか知らないが、私に聞かせるのだからいっしょに行って下すっても宜(い)いでしょうと手詰(てづめ)の談判をする。御前がそんなに行きたいなら行っても宜(よ)ろしい、しかし一世一代と云うので大変な大入だから到底(とうてい)突懸(つっか)けに行ったって這入(はい)れる気遣(きづか)いはない。元来ああ云う場所へ行くには茶屋と云うものが在(あ)ってそれと交渉して相当の席を予約するのが正当の手続きだから、それを踏まないで常規を脱した事をするのはよくない、残念だが今日はやめようと云うと、細君は凄(すご)い眼付をして、私は女ですからそんなむずかしい手続きなんか知りませんが、大原のお母あさんも、鈴木の君代さんも正当の手続きを踏まないで立派に聞いて来たんですから、いくらあなたが教師だからって、そう手数(てすう)のかかる見物をしないでもすみましょう、あなたはあんまりだと泣くような声を出す。それじゃ駄目でもまあ行く事にしよう。晩飯をくって電車で行こうと降参をすると、行くなら四時までに向うへ着くようにしなくっちゃいけません、そんなぐずぐずしてはいられませんと急に勢がいい。なぜ四時までに行かなくては駄目なんだと聞き返すと、そのくらい早く行って場所をとらなくちゃ這入れないからですと鈴木の君代さんから教えられた通りを述べる。それじゃ四時を過ぎればもう駄目なんだねと念を押して見たら、ええ駄目ですともと答える。すると君不思議な事にはその時から急に悪寒(おかん)がし出してね」\\
「奥さんがですか」と寒月が聞く。\\
「なに細君はぴんぴんしていらあね。僕がさ。何だか穴の明いた風船玉のように一度に萎縮(いしゅく)する感じが起ると思うと、もう眼がぐらぐらして動けなくなった」\\
「急病だね」と迷亭が註釈を加える。\\
「ああ困った事になった。細君が年に一度の願だから是非叶(かな)えてやりたい。平生(いつも)叱りつけたり、口を聞かなかったり、身上(しんしょう)の苦労をさせたり、小供の世話をさせたりするばかりで何一つ洒掃薪水(さいそうしんすい)の労に酬(むく)いた事はない。今日は幸い時間もある、嚢中(のうちゅう)には四五枚の堵物(とぶつ)もある。連れて行けば行かれる。細君も行きたいだろう、僕も連れて行ってやりたい。是非連れて行ってやりたいがこう悪寒がして眼がくらんでは電車へ乗るどころか、靴脱(くつぬぎ)へ降りる事も出来ない。ああ気の毒だ気の毒だと思うとなお悪寒がしてなお眼がくらんでくる。早く医者に見てもらって服薬でもしたら四時前には全快するだろうと、それから細君と相談をして甘木(あまき)医学士を迎いにやると生憎(あいにく)昨夜(ゆうべ)が当番でまだ大学から帰らない。二時頃には御帰りになりますから、帰り次第すぐ上げますと云う返事である。困ったなあ、今杏仁水(きょうにんすい)でも飲めば四時前にはきっと癒(なお)るに極(きま)っているんだが、運の悪い時には何事も思うように行かんもので、たまさか妻君の喜ぶ笑顔を見て楽もうと云う予算も、がらりと外(はず)れそうになって来る。細君は恨(うら)めしい顔付をして、到底(とうてい)いらっしゃれませんかと聞く。行くよ必ず行くよ。四時までにはきっと直って見せるから安心しているがいい。早く顔でも洗って着物でも着換えて待っているがいい、と口では云ったようなものの胸中は無限の感慨である。悪寒はますます劇(はげ)しくなる、眼はいよいよぐらぐらする。もしや四時までに全快して約束を履行(りこう)する事が出来なかったら、気の狭い女の事だから何をするかも知れない。情(なさ)けない仕儀になって来た。どうしたら善かろう。万一の事を考えると今の内に有為転変(ういてんぺん)の理、生者必滅(しょうじゃひつめつ)の道を説き聞かして、もしもの変が起った時取り乱さないくらいの覚悟をさせるのも、夫(おっと)の妻(つま)に対する義務ではあるまいかと考え出した。僕は速(すみや)かに細君を書斎へ呼んだよ。呼んで御前は女だけれども
many a slip 'twixt the cup and the lip
と云う西洋の諺(ことわざ)くらいは心得ているだろうと聞くと、そんな横文字なんか誰が知るもんですか、あなたは人が英語を知らないのを御存じの癖にわざと英語を使って人にからかうのだから、宜(よろ)しゅうございます、どうせ英語なんかは出来ないんですから、そんなに英語が御好きなら、なぜ耶蘇学校(ヤソがっこう)の卒業生かなんかをお貰いなさらなかったんです。あなたくらい冷酷な人はありはしないと非常な権幕(けんまく)なんで、僕もせっかくの計画の腰を折られてしまった。君等にも弁解するが僕の英語は決して悪意で使った訳じゃない。全く妻(さい)を愛する至情から出たので、それを妻のように解釈されては僕も立つ瀬がない。それにさっきからの悪寒(おかん)と眩暈(めまい)で少し脳が乱れていたところへもって来て、早く有為転変、生者必滅の理を呑み込ませようと少し急(せ)き込んだものだから、つい細君の英語を知らないと云う事を忘れて、何の気も付かずに使ってしまった訳さ。考えるとこれは僕が悪(わ)るい、全く手落ちであった。この失敗で悪寒はますます強くなる。眼はいよいよぐらぐらする。妻君は命ぜられた通り風呂場へ行って両肌(もろはだ)を脱いで御化粧をして、箪笥(たんす)から着物を出して着換える。もういつでも出掛けられますと云う風情(ふぜい)で待ち構えている。僕は気が気でない。早く甘木君が来てくれれば善いがと思って時計を見るともう三時だ。四時にはもう一時間しかない。「そろそろ出掛けましょうか」と妻君が書斎の開き戸を明けて顔を出す。自分の妻(さい)を褒(ほ)めるのはおかしいようであるが、僕はこの時ほど細君を美しいと思った事はなかった。もろ肌を脱いで石鹸で磨(みが)き上げた皮膚がぴかついて黒縮緬(くろちりめん)の羽織と反映している。その顔が石鹸と摂津大掾(せっつだいじょう)を聞こうと云う希望との二つで、有形無形の両方面から輝やいて見える。どうしてもその希望を満足させて出掛けてやろうと云う気になる。それじゃ奮発して行こうかな、と一ぷくふかしているとようやく甘木先生が来た。うまい注文通りに行った。が容体をはなすと、甘木先生は僕の舌を眺(なが)めて、手を握って、胸を敲(たた)いて背を撫(な)でて、目縁(まぶち)を引っ繰り返して、頭蓋骨(ずがいこつ)をさすって、しばらく考え込んでいる。「どうも少し険呑(けんのん)のような気がしまして」と僕が云うと、先生は落ちついて、「いえ格別の事もございますまい」と云う。「あのちょっとくらい外出致しても差支(さしつか)えはございますまいね」と細君が聞く。「さよう」と先生はまた考え込む。「御気分さえ御悪くなければ\ldots{}\ldots{}」「気分は悪いですよ」と僕がいう。「じゃともかくも頓服(とんぷく)と水薬(すいやく)を上げますから」「へえどうか、何だかちと、危(あぶ)ないようになりそうですな」「いや決して御心配になるほどの事じゃございません、神経を御起しになるといけませんよ」と先生が帰る。三時は三十分過ぎた。下女を薬取りにやる。細君の厳命で馳(か)け出して行って、馳(か)け出して返ってくる。四時十五分前である。四時にはまだ十五分ある。すると四時十五分前頃から、今まで何とも無かったのに、急に嘔気(はきけ)を催(もよ)おして来た。細君は水薬(すいやく)を茶碗へ注(つ)いで僕の前へ置いてくれたから、茶碗を取り上げて飲もうとすると、胃の中からげーと云う者が吶喊(とっかん)して出てくる。やむをえず茶碗を下へ置く。細君は「早く御飲(おの)みになったら宜(い)いでしょう」と逼(せま)る。早く飲んで早く出掛けなくては義理が悪い。思い切って飲んでしまおうとまた茶碗を唇へつけるとまたゲーが執念深(しゅうねんぶか)く妨害をする。飲もうとしては茶碗を置き、飲もうとしては茶碗を置いていると茶の間の柱時計がチンチンチンチンと四時を打った。さあ四時だ愚図愚図してはおられんと茶碗をまた取り上げると、不思議だねえ君、実に不思議とはこの事だろう、四時の音と共に吐(は)き気(け)がすっかり留まって水薬が何の苦なしに飲めたよ。それから四時十分頃になると、甘木先生の名医という事も始めて理解する事が出来たんだが、背中がぞくぞくするのも、眼がぐらぐらするのも夢のように消えて、当分立つ事も出来まいと思った病気がたちまち全快したのは嬉しかった」\\
「それから歌舞伎座へいっしょに行ったのかい」と迷亭が要領を得んと云う顔付をして聞く。\\
「行きたかったが四時を過ぎちゃ、這入(はい)れないと云う細君の意見なんだから仕方がない、やめにしたさ。もう十五分ばかり早く甘木先生が来てくれたら僕の義理も立つし、妻(さい)も満足したろうに、わずか十五分の差でね、実に残念な事をした。考え出すとあぶないところだったと今でも思うのさ」\\
 語り了(おわ)った主人はようやく自分の義務をすましたような風をする。これで両人に対して顔が立つと云う気かも知れん。\\
 寒月は例のごとく欠けた歯を出して笑いながら「それは残念でしたな」と云う。\\
 迷亭はとぼけた顔をして「君のような親切な夫(おっと)を持った妻君は実に仕合せだな」と独(ひと)り言(ごと)のようにいう。障子の蔭でエヘンと云う細君の咳払(せきばら)いが聞える。\\
 吾輩はおとなしく三人の話しを順番に聞いていたがおかしくも悲しくもなかった。人間というものは時間を潰(つぶ)すために強(し)いて口を運動させて、おかしくもない事を笑ったり、面白くもない事を嬉しがったりするほかに能もない者だと思った。吾輩の主人の我儘(わがまま)で偏狭(へんきょう)な事は前から承知していたが、平常(ふだん)は言葉数を使わないので何だか了解しかねる点があるように思われていた。その了解しかねる点に少しは恐しいと云う感じもあったが、今の話を聞いてから急に軽蔑(けいべつ)したくなった。かれはなぜ両人の話しを沈黙して聞いていられないのだろう。負けぬ気になって愚(ぐ)にもつかぬ駄弁を弄(ろう)すれば何の所得があるだろう。エピクテタスにそんな事をしろと書いてあるのか知らん。要するに主人も寒月も迷亭も太平(たいへい)の逸民(いつみん)で、彼等は糸瓜(へちま)のごとく風に吹かれて超然と澄(すま)し切っているようなものの、その実はやはり娑婆気(しゃばけ)もあり慾気(よくけ)もある。競争の念、勝とう勝とうの心は彼等が日常の談笑中にもちらちらとほのめいて、一歩進めば彼等が平常罵倒(ばとう)している俗骨共(ぞっこつども)と一つ穴の動物になるのは猫より見て気の毒の至りである。ただその言語動作が普通の半可通(はんかつう)のごとく、文切(もんき)り形(がた)の厭味を帯びてないのはいささかの取(と)り得(え)でもあろう。\\
 こう考えると急に三人の談話が面白くなくなったので、三毛子の様子でも見て来(き)ようかと二絃琴(にげんきん)の御師匠さんの庭口へ廻る。門松(かどまつ)注目飾(しめかざ)りはすでに取り払われて正月も早(は)や十日となったが、うららかな春日(はるび)は一流れの雲も見えぬ深き空より四海天下を一度に照らして、十坪に足らぬ庭の面(おも)も元日の曙光(しょこう)を受けた時より鮮(あざや)かな活気を呈している。椽側に座蒲団(ざぶとん)が一つあって人影も見えず、障子も立て切ってあるのは御師匠さんは湯にでも行ったのか知らん。御師匠さんは留守でも構わんが、三毛子は少しは宜(い)い方か、それが気掛りである。ひっそりして人の気合(けわい)もしないから、泥足のまま椽側(えんがわ)へ上(あが)って座蒲団の真中へ寝転(ねこ)ろんで見るといい心持ちだ。ついうとうととして、三毛子の事も忘れてうたた寝をしていると、急に障子のうちで人声がする。\\
「御苦労だった。出来たかえ」御師匠さんはやはり留守ではなかったのだ。\\
「はい遅くなりまして、仏師屋(ぶっしや)へ参りましたらちょうど出来上ったところだと申しまして」「どれお見せなさい。ああ奇麗に出来た、これで三毛も浮かばれましょう。金(きん)は剥(は)げる事はあるまいね」「ええ念を押しましたら上等を使ったからこれなら人間の位牌(いはい)よりも持つと申しておりました。\ldots{}\ldots{}それから猫誉信女(みょうよしんにょ)の誉の字は崩(くず)した方が恰好(かっこう)がいいから少し劃(かく)を易(か)えたと申しました」「どれどれ早速御仏壇へ上げて御線香でもあげましょう」\\
 三毛子は、どうかしたのかな、何だか様子が変だと蒲団の上へ立ち上る。チーン南無猫誉信女(なむみょうよしんにょ)、南無阿弥陀仏(なむあみだぶつ)南無阿弥陀仏と御師匠さんの声がする。\\
「御前も回向(えこう)をしておやりなさい」\\
 チーン南無猫誉信女南無阿弥陀仏南無阿弥陀仏と今度は下女の声がする。吾輩は急に動悸(どうき)がして来た。座蒲団の上に立ったまま、木彫(きぼり)の猫のように眼も動かさない。\\
「ほんとに残念な事を致しましたね。始めはちょいと風邪(かぜ)を引いたんでございましょうがねえ」「甘木さんが薬でも下さると、よかったかも知れないよ」「一体あの甘木さんが悪うございますよ、あんまり三毛を馬鹿にし過ぎまさあね」「そう人様(ひとさま)の事を悪く云うものではない。これも寿命(じゅみょう)だから」\\
 三毛子も甘木先生に診察して貰ったものと見える。\\
「つまるところ表通りの教師のうちの野良猫(のらねこ)が無暗(むやみ)に誘い出したからだと、わたしは思うよ」「ええあの畜生(ちきしょう)が三毛のかたきでございますよ」\\
 少し弁解したかったが、ここが我慢のしどころと唾(つば)を呑んで聞いている。話しはしばし途切(とぎ)れる。\\
「世の中は自由にならん者でのう。三毛のような器量よしは早死(はやじに)をするし。不器量な野良猫は達者でいたずらをしているし\ldots{}\ldots{}」「その通りでございますよ。三毛のような可愛らしい猫は鐘と太鼓で探してあるいたって、二人(ふたり)とはおりませんからね」\\
 二匹と云う代りに二(ふ)たりといった。下女の考えでは猫と人間とは同種族ものと思っているらしい。そう云えばこの下女の顔は吾等猫属(ねこぞく)とはなはだ類似している。\\
「出来るものなら三毛の代りに\ldots{}\ldots{}」「あの教師の所の野良(のら)が死ぬと御誂(おあつら)え通りに参ったんでございますがねえ」\\
 御誂え通りになっては、ちと困る。死ぬと云う事はどんなものか、まだ経験した事がないから好きとも嫌いとも云えないが、先日あまり寒いので火消壺(ひけしつぼ)の中へもぐり込んでいたら、下女が吾輩がいるのも知らんで上から蓋(ふた)をした事があった。その時の苦しさは考えても恐しくなるほどであった。白君の説明によるとあの苦しみが今少し続くと死ぬのであるそうだ。三毛子の身代(みがわ)りになるのなら苦情もないが、あの苦しみを受けなくては死ぬ事が出来ないのなら、誰のためでも死にたくはない。\\
「しかし猫でも坊さんの御経を読んでもらったり、戒名(かいみょう)をこしらえてもらったのだから心残りはあるまい」「そうでございますとも、全く果報者(かほうもの)でございますよ。ただ慾を云うとあの坊さんの御経があまり軽少だったようでございますね」「少し短か過ぎたようだったから、大変御早うございますねと御尋ねをしたら、月桂寺(げっけいじ)さんは、ええ利目(ききめ)のあるところをちょいとやっておきました、なに猫だからあのくらいで充分浄土へ行かれますとおっしゃったよ」「あらまあ\ldots{}\ldots{}しかしあの野良なんかは\ldots{}\ldots{}」\\
 吾輩は名前はないとしばしば断っておくのに、この下女は野良野良と吾輩を呼ぶ。失敬な奴だ。\\
「罪が深いんですから、いくらありがたい御経だって浮かばれる事はございませんよ」\\
 吾輩はその後(ご)野良が何百遍繰り返されたかを知らぬ。吾輩はこの際限なき談話を中途で聞き棄てて、布団(ふとん)をすべり落ちて椽側から飛び下りた時、八万八千八百八十本の毛髪を一度にたてて身震(みぶる)いをした。その後(ご)二絃琴(にげんきん)の御師匠さんの近所へは寄りついた事がない。今頃は御師匠さん自身が月桂寺さんから軽少な御回向(ごえこう)を受けているだろう。\\
 近頃は外出する勇気もない。何だか世間が慵(もの)うく感ぜらるる。主人に劣らぬほどの無性猫(ぶしょうねこ)となった。主人が書斎にのみ閉じ籠(こも)っているのを人が失恋だ失恋だと評するのも無理はないと思うようになった。\\
 鼠(ねずみ)はまだ取った事がないので、一時は御三(おさん)から放逐論(ほうちくろん)さえ呈出(ていしゅつ)された事もあったが、主人は吾輩の普通一般の猫でないと云う事を知っているものだから吾輩はやはりのらくらしてこの家(や)に起臥(きが)している。この点については深く主人の恩を感謝すると同時にその活眼(かつがん)に対して敬服の意を表するに躊躇(ちゅうちょ)しないつもりである。御三が吾輩を知らずして虐待をするのは別に腹も立たない。今に左甚五郎(ひだりじんごろう)が出て来て、吾輩の肖像を楼門(ろうもん)の柱に刻(きざ)み、日本のスタンランが好んで吾輩の似顔をカンヴァスの上に描(えが)くようになったら、彼等鈍瞎漢(どんかつかん)は始めて自己の不明を恥(は)ずるであろう。\\

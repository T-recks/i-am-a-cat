\documentclass{book}

\begin{document}
\chapter*{I}

I am a cat. As yet I have no name. I've no idea where I was born. All I
remember is that I was miaowing in a dampish dark place when, for the
first time, I saw a human being. This human being, I heard afterwards,
was a member of the most ferocious human species; a shosei, one of those
students who, in return for board and lodging, perform small chores
about the house. I hear that, on occasion, this species catches, boils,
and eats us. However as at that time I lacked all knowledge of such
creatures, I did not feel particularly frightened. I simply felt myself
floating in the air as I was lifted up lightly on his palm. When I
accustomed myself to that position, I looked at his face. This must have
been the very first time that ever I set eyes on a human being. The
impression of oddity, which I then received, still remains today. First
of all, the face that should be decorated with hair is as bald as a
kettle. Since that day I have met many a cat but never have I come
across such deformity. The center of the face protrudes excessively and
sometimes, from the holes in that protuberance, smoke comes out in
little puffs. I was originally somewhat troubled by such exhalations for
they made me choke, but I learnt only recently that it was the smoke of
burnt tobacco which humans like to breathe.

For a little while I sat comfortably in that creature's palm, but things
soon developed at a tremendous speed. I could not tell whether the
shosei was in movement or whether it was only I that moved; but anyway I
began to grow quite giddy, to feel sick. And just as I was thinking that
the giddiness would kill me, I heard a thud and saw a million stars.
Thus far I can remember but, however hard I try, I cannot recollect
anything thereafter.

When I came to myself, the creature had gone. I had at one time had a
basketful of brothers, but now not one could be seen. Even my precious
mother had disappeared. Moreover I now found myself in a painfully
bright place most unlike that nook where once I'd sheltered. It was in
fact so bright that I could hardly keep my eyes open. Sure that there
was something wrong, I began to crawl about. Which proved painful. I had
been snatched away from softest straw only to be pitched with violence
into a prickly clump of bamboo grass.

After a struggle, I managed to scramble clear of the clump and emerged
to find a wide pond stretching beyond it. I sat at the edge of the pond
and wondered what to do. No helpful thought occurred. After a while it
struck me that, if I cried, perhaps the shosei might come back to fetch
me. I tried some feeble mewing, but no one came. Soon a light wind blew
across the pond and it began to grow dark. I felt extremely hungry. I
wanted to cry, but I was too weak to do so. There was nothing to be
done. However, having decided that I simply must find food, I turned,
very, very slowly, left around the pond. It was extremely painful going.
Nevertheless, I persevered and crawled on somehow until at long last I
reached a place where my nose picked up some trace of human presence. I
slipped into a property through a gap in a broken bamboo fence, thinking
that something might turn up once I got inside. It was sheer chance; if
the bamboo fence had not been broken just at that point, I might have
starved to death at the roadside. I realize now how true the adage is
that what is to be will be. To this very day that gap has served as my
shortcut to the neighbor's tortoiseshell.

Well, though I had managed to creep into the property, I had no idea
what to do next. Soon it got really dark. I was hungry, it was cold and
rain began to fall. I could not afford to lose any more time. I had no
choice but to struggle toward a place which seemed, since brighter,
warmer. I did not know it then, but I was in fact already inside the
house where I now had a chance to observe further specimens of
humankind. The first one that I met was O-san, the servant-woman, one of
a species yet more savage than the shosei. No sooner had she seen me
than she grabbed me by the scruff of the neck and flung me out of the
house. Accepting that I had no hope, I lay stone-still, my eyes shut
tight and trusting to Providence. But the hunger and the cold were more
than I could bear. Seizing a moment when O-san had relaxed her watch, I
crawled up once again to flop into the kitchen. I was soon flung out
again. I crawled up yet again, only to be flung out yet again. I
remember that the process was several times repeated. Ever since that
time, I have been utterly disgusted with this O-san person. The other
day I managed at long last to rid myself of my sense of grievance, for I
squared accounts by stealing her dinner of mackerel-pike. As I was about
to be flung out for the last time, the master of the house appeared,
complaining of the noise and demanding an explanation. The servant
lifted me up, turned my face to the master and said, ``This little stray
kitten is being a nuisance. I keep putting it out and it keeps crawling
back into the kitchen.'' The master briefly studied my face, twisting
the black hairs under his nostrils. Then, ``In that case, let it stay,''
he said; and turned and went inside. The master seemed to be a person of
few words. The servant resentfully threw me down in the kitchen. And it
was thus that I came to make this house my dwelling.

My master seldom comes face-to-face with me. I hear he is a
schoolteacher. As soon as he comes home from school, he shuts himself up
in the study for the rest of the day; and he seldom emerges. The others
in the house think that he is terribly hard-working. He himself pretends
to be hard-working. But actually he works less hard than any of them
think. Sometimes I tiptoe to his study for a peep and find him taking a
snooze. Occasionally his mouth is drooling onto some book he has begun
to read. He has a weak stomach and his skin is of a pale yellowish
color, inelastic and lacking in vitality. Nevertheless he is an enormous
gormandiser. After eating a great deal, he takes some taka-diastase for
his stomach and, after that, he opens a book. When he has read a few
pages, he becomes sleepy. He drools onto the book. This is the routine
religiously observed each evening. There are times when even I, a mere
cat, can put two thoughts together. ``Teachers have it easy. If you are
born a human, it's best to become a teacher. For if it's possible to
sleep this much and still to be a teacher, why, even a cat could
teach.'' However, according to the master, there's nothing harder than a
teacher's life and every time his friends come round to see him, he
grumbles on and on.

During my early days in the house, I was terribly unpopular with
everyone except the master. Everywhere I was unwelcome, and no one would
have anything to do with me. The fact that nobody, even to this day, has
given me a name indicates quite clearly how very little they have
thought about me. Resigned, I try to spend as much of my time as
possible with the master, the man who had taken me in. In the morning,
while he reads the newspaper, I jump to curl up on his knees. Throughout
his afternoon siesta, I sit upon his back. This is not because I have
any particular fondness for the master, but because I have no other
choice; no one else to turn to. Additionally, and in the light of other
experiments, I have decided to sleep on the boiled-rice container, which
stays warm through the morning, on the quilted foot-warmer during the
evening, and out on the veranda when it is fine. But what I find
especially agreeable is to creep into the children's bed and snuggle
down between them. There are two children, one of five and one of three:
they sleep in their own room, sharing a bed. I can always find a space
between their bodies, and I manage somehow to squeeze myself quietly in.
But if, by great ill-luck, one of the children wakes, then I am in
trouble. For the children have nasty natures, especially the younger
one. They start to cry out noisily, regardless of the time, even in the
middle of the night, shouting, ``Here's the cat!'' Then invariably the
neurotic dyspeptic in the next room wakes and comes rushing in. Why,
only the other day, my master beat my backside black and blue with a
wooden ruler.

Living as I do with human beings, the more that I observe them, the more
I am forced to conclude that they are selfish. Especially those
children. I find my bedmates utterly unspeakable. When the fancy takes
them, they hang me upside-down, they stuff my face into a paper-bag,
they fling me about, they ram me into the kitchen range. Furthermore, if
I do commit so much as the smallest mischief, the entire household
unites to chase me around and persecute me. The other day when I
happened to be sharpening my claws on some straw floor-matting, the
mistress of the house became so unreasonably incensed that now it is
only with the greatest reluctance that she'll even let me enter a matted
room. Though I'm shivering on the wooden floor in the kitchen,
heartlessly she remains indifferent. Miss Blanche, the white cat who
lives opposite and whom I much admire, tells me whenever I see her that
there is no living creature quite so heartless as a human. The other
day, she gave birth to four beautiful kittens. But three days later, the
shosei of her house removed all four and tossed them away into the
backyard pond. Miss Blanche, having given through her tears a complete
account of this event, assured me that, to maintain our own parental
love and to enjoy our beautiful family life, we, the cat-race, must
engage in total war upon all humans. We have no choice but to
exterminate them. I think it is a very reasonable proposition. And the
three-colored tomcat living next door is especially indignant that human
beings do not understand the nature of proprietary rights. Among our
kind it is taken for granted that he who first finds something, be it
the head of a dried sardine or a gray mullet's navel, acquires thereby
the right to eat it. And if this rule be flouted, one may well resort to
violence. But human beings do not seem to understand the rights of
property. Every time we come on something good to eat, invariably they
descend and take it from us. Relying on their naked strength, they
coolly rob us of things which are rightly ours to eat. Miss Blanche
lives in the house of a military man, and the tomcat's master is a
lawyer. But since I live in a teacher's house, I take matters of this
sort rather more lightly than they. I feel that life is not unreasonable
so long as one can scrape along from day to day. For surely even human
beings will not flourish forever. I think it best to wait in patience
for the Day of the Cats.

Talking of selfishness reminds me that my master once made a fool of
himself by reason of this failing. I'll tell you all about it. First you
must understand that this master of mine lacks the talent to be more
than average at anything at all; but nonetheless he can't refrain from
trying his hand at everything and anything. He's always writing haiku
and submitting them to Cuckoo; he sends off new-style poetry to Morning
Star; he has a shot at English prose peppered with gross mistakes; he
develops a passion for archery; he takes lessons in chanting No
play-texts; and sometimes he devotes himself to making hideous noises
with a violin. But I am sorry to say that none of these activities has
led to anything whatsoever. Yet, though he is dyspeptic, he gets
terribly keen once he has embarked upon a project. He once got himself
nicknamed ``The Maestro of the Water-closet'' through chanting in the
lavatory, but he remains entirely unconcerned and can still be heard
there chanting ``I am Taira-no-Munemori.'' We all say, ``There goes
Munemori,'' and titter at his antics. I do not know why it happened, but
one fine day (a payday roughly four weeks after I'd taken up residence)
this master of mine came hurrying home with a large parcel under his
arm. I wondered what he'd bought. It turned out that he'd purchased
watercolor paints, brushes, and some special ``Whatman'' paper. It
looked to me as if haiku-writing, and mediaeval chanting were going to
be abandoned in favor of watercolor painting. Sure enough, from the next
day on and every day for some long time, he did nothing but paint
pictures in his study. He even went without his afternoon siestas.
However, no one could tell what he had painted by looking at the result.
Possibly he himself thought little of his work; for, one day when his
friend who specializes in matters of aesthetics came to visit him, I
heard the following conversation.

``Do you know it's quite difficult? When one sees someone else painting,
it looks easy enough; but not till one takes a brush oneself, does one
realize just how difficult it is.'' So said my noble master, and it was
true enough.

His friend, looking at my master over his gold-rimmed spectacles,
observed, ``It's only natural that one cannot paint particularly well
the moment one starts. Besides, one cannot paint a picture indoors by
force of the imagination only. The Italian Master, Andrea del Sarto,
remarked that if you want to paint a picture, always depict nature as
she is. In the sky, there are stars. On earth, there are sparkling dews.
Birds are flying. Animals are running. In a pond there are goldfish. On
an old tree one sees winter crows. Nature herself is one vast living
picture. D'you understand? If you want to paint a picturesque picture,
why not try some preliminary sketching?''

``Oh, so Andrea del Sarto said that? I didn't know that at all. Come to
think of it, it's quite true. Indeed, it's very true.'' The master was
unduly impressed. I saw a mocking smile behind the gold-rimmed glasses.

The next day when, as always, I was having a pleasant nap on the
veranda, the master emerged from his study (an act unusual in itself)
and began behind my back to busy himself with something. At this point I
happened to wake up and, wondering what he was up to, opened my eyes
just one slit the tenth of an inch. And there he was, fairly killing
himself at being Andrea del Sarto. I could not help but laugh. He's
starting to sketch me just because he's had his leg pulled by a friend.
I have already slept enough, and I'm itching to yawn. But seeing my
master sketching away so earnestly, I hadn't the heart to move: so I
bore it all with resignation. Having drawn my outline, he's started
painting the face. I confess that, considering cats as works of art, I'm
far from being a collector's piece. I certainly do not think that my
figure, my fur, or my features are superior to those of other cats. But
however ugly I may be, there's no conceivable resemblance between myself
and that queer thing which my master is creating. First of all, the
coloring is wrong. My fur, like that of a Persian, bears tortoiseshell
markings on a ground of a yellowish pale grey. It is a fact beyond all
argument. Yet the color which my master has employed is neither yellow
nor black; neither grey nor brown; nor is it any mixture of those four
distinctive colors. All one can say is that the color used is a sort of
color. Furthermore, and very oddly, my face lacks eyes. The lack might
be excused on the grounds that the sketch is a sketch of a sleeping cat;
but, all the same, since one cannot find even a hint of an eye's
location, it is not all clear whether the sketch is of a sleeping cat or
of a blind cat. Secretly I thought to myself that this would never do,
even for Andrea del Sarto. However, I could not help being struck with
admiration for my master's grim determination. Had it been solely up to
me, I would gladly have maintained my pose for him, but Nature has now
been calling for some time. The muscles in my body are getting pins and
needles. When the tingling reached a point where I couldn't stand it
another minute, I was obliged to claim my liberty. I stretched my front
paws far out in front of me, pushed my neck out low and yawned
cavernously. Having done all that, there's no further point in trying to
stay still. My master's sketch is spoilt anyway, so I might as well pad
round to the backyard and do my business. Moved by these thoughts, I
start to crawl sluggishly away. Immediately, ``You fool'' came shouted
in my master's voice, a mixture of wrath and disappointment, out of the
inner room. He has a fixed habit of saying, ``You fool'' whenever he
curses anyone. He cannot help it since he knows no other swear words.
But I thought it rather impertinent of him thus unjustifiably to call me
``a fool.'' After all, I had been very patient up to this point. Of
course, had it been his custom to show even the smallest pleasure
whenever I jump on his back, I would have tamely endured his
imprecations: but it is a bit thick to be called ``a fool'' by someone
who has never once with good grace done me a kindness just because I get
up to go and urinate. The prime fact is that all humans are puffed up by
their extreme self-satisfaction with their own brute power. Unless some
creatures more powerful than humans arrive on earth to bully them,
there's just no knowing to what dire lengths their fool presumptuousness
will eventually carry them.

One could put up with this degree of selfishness, but I once heard a
report concerning the unworthiness of humans, which is several times
more ugly and deplorable.

At the back of my house there is a small tea-plantation, perhaps some
six yards square. Though certainly not large, it is a neat and
pleasantly sunny spot. It is my custom to go there whenever my morale
needs strengthening; when, for instance, the children are making so much
noise that I cannot doze in peace, or when boredom has disrupted my
digestion. One day, a day of Indian summer, at about two o'clock in the
afternoon, I woke from a pleasant after-luncheon nap and strolled out to
this tea-plantation by way of taking exercise. Sniffing, one after
another, at the roots of the tea plants, I came to the cypress fence at
the western end; and there I saw an enormous cat fast asleep on a bed of
withered chrysanthemums, which his weight had flattened down. He did not
seem to notice my approach. Perhaps he noticed but did not care. Anyway,
there he was, stretched out at full length and snoring loudly. I was
amazed at the daring courage that permitted him, a trespasser, to sleep
so unconcernedly in someone else's garden. He was a pure black cat. The
sun of earliest afternoon was pouring its most brilliant rays upon him,
and it seemed as if invisible flames were blazing out from his glossy
fur. He had a magnificent physique; the physique, one might say, of the
Emperor of Catdom. He was easily twice my size. Filled with admiration
and curiosity, I quite forgot myself. I stood stock-still, entranced,
all eyes in front of him. The quiet zephyrs of that Indian summer set
gently nodding a branch of Sultan's Parasol, which showed above the
cypress fence, and a few leaves pattered down upon the couch of crushed
chrysanthemums. The Emperor suddenly opened his huge round eyes. I
remember that moment to this day. His eyes gleamed far more beautifully
than that dull amber stuff which humans so inordinately value. He lay
dead still. Focussing the piercing light that shone from his eyes'
interior upon my dwarfish forehead, he remarked, ``And who the hell are
you?''

I thought his turn of phrase a shade inelegant for an Emperor, but
because the voice was deep and filled with a power that could suppress a
bulldog, I remained dumb-struck with pure awe. Reflecting, however, that
I might get into trouble if I failed to exchange civilities, I answered
frigidly, with a false sang froid as cold as I could make it, ``I, sir,
am a cat. I have as yet no name.'' My heart at that moment was beating a
great deal faster than usual.

In a tone of enormous scorn, the Emperor observed, ``You\ldots{} a cat?
Well, I'm damned. Anyway, where the devil do you hang out?'' I thought
this cat excessively blunt-spoken.

``I live here, in the teacher's house.''

``Huh, I thought as much. 'Orrible scrawny aren't you.'' Like a true
Emperor, he spoke with great vehemence.

Judged by his manner of speech, he could not be a cat of respectable
background. On the other hand, he seemed well fed and positively
prosperous, almost obese, in his oily glossiness. I had to ask him ``And
you, who on earth are you?''

``Me? I'm Rickshaw Blacky.'' He gave his answer with spirit and some
pride: for Rickshaw Blacky is well-known in the neighborhood as a real
rough customer. As one would expect of those brought up in a
rickshaw-garage, he's tough but quite uneducated. Hence very few of us
mix with him, and it is our common policy to ``keep him at a respectful
distance.'' Consequently when I heard his name, I felt a trifle jittery
and uneasy but at the same time a little disdainful of him. Accordingly,
and in order to establish just how illiterate he was, I pursued the
conversation by enquiring, ``Which do you think is superior, a
rickshaw-owner or a teacher?''

``Why, a rickshaw-owner, of course. He's the stronger. Just look at your
master, almost skin and bones.''

``You, being the cat of a rickshaw-owner, naturally look very tough. I
can see that one eats well at your establishment.''

``Ah well, as far as I'm concerned, I never want for decent grub
wherever I go. You too, instead of creeping around in a tea-plantation,
why not follow along with me? Within a month, you'd get so fat nobody'd
recognize you.''

``In due course I might come and ask to join you. But it seems that the
teacher's house is larger than your boss's.''

``You dimwit! A house, however big it is, won't help fill an empty
belly.'' He looked quite huffed. Savagely twitching his ears, ears as
sharp as slant-sliced stems of the solid bamboo, he took off rowdily.

This was how I first made the acquaintance of Rickshaw Blacky, and since
that day I've run across him many times. Whenever we meet he talks big,
as might be expected from a rickshaw-owner's cat; but that deplorable
incident which I mentioned earlier was a tale he told me.

One day Blacky and I were lying as usual, sunning ourselves in the
tea-garden. We were chatting about this and that when, having made his
usual boasts as if they were all brandnew, he asked me, ``How many rats
have you caught so far?''

While I flatter myself that my general knowledge is wider and deeper
than Blacky's, I readily admit that my physical strength and courage are
nothing compared with his. All the same, his point-blank question
naturally left me feeling a bit confused. Nevertheless, a fact's a fact,
and one should face the truth. So I answered ``Actually, though I'm
always thinking of catching one, I've never yet caught any.''

Blacky laughed immoderately, quivering the long whiskers, which stuck
out stiffly round his muzzle. Blacky, like all true braggarts, is
somewhat weak in the head. As long as you purr and listen attentively,
pretending to be impressed by his rhodomontade, he is a more or less
manageable cat. Soon after getting to know him, I learnt this way to
handle him. Consequently on this particular occasion I also thought it
would be unwise to further weaken my position by trying to defend
myself, and that it would be more prudent to dodge the issue by inducing
him to brag about his own successes. So without making a fuss, I sought
to lead him on by saying, ``You, judging by your age, must have caught a
notable number of rats?'' Sure enough, he swallowed the bait with gusto.

``Well, not too many, but I must've caught thirty or forty,'' was his
triumphant answer. ``I can cope,'' he went on, ``with a hundred or two
hundred rats, any time and by myself. But a weasel, no. That I just
can't take. Once I had a hellish time with a weasel.''

``Did you really?'' I innocently offered. Blacky blinked his saucer eyes
but did not discontinue.

``It was last year, the day for the general housecleaning. As my master
was crawling in under the floorboards with a bag of lime, suddenly a
great, dirty weasel came whizzing out.''

``Really?'' I make myself look impressed.

``I say to myself, `So what's a weasel? Only a wee bit bigger than a
rat.' So I chase after it, feeling quite excited and finally I got it
cornered in a ditch.''

``That was well done,'' I applaud him.

``Not in the least. As a last resort it upped its tail and blew a filthy
fart. Ugh! The smell of it! Since that time, whenever I see a weasel, I
feel poorly.'' At this point, he raised a front paw and stroked his
muzzle two or three times as if he were still suffering from last year's
stench.

I felt rather sorry for him and, in an effort to cheer him up, said,
``But when it comes to rats, I expect you just pin them down with one
hypnotic glare. And I suppose that it's because you're such a marvelous
ratter, a cat well nourished by plenty of rats, that you are so
splendidly fat and have such a good complexion.'' Though this speech was
meant to flatter Blacky, strangely enough it had precisely the opposite
effect. He looked distinctly cast down and replied with a heavy sigh.

``It's depressing,'' he said, ``when you come to think of it. However
hard one slaves at catching rats\ldots{} In the whole wide world there's
no creature more brazen-faced than a human being. Every rat I catch they
confiscate, and they tote them off to the nearest police-box. Since the
copper can't tell who caught the rats, he just pays up a penny a tail to
anyone that brings them in. My master, for instance, has already earned
about half a crown purely through my efforts, but he's never yet stood
me a decent meal. The plain fact is that humans, one and all, are merely
thieves at heart.''

Though Blacky's far from bright, one cannot fault him in this
conclusion. He begins to look extremely angry and the fur on his back
stands up in bristles. Somewhat disturbed by Blacky's story and
reactions, I made some vague excuse and went off home. But ever since
then I've been determined never to catch a rat. However, I did not take
up Blacky's invitation to become his associate in prowling after
dainties other than rodents. I prefer the cozy life, and it's certainly
easier to sleep than to hunt for titbits. Living in a teacher's house,
it seems that even a cat acquires the character of teachers. I'd best
watch out lest, one of these days, I, too, become dyspeptic.

Talking of teachers reminds me that my master seems to have recently
realized his total incapacity as a painter of watercolors; for under the
date of December 1st his diary contains the following passage: At
today's gathering I met for the first time a man who shall be nameless.
He is said to have led a fast life. Indeed he looks very much a man of
the world. Since women like this type of person, it might be more
appropriate to say that he has been forced to lead, rather than that he
has led, a fast life. I hear his wife was originally a geisha. He is to
be envied. For the most part, those who carp at rakes are those
incapable of debauchery. Further, many of those who fancy themselves as
rakehells are equally incapable of debauchery. Such folk are under no
obligation to live fast lives, but do so of their own volition. So I in
the matter of watercolors. Neither of us will ever make the grade. And
yet this type of debauchee is calmly certain that only he is truly a man
of the world. If it is to be accepted that a man can become a man of the
world by drinking saké in restaurants, or by frequenting houses of
assignation, then it would seem to follow that I could acquire a name as
a painter of watercolors. The notion that my watercolor pictures will be
better if I don't actually paint them leads me to conclude that a
boorish country-bumpkin is in fact far superior to such foolish men of
the world.

His observations about men of the world strike me as somewhat
unconvincing. In particular his confession of envy in respect of that
wife who'd worked as a geisha is positively imbecile and unworthy of a
teacher. Nevertheless his assessment of the value of his own watercolor
painting is certainly just. Indeed my master is a very good judge of his
own character but still manages to retain his vanity. Three days later,
on December 4th, he wrote in his diary:

Last night I dreamt that someone picked up one of my watercolor
paintings which I, thinking it worthless, had tossed aside. This person
in my dream put the painting in a splendid frame and hung it up on a
transom. Staring at my work thus framed, I realized that I have suddenly
become a true artist. I feel exceedingly pleased. I spend whole days
just staring at my handiwork, happy in the conviction that the picture
is a masterpiece. Dawn broke and I woke up, and in the morning sunlight
it was obvious that the picture was still as pitiful an object as when I
painted it.

The master, even in his dreams, seems burdened with regrets about his
watercolors. And men who accept the burdens of regret, whether in
respect of watercolors or of anything else, are not the stuff that men
of the world are made of.

The day after my master dreamt about the picture, the aesthete in the
gold-rimmed spectacles paid a call upon him. He had not visited for some
long time. As soon as he was seated he inquired, ``And how is the
painting coming along?''

My master assumed a nonchalant air and answered, ``Well, I took your
advice and I am now busily engaged in sketching. And I must say that
when one sketches one seems to apprehend those shapes of things, those
delicate changes of color, which hitherto had gone unnoticed. I take it
that sketching has developed in the West to its present remarkable
condition solely as the result of the emphasis which, historically, has
always there been placed upon the essentiality thereof. Precisely as
Andrea del Sarto once observed.'' Without even so much as alluding to
the passage in his diary, he speaks approvingly of Andrea del Sarto.

The aesthete scratched his head, and remarked with a laugh, ``Well
actually that bit about del Sarto was my own invention.''

``What was?'' My master still fails to grasp that he's been tricked into
making a fool of himself.

``Why, all that stuff about Andrea del Sarto whom you so particularly
admire, I made it all up. I never thought you'd take it seriously.'' He
laughed and laughed, enraptured with the situation.

I overheard their conversation from my place on the veranda and I could
not help wondering what sort of entry would appear in the diary for
today. This aesthete is the sort of man whose sole pleasure lies in
bamboozling people by conversation consisting entirely of humbug. He
seems not to have thought of the effect his twaddle about Andrea del
Sarto must have on my master's feelings, for he rattled on proudly,
``Sometimes I cook up a little nonsense and people take it seriously,
which generates an aesthetic sensation of extreme comicality which I
find interesting. The other day, I told a certain undergraduate that
Nicholas Nickleby had advised Gibbon to cease using French for the
writing of his masterpiece, The History of the French Revolution, and
had indeed persuaded Gibbon to publish it in English. Now this
undergraduate was a man of almost eidetic memory, and it was especially
amusing to hear him repeating what I told him, word for word and in all
seriousness, to a debating session of the Japan Literary Society. And
d'you know, there were nearly a hundred in his audience, and all of them
sat listening to his drivel with the greatest enthusiasm! In fact, I've
another, even better, story. The other day, when I was in the company of
some men of letters, one of them happened to mention Theofano,
Ainsworth's historical novel of the Crusades. I took the occasion to
remark that it was a quite outstanding romantic monograph and added the
comment that the account of the heroine's death was the epitome of the
spectral. The man sitting opposite to me, one who has never uttered the
three words `I don't know,' promptly responded that those particular
paragraphs were indeed especially fine writing. From which observation I
became aware that he, no more than I, had ever read the book.''

Wide-eyed, my poor dyspeptic master asked him, ``Fair enough, but what
would you do if the other party had in fact read the book?'' It appears
that my master is not worried about the dishonesty of the deception,
merely about the possible embarrassment of being caught out in a lie.
The question leaves the aesthete utterly unfazed.

``Well, if that should happen, I'd say I'd mistaken the title or
something like that,'' and again, quite unconcerned, he gave himself to
laughter.

Though nattily tricked out in gold-rimmed spectacles, his nature is
uncommonly akin to that of Rickshaw Blacky. My master said nothing, but
blew out smoke rings as if in confession of his own lack of such
audacity. The aesthete (the glitter of whose eyes seemed to be
answering, ``and no wonder; you, being you, could not even cope with
watercolors'') went on aloud. ``But, joking apart, painting a picture's
a difficult thing. Leonardo da Vinci is supposed to have once told his
pupils to make drawings of a stain on the Cathedral wall. The words of a
great teacher. In a lavatory for instance, if absorbedly one studies the
pattern of the rain leaks on the wall, a staggering design, a natural
creation, invariably emerges. You should keep your eyes open and try
drawing from nature. I'm sure you could make something interesting.''

``Is this another of your tricks?''

``No; this one, I promise, is seriously meant. Indeed, I think that that
image of the lavatory wall is really rather witty, don't you? Quite the
sort of thing da Vinci would have said.''

``Yes, it's certainly witty,'' my master somewhat reluctantly conceded.
But I do not think he has so far made a drawing in a lavatory.

Rickshaw Blacky has recently gone lame. His glossy fur has thinned and
gradually grown dull. His eyes, which I once praised as more beautiful
than amber, are now bleared with mucus. What I notice most is his loss
of all vitality and his sheer physical deterioration. When last I saw
him in the tea garden and asked him how he was, the answer was
depressingly precise: ``I've had enough of being farted at by weasels
and crippled with side-swipes from the fishmonger's pole.''

The autumn leaves, arranged in two or three scarlet terraces among the
pine trees, have fallen like ancient dreams. The red and white
sasan-quas near the garden's ornamental basin, dropping their petals,
now a white and now a red one, are finally left bare. The wintry sun
along the ten-foot length of the southwards-facing veranda goes down
daily earlier than yesterday. Seldom a day goes by but a cold wind
blows. So my snoozes have been painfully curtailed.

The master goes to school every day and, as soon as he returns, shuts
himself up in the study. He tells all visitors that he's tired of being
a teacher. He seldom paints. He's stopped taking his taka-diastase,
saying it does no good. The children, dear little things, now trot off,
day after day, to kindergarten: but on their return, they sing songs,
bounce balls and sometimes hang me up by the tail.

Since I do not receive any particularly nourishing food, I have not
grown particularly fat; but I struggle on from day to day keeping myself
more or less fit and, so far, without getting crippled. I catch no rats.
I still detest that O-san. No one has yet named me but, since it's no
use crying for the moon, I have resolved to remain for the rest of my
life a nameless cat in the house of this teacher.
\chapter*{II}
SINCE New Year's Day I have acquired a certain modest celebrity: so
that, though only a cat, I am feeling quietly proud of myself. Which is
not unpleasing.

On the morning of New Year's Day, my master received a picture-postcard,
a card of New Year greetings from a certain painter-friend of his. The
upper part was painted red, the lower deep green; and right in the
center was a crouching animal painted in pastel. The master, sitting in
his study, looked at this picture first one way up and then the other.
``What fine coloring!'' he observed. Having thus expressed his
admiration, I thought he had finished with the matter. But no, he
continued studying it, first sideways and then longways. In order to
examine the object he twists his body, then stretches out his arms like
an ancient studying the Book of Divinations and then, turning to face
the window, he brings it in to the tip of his nose. I wish he would soon
terminate this curious performance, for the action sets his knees asway
and I find it hard to keep my balance. When at long last the wobbling
began to diminish, I heard him mutter in a tiny voice, ``I wonder what
it is.'' Though full of admiration for the colors on the
picture-postcard, he couldn't identify the animal painted in its center.
Which explained his extraordinary antics. Could it perhaps really be a
picture more difficult to interpret than my own first glance had
suggested? I half-opened my eyes and looked at the painting with an
imperturbable calmness. There could be no shadow of a doubt: it was a
portrait of myself. I do not suppose that the painter considered himself
an Andrea del Sarto, as did my master; but, being a painter, what he had
painted, both in respect of form and of color, was perfectly harmonious.
Any fool could see it was a cat. And so skillfully painted that anyone
with eyes in his head and the mangiest scrap of discernment would
immediately recognize that it was a picture of no other cat but me. To
think that anyone should need to go to such painful lengths over such a
blatantly simple matter\ldots{} I felt a little sorry for the human
race. I would have liked to have let him know that the picture is of me.
Even if it were too difficult for him to grasp that particularity, I
would still have liked to help him see that the painting is of a cat.
But since heaven has not seen fit to dower the human animal with an
ability to understand cat language, I regret to say that I let the
matter be.

Incidentally, I would like to take the occasion of this incident to
advise my readers that the human habit of referring to me in a scornful
tone of voice as some mere trifling ``cat'' is an extremely bad one.
Humans appear to think that cows and horses are constructed from
rejected human material, and that cats are constructed from cow pats and
horse dung. Such thoughts, objectively regarded, are in very poor taste
though they are no doubt not uncommon among teachers who, ignorant even
of their ignorance, remain self-satisfied with their quaint puffed-up
ideas of their own unreal importance. Even cats must not be treated
roughly or taken for granted. To the casual observer it may appear that
all cats are the same, facsimiles in form and substance, as
indistinguishable as peas in a pod; and that no cat can lay claim to
individuality. But once admitted to feline society, that casual observer
would very quickly realize that things are not so simple, and that the
human saying that ``people are freaks'' is equally true in the world of
cats. Our eyes, noses, fur, paws---all of them differ. From the tilt of
one's whiskers to the set of one's ears, down to the very hang of one's
tail, we cats are sharply differentiated. In our good looks and our poor
looks, in our likes and dislikes, in our refinement and our
coarsenesses, one may fairly say that cats occur in infinite variety.
Despite the fact of such obvious differentiation, humans, their eyes
turned up to heaven by reason of the elevation of their minds or some
such other rubbish, fail to notice even obvious differences in our
external features, that our characters might be characteristic is beyond
their comprehension. Which is to be pitied. I understand and endorse the
thought behind such sayings as, the cobbler should stick to his last,
that birds of a feather flock together, that rice-cakes are for
rice-cake makers. For cats, indeed, are for cats. And should you wish to
learn about cats, only a cat can tell you. Humans, however advanced, can
tell you nothing on this subject. And inasmuch as humans are, in fact,
far less advanced than they fancy themselves, they will find it
difficult even to start learning about cats. And for an unsympathetic
man like my master there's really no hope at all. He does not even
understand that love can never grow unless there is at least a complete
and mutual understanding. Like an ill-natured oyster, he secretes
himself in his study and has never once opened his mouth to the outside
world. And to see him there, looking as though he alone has truly
attained enlightenment, is enough to make a cat laugh. The proof that he
has not attained enlightenment is that, although he has my portrait
under his nose, he shows no sign of comprehension but coolly offers such
crazy comment as, ``perhaps, this being the second year of the war
against the Russians, it is a painting of a bear.''

As, with my eyes closed, I sat thinking these thoughts on my master's
knees, the servant-woman brought in a second picture-postcard. It is a
printed picture of a line of four or five European cats all engaged in
study, holding pens or reading books. One has broken away from the line
to perform a simple Western dance at the corner of their common desk.
Above this picture ``I am a cat'' is written thickly in Japanese black
ink. And down the right-hand side there is even a haiku stating that
``on spring days cats read books or dance.'' The card is from one of the
master's old pupils and its meaning should be obvious to anyone. However
my dimwitted master seems not to understand, for he looked puzzled and
said to himself, ``Can this be a Year of the Cat?'' He just doesn't seem
to have grasped that these postcards are manifestations of my growing
fame.

At that moment the servant brought in yet a third postcard. This time
the postcard has no picture, but alongside the characters wishing my
master a happy New Year, the correspondent has added those for, ``Please
be so kind as to give my best regards to the cat.'' Bone-headed though
he is, my master does appear to get the message when it's written out
thus unequivocally: for he glanced down at my face and, as if he really
had at last comprehended the situation, said, ``hmm.'' And his glance,
unlike his usual ones, did seem to contain a new modicum of respect.
Which was quite right and proper considering the fact that it is
entirely due to me that my master, hitherto a nobody, has suddenly begun
to get a name and to attract attention.

Just then the gate-bell sounded: tinkle-tinkle, possibly even ting-ting.
Probably a visitor. If so, the servant will answer. Since I never go out
of my way to investigate callers, except the fishmonger's errand-boy, I
remained quietly on my master's knees. The master, however, peered
worriedly toward the entrance as if duns were at the door. I deduce that
he just doesn't like receiving New Year's callers and sharing a
convivial tot. What a marvellous way to be. How much further can pure
bigotry go? If he doesn't like visitors, he should have gone out
himself, but he lacks even that much enterprise. The inaudacity of his
clam-like character grows daily more apparent. A few moments later the
servant comes in to say that Mr.~Coldmoon has called. I understand that
this Coldmoon person was also once a pupil of my master's and that,
after leaving school, he so rose in the world to be far better known
than his teacher. I don't know why, but this fellow often comes round
for a chat. On every such visit he babbles on, with a dreadful sort of
coquettishness, about being in love or not in love with somebody or
other; about how much he enjoys life or how desperately he is tired of
it. And then he leaves. It is quaint enough that to discuss such matters
he should seek the company of a withered old nut like my master, but
it's quainter still to see my mollusk opening up to comment, now and
again, on Coldmoon's mawkish maunderings.

``I'm afraid I haven't been round for quite some time. Actually, I've
been as busy as, busy since the end of last year, and, though I've
thought of going out often enough, somehow shanks' pony has just not
headed here.'' Thus, twisting and untwisting the fastening-strings of
his short surcoat, Coldmoon babbled on.

``Where then did shanks' pony go?'' my master enquired with a serious
look as he tugged at the cuffs of his worn, black, crested surcoat. It
is a cotton garment unduly short in the sleeves, and some of its
nonde-script, thin, silk lining sticks out about a half an inch at the
cuffs.

``As it were in various directions,'' Coldmoon answered, and then
laughed. I notice that one of his front teeth is missing.

``What's happened to your teeth?'' asks my master, changing the subject.

``Well, actually, at a certain place I ate mushrooms.''

``What did you say you ate?''

``A bit of mushroom. As I tried to bite off a mushroom's umbrella with
my front teeth, a tooth just broke off.''

``Breaking teeth on a mushroom sounds somewhat senile. An image possibly
appropriate to a haiku but scarcely appropriate to the pursuit of
love,'' remarked my master as he tapped lightly on my head with the palm
of his hand.

``Ah! Is that the cat? But he's quite plump! Sturdy as that, not even
Rickshaw Blacky could beat him up. He certainly is a most splendid
beast.'' Coldmoon offers me his homage.

``He's grown quite big lately,'' responds my master, and proudly smacks
me twice upon the head. I am flattered by the compliment but my head
feels slightly sore.

``The night before last, what's more, we had a little concert,'' said
Coldmoon going back to his story.

``Where?''

``Surely you don't have to know where. But it was quite interesting,
three violins to a piano accompaniment. However unskilled, when there
are three of them, violins sound fairly good. Two of them were women and
I managed to place myself between them. And I myself, I thought, played
rather well.''

``Ah, and who were the women?'' enviously my master asks. At first
glance my master usually looks cold and hard; but, to tell the truth, he
is by no means indifferent to women. He once read in a Western novel of
a man who invariably fell partially in love with practically every woman
that he met. Another character in the book somewhat sarcastically
observed that, as a rough calculation, that fellow fell in love with
just under seven-tenths of the women he passed in the street. On reading
this, my master was struck by its essential truth and remained deeply
impressed. Why should a man so impressionable lead such an oysterish
existence? A mere cat such as I cannot possibly understand it. Some say
it is the result of a love affair that went wrong; some say it is due to
his weak stomach; while others simply state that it's because he lacks
both money and audacity. Whatever the truth, it doesn't much matter
since he's a person of insufficient importance to affect the history of
his period. What is certain is that he did enquire enviously about
Coldmoon's female fiddlers. Coldmoon, looking amused, picked up a sliver
of boiled fishpaste in his chopsticks and nipped at it with his
remaining front teeth. I was worried lest another should fall out. But
this time it was all right.

``Well, both of them are daughters of good families. You don't know
them,'' Coldmoon coldly answered.

The master drawled ``Is---th-a-t---,'' but omitted the final ``so''
which he'd intended.

Coldmoon probably considered it was about time to be off, for he said,
``What marvellous weather. If you've nothing better to do, shall we go
out for a walk? As a result of the fall of Port Arthur,'' he added
encouragingly, ``the town's unusually lively.''

My master, looking as though he would sooner discuss the identity of the
female fiddlers than the fall of Port Arthur, hesitated for a moment's
thought. But he seemed finally to reach a decision, for he stood up
resolutely and said, ``All right, let's go out.'' He continues to wear
his black cotton crested surcoat and, thereunder, a quilted kimono of
hand-woven silk which, supposedly a keep-sake of his elder brother, he
has worn continuously for twenty years. Even the most strongly woven
silk, cannot survive such unremitting, such preternaturally, perennial
wear. The material has been worn so thin that, held against the light,
one can see the patches sewn on here and there from the inner side. My
master wears the same clothes throughout December and January, not
bothering to observe the traditional New Year change. He makes, indeed,
no distinction between workaday and Sunday clothes. When he leaves the
house he saunters out in whatever dress he happens to have on. I do not
know whether this is because he has no other clothes to wear or whether,
having such clothes, he finds it too much of a bore to change into them.
Whatever the case, I can't conceive that these uncouth habits are in any
way connected with disappointment in love.

After the two men left, I took the liberty of eating such of the boiled
fishpaste as Coldmoon had not already devoured. I am, these days, no
longer just a common, old cat. I consider myself at least as good as
those celebrated in the tales of Momokawa Joen or as that cat of Thomas
Gray's, which trawled for goldfish. Brawlers such as Rickshaw Blacky are
now beneath my notice. I don't suppose anyone will make a fuss if I
sneak a bit of fishpaste. Besides, this habit of taking secret snacks
between meals is by no means a purely feline custom. O-san, for
instance, is always pinching cakes and things, which she gobbles down
whenever the mistress leaves the house. Nor is O-san the only offender:
even the children, of whose refined upbringing the mistress is
continually bragging, display the selfsame tendency. Only a few days ago
that precious pair woke at some ungodly hour, and, though their parents
were still sound asleep, took it upon themselves to sit down,
face-to-face, at the dining-table. Now it is my master's habit every
morning to consume most of a loaf of bread, and to give the children
scraps thereof which they eat with a dusting of sugar. It so happened
that on this day the sugar basin was already on the table, even a spoon
stuck in it. Since there was no one there to dole them out their sugar,
the elder child scooped up a spoonful and dumped it on her plate. The
younger followed her elder's fine example and spooned an equal pile of
sugar onto another plate. For a brief while these charming creatures
just sat and glared at each other. Then the elder girl scooped a second
spoonful onto her plate, and the younger one proceeded to equalize the
position. The elder sister took a third spoonful and the younger, in a
splendid spirit of rivalry, followed suit. And so it went on until both
plates were piled high with sugar and not one single grain remained in
the basin. My master thereupon emerged from his bedroom rubbing
half-sleepy eyes and proceeded to return the sugar, so laboriously
extracted by his daughters, back into the sugar-basin. This incident
suggests that, though egotistical egalitarianism may be more highly
developed among humans than among cats, cats are the wiser creatures. My
advice to the children would have been to lick the sugar up quickly
before it became massed into such senseless pyramids, but, because they
cannot understand what I say, I merely watched them in silence from my
warm, morning place on top of the container for boiled rice.

My master came home late last night from his expedition with Coldmoon.
God knows where he went, but it was already past nine before he sat down
at the breakfast table. From my same old place I watched his morose
consumption of a typical New Year's breakfast of rice-cakes boiled with
vegetables, all served up in soup. He takes endless helpings. Though the
rice-cakes are admittedly small, he must have eaten some six or seven
before leaving the last one floating in the bowl. ``I'll stop now,'' he
remarked and laid his chopsticks down. Should anyone else behave in such
a spoilt manner, he could be relied upon to put his foot down: but, vain
in the exercise of his petty authority as master of the house, he seems
quite unconcerned by the sight of the corpse of a scorched rice-cake
drowning in turbid soup. When his wife took taka-diastase from the back
of a small cupboard and put it on the table, my master said, ``I won't
take it, it does me no good.''

``But they say it's very good after eating starchy things. I think you
should take some.'' His wife wants him to take it.

``Starchy or not, the stuff's no good.'' He remains stubborn.

``Really, you are a most capricious man,'' the mistress mutters as
though to herself.

``I'm not capricious, the medicine doesn't work.''

``But until the other day you used to say it worked very well and you
used to take it every day, didn't you?''

``Yes, it did work until that other day, but it hasn't worked since
then,'' an antithetical answer.

``If you continue in these inconsistencies, taking it one day and
stopping it the next, however efficacious the medicine may be, it will
never do you any good. Unless you try to be a little more patient,
dyspepsia, unlike other illnesses, won't get cured, will it?'' and she
turns to O-san who was serving at the table.

``Quite so, madam. Unless one takes it regularly, one cannot find out
whether a medicine is a good one or a bad one.'' O-san readily sides
with the mistress.

``I don't care. I don't take it because I don't take it. How can a mere
woman understand such things? Keep quiet.''

``All right. I'm merely a woman,'' she says pushing the taka-diastase
toward him, quite determined to make him see he is beaten. My master
stands up without saying a word and goes off into his study. His wife
and servant exchange looks and giggle. If on such occasions I follow him
and jump up onto his knees, experience tells me that I shall pay dearly
for my folly. Accordingly, I go quietly round through the garden and hop
up onto the veranda outside his study. I peeped through the slit between
the paper sliding doors and found my master examining a book by somebody
called Epictetus. If he could actually understand what he's reading,
then he would indeed be worthy of praise. But within five or six minutes
he slams the book down on the table, which is just what I'd suspected.
As I sat there watching him, he took out his diary and made the
following entry.

Took a stroll with Coldmoon round Nezu, Ueno, Ikenohata and Kanda. At
Ikenohata, geishas in formal spring kimono were playing battledore and
shuttlecock in front of a house of assignation. Their clothes beautiful,
but their faces extremely plain. It occurs to me that they resemble the
cat at home.

I don't see why he should single me out as an example of plain features.
If I went to a barber and had my face shaved, I wouldn't look much
different from a human. But, there you are, humans are conceited and
that's the trouble with them.

As we turned at Hotan's corner another geisha appeared. She was slim,
well-shaped and her shoulders were most beautifully sloped. The way she
wore her mauve kimono gave her a genuine elegance. ``Sorry about last
night, Gen-chan---I was so busy\ldots{}'' She laughed and one glimpsed
white teeth. Her voice was so harsh, as harsh as that of a roving crow,
that her otherwise fine appearance diminished in enchantment. So much so
that I didn't even bother to turn around to see what sort of person this
Gen-chan was, but sauntered on toward Onarimachi with my hands tucked
inside the breast-fold of my kimono. Coldmoon, however, seemed to have
become a trifle fidgety.

There is nothing more difficult than understanding human mentality. My
master's present mental state is very far from clear; is he feeling
angry or lighthearted, or simply seeking solace in the scribblings of
some dead philosopher? One just can't tell whether he's mocking the
world or yearning to be accepted into its frivolous company; whether he
is getting furious over some piddling little matter or holding himself
aloof from worldly things. Compared with such complexities, cats are
truly simple. If we want to eat, we eat; if we want to sleep, we sleep;
when we are angry, we are angry utterly; when we cry, we cry with all
the desperation of extreme commitment to our grief. Thus we never keep
things like diaries. For what would be the point? No doubt human beings
like my two-faced master find it necessary to keep diaries in order to
display in a darkened room that true character so assiduously hidden
from the world. But among cats both our four main occupations (walking,
standing, sitting, and lying down) and such incidental activities as
excreting waste are pursued quite openly. We live our diaries, and
consequently have no need to keep a daily record as a means of
maintaining our real characters. Had I the time to keep a diary, I'd use
that time to better effect; sleeping on the veranda.

We dined somewhere in Kanda. Because I allowed myself one or two cups of
saké (which I had not tasted for quite a time), my stomach this morning
feels extremely well. I conclude that the best remedy for a stomach
ailment is saké at suppertime. Taka-diastase just won't do. Whatever
claims are made for it, it's just no good. That which lacks effect will
continue to lack effect.

Thus with his brush he smears the good name of taka-diastase. It is as
though he quarreled with himself, and in this entry one can see a last
flash of this morning's ugly mood. Such entries are perhaps most
characteristic of human mores.

The other day, Mr.~X claimed that going without one's breakfast helped
the stomach. So I took no breakfast for two or three days but the only
effect was to make my stomach grumble. Mr.~Y strongly advised me to
refrain from eating pickles. According to him, all disorders of the
stomach originate in pickles. His thesis was that abstinence from
pickles so dessicates the sources of all stomach trouble, that a
complete cure must follow. For at least a week no pickle crossed my
lips, but, since that banishment produced no noticeable effect, I have
resumed consuming them. According to Mr.~Z, the one true remedy is
ventral massage. But no ordinary massage of the stomach would suffice.
It must be massage in accordance with the old-world methods of the
Minagawa School. Massaged thus once, or at most twice, the stomach would
be rid of every ill. The wisest scholars, such as Yasui Sokuken, and the
most resourceful heroes, such as Sakamoto Ryoma, all relied upon this
treatment. So off I went to Kaminegishi for an immediate massage. But
the methods used were of inordinate cruelty. They told me, for instance,
that no good could be hoped for unless one's bones were massaged; that
it would be difficult properly to eradicate my troubles unless, at least
once, my viscera were totally inverted. At all events, a single session
reduced my body to the condition of cotton-wool and I felt as though I
had become a lifelong sufferer from sleeping sickness. I never went
there again. Once was more than enough. Then Mr.~A assured me that one
shouldn't eat solids. So I spent a whole day drinking nothing but milk.
My bowels gave forth heavy plopping noises as though they had been
swamped, and I could not sleep all night. Mr.~B states that exercising
one's intestines by diaphragmic breathing produces a naturally healthy
stomach and he counsels me to follow his advice. And I did try. For a
time. But it proved no good for it made my bowels queasy. Besides,
though every now and again I strive with all my heart and soul to
control my breathing with the diaphragm, in five or six minutes I forget
to discipline my muscles. And if I concentrate on maintaining that
discipline I get so midriff-minded that I can neither read nor write.
Waverhouse, my aesthete friend, once found me thus breathing in pursuit
of a naturally healthy stomach and, rather unkindly, urged me, as a man,
to terminate my labor-pangs. So diaphragmic breathing is now also a
thing of the past. Dr.~C recommends a diet of buckwheat noodles. So
buckwheat noodles it was, alternately in soup and served cold after
boiling. It did nothing, except loosen my bowels. I have tried every
possible means to cure my ancient ailment, but all of them are useless.
But those three cups of saké which I drank last night with Coldmoon have
certainly done some good. From now on, I will drink two or three cups
each evening.

I doubt whether this saké treatment will be kept up very long. My
master's mind exhibits the same incessant changeability as can be seen
in the eyes of cats. He has no sense of perseverance. It is, moreover,
idiotic that, while he fills his diary with lamentation over his stomach
troubles, he does his best to present a brave face to the world; to grin
and bear it.

The other day his scholar friend, Professor Whatnot, paid a visit and
advanced the theory that it was at least arguable that every illness is
the direct result of both ancestral and personal malefaction. He seemed
to have studied the matter pretty deeply for the sequence of his logic
was clear, consistent, and orderly. Altogether it was a fine theory. I
am sorry to say that my master has neither the brain nor the erudition
to rebut such theories. However, perhaps because he himself was actually
suffering from stomach trouble, he felt obliged to make all sorts of
face-saving excuses. He irrelevantly retorted, ``Your theory is
interesting, but are you aware that Carlyle was dyspeptic?'' as if
claiming that because Carlyle was dyspeptic his own dyspepsia was an
intellectual honor. His friend replied,

``It does not follow that because Carlyle was a dyspeptic, all
dyspeptics are Carlyles.'' My master, reprimanded, held his tongue, but
the incident revealed his curious vanity. It's all the more amusing when
one recalls that he would probably prefer not to be dyspeptic, for just
this morning he recorded in his diary an intention to take treatment by
saké as from tonight. Now that I've come to think of it, his inordinate
consumption of rice-cakes this morning must have been the effect of last
night's saké session with Coldmoon. I could have eaten those cakes
myself.

Though I am a cat, I eat practically anything. Unlike Rickshaw Blacky, I
lack the energy to go off raiding fishshops up distant alleys. Further,
my social status is such that I cannot expect the luxury enjoyed by
Tortoiseshell whose mistress teaches the idle rich to play on the
two-stringed harp. Therefore I don't, as others can, indulge myself in
likes and dislikes. I eat small bits of bread left over by the children,
and I lick the jam from bean-jam cakes. Pickles taste awful, but to
broaden my experience I once tried a couple of slices of pickled radish.
It's a strange thing but once I've tried it, almost anything turns out
edible. To say, ``I don't like that'' or ``I don't like this'' is mere
extravagant willfulness, and a cat that lives in a teacher's house
should eschew such foolish remarks.

According to my master, there was once a novelist whose name was Balzac
and he lived in France. He was an extremely extravagant man. I do not
mean an extravagant eater but that, being a novelist, he was extravagant
in his writing. One day he was trying to find a suitable name for a
character in the novel he was writing, but, for whatever reason, could
not think of a name that pleased him. Just then one of his friends
called by, and Balzac suggested they should go out for a walk. This
friend had, of course, no idea why, still less that Balzac was
determined to find the name he needed. Out on the streets, Balzac did
nothing but stare at shop signboards, but still he couldn't find a
suitable name. He marched on endlessly, while his puzzled friend, still
ignorant of the object of the expedition, tagged along behind him.
Having fruitlessly explored Paris from morning till evening, they were
on their way home when Balzac happened to notice a tailor's signboard
bearing the name ``Marcus.'' He clapped his hands. ``This is it,'' he
shouted. ``It just has to be this. Marcus is a good name, but with a Z
in front of Marcus it becomes a perfect name. It has to be a Z. Z.
Marcus is remarkably good. Names that I invent are never good. They
sound unnatural however cleverly constructed. But now, at long, long
last, I've got the name I like.'' Balzac, extremely pleased with
himself, was totally oblivious to the inconvenience he had caused his
friend. It would seem unduly troublesome that one should have to spend a
whole day trudging around Paris merely to find a name for a character in
a novel. Extravagance of such enormity acquires a certain splendor, but
folk like me, a cat kept by a clam-like introvert, cannot even envisage
such inordinate behavior. That I should not much care what, so long as
it's edible, I eat is probably an inevitable result of my circumstances.
Thus it was in no way as an expression of extravagance that I expressed
just now my feeling of wishing to eat a rice-cake. I simply thought that
I'd better eat while the chance offered, and I then remembered that the
piece of rice-cake which my master had left in his breakfast bowl was
possibly still in the kitchen. So round to the kitchen I went.

The rice-cake was stuck, just as I saw it this morning, at the bottom of
the bowl and its color was still as I remembered it. I must confess that
I've never previously tasted rice-cake. Yet, though I felt a shade
uncertain, it looks quite good to eat. With a tentative front paw I rake
at the green vegetables adhering to the rice-cake. My claws, having
touched the outer part of the rice-cake, become sticky. I sniff at them
and recognize the smell that can be smelt when rice stuck at the bottom
of a cooking-pot is transferred into the boiled-rice container. I look
around, wondering, ``Shall I eat it, shall I not?'' Fortunately, or
unfortunately, there's nobody about. O-san, with a face that shows no
change between year end and the spring, is playing battledore and
shuttlecock. The children in the inner room are singing something about
a rabbit and a tortoise. If I am to eat this New Year speciality, now's
the moment. If I miss this chance I shall have to spend a whole, long
year not knowing how a rice-cake tastes. At this point, though a mere
cat, I perceived a truth: that golden opportunity makes all animals
venture to do even those things they do not want to do. To tell the
truth, I do not particularly want to eat the rice-cake. In fact the more
I examined the thing at the bottom of the bowl the more nervous I became
and the more keenly disinclined to eat it. If only O-san would open the
kitchen door, or if I could hear the children's footsteps coming toward
me, I would unhesitatingly abandon the bowl; not only that, I would have
put away all thought of rice-cakes for another year. But no one comes.
I've hesitated long enough. Still no one comes. I feel as if someone
were hotly urging me on, someone whispering, ``Eat it, quickly!'' I
looked into the bowl and prayed that someone would appear. But no one
did. I shall have to eat the rice-cake after all. In the end, lowering
the entire weight of my body into the bottom of the bowl, I bit about an
inch deep into a corner of the rice-cake.

Most things that I bite that hard come clean off in my mouth. But what a
surprise! For I found when I tried to reopen my jaw that it would not
budge. I try once again to bite my way free, but find I'm stuck. Too
late I realize that the rice-cake is a fiend. When a man who has fallen
into a marsh struggles to escape, the more he thrashes about trying to
extract his legs, the deeper in he sinks. Just so, the harder I clamp my
jaws, the more my mouth grows heavy and my teeth immobilized. I can feel
the resistance to my teeth, but that's all. I cannot dispose of it.
Waverhouse, the aesthete, once described my master as an aliquant man
and I must say it's rather a good description. This rice-cake too, like
my master, is aliquant. It looked to me that, however much I continued
biting, nothing could ever result: the process could go on and on
eternally like the division of ten by three. In the middle of this
anguish I found my second truth: that all animals can tell by instinct
what is or is not good for them. Although I have now discovered two
great truths, I remain unhappy by reason of the adherent rice-cake. My
teeth are being sucked into its body, and are becoming excruciatingly
painful. Unless I can complete my bite and run away quickly, O-san will
be on me. The children seem to have stopped singing, and I'm sure
they'll soon come running into the kitchen. In an extremity of anguish,
I lashed about with my tail, but to no effect. I made my ears stand up
and then lie flat, but this didn't help either. Come to think of it, my
ears and tail have nothing to do with the rice-cake. In short, I had
indulged in a waste of wagging, a waste of ear-erection, and a waste of
ear-flattening. So I stopped.

At long last it dawned on me that the best thing to do is to force the
rice-cake down by using my two front paws. First I raised my right paw
and stroked it around my mouth. Naturally, this mere stroking brought no
relief whatsoever. Next, I stretched out my left paw and with it scraped
quick circles around my mouth. These ineffectual passes failed to
exorcize the fiend in the rice-cake. Realizing that it was essential to
proceed with patience, I scraped alternatively with my right and left
paws, but my teeth stayed stuck in the rice-cake. Growing impatient, I
now used both front paws simultaneously. Then, only then, I found to my
amazement that I could actually stand up on my hind legs. Somehow I feel
un-catlike. But not caring whether I am a cat or not, I scratch away
like mad at my whole face in frenzied determination to keep on
scratching until the fiend in the rice-cake has been driven out. Since
the movements of my front paws are so vigorous I am in danger of losing
my balance and falling down. To keep my equilibrium I find myself
marking time with my hind legs. I begin to tittup from one spot to
another, and I finish up prancing madly all over the kitchen. It gives
me great pride to realize that I can so dextrously maintain an upright
position, and the revelation of a third great truth is thus vouchsafed
me: that in conditions of exceptional danger one can surpass one's
normal level of achievement. This is the real meaning of Special
Providence.

Sustained by Special Providence, I am fighting for dear life against
that demonic rice-cake when I hear footsteps. Someone seems to be
approaching. Thinking it would be fatal to be caught in this
predicament, I redouble my efforts and am positively running around the
kitchen. The footsteps come closer and closer. Alas, that Special
Providence seems not to last forever. In the end I am discovered by the
children who loudly shout, ``Why look! The cat's been eating rice-cakes
and is dancing.'' The first to hear their announcement was that O-san
person. Abandoning her shuttlecock and battledore, she flew in through
the kitchen door crying, ``Gracious me!'' Then the mistress, sedate in
her formal silk kimono, deigns to remark, ``What a naughty cat.'' And my
master, drawn from his study by the general hubbub, shouts, ``You
fool!'' The children find me funniest, but by general agreement the
whole household is having a good old laugh. It is annoying, it is
painful, it is impossible to stop dancing. Hell and damnation! When at
long last the laughter began to die down, the dear, little five-year-old
piped up with an, ``Oh what a comical cat,'' which had the effect of
renewing the tide of their ebbing laughter. They fairly split their
sides. I have already heard and seen quite a lot of heartless human
behavior, but never before have I felt so bitterly critical of their
conduct. Special Providence having vanished into thin air, I was back in
my customary position on all fours, finally at my wit's end, and, by
reason of giddiness, cutting a quite ridiculous figure. My master seems
to have felt it would be perhaps a pity to let me die before his very
eyes, for he said to O-san, ``Help him get rid of that rice-cake.''
O-san looks at the mistress as if to say, ``Why not make him go on
dancing?'' The mistress would gladly see my minuet continued, but, since
she would not go so far as wanting me to dance myself to death, says
nothing. My master turned somewhat sharply to the servant and ordered,
``Hurry it up, if you don't help quickly the cat will be dead.'' O-san,
with a vacant look on her face, as though she had been roughly wakened
from some peculiarly delicious dream, took a firm grip on the rice-cake
and yanked it out of my mouth. I am not quite as feeble-fanged as
Coldmoon, but I really did think my entire front toothwork was about to
break off. The pain was indescribable. The teeth embedded in the
rice-cake are being pitilessly wrenched. You can't imagine what it was
like. It was then that the fourth enlightenment burst upon me: that all
comfort is achieved through hardship. When at last I came to myself and
looked around at a world restored to normality, all the members of the
household had disappeared into the inner room.

Having made such a fool of myself, I feel quite unable to face such
hostile critics as O-san. It would, I think, unhinge my mind. To restore
my mental tranquillity, I decided to visit Tortoiseshell, so I left the
kitchen and set off through the backyard to the house of the
two-stringed harp. Tortoiseshell is a celebrated beauty in our district.
Though I am undoubtedly a cat, I possess a wide general knowledge of the
nature of compassion and am deeply sensitive to affection,
kind-heartedness, tenderness, and love. Merely to observe the bitterness
in my master's face, just to be snubbed by O-san, leaves me out of
sorts. At such times I visit this fair, lady friend of mine and our
conversation ranges over many things. Then, before I am aware of it, I
find myself refreshed. I forget my worries, hardships, everything. I
feel as if reborn. Female influence is indeed a most potent thing.
Through a gap in the cedar-hedge, I peer to see if she is anywhere
about. Tortoiseshell, wearing a smart new collar in celebration of the
season, is sitting very neatly on her veranda. The rondure of her back
is indescribably beautiful. It is the most beautiful of all curved
lines. The way her tail curves, the way she folds her legs, the
charmingly lazy shake of her ears---all these are quite beyond
description. She looks so warm sitting there so gracefully in the very
sunniest spot. Her body holds an attitude of utter stillness and
correctness. And her fur, glossy as velvet that reflects the rays of
spring, seems suddenly to quiver although the air is still. For a while
I stood, completely enraptured, gazing at her. Then as I came to myself,
I softly called, ``Miss Tortoiseshell, Miss Tortoiseshell,'' and
beckoned with my paw.

``Why, Professor,'' she greeted me as she stepped down from the veranda.
A tiny bell attached to her scarlet collar made little tinkling sounds.
I say to myself, ``Ah, it's for the New Year that she's wearing a
bell,'' and, while I am still admiring its lively tinkle, find she has
arrived beside me. ``A happy NewYear, Professor,'' and she waves her
tail to the left; for when cats exchange greetings one first holds one's
tail upright like a pole, then twists it round to the left. In our
neighborhood it is only Tortoiseshell who calls me Professor. Now, I
have already mentioned that I have, as yet, no name; it is
Tortoiseshell, and she alone, who pays me the respect due to one that
lives in a teacher's house. Indeed, I am not altogether displeased to be
addressed as a Professor, and respond willingly to her apostrophe.

``And a happy New Year to you,'' I say. ``How beautifully you're done
up!''

``Yes, the mistress bought it for me at the end of last year. Isn't it
nice?'' and she makes it tinkle for me.

``Yes indeed, it has a lovely sound. I've never seen such a wonderful
thing in my life.''

``No! Everyone's using them,'' and she tinkle-tinkles. ``But isn't it a
lovely sound? I'm so happy.'' She tinkle-tinkle-tinkles continuously.

``I can see your mistress loves you very dearly.'' Comparing my lot with
hers, I hinted at my envy of a pampered life.

Tortoiseshell is a simple creature. ``Yes,'' she says, ``that's true;
she treats me as if I were her own child.'' And she laughs innocently.
It is not true that cats never laugh. Human beings are mistaken in their
belief that only they are capable of laughter. When I laugh my nostrils
grow triangular and my Adam's apple trembles. No wonder human beings
fail to understand it.

``What is your master really like?''

``My master? That sounds strange. Mine is a mistress. A mistress of the
two stringed harp.''

``I know that. But what is her background? I imagine she's a person of
high birth?''

``Ah, yes.''

A small Princess-pine

While waiting for you\ldots{}

Beyond the sliding paper-door the mistress begins to play on her
two-stringed harp.

``Isn't that a splendid voice?'' Tortoiseshell is proud of it.

``It seems extremely good, but I don't understand what she's singing.
What's the name of the piece?''

``That? Oh, it's called something or other. The mistress is especially
fond of it. D'you know, she's actually sixty-two? But in excellent
condition, don't you think?''

I suppose one has to admit that she's in excellent condition if she's
still alive at sixty-two. So I answered, ``Yes.'' I thought to myself
that I'd given a silly answer, but I could do no other since I couldn't
think of anything brighter to say.

``You may not think so, but she used to be a person of high standing.
She always tells me so.''

``What was she originally?''

``I understand that she's the thirteenth Shogun's widowed wife's
private-secretary's younger sister's husband's mother's nephew's
daughter.''

``What?''

``The thirteenth Shogun's widowed wife's private-secretary's younger
sister's\ldots{}''

``Ah! But, please, not quite so fast. The thirteenth Shogun's widowed
wife's younger sister's private-secretary's\ldots{}''

``No, no, no. The thirteenth Shogun's widowed wife's private-secretary's
younger sister's\ldots{}''

``The thirteenth Shogun's widowed wife's\ldots{}''

``Right.''

``Private-secretary's. Right?''

``Right.''

``Husband's\ldots{}''

``No, younger sister's husband's.''

``Of course. How could I? Younger sister's husband's\ldots{}''

``Mother's nephew's daughter. There you are.''

``Mother's nephew's daughter?''

``Yes, you've got it.''

``Not really. It's so terribly involved that I still can't get the hang
of it.

What exactly is her relation to the thirteenth Shogun's widowed wife?''

``Oh, but you are so stupid! I've just been telling you what she is.

She's the thirteenth Shogun's widowed wife's private-secretary's younger
sister's husband's mother's\ldots{}''

``That much I've followed, but\ldots{}''

``Then, you've got it, haven't you?''

``Yes.'' I had to give in. There are times for little white lies.

Beyond the sliding paper-door the sound of the two-stringed harp came to
a sudden stop and the mistress' voice called, ``Tortoiseshell,
Tortoiseshell, your lunch is ready.'' Tortoiseshell looked happy and
remarked, ``There, she's calling, so I must go home. I hope you'll
forgive me?'' What would be the good of my saying that I mind? ``Come
and see me again,'' she said; and she ran off through the garden
tinkling her bell. But suddenly she turned and came back to ask me
anxiously, ``You're looking far from well. Is anything wrong?'' I
couldn't very well tell her that I'd eaten a rice-cake and gone dancing;
so, ``No,'' I said, ``nothing in particular. I did some weighty
thinking, which brought on something of a headache. Indeed I called
today because I fancied that just to talk with you would help me to feel
better.''

``Really? Well, take good care of yourself. Good-bye now.'' She seemed a
tiny bit sorry to leave me, which has completely restored me to the
liveliness I'd felt before the rice-cake bit me. I now felt wonderful
and decided to go home through that tea-plantation where one could have
the pleasure of treading down lumps of half-melted frost. I put my face
through the broken bamboo hedge, and there was Rickshaw Blacky, back
again on the dry chrysanthemums, yawning his spine into a high, black
arch. Nowadays I'm no longer scared of Blacky, but, since any
conversation with him involves the risk of trouble, I endeavor to pass,
cutting him off. But it's not in Blacky's nature to contain his feelings
if he believes himself looked down upon. ``Hey you, Mr.~No-name. You're
very stuck-up these days, now aren't you? You may be living in a
teacher's house, but don't go giving yourself such airs. And stop, I
warn you, trying to make a fool of me.'' Blacky doesn't seem to know
that I am now a celebrity. I wish I could explain the situation to him,
but, since he's not the kind who can understand such things, I decide
simply to offer him the briefest of greetings and then to take my leave
as soon as I decently can.

``A happy New Year, Mr.~Blacky. You do look well, as usual.'' And I lift
up my tail and twist it to the left. Blacky, keeping his tail straight
up, refused to return my salutation.

``What! Happy? If the New Year's happy, then you should be out of your
tiny mind the whole year round. Now push off sharp, you back-end of a
bellows.''

That turn of phrase about the back-end of a bellows sounds distinctly
derogatory, but its semantic content happened to escape me. ``What,'' I
enquired, ``do you mean by the back-end of a bellows?''

``You're being sworn at and you stand there asking its meaning. I give
up! I really do! You really are a New Year's nit.''

A New Year's nit sounds somewhat poetic, but its meaning is even more
obscure than that bit about the bellows. I would have liked to ask the
meaning for my future reference, but, as it was obvious I'd get no clear
answer, I just stood facing him without a word. I was actually feeling
rather awkward, but just then the wife of Blacky's master suddenly
screamed out, ``Where in hell is that cut of salmon I left here on the
shelf? My God, I do declare that hellcat's been here and snitched it
once again! That's the nastiest cat I've ever seen. See what he'll get
when he comes back!'' Her raucous voice unceremoniously shakes the mild
air of the season, vulgarizing its natural peacefulness. Blacky puts on
an impudent look as if to say, ``If you want to scream your head off,
scream away,'' and he jerked his square chin forward at me as if to say,
``Did you hear that hullaballoo?'' Up to this point I've been too busy
talking to Blacky to notice or think about anything else; but now,
glancing down, I see between his legs a mud-covered bone from the
cheapest cut of salmon.

``So you've been at it again!'' Forgetting our recent exchanges, I
offered Blacky my usual flattering exclamation. But it was not enough to
restore him to good humor.

``Been at it! What the hell d'you mean, you saucy blockhead? And what do
you mean by saying `again' when this is nothing but a skinny slice of
the cheapest fish? Don't you know who I am! I'm Rickshaw Blacky, damn
you.'' And, having no shirtsleeves to roll up, he lifts an aggressive
right front-paw as high as his shoulder.

``I've always known you were Mr.~Rickshaw Blacky.''

``If you knew, why the hell did you say I'd been at it again? Answer
me!'' And he blows out over me great gusts of oven breath. Were we
humans, I would be shaken by the collar of my coat. I am somewhat taken
aback and am indeed wondering how to get out of the situation, when that
woman's fearful voice is heard again.

``Hey! Mr.~Westbrook. You there, Westbrook, can you hear me? Listen, I
got something to say. Bring me a pound of beef, and quick. O.K.?
Understand? Beef that isn't tough. A pound of it. See?'' Her
beef-demanding tones shatter the peace of the whole neighborhood.

``It's only once a year she orders beef and that's why she shouts so
loud. She wants the entire neighborhood to know about her marvellous
pound of beef. What can one do with a woman like that!'' asked Blacky
jeeringly as he stretched all four of his legs. As I can find nothing to
say in reply, I keep silent and watch.

``A miserable pound just simply will not do. But I reckon it can't be
helped. Hang on to that beef. I'll have it later.'' Blacky communes with
himself as though the beef had been ordered specially for him.

``This time you're in for a real treat. That's wonderful!'' With these
words I'd hoped to pack him off to his home.

But Blacky snarled, ``That's nothing to do with you. Just shut your big
mouth, you!'' and using his strong hind-legs, he suddenly scrabbles up a
torrent of fallen icicles which thuds down on my head. I was taken
completely aback, and, while I was still busy shaking the muddy debris
off my body, Blacky slid off through the hedge and disappeared.
Presumably to possess himself of Westbrook's beef.

When I get home I find the place unusually springlike and even the
master is laughing gaily. Wondering why, I hopped onto the veranda, and,
as I padded to sit beside the master, noticed an unfamiliar guest. His
hair is parted neatly and he wears a crested cotton surcoat and a
duck-cloth hakama. He looks like a student and, at that, an extremely
serious one. Lying on the corner of my master's small hand-warming
brazier, right beside the lacquer cigarette-box, there's a visiting card
on which is written, ``To introduce Mr.~Beauchamp Blowlamp: from
Coldmoon.''

Which tells me both the name of this guest and the fact that he's a
friend of Coldmoon. The conversation going on between host and guest
sounds enigmatic because I missed the start of it. But I gather that it
has something to do with Waverhouse, the aesthete whom I have had
previous occasion to mention.

``And he urged me to come along with him because it would involve an
ingenious idea, he said.'' The guest is talking calmly.

``Do you mean there was some ingenious idea involved in lunching at
aWestern style restaurant?'' My master pours more tea for the guest and
pushes the cup toward him.

``Well, at the time I did not understand what this ingenious idea could
be, but, since it was his idea, I thought it bound to be something
interesting and\ldots{}''

``So you accompanied him. I see.''

``Yes, but I got a surprise.''

The master, looking as if to say, ``I told you so,'' gives me a whack on
the head. Which hurts a little. ``I expect it proved somewhat farcical.
He's rather that way inclined.'' Clearly, he has suddenly remembered
that business with Andrea del Sarto.

``Ah yes? Well, as he suggested we would be eating something
special\ldots{}''

``What did you have?''

``First of all, while studying the menu, he gave me all sorts of
information about food.''

``Before ordering any?''

``Yes.''

``And then?''

``And then, turning to a waiter, he said, `There doesn't seem to be
anything special on the card.' The waiter, not to be outdone, suggested
roast duck or veal chops. Whereupon Waverhouse remarked quite sharply
that we hadn't come a very considerable distance just for common or
garden fare. The waiter, who didn't understand the significance of
common or garden, looked puzzled and said nothing.''

``So I would imagine.''

``Then, turning to me, Waverhouse observed that in France or in England
one can obtain any amount of dishes cooked à la Tenmei or à la Manyō but
that in Japan, wherever you go, the food is all so stereotyped that one
doesn't even feel tempted to enter a restaurant of the so-called Western
style. And so on and so on. He was in tremendous form. But has he ever
been abroad?''

``Waverhouse abroad? Of course not. He's got the money and the time. If
he wanted to, he could go off anytime. He probably just converted his
future intention to travel into the past tense of widely traveled
experience as a sort of joke.'' The master flatters himself that he has
said something witty and laughs invitingly. His guest looks largely
unimpressed.

``I see. I wondered when he'd been abroad. I took everything he said
quite seriously. Besides, he described such things as snail soup and
stewed frogs as though he'd really seen them with his own two eyes.''

``He must have heard about them from someone. He's adept at such
terminological inexactitudes.''

``So it would seem,'' and Beauchamp stares down at the narcissus in a
vase. He seems a little disappointed.

``So, that then was his ingenious idea, I take it?'' asks the master
still in quest of certainties.

``No, that was only the beginning. The main part's still to come.''

``Ah!'' The master utters an interjection mingled with curiosity.

``Having finished his dissertation on matters gastronomical and
European, he proposed `since it's quite impossible to obtain snails or
frogs, however much we may desire them, let's at least have moat-bells.

What do you say?' And without really giving the matter any thought at
all, I answered, `Yes, that would be fine.'''

``Moat-bells sound a little odd.''

``Yes, very odd, but because Waverhouse was speaking so seriously, I
didn't then notice the oddity.'' He seems to be apologizing to my master
for his carelessness.

``What happened next?'' asks my master quite indifferently and without
any sign of sympathetic response to his guest's implied apology.

``Well, then he told the waiter to bring moat-bells for two. The waiter
said, `Do you mean meatballs, sir?' but Waverhouse, assuming an ever
more serious expression, corrected him with gravity. `No, not meatballs,
moat-bells.'\,''

``Really? But is there any such dish as moat-bells?''

``Well I thought it sounded somewhat strange, but as Waverhouse was so
calmly sure and is so great an authority on all things
Occidental---remember it was then my firm belief that he was widely
traveled---I too joined in and explained to the waiter, `Moat-bells, my
good man, moat-bells.'\,''

``What did the waiter do?''

``The waiter---it's really rather funny now one comes to think back on
it---looked thoughtful for a while and then said, `I'm terribly sorry
sir, but today, unfortunately, we have no moat-bells. Though should you
care for meatballs we could serve you, sir, immediately.' Waverhouse
thereupon looked extremely put out and said, `So we've come all this
long way for nothing. Couldn't you really manage moat-bells? Please do
see what can be done,' and he slipped a small tip to the waiter. The
waiter said he would ask the cook again and went off into the kitchen.''

``He must have had his mind dead set on eating moat-bells.''

``After a brief interval the waiter returned to say that if moat-bells
were ordered specially they could be provided, but that it would take a
long time. Waverhouse was quite composed. He said, `It's the New Year
and we are in no kind of hurry. So let's wait for it?' He drew a cigar
from the inside of his Western suit and lighted up in the most leisurely
manner. I felt called upon to match his cool composure so, taking the
Japan News from my kimono pocket, I started reading it. The waiter
retired for further consultations.''

``What a business!'' My master leans forward, showing quite as much
enthusiasm as he does when reading war news in the dailies.

``The waiter re-emerged with apologies and the confession that, of late,
the ingredients of moat-bells were in such short supply that one could
not get them at Kameya's nor even down at No.~15 in Yokohama.

He expressed regret, but it seemed certain that the material for
moat-bells would not be back in stock for some considerable time.

Waverhouse then turned to me and repeated, over and over again,

`What a pity, and we came especially for that dish.' I felt that I had
to say something, so I joined him in saying, `Yes, it's a terrible
shame! Really, a great, great pity!'''

``Quite so,'' agrees my master, though I myself don't follow his
reasoning.

``These observations must have made the waiter feel quite sorry, for he
said, `When, one of these days, we do have the necessary ingredients,
we'd be happy if you would come, sir, and sample our fare.' But when
Waverhouse proceeded to ask him what ingredients the restaurant did use,
the waiter just laughed and gave no answer. Waverhouse then pressingly
enquired if the key-ingredient happened to be Tochian (who, as you know,
is a haiku poet of the Nihon School); and d'you know, the waiter
answered, `Yes, it is, sir, and that is precisely why none is currently
available even in Yokohama. I am indeed,' he added, `most regretful,
sir.'\,''

``Ha-ha-ha! So that's the point of the story? How very funny!'' and the
master, quite unlike his usual self, roars with laughter. His knees
shake so much that I nearly tumble off. Paying no regard to my
predicament, the master laughs and laughs. He seems suddenly deeply
pleased to realize that he is not alone in being gulled by Andrea del
Sarto.

``And then, as soon as we were out in the street, he said `You see,
we've done well. That ploy about the moat-bells was really rather good,
wasn't it?' and he looked as pleased as punch. I let it be known that I
was lost in admiration, and so we parted. However, since by then it was
well past the lunch-hour, I was nearly starving.''

``That must have been very trying for you.'' My master shows, for the
first time, a sympathy to which I have no objection. For a while there
was a pause in the conversation and my purring could be heard by host
and guest.

Mr.~Beauchamp drains his cup of tea, now quite cold, in one quick gulp
and with some formality remarks, ``As a matter-of-fact I've come today
to ask a favor from you.''

``Yes? And what can I do for you?'' My master, too, assumes a formal
face.

``As you know, I am a devotee of literature and art\ldots{}''

``That's a good thing,'' replies my master quite encouragingly.

``Since a little while back, I and a few like-minded friends have got
together and organized a reading group. The idea is to meet once a month
for the purpose of continued studying in this field. In fact, we've
already had the first meeting at the end of last year.''

``May I ask you a question? When you say, like that, a reading group, it
suggests that you engage in reading poetry and prose in a singsong tone.
But in what sort of manner do you, in fact, proceed?''

``Well, we are beginning with ancient works but we intend to consider
the works of our fellow members.''

``When you speak of ancient works, do you mean something like Po Chu-i's
Lute Song?''

``No.''

``Perhaps things like Buson's mixture of haiku and Chinese verse?''

``No.''

``What kinds of thing do you then do?''

``The other day, we did one of Chikamatsu's lovers' suicides.''

``Chikamatsu? You mean the Chikamatsu who wrote jōruri plays?''

There are not two Chikamatsus. When one says Chikamatsu, one does indeed
mean Chikamatsu the playwright and could mean nobody else. I thought my
master really stupid to ask so fool a question. However, oblivious to my
natural reactions, he gently strokes my head. I calmly let him go on
stroking me, justifying my compliance with the reflection that so small
a weakness is permissible when there are those in the world who admit to
thinking themselves under loving observation by persons who merely
happen to be cross-eyed.

Beauchamp answers, ``Yes,'' and tries to read the reaction on my
master's face.

``Then is it one person who reads or do you allot parts among you?''

``We allot parts and each reads out the appropriate dialogue. The idea
is to empathize with the characters in the play and, above all, to bring
out their individual personalities. We do gestures as well. The main
thing is to catch the essential character of the era of the play.
Accordingly, the lines are read out as if spoken by each character,
which may perhaps be a young lady or possibly an errand-boy.''

``In that case it must be like a play.''

``Yes, almost the only things missing are the costumes and the
scenery.''

``May I ask if your reading was a success?''

``For a first attempt, I think one might claim that it was, if anything,
a success.''

``And which lovers' suicide play did you perform on the last occasion?''

``We did a scene in which a boatman takes a fare to the red light
quarter of Yoshiwara.''

``You certainly picked on a most irregular incident, didn't you?'' My
master, being a teacher, tilts his head a little sideways as if
regarding something slightly doubtful. The cigarette smoke drifting from
his nose passes up by his ear and along the side of his head.

``No, it isn't that irregular. The characters are a passenger, a
boatman, a high-class prostitute, a serving-girl, an ancient crone of a
brothel-attendant, and, of course, a geisha-registrar. But that's all.''
Beauchamp seems utterly unperturbed. My master, on hearing the words ``a
high-class prostitute,'' winces slightly but probably only because he's
not well up in the meanings of such technical terms as nakai, yarite,
and kemban.

He seeks to clear the ground with a question. ``Does not nakai signify
something like a maid-servant in a brothel?''

``Though I have not yet given the matter my full attention, I believe
that nakai signifies a serving-girl in a teahouse and that yarite is
some sort of an assistant in the women's quarters.'' Although Beauchamp
recently claimed that his group seeks to impersonate the actual voices
of the characters in the plays, he does not seem to have fully grasped
the real nature of yarite and nakai.

``I see, nakai belong to a teahouse while yarite live in a brothel.
Next, are kemban human beings or is it the name of a place? If human,
are they men or women?''

``Kemban, I rather think, is a male human being.''

``What is his function?''

``I've not yet studied that far. But I'll make inquiries, one of these
days.''

Thinking, in the light of these revelations, that the play-readings must
be affairs extraordinarily ill-conducted, I glance up at my master's
face.

Surprisingly, I find him looking serious. ``Apart from yourself, who
were the other readers taking part?''

``A wide variety of people. Mr.~K, a Bachelor of Law, played the
high-class prostitute, but his delivery of that woman's sugary dialogue
through his very male mustache did, I confess, create a slightly queer
impression. And then there was a scene in which this oiran was seized
with spasms\ldots{}''

``Do your readers extend their reading activities to the simulation of
spasms?'' asked my master anxiously.

``Yes indeed; for expression is, after all, important.'' Beauchamp
clearly considers himself a literary artist à l'outrance.

``Did he manage to have his spasms nicely?'' My master has made a witty
remark.

``The spasms were perhaps the only thing beyond our capability at such a
first endeavor.'' Beauchamp, too, is capable of wit.

``By the way,'' asks my master, ``what part did you take?''

``I was the boatman.''

``Really? You, the boatman!'' My master's tone was such as to suggest
that, if Beauchamp could be a boatman, he himself could be a
geisha-registrar. Switching his tone to one of simple candor, he then
asks: ``Was the role of the boatman too much for you?''

Beauchamp does not seem particularly offended. Maintaining the same calm
voice, he replies, ``As a matter of fact, it was because of this boatman
that our precious gathering, though it went up like a rocket, came down
like a stick. It so happened that four or five girl students are living
in the boarding house next door to our meeting hall. I don't know how,
but they found out when our reading was to take place. Anyway, it
appears that they came and listened to us under the window of the hall.

I was doing the boatman's voice, and, just when I had warmed up nicely
and was really getting into the swing of it---perhaps my gestures were a
little over-exaggerated---the girl students, all of whom had managed to
control their feelings up to that point, thereupon burst out into
simultaneous cachinnations. I was of course surprised, and I was of
course embarrassed: indeed, thus dampened, I could not find it in me to
continue. So our meeting came to an end.''

If this were considered a success, even for a first meeting, what would
failure have been like? I could not help laughing. Involuntarily, my
Adam's apple made a rumbling noise. My master, who likes what he takes
to be purring, strokes my head ever more and more gently. I'm thankful
to be loved just because I laugh at someone, but at the same time I feel
a bit uneasy.

``What very bad luck!'' My master offers condolences despite the fact
that we are still in the congratulatory season of the New Year.

``As for our second meeting, we intend to make a great advance and
manage things in the grand style. That, in fact, is the very reason for
my call today: we'd like you to join our group and help us.''

``I can't possibly have spasms.'' My negative-minded master is already
poised to refuse.

``No, you don't have to have spasms or anything like that. Here's a list
of the patron members.'' So saying, Beauchamp very carefully produced a
small notebook from a purple-colour carrying-wrapper. He opened the
notebook and placed it in front of my master's knees. ``Will you please
sign and make your seal-mark here?'' I see that the book contains the
names of distinguished Doctors of Literature and Bachelors of Arts of
this present day, all neatly mustered in full force.

``Well, I wouldn't say I object to becoming a supporter, but what sort
of obligations would I have to meet?'' My oyster-like master displays
his apprehensions\ldots{}

``There's hardly any obligation. We ask nothing from you except a
signature expressing your approval.''

``Well, in that case, I'll join.'' As he realizes that there is no real
obligation involved, he suddenly becomes lighthearted. His face assumes
the expression of one who would sign even a secret commitment to engage
in rebellion, provided it was clear that the signature carried no
binding obligation. Besides, it is understandable that he should assent
so eagerly: for to be included, even by name only, among so many names
of celebrated scholars is a supreme honor for one who has never before
had such an opportunity. ``Excuse me,'' and my master goes off to the
study to fetch his seal. I am tipped to fall unceremoniously onto the
matting.

Beauchamp helps himself to a slice of sponge cake from the cake-bowl and
crams it into his mouth. For a while he seems to be in pain, mumbling.
Just for a second I am reminded of my morning experience with the
rice-cake. My master reappears with his seal just as the sponge cake
settles down in Beauchamp's bowels. My master does not seem to notice
that a piece of sponge cake is missing from the cake-bowl. If he does, I
shall be the first to be suspected.

Mr.~Beauchamp having taken his departure, my master reenters the study
where he finds on his desk a letter from friend Waverhouse.

``I wish you a very happy New Year\ldots{}''

My master considers the letter to have started with an unusual
seriousness. Letters from Waverhouse are seldom serious. The other day,
for instance, he wrote: ``Of late, as I am not in love with any woman, I
receive no love letters from anywhere. As I am more or less alive,
please set your mind at ease.'' Compared with which, this New Year's
letter is exceptionally matter-of-fact:

I would like to come and see you, but I am so very extremely busy every
day because, contrary to your negativism, I am planning to greet this
New Year, a year unprecedented in all history, with as positive an
attitude as is possible. Hoping you will understand\ldots{}

My master quite understands, thinking that Waverhouse, being Waverhouse,
must be busy having fun during the New Year season.

Yesterday, finding a minute to spare, I sought to treat Mr.~Beauchamp to
a dish of moat-bells. Unfortunately, due to a shortage of their
ingredients, I could not carry out my intention. It was most
regrettable\ldots{}

My master smiles, thinking that the letter is falling more into the
usual pattern.

Tomorrow there will be a card party at a certain Baron's house; the day
after tomorrow a New Year's banquet at the Society of Aesthetes; and the
day after that, a welcoming party for Professor Toribe; and on the day
thereafter\ldots{} My master, finding it rather a bore, skips a few
lines.

So you see, because of these incessant parties--- nō song parties, haiku
parties, tanka parties, even parties for New Style Poetry, and so on and
so on, I am perpetually occupied for quite some time. And that is why I
am obliged to send you this New Year's letter instead of calling on you
in person. I pray you will forgive me\ldots{}

``Of course you do not have to call on me.'' My master voices his answer
to the letter.

Next time that you are kind enough to visit me, I would like you to stay
and dine. Though there is no special delicacy in my poor larder, at
least I hope to be able to offer you some moat-bells, and I am indeed
looking forward to that pleasure\ldots{}

``He's still brandishing his moat-bells,'' muttered my master, who,
thinking the invitation an insult, begins to feel indignant.

However, because the ingredients necessary for the preparation of
moat-bells are currently in rather short supply, it may not be possible
to arrange it. In which case, I will offer you some peacocks'
tongues\ldots{}

``Aha! So he's got two strings to his bow,'' thinks my master and cannot
resist reading the rest of the letter.

As you know, the tongue meat per peacock amounts to less than half the
bulk of the small finger. Therefore, in order to satisfy your gluttonous
stomach\ldots{}

``What a pack of lies,'' remarks my master in a tone of resignation.

I think one needs to catch at least twenty or thirty peacocks. However,
though one sees an occasional peacock, maybe two, at the zoo or at the
Asakusa Amusement Center, there are none to be found at my poulterer's,
which is occasioning me pain, great pain\ldots{}

``You're having that pain of your own free will.'' My master shows no
evidence of gratitude.

The dish of peacocks' tongues was once extremely fashionable in Rome
when the Roman Empire was in the full pride of its prosperity. How I
have always secretly coveted after peacocks' tongues, that acme of
gastronomical luxury and elegance, you may well imagine\ldots{}

``I may well imagine, may I? How ridiculous.'' My master is extremely
cold.

From that time forward until about the sixteenth century, peacock was an
indispensable delicacy at all banquets. If my memory serves me, when the
Earl of Leicester invited Queen Elizabeth to Kenilworth, peacocks'
tongues were on the menu. And in one of Rembrandt's banquet scenes, a
peacock is clearly to be seen, lying in its pride upon the table\ldots{}

My master grumbles that if Waverhouse can find time to compose a history
of the eating of peacocks, he cannot really be so busy.

Anyway, if I go on eating good food as I have been doing recently, I
will doubtless end up one of these days with a stomach weak as
yours\ldots{}

``\,`Like yours' is quite unnecessary. He has no need to establish me as
the prototypical dyspeptic,'' grumbles my master.

According to historians, the Romans held two or three banquets every
day. But the consumption of so much good food, while sitting at a large
table two or three times a day, must produce in any man, however sturdy
his stomach, disorders in the digestive functions. Thus nature has, like
you\ldots{}

``\,`Like you,' again, what impudence!''

But they, who studied long and hard simultaneously to enjoy both luxury
and exuberant health, considered it vital not only to devour
disproportionately large quantities of delicacies, but also to maintain
the bowels in full working order. They accordingly devised a secret
formula\ldots{}

``Really?'' My master suddenly becomes enthusiastic.

They invariably took a post-prandial bath. After the bath, utilizing
methods whose secret has long been lost, they proceeded to vomit up
everything they had swallowed before the bath. Thus were the insides of
their stomachs kept scrupulously clean. Having so cleansed their
stomachs, they would sit down again at the table and there savor to the
uttermost the delicacies of their choice. Then they took a bath again
and vomited once more. In this way, though they gorged on their favorite
dishes to their hearts' content, none of their internal organs suffered
the least damage. In my humble opinion, this was indeed a case of having
one's cake and eating it.

``They certainly seem to have killed two or more birds with one stone.''
My master's expression is one of envy.

Today, this twentieth century, quite apart from the heavy traffic and
the increased number of banquets, when our nation is in the second year
of a war against Russia, is indeed eventful. I, consequently, firmly
believe that the time has come for us, the people of this victorious
country, to bend our minds to study of the truly Roman art of bathing
and vomiting. Otherwise, I am afraid that even the precious people of
this mighty nation will, in the very near future, become, like you,
dyspeptic\ldots{}

``What, again like me? An annoying fellow,'' thinks my master.

Now suppose that we, who are familiar with all things Occidental, by
study of ancient history and legend contrive to discover the secret
formula that has long been lost; then to make use of it now in our Meiji
Era would be an act of virtue. It would nip potential misfortune in the
bud, and, moreover, it would justify my own everyday life which has been
one of constant indulgence in pleasure.

My master thinks all this a trifle odd.

Accordingly, I have now, for some time, been digging into the relevant
works of Gibbon, Mommsen, and Goldwin Smith, but I am extremely sorry to
report that, so far, I have gained not even the slightest clue to the
secret. However, as you know, I am a man who, once set upon a course,
will not abandon it until my object is achieved. Therefore my belief is
that a rediscovery of the vomiting method is not far off. I will let you
know when it happens. Incidentally, I would prefer postponing that feast
of moat-bells and peacocks' tongues, which I've mentioned above, until
the discovery has actually been made. Which would not only be convenient
to me, but also to you who suffer from a weak stomach.

``So, he's been pulling my leg all along. The style of writing was so
sober that I have read it all, and took the whole thing seriously.

Waverhouse must indeed be a man of leisure to play such a practical joke
on me,'' said my master through his laughter.

Several days then passed without any particular event. Thinking it too
boring to spend one's time just watching the narcissus in a white vase
gradually wither, and the slow blossoming of a branch of the
blue-stemmed plum in another vase, I have gone around twice to look for
Tortoiseshell, but both times unsuccessfully. On the first occasion I
thought she was just out, but on my second visit I learnt that she was
ill.

Hiding myself behind the aspidistra beside a wash-basin, I heard the
following conversation which took place between the mistress and her
maid on the other side of the sliding paper-door.

``Is Tortoiseshell taking her meal?''

``No, madam, she's eaten nothing this morning. I've let her sleep on the
quilt of the foot-warmer, well wrapped up.'' It does not sound as if
they spoke about a cat. Tortoiseshell is being treated as if she were a
human.

As I compare this situation with my own lot, I feel a little envious but
at the same time I am not displeased that my beloved cat should be
treated with such kindness.

``That's bad. If she doesn't eat she will only get weaker.''

``Yes indeed, madam. Even me, if I don't eat for a whole day, I couldn't
work at all the next day.''

The maid answers as though she recognized the cat as an animal superior
to herself. Indeed, in this particular household the cat may well be
more important than the maid.

``Have you taken her to see a doctor?''

``Yes, and the doctor was really strange. When I went into his
consulting room carrying Tortoiseshell in my arms, he asked me if I'd
caught a cold and tried to take my pulse. I said `No, Doctor, it is not
I who am the patient, this is the patient,' and I placed Tortoiseshell
on my knees.

The doctor grinned and said he had no knowledge of the sicknesses of
cats, and that if I just left it, perhaps it would get better. Isn't he
too terrible? I was so angry that I told him, `Then, please don't bother
to examine her, she happens to be our precious cat.' And I snuggled
Tortoiseshell back into the breast of my kimono and came straight
home.''

``Truly so.''

``Truly so'' is one of those elegant expressions that one would never
hear in my house. One has to be the thirteenth Shogun's widowed wife's
somebody's something to be able to use such a phrase. I was much
impressed by its refinement.

``She seems to be sniffling\ldots{}''

``Yes, I'm sure she's got a cold and a sore throat; whenever one has a
cold, one suffers from an honorable cough.''

As might be expected from the maid of the thirteenth Shogun's somebody's
something, she's quick with honorifics.

``Besides, recently, there's a thing they call consumption\ldots{}''

``Indeed these days one cannot be too careful. What with the increase in
all these new diseases like tuberculosis and the black plague.''

``Things that did not exist in the days of the Shogunate are all no good
to anyone. So you be careful too!''

``Is that so, madam?''

The maid is much moved.

``I don't see how she could have caught a cold, she hardly ever went
out\ldots{}''

``No, but you see she's recently acquired a bad friend.''

The maid is as highly elated as if she were telling a State secret.

``A bad friend?''

``Yes, that tatty-looking tom at the teacher's house in the main
street.''

``D'you mean that teacher who makes rude noises every morning?''

``Yes, the one who makes the sounds like a goose being strangled every
time he washes his face.''

The sound of a goose being strangled is a clever description. Every
morning when my master gargles in the bathroom he has an odd habit of
making a strange, unceremonious noise by tapping his throat with his
toothbrush. When he is in a bad temper he croaks with a vengeance; when
he is in a good temper, he gets so pepped up that he croaks even more
vigorously. In short, whether he is in a good or a bad temper, he croaks
continually and vigorously. According to his wife, until they moved to
this house he never had the habit; but he's done it every day since the
day he first happened to do it. It is rather a trying habit. We cats
cannot even imagine why he should persist in such behavior. Well, let
that pass. But what a scathing remark that was about ``a tatty-looking
tom.'' I continue to eavesdrop.

``What good can he do making that noise! Under the Shogunate even a
lackey or a sandal-carrier knew how to behave; and in a residential
quarter there was no one who washed his face in such a manner.''

``I'm sure there wasn't, madam.''

That maid is all too easily influenced, and she uses ``madam'' far too
often.

``With a master like that what's to be expected from his cat? It can
only be a stray. If he comes round here again, beat him.''

``Most certainly I'll beat him. It must be all his fault that
Tortoiseshell's so poorly. I'll take it out on him, that I will.''

How false these accusations laid against me! But judging it rash to
approach too closely, I came home without seeing Tortoiseshell.

When I return, my master is in the study meditating in the middle of
writing something. If I told him what they say about him in the house of
the two-stringed harp, he would be very angry; but, as the saying goes,
ignorance is bliss. There he sits, posing like a sacred poet, groaning.

Just then, Waverhouse, who has expressly stated in his New Year letter
that he would be too busy to call for some long time, dropped in.

``Are you composing a new-style poem or something? Show it to me if it's
interesting.''

``I considered it rather impressive prose, so I thought I'd translate
it,'' answers my master somewhat reluctantly.

``Prose? Whose prose?''

``Don't know whose.''

``I see, an anonymous author. Among anonymous works, there are indeed
some extremely good ones. They are not to be slighted. Where did you
find it?''

``The Second Reader,'' answers my master with imperturbable calmness.

``The Second Reader? What's this got to do with the Second Reader?''

``The connection is that the beautifully written article which I'm now
translating appears in the Second Reader.''

``Stop talking rubbish. I suppose this is your idea of a last minute
squaring of accounts for the peacocks' tongues?''

``I'm not a braggart like you,'' says my master and twists his mustache.
He is perfectly composed.

``Once when someone asked Sanyo whether he'd lately seen any fine pieces
of prose, that celebrated scholar of the Chinese classics produced a
dunning letter from a packhorse man and said, `This is easily the finest
piece of prose that has recently come to my attention.' Which implies
that your eye for the beautiful might, contrary to one's expectations,
actually be accurate. Read your piece aloud. I'll review it for you,''
says Waverhouse as if he were the originator of all aesthetic theories
and practice. My master starts to read in the voice of a Zen priest,
reading that injunction left by the Most Reverend Priest Daitō.
``\,`Giant Gravitation,'\,'' he intoned.

``What on earth is giant gravitation?''

``\,`Giant Gravitation' is the title.''

``An odd title. I don't quite understand.''

``The idea is that there's a giant whose name is Gravitation.''

``A somewhat unreasonable idea but, since it's a title, I'll let that
pass.

All right, carry on with the text. You have a good voice. Which makes it
rather interesting.''

``Right, but no more interruptions.'' My master, having laid down his
prior conditions, begins to read again.

Kate looks out of the window. Children are playing ball. They throw the
ball high up in the sky. The ball rises up and up. After a while the
ball comes down. They throw it high again: twice, three times. Every
time they throw it up, the ball comes down. Kate asks why it comes down
instead of rising up and up. ``It is because a giant lives in the
earth,'' replies her mother. ``He is the Giant Gravitation. He is
strong. He pulls everything toward him. He pulls the houses to the
earth. If he didn't they would fly away. Children, too, would fly away.
You've seen the leaves fall, haven't you? That's because the Giant
called them. Sometimes you drop a book. It's because the Giant
Gravitation asks for it. A ball goes up in the sky. The giant calls for
it. Down it falls.

``Is that all?''

``Yes, isn't it good?''

``All right, you win. I wasn't expecting such a present in return for
the moat-bells.''

``It wasn't meant as a return present, or anything like that. I
translated it because I thought it was good. Don't you think it's
good?'' My master stares deep into the gold-rimmed spectacles.

``What a surprise! To think that you of all people had this
talent\ldots{} Well, well! I've certainly been taken in right and proper
this time. I take my hat off to you.'' He is alone in his understanding.
He's talking to himself. The situation is quite beyond my master's
grasp.

``I've no intention of making you doff your cap. I translated this text
simply because I thought it was an interesting piece of writing.''

``Indeed, yes! Most interesting! Quite as it should be! Smashing! I feel
small.''

``You don't have to feel small. Since I recently gave up painting in
watercolors, I've been thinking of trying my hand at writing.''

``And compared with your watercolors, which showed no sense of
perspective, no appreciation of differences in tone, your writings are
superb. I am lost in admiration.''

``Such encouraging words from you are making me positively enthusiastic
about it,'' says my master, speaking from under his continuing
mis-apprehension.

Just then Mr.~Coldmoon enters with the usual greeting.

``Why, hello,'' responds Waverhouse, ``I've just been listening to a
terrifically fine article and the curtain has been rung down upon my
moat-bells.'' He speaks obliquely about something incomprehensible.

``Have you really?'' The reply is equally incomprehensible. It is only
my master who seems not to be in any particularly light humor.

``The other day,'' he remarked, ``a man called Beauchamp Blowlamp came
to see me with an introduction from you.''

``Ah, did he? Beauchamp's an uncommonly honest person, but, as he is
also somewhat odd, I was afraid that he might make himself a nuisance to
you. However, since he had pressed me so hard to be introduced to
you\ldots{}''

``Not especially a nuisance\ldots{}''

``Didn't he, during his visit, go on at length about his name?''

``No, I don't recall him doing so.''

``No? He's got a habit at first meeting of expatiating upon the
singularity of his name.''

``What is the nature of that singularity?'' butts in Waverhouse, who has
been waiting for something to happen.

``He gets terribly upset if someone pronounces Beauchamp as Beecham.''

``Odd!'' said Waverhouse, taking a pinch of tobacco from his
gold-painted, leather tobacco pouch.

``Invariably he makes the immediate point that his name is not Beecham
Blowlamp but Bo-champ Blowlamp.''

``That's strange,'' and Waverhouse inhales pricey tobacco-smoke deep
into his stomach.

``It comes entirely from his craze for literature. He likes the effect
and is inexplicably proud of the fact that his personal name and his
family name can be made to rhyme with each other. That's why when one
pronounces Beauchamp incorrectly, he grumbles that one does not
appreciate what he is trying to get across.''

``He certainly is extraordinary.'' Getting more and more interested,
Waverhouse hauls back the pipe smoke from the bottom of his stomach to
let it loose at his nostrils. The smoke gets lost en route and seems to
be snagged in his gullet. Transferring the pipe to his hand, he coughs
chokingly.

``When he was here the other day, he said he'd taken the part of a
boatman at a meeting of his Reading Society, and that he'd gotten
himself laughed at by a gaggle of schoolgirls,'' says my master with a
laugh.

``Ah, that's it, I remember.'' Waverhouse taps his pipe upon his knees.

This strikes me as likely to prove dangerous, so I move a little way
farther off. ``That Reading Society, now. The other day when I treated
him to moat-bells, he mentioned it. He said they were going to make
their second meeting a grand affair by inviting well-known literary men,
and he cordially invited me to attend. When I asked him if they would
again try another of Chikamatsu's dramas of popular life, he said no and
that they'd decided on a fairly modern play, The Golden Demon. I asked
him what role he would take and he said, `I'm going to play O-miya.'
Beauchamp as O-miya would certainly be worth seeing. I'm determined to
attend the meeting in his support.''

``It's going to be interesting, I think,'' says Coldmoon and he laughs
in an odd way.

``But he is so thoroughly sincere, which is good, and has no hint of
frivolousness about him. Quite different from Waverhouse, for
instance.'' My master is revenged for Andrea del Sarto, for peacocks'
tongues, and for moat-bells all in one go. Waverhouse appears to take no
notice of the remark.

``Ah well, when all's said and done, I'm nothing but a chopping board at
Gyōtoku.''

``Yes, that's about it,'' observes my master, although in fact he does
not understand Waverhouse's involved method of describing himself as a
highly sophisticated simpleton. But not for nothing has he been so many
years a schoolteacher. He is skilled in prevarication, and his long
experience in the classrooms can be usefully applied at such awkward
moments in his social life.

``What is a chopping board at Gyōtoku?'' asks the guileless Coldmoon.

My master looks toward the alcove and pulverizes that chopping board at
Gyōtoku by saying, ``Those narcissi are lasting well. I bought them on
my way home from the public baths toward the end of last year.''

``Which reminds me,'' says Waverhouse, twirling his pipe, ``that at the
end of last year I had a really most extraordinary experience.''

``Tell us about it.'' My master, confident that the chopping board is
now safely back in Gyōtoku, heaves a sigh of relief. The extraordinary
experience of Mr.~Waverhouse fell thus upon our ears:

``If I remember correctly, it was on the twenty-seventh of December.

Beauchamp had said he would like to come and hear me talk upon matters
literary, and had asked me to be sure to be in. Accordingly, I waited
for him all the morning but he failed to turn up. I had lunch and was
seated in front of the stove reading one of Pain's humorous books, when
a letter arrived from my mother in Shizuoka. She, like all old women,
still thinks of me as a child. She gives me all sorts of advice; that I
mustn't go out at night when the weather's cold; that unless the room is
first well-heated by a stove, I'll catch my death of cold every time I
take a bath. We owe much to our parents. Who but a parent would think of
me with such solicitude? Though normally I take things lightly and as
they come, I confess that at that juncture the letter affected me
deeply. For it struck me that to idle my life away, as indeed I do, was
rather a waste. I felt that I must win honor for my family by producing
a masterwork of literature or something like that. I felt I would like
the name of Doctor Waverhouse to become renowned, that I should be
acclaimed as a leading figure in Meiji literary circles, while my mother
is still alive.

Continuing my perusal of the letter, I read, `You are indeed lucky.
While our young people are suffering great hardships for the country in
the war against Russia, you are living in happy-go-lucky idleness as if
life were one long New Year's party organized for your particular
benefit!'

Actually, I'm not as idle as my mother thinks. But she then proceeded to
list the names of my classmates at elementary school who had either died
or had been wounded in the present war. As, one after another, I read
those names, the world grew hollow, all human life quite futile.

And she ended her letter by saying, `since I am getting old, perhaps
this NewYear's rice-cakes will be my last\ldots{}' You will understand
that, as she wrote so very dishearteningly, I grew more and more
depressed. I began to yearn for Beauchamp to come soon, but somehow he
didn't. And at last it was time for supper. I thought of writing in
reply to my mother, and I actually wrote about a dozen lines. My
mother's letter was more than six feet long, but, unable myself to match
such a prodigious performance, I usually excuse myself after writing
some ten lines. As I had been sitting down for the whole of the day, my
stomach felt strange and heavy. Thinking that if Beauchamp did turn up
he could jolly well wait, I went out for a walk to post my letter.
Instead of going toward Fujimicho, which is my usual course, I went,
without my knowing it, out toward the third embankment. It was a little
cloudy that evening and a dry wind was blowing across from the other
side of the moat. It was terribly cold. A train coming from the
direction of Kagurazaka passed with a whistle along the lower part of
the bank. I felt very lonely. The end of the year, those deaths on the
battlefield, senility, life's insecurity, that time and tide wait for no
man, and other thoughts of a similar nature ran around in my head. One
often talks about hanging oneself.

But I was beginning to think that one could be tempted to commit suicide
just at such a time as this. It so happened that at that moment I raised
my head slightly, and, as I looked up to the top of the bank, I found
myself standing right below that very pine tree.''

``That very pine tree? What's that?'' cuts in my master.

``The pine for hanging heads,'' says Waverhouse ducking his noddle.

``Isn't the pine for hanging heads that one at Ko-nodai?'' Coldmoon
amplifies the ripple.

``The pine at Kōnodai is the pine for hanging temple bells. The pine at
Dotesambanchō is the one for hanging heads. The reason why it has
acquired this name is that an old legend says that anyone who finds
himself under this pine tree is stricken with a desire to hang himself.
Though there are several dozen pine trees on the bank, every time
someone hangs himself, it is invariably on this particular tree that the
body is found dangling. I can assure you there are at least two or three
such danglings every year. It would be unthinkable to go and dangle on
any other pine. As I stared at the tree I noted that a branch stuck out
conveniently toward the pavement. Ah! What an exquisitely fashioned
branch. It would be a real pity to leave it as it is. I wish so much
that I could arrange for some human body to be suspended there. I look
around to see if anyone is coming. Unfortunately, no one comes. It can't
be helped.

Shall I hang myself? No, no, if I hang myself, I'll lose my life. I
won't because it's dangerous. But I've heard a story that an ancient
Greek used to entertain banquet parties by giving demonstrations of how
to hang oneself. A man would stand on a stool and the very second that
he put his head through a noose, a second man would kick the stool from
under him. The trick was that the first man would loosen the knot in the
rope just as his stool was kicked away, and so drop down unharmed. If
this story is really true, I've no need to be frightened. So thinking I
might try the trick myself, I place my hand on the branch and find it
bends in a manner precisely appropriate. Indeed the way it bends is
positively aesthetic. I feel extraordinarily happy as I try to picture
myself floating on this branch. I felt I simply must try it, but then I
began to think that it would be inconsiderate if Beauchamp were waiting
for me. Right, I would first see Beauchamp and have the chat I'd
promised; thereafter I could come out again. So thinking, I went home.''

``And is that the happy ending to your story?'' asks my master.

``Very interesting,'' says Coldmoon with a broad grin.

``When I got home, Beauchamp had not arrived. Instead, I found a
postcard from him saying that he was sorry he could not keep our
appointment because of some pressing but unexpected happening, and that
he was looking forward to having a long interview with me in the near
future. I was relieved, and I felt happy, for now I could hang myself
with an easy mind. Accordingly, I hurry back to the same spot, and
then\ldots{}'' Waverhouse, assuming a nonchalant air, gazes at Coldmoon
and my master.

``And then, what happened?'' My master is becoming a little impatient.

``We've now come to the climax,'' says Coldmoon as he twists the strings
of his surcoat.

``And then, somebody had beaten me to it and had already hanged himself.
I'm afraid I missed the chance just by a second. I see now that I had
been in the grip of the God of Death. William James, that eminent
philosopher, would no doubt explain that the region of the dead in the
world of one's subliminal consciousness and the real world in which I
actually exist, must have interacted in mutual response in accordance
with some kind of law of cause and effect. But it really was
extraordinary, wasn't it?'' Waverhouse looks quite demure.

My master, thinking that he has again been taken in, says nothing but
crams his mouth with bean-jam cake and mumbles incoherently.

Coldmoon carefully rakes smooth the ashes in the brazier and casts down
his eyes, grinning; eventually he opens his mouth. He speaks in an
extremely quiet tone.

``It is indeed so strange that it does not seem a thing likely to
happen. On the other hand, because I myself have recently had a similar
kind of experience, I can readily believe it.''

``What! Did you too want to stretch your neck?''

``No, mine wasn't a hanging matter. It seems all the more strange in
that it also happened at the end of last year, at about the same time
and on the same day as the extraordinary experience of Mr.~Waverhouse.''

``That's interesting,'' says Waverhouse. And he, too, stuffs his mouth
with bean-jam cake.

``On that day, there was a year-end party combined with a concert given
at the house of a friend of mine at Mūkōjima. I went there taking my
violin with me. It was a grand affair with fifteen or sixteen young or
married ladies. Everything was so perfectly arranged that one felt it
was the most brilliant event of recent times. When the dinner and the
concert were over, we sat and talked late, and as I was about to take my
leave, the wife of a certain doctor came up to me and asked in whisper
if I knew that Miss O was unwell. A few days earlier, when last I saw
Miss O, she had been looking well and normal. So I was surprised to hear
this news, and my immediate questions elicited the information that she
had become feverish on the very evening of the day when I'd last seen
her, and that she was saying all sorts of curious things in her
delirium. What was worse, every now and again in that delirium, she was
calling my name.''

Not only my master but even Waverhouse refrain from making any such
hackneyed remark as ``you lucky fellow.'' They just listen in silence.

``They fetched a doctor who examined her. According to the doctor's
diagnosis, though the name of the disease was unknown, the high fever
affecting the brain made her condition dangerous unless the
administration of soporifics worked as effectively as was to be hoped
for. As soon as I heard this news, a feeling of something awful grew
within me. It was a heavy feeling, as though one were having a
nightmare, and all the surrounding air seemed suddenly to be solidifying
like a clamp upon my body. On my way home, moreover, I found I could
think of nothing else, and it hurt. That beautiful, that gay, that so
healthy Miss O\ldots{}''

``Just a minute, please. You've mentioned Miss O about two times. If
you've no objection, we'd like to know her name wouldn't we?'' asks
Waverhouse turning to look at my master. The latter evades the question
and says, ``Hmm.''

``No, I won't tell you her name since it might compromise the person in
question.''

``Do you then propose to recount your entire story in such vague,
ambiguous, equivocal, and noncommittal terms?''

``You mustn't sneer. This is a serious story. Anyway, the thought of
that young lady suffering from so odd an ailment filled my heart with
mournful emotion, and my mind with sad reflections on the ephemerality
of life. I felt suddenly depressed beyond all saying, as if every last
ounce of my vitality had, just like that, evaporated from my body. I
staggered on, tottering and wobbling, until I came to the Azuma Bridge.
As I looked down, leaning on the parapet, the black waters---at neap or
ebb, I don't know which---seemed to be coagulating, only just barely
moving. A rickshaw coming from the direction of Hanakawado ran over the
bridge. I watched its lamp grow smaller and smaller until it disappeared
at the Sapporo Beer factory. Again, I looked down at the water.

And at that moment I heard a voice from upstream calling my name. It is
most improbable that anyone should be calling after me at this unlikely
time of night, and, wondering whom it could possibly be, I peered down
to the surface of the water, but I could see nothing in the darkness.
Thinking it must have been my imagination, I had decided to go home,
when I again heard the voice calling my name. I stood dead-still and
listened. When I heard it calling me for the third time, though I was
gripping the parapet firmly, my knees began to tremble uncontrollably.

The voice seemed to be coming either from far away or from the bottom of
the river, but it was unmistakably the voice of Miss O. In spite of
myself I answered, `Yes.' My answer was so loud that it echoed back from
the still water, and, surprised by my own voice, I looked around me in a
startled manner. There was no one to be seen. No dog. No moon. Nothing.
At this very second I experienced a sudden urge to immerse myself in
that total darkness from which the voice had summoned me. And, once
again, the voice of Miss O pierced my ears painfully, appealingly, as if
begging for help. This time I cried, `I'm coming now,' and, leaning well
out over the parapet, I looked down into the somber depths. For, it
seemed to me that the summoning voice was surging powerfully up from
beneath the waves. Thinking that the source of the pleading must lie in
the water directly below me, I at last managed to clamber onto the
parapet. I was determined that, next time the voice called out to me, I
would dive straight in; and, as I stood watching the stream, once again
the thin thread of that pitiful voice came floating up to me. This, I
thought, is it; jumping high with all my strength, I came dropping down
without regret like a pebble, or something.''

``So, you actually did dive in?'' asks my master, blinking his eyes.

``I never thought you'd go as far as that,'' says Waverhouse pinching
the tip of his nose.

``After my dive I became unconscious, and for a while I seemed to be
living in a dream. But eventually I woke up, and, though I felt cold, I
was not at all wet and did not feel as if I had swallowed any water. Yet
I was sure that I had dived. How very strange! Realizing that something
peculiar must have taken place, I looked around me and received a real
shock.

I'd meant to dive into the water but apparently I'd accidentally landed
in the middle of the bridge itself. I felt abysmally regretful. Having,
by sheer mistake, jumped backwards instead of forwards, I'd lost my
chance to answer the summons of the voice.'' Coldmoon smirks and fiddles
with the strings of his surcoat as if they were in some way irksome.

``Ha-ha-ha, how very comical. It's odd that your experience so much
resembles mine. It, too, could be adduced in support of the theories of
Professor James. If you were to write it up in an article entitled `The
Human Response,' it would astound the whole literary world. But what,''
persisted Waverhouse, ``became of the ailing Miss O?''

``When I called at her house a few days ago, I saw her just inside the
gate playing battledore and shuttlecock with her maid. So I expect she
has completely recovered from her illness.''

My master, who for some time has been deep in thought, finally opens his
mouth, and, in a spirit of unnecessary rivalry, remarks, ``I too have a
strange experience to relate.''

``You've got what?'' In Waverhouse's view, my master counts for so
little that he is scarcely entitled to have experiences.

``Mine also occurred at the end of last year.''

``It's queer,'' observed Coldmoon, ``that all last year,'' and he
sniggers. A piece of bean-jam cake adheres to the corner of his chipped
front tooth.

``And it took place, doubtless,'' added Waverhouse, ``at the very same
time on the very same day.''

``No, I think the date is different: it was about the 20th. My wife had
earlier asked me, as a year's-end present to herself, to take her to
hear Settsu Daijō. I'd replied that I wouldn't say no, and asked her the
nature of the program for that day. She consulted the newspapers and
answered that it was one of Chikamatsu's suicide dramas, Unagidani.
`Let's not go today, I don't like Unagidani,' said I. So we did not go
that day. The next day my wife, bringing out the newspaper again, said,
`Today he's doing the Monkey Man at Horikawa, so, let's go.' I said
let's not, because Horikawa was so frivolous, just samisen-playing with
no meat in it. My wife went away looking discontented. The following
day, she stated almost as a demand, `Today's program is The Temple With
Thirty-Three Pillars. You may dislike the Temple quite as strongly as
you disliked all the others, but since the treat is intended to be for
me, surely you won't object to taking me there.' I responded, `If you've
set your heart on it so firmly, then we'll go, but since the performance
has been announced as Settsu's farewell appearance on the stage, the
house is bound to be packed full, and since we haven't booked in
advance, it will obviously be impossible to get in. To start with, in
order to attend such performances there's an established procedure to be
observed. You have to go to the theatre-teahouse and there negotiate for
seat reservations. It would be hopeless to try going about it in the
wrong way. You just can't dodge this proper procedure. So, sorry though
I am, we simply cannot go today.' My wife's eyes glittered fiercely.
`Since I am a mere woman, I do not understand your complicated
procedures, but both Ohare's mother and Kimiyo of the Suzuki family
managed to get in without observance of any such formalities, and they
heard everything very well. I realize that you are a teacher, but surely
you don't have to go through all that troublesome rigmarole just to
visit a theatre? It's too bad\ldots{} you are so\ldots{}' and her voice
became tearful. I gave in. `All right. We'll go to the theatre even if
we can't get into it. After an early supper we'll take the tram.' She
suddenly became quite lively. `If we're going, we must be there by four
o'clock, so we mustn't dilly-dally.' When I asked her why one had to be
there by four o'clock, she explained that Kimiyo had told her that, if
one arrived any later, all the seats would be taken. I asked her again,
to make quite sure, if it would be fruitless to turn up later than four
o'clock; and she answered briskly, `Of course it would be no good.'
Then, d'you know, at that very moment the shivering set in.''

``Do you mean your wife?'' asks Coldmoon.

``Oh, no, my wife was as fit as a fiddle. It was me. I had a sudden
feeling that I was shriveling like a pricked balloon. Then I grew giddy
and unable even to move.''

``You were taken ill with a most remarkable suddenness,'' commented
Waverhouse.

``This is terrible. What shall I do? I'd like so much to grant my wife
her wish, her one and only request in the whole long year. All I ever do
is scold her fiercely, or not speak to her, or nag her about household
expenses, or insist that she cares more carefully for the children; yet
I have never rewarded her for all her efforts in the domestic field.
Today, luckily, I have the time and the money available. I could easily
take her on some little outing. And she very much wants to go. Just as I
very much want to take her. But much indeed as I want to take her, this
icy shivering and frightful giddiness make it impossible for me even to
step down from the entrance of my own house, let alone to climb up into
a tram. The more I think how deeply I grieve for her, the poor thing,
the worse my shivering grows and the more giddy I become. I thought if I
consulted a doctor and took some medicine, I might get well before four
o'clock. I discussed the matter with my wife and sent for Mr.~Amaki,
Bachelor of Medicine. Unfortunately, he had been on night duty at the
university hospital and hadn't yet come home. However, we received every
assurance that he was expected home by about two o'clock and that he
would hurry round to see me the minute he returned. What a nuisance. If
only I could get some sedative, I know I could be cured before four. But
when luck is running against one, nothing goes well.

Here I am, just this once in a long, long time, looking forward to
seeing my wife's happy smile, and to be sharing in that happiness. My
expectations seem sadly unlikely to be fulfilled. My wife, with a most
reproachful look, enquires whether it really is impossible for me to go
out. `I'll go; certainly I'll go. Don't worry. I'm sure I'll be all
right by four. Wash your face, get ready to go out and wait for me.'
Though I uttered all these reassurance, my mind was shaken with profound
emotions. The shivering strengthens and accelerates, and my giddiness
grows worse and worse. Unless I do get well by four o'clock and
implement my promise, one can never tell what such a pusillanimous woman
might do.

What a wretched business. What should I do? As I thought it possible
that the very worst could happen, I began to consider whether perhaps it
might be my duty as a husband to explain to my wife, now while I was
still in possession of my faculties, the dread truths concerning
mortality and the vicissitudes of life. For if the worst should happen,
she would then at least be prepared and less liable to be overcome by
the paroxysms of her grief. I accordingly summoned my wife to come
immediately to my study. But when I began by saying, `Though but a woman
you must be aware of that Western proverb which states that there is
many a slip ``twixt the cup and the lip,''\,' she flew into a fury. `How
should I know anything at all about such sideways-written words? You're
deliberately making a fool of me by choosing to speak English when you
know perfectly well that I don't understand a word of it. All right. So
I can't understand English. But if you're so besotted about English, why
didn't you marry one of those girls from the mission schools? I've never
come across anyone quite so cruel as you.' In the face of this tirade,
my kindly feelings, my husbandly anxiety to prepare her for extremities,
were naturally damped down. I'd like you two to understand that it was
not out of malice that I spoke in English. The words sprang solely from
a sincere sentiment of love for my wife. Consequently, my wife's malign
interpretation of my motives left me feeling helpless. Besides, my brain
was somewhat disturbed by reason of the cold shivering and the
giddiness; on top of all that, I was understandably distraught by the
effort of trying quickly to explain to her the truths of mortality and
the nature of the vicissitudes of life. That was why, quite
unconsciously and forgetting that my wife could not understand the
tongue, I spoke in English. I immediately realized I was in the wrong.
It was all entirely my fault. But as a result of my blunder, the cold
shivering intensified its violence and my giddiness grew ever more
viciously vertiginous. My wife, in accordance with my instructions,
proceeds to the bathroom and, stripping herself to the waist, completes
her make-up. Then, taking a kimono from a drawer, she puts it on. Her
attitudes make it quite clear that she is now ready to go out any time,
and is simply waiting for me. I begin to get nervous. Wishing that
Mr.~Amaki would arrive quickly, I look at my watch. It's already three
o'clock. Only one hour to go. My wife slides open the study door and
putting her head in, asks, `Shall we go now?' It may sound silly to
praise one's own wife, but I had never thought her quite so beautiful as
she was at that moment. Her skin, thoroughly polished with soap, gleams
deliciously and makes a marvelous contrast with the blackness of her
silken surcoat. Her face has a kind of radiance both externally and
shining from within; partly because of the soap and partly because of
her intense longing to listen to Settsu Daijō. I feel I must, come what
may, take her out to satisfy that yearning. All right, perhaps I will
make the awful effort to go out. I was smoking and thinking along these
lines when at long last Mr.~Amaki arrived. Excellent. Things are turning
out as one would wish. However, when I told him about my condition,
Amaki examined my tongue, took my pulse, tapped my chest, stroked my
back, turned my eyelids inside out, patted my skull, and thereafter sank
into deep thought for quite some time. I said to him, `It is my
impression that there may be some danger\ldots{}' but he replied, `No, I
don't think there's anything seriously wrong.'

`I imagine it would be perfectly all right for him to go out for a
little while?' asked my wife.

`Let me think.' Amaki sank back into the profundities of thought,
reemerging to remark, `Well, so long as he doesn't feel unwell\ldots{}'

`Oh, but I do feel unwell,' I said.

`In that case I'll give you a mild sedative and some liquid
medicine\ldots{}'

`Yes please. This is going to be something serious, isn't it?'

`Oh no, there's nothing to worry about. You mustn't get nervous,' said
Amaki, and thereupon departed. It is now half past three.

The maid was sent to fetch the medicine. In accordance with my wife's
imperative instructions, the wretched girl not only ran the whole way
there, but also the whole way back. It is now a quarter to four. Fifteen
minutes still to go. Then, quite suddenly, just about that time, I began
to feel sick. It came on with a quite extraordinary suddenness. All
totally unexpected. My wife had poured the medicine into a teacup and
placed it in front of me, but as soon as I tried to lift the teacup,
some keck-keck thing stormed up from within the stomach. I am compelled
to put the teacup down. `Drink it up quickly' urges my wife. Yes,
indeed, I must drink it quickly and go out quickly. Mustering all my
courage to imbibe the potion I bring the teacup to my lips, when again
that insuppressible keck-keck thing prevents my drinking it. While this
process of raising the cup and putting it down is being several times
repeated, the minutes crept on until the wall clock in the living room
struck four o'clock.

Ting-ting-ting-ting. Four o'clock it is. I can no longer dilly-dally and
I raise the teacup once again. D'you know, it really was most strange.
I'd say that it was certainly the uncanniest thing I've ever
experienced. At the fourth stroke my sickliness just vanished, and I was
able to take the medicine without any trouble at all. And, by about ten
past four---here I must add that I now realized for the first time how
truly skilled a physician we have in Dr.~Amaki---the shivering of my
back and the giddiness in my head both disappeared like a dream. Up to
that point I had expected that I was bound to be laid up for days, but
to my great pleasure the illness proved to have been completely cured.''

``And did you two then go out to the theatre?'' asks Waverhouse with the
puzzled expression of one who cannot see the point of a story.

``We certainly both wanted to go, but since it had been my wife's
reiterated view that there was no hope of getting in after four o'clock,
what could we do? We didn't go. If only Amaki had arrived fifteen
minutes earlier, I could have kept my promise and my wife would have
been satisfied. Just that fifteen minute difference. I was indeed
distressed. Even now, when I think how narrow the margin was, I am again
distressed.''

My master, having told his shabby tale, contrives to look like a person
who has done his duty. I imagine he feels he's gotten even with the
other two.

``How very vexing,'' says Coldmoon. His laugh, as usual, displays his
broken tooth.

Waverhouse, with a false naivety, remarks as if to himself. ``Your wife,
with a husband so thoughtful and kind-hearted, is indeed a lucky
woman.'' Behind the sliding paper-door, we heard the master's wife make
an harumphing noise as though clearing her throat.

I had been quietly listening to the successive stories of these three
precious humans, but I was neither amused nor saddened by what I'd
heard. I merely concluded that human beings were good for nothing,
except for the strenuous employment of their mouths for the purpose of
whiling away their time in laughter at things which are not funny, and
in the enjoyment of amusements which are not amusing. I have long known
of my master's selfishness and narrowmindedness, but, because he usually
has little to say, there was always something about him which I could
not understand. I'd felt a certain caution, a certain fear, even a
certain respect toward him on account of that aspect of his nature that
I did not understand. But having heard his story, my uncertainties
suddenly coalesced into a mere contempt for him. Why can't he listen to
the stories of the other two in silence? What good purpose can he serve
by talking such utter rubbish just because his competitive spirit has
been roused? I wonder if, in his portentous writings, Epictetus
advocated any such course of action. In short, my master, Waverhouse,
and Coldmoon are all like hermits in a peaceful reign. Though they adopt
a nonchalant attitude, keeping themselves aloof from the crowd,
segegrated like so many snake-gourds swayed lightly by the wind, in
reality they, too, are shaken by just the same greed and worldly
ambition as their fellow men.

The urge to compete and their anxiety to win are revealed flickeringly
in their everyday conversation, and only a hair's breadth separates them
from the Philistines whom they spend their idle days denouncing. They
are all animals from the same den. Which fact, from a feline viewpoint,
is infinitely regrettable. Their only moderately redeeming feature is
that their speech and conduct are less tediously uninventive than those
of less subtle creatures.

As I thus summed up the nature of the human race, I suddenly felt the
conversation of these specimens to be intolerably boring, so I went
around to the garden of the mistress of the two-stringed harp to see how
Tortoiseshell was getting on. Already the pine tree decorations for the
New Year and that season's sacred festoons have been taken down. It is
the 10th of January. From a deep sky containing not even a single streak
of cloud the glorious springtime sun shines down upon the lands and seas
of the whole wide world, so that even her tiny garden seems yet more
brilliantly lively than when it saw the dawn of New Year's Day.

There is a cushion on the veranda, the sliding paper-door is closed, and
there's nobody about. Which probably means that the mistress has gone
off to the public baths. I'm not at all concerned if the mistress should
be out, but I do very much worry about whether Tortoiseshell is any
better. Since everything's so quiet and not a sign of a soul, I hop up
onto the veranda with my muddy paws and curl up right in the middle of
the cushion, which I find comfortable. A drowsiness came over me, and,
forgetting all about Tortoiseshell, I was about to drop off into a doze
when suddenly I heard voices beyond the paper-door.

``Ah, thanks. Was it ready?'' The mistress has not gone out after all.

``Yes, madam. I'm sorry to have taken such a long time. When I got
there, the man who makes Buddhist altar furniture told me he'd only just
finished it.''

``Well, let me see it. Ah, but it's beautifully done. With this,
Tortoiseshell can surely rest in peace. Are you sure the gold won't peel
away?''

``Yes, I've made sure of it. They said that as they had used the very
best quality, it would last longer than most human memorial tablets.
They also said that the character for `honor' in Tortoiseshell's
posthumous name would look better if written in the cursive style, so
they had added the appropriate strokes.''

``Is that so? Well, let's put Myōyoshinnyo's tablet in the family shrine
and offer incense sticks.''

Has anything happened to Tortoiseshell? Thinking something must be
wrong, I stand up on the cushion. Ting! ``Amen! Myōyoshinnyo. Save us,
merciful Buddha! May she rest in peace.'' It is the voice of the
mistress.

``You, too, say prayers for her.''

Ting! ``Amen! Myōyoshinnyo. Save us, merciful Buddha! May she rest in
peace.'' Suddenly my heart throbs violently. I stand dead-still upon the
cushion, like a wooden cat; not even my eyes are moving.

``It really was a pity. It was only a cold at first.''

``Perhaps if Dr.~Amaki had given her some medicine, it might have
helped.''

``It was indeed Amaki's fault. He paid too little regard to
Tortoiseshell.''

``You must not speak ill of other persons. After all, everyone dies when
their allotted span is over.''

It seems thatTortoiseshell was also attended by that skilled physician,
Dr.~Amaki.

``When all's said and done, I believe the root cause was that the stray
cat at the teacher's in the main street took her out too often.''

``Yes, that brute.''

I would like to exculpate myself, but realizing that at this juncture it
behoves me to be patient, I swallow hard and continue listening. There
is a pause in the conversation.

``Life does not always turn out as one wishes. A beauty like
Tortoiseshell dies young. That ugly stray remains healthy and flourishes
in devilment\ldots{}''

``It is indeed so, Madam. Even if one searched high and low for a cat as
charming as Tortoiseshell, one would never find another person like
her.''

She didn't say `another cat,' she said `another person.' The maid seems
to think that cats and human beings are of one race. Which reminds me
that the face of this particular maid is strangely like a cat's.

``If only instead of our dear Tortoiseshell\ldots{}''

``\ldots{}that wretched stray at the teacher's had been taken. Then,
Madam, how perfectly everything would have gone\ldots{}''

If everything had gone that perfectly, I would have been in deep
trouble. Since I have not yet had the experience of being dead, I cannot
say whether or not I would like it. But the other day, it happening to
be unpleasantly chilly, I crept into the tub for conserving half-used
charcoal and settled down upon its still-warm contents. The maid, not
realizing I was in there, popped on the lid. I shudder even now at the
mere thought of the agony I then suffered. According to Miss Blanche,
the cat across the road, one dies if that agony continues for even a
very short stretch. I wouldn't complain if I were asked to substitute
for Tortoiseshell; but if one cannot die without going through that kind
of agony, I frankly would not care to die on anyone's behalf.

``Though a cat, she had her funeral service conducted by a priest and
now she's been given a posthumous Buddhist name. I don't think she would
expect us to do more.''

``Of course not, madam. She is indeed thrice blessed. The only comment
that one might make is that the funeral service read by the priest was,
perhaps, a little wanting in gravity.''

``Yes, and I thought it rather too brief. But when I remarked to the
priest from the Gekkei Temple `you've finished very quickly, haven't
you?' he answered `I've done sufficient of the effective parts, quite
enough to get a kitty into Paradise.'\,''

``Dear me! But if the cat in question were that unpleasant
stray\ldots{}''

I have pointed out often enough that I have no name, but this maid keeps
calling me ``that stray.'' She is a vulgar creature.

``So very sinful a creature. Madam, would never be able to rest in
peace, however many edifying texts were read for its salvation.''

I do not know how many hundreds of times I was thereafter stigmatized as
a stray. I stopped listening to their endless babble while it was still
only half-run, and, slipping down from the cushion, I jumped off the
veranda. Then, simultaneously erecting every single one of my
eighty-eight thousand, eight hundred, eighty hairs, I shook my whole
body.

Since that day I have not ventured near the mistress of the two-stringed
harp. No doubt by now she herself is having texts of inadequate gravity
read on her behalf by the priest from the Gekkei Temple.

Nowadays I haven't even energy to go out. Somehow life seems weary. I
have become as indolent a cat as my master is an indolent human. I have
come to understand that it is only natural that people should so often
explain my master's self-immurement in his study as the result of a love
affair gone wrong.

As I have never caught a rat, that O-san person once proposed that I
should be expelled; but my master knows that I'm no ordinary common or
garden cat, and that is why I continue to lead an idle existence in this
house. For that understanding I am deeply grateful to my master. What's
more, I take every opportunity to show the respect due to his
perspicacity. I do not get particularly angry with O-san's ill-treatment
of me, for she does not understand why I am as I now am. But when, one
of these days, some master sculptor, some regular Hidari Jingorō, comes
and carves my image on a temple gate; when some Japanese equivalent of
the French master portraitist, Steinlein, immortalizes my features on a
canvas, then at last will the silly purblind beings in shame regret
their lack of insight.
\chapter*{III}
TORTOISESHELL is dead, one cannot consort with Rickshaw Blacky, and I
feel a little lonely. Luckily I have made acquaintances among humankind
so I do not suffer from any real sense of boredom. Someone wrote
recently asking my master to have my photograph taken and the picture
sent to him. And then the other day somebody else presented some millet
dumplings, that speciality of Okayama, specifically addressed to me. The
more that humans show me sympathy, the more I am inclined to forget that
I am a cat. Feeling that I am now closer to humans than to cats, the
idea of rallying my own race in an effort to wrest supremacy from the
bipeds no longer has the least appeal. Moreover, I have developed,
indeed evolved, to such an extent that there are now times when I think
of myself as just another human in the human world; which I find very
encouraging. It is not that I look down on my own race, but it is no
more than natural to feel most at ease among those whose attitudes are
similar to one's own. I would consequently feel somewhat piqued if my
growing penchant for mankind were stigmatized as fickleness or flippancy
or treachery. It is precisely those who sling such words about in
slanderous attacks on others who are usually both drearily
straight-laced and born unlucky. Having thus graduated from felinity to
humanity, I find myself no longer able to confine my interests to the
world of Tortoiseshell and Blacky. With a haughtiness not less prideful
than that of human beings, I, too, now like to judge and criticize their
thoughts and words and deeds. This, surely, is equally natural. Yet,
though I have become thus proudly conscious of my own dignity, my master
still regards me as a cat only slightly superior to any other common or
garden moggy. For, as if they were his own and without so much as a
by-your-leave to me, he has eaten all the millet dumplings; which is, I
find, regrettable. Nor does he seem yet to have dispatched my
photograph. I suppose I would be justified if I made this fact a cause
for grumbling, but after all, if our opinions---my master's and
mine---are naturally at difference, the consequences of that difference
cannot be helped. Since I am seeking to behave with total humanity, I'm
finding it increasingly difficult to write about the activities of cats
with whom I no longer associate. I must accordingly seek the indulgence
of my readers if I now confine my writing to reports about such
respected figures as Waverhouse and Coldmoon.

Today is a Sunday and the weather fine. The master has therefore crept
out of his study, and, placing a brush, an ink stone, and a writing pad
in a row before him, he now lies flat on his belly beside me, and is
groaning hard. I watch him, thinking that he is perhaps making this
peculiar noise in the birth pangs of some literary effort. After a
while, and in thick black strokes, he wrote, ``Burn incense.'' Is it
going to be a poem or a haiku? Just when I was thinking that the phrase
was rather too witty for my master, he abandons it, and, his brush
running quickly over the paper, writes an entirely new line: ``Now for
some little time I have been thinking of writing an article about
Mr.~the-late-and-sainted Natural Man.'' At this point the brush stops
dead. My master, brush in hand, racks his brains, but no bright notions
seem to emerge for he now starts licking the head of his brush. I
watched his lips acquire a curious inkiness. Then, underneath what he
had just written, he drew a circle, put in two dots as eyes, added a
nostrilled nose in the center, and finally drew a single sideways line
for a mouth. One could not call such creations either haiku or prose.
Even my master must have been disgusted with himself, for he quickly
smeared away the face. He then starts a new line. He seems to have some
vague notion that, provided he himself produces a new line, maybe some
kind of a Chinese poem will evolve itself.

After further moonings, he suddenly started writing briskly in the
colloquial style. ``Mr.~the-late-and-sainted Natural Man is one who
studies Infinity, reads the Analects of Confucius, eats baked yams, and
has a runny nose.'' A somewhat muddled phrase. He thereupon read the
phrase aloud in a declamatory manner and, quite unlike his usual self,
laughed. ``Ha-ha-ha. Interesting! But that `runny nose' is a shade
cruel, so I'll cross it out,'' and he proceeds to draw lines across that
phrase.

``Though a single line would clearly have sufficed, he draws two lines
and then three lines. He goes on drawing more and more lines regardless
of their crowding into the neighboring line of writing. When he has
drawn eight such obliterations, he seems unable to think of anything to
add to his opening outburst. So he takes to twirling his mustache,
determined to wring some telling sentence from his whiskers. He is still
twisting them up and twirling them down when his wife appears from the
living room, and sitting herself down immediately before my master's
nose, remarks, ``My dear.''

``What is it?'' My master's voice sounds dully like a gong struck under
water. His wife seems not to like the answer, for she starts all over
again.

``My dear!'' she says.

``Well, what is it?''

This time, cramming a thumb and index finger into a nostril, he yanks
out nostril hairs.

``We are a bit short this month\ldots{}''

``Couldn't possibly be short. We've settled the doctor's fee and we paid
off the bookshop's bill last month. So this month, there ought in fact
to be something left over.'' He coolly examines his uprooted nostril
hairs as though they were some wonder of the world.

``But because you, instead of eating rice, have taken to bread and
jam\ldots{}''

``Well, how many tins of jam have I gone through?''

``This month, eight tins were emptied.''

``Eight? I certainly haven't eaten that much.''

``It wasn't only you. The children also lick it.''

``However much one licks, one couldn't lick more than two or three
shillings worth.'' My master calmly plants his nostril hairs, one by
one, on the writing pad. The sticky-rooted bristles stand upright on the
paper like a little copse of needles. My master seems impressed by this
unexpected discovery and he blows upon them. Being so sticky, they do
not fly away.

``Aren't they obstinate?'' he says and blows upon them frantically.

``It is not only the jam. There's other things we have to buy.'' The
lady of the house expresses her extreme dissatisfaction by pouting
sulkily.

``Maybe.'' Again inserting his thumb and finger, he extracts some hairs
with a jerk. Among these hairs of various hue, red ones and black ones,
there is a single pure white bristle. My master who, with a look of
great surprise, has been staring at this object, proceeds to show it to
his wife, holding it up between his fingers right in front of her face.

``No, don't.'' She pushes his hand away with a grimace of distaste.

``Look at it! A white hair from the nostrils.'' My master seems to be
immensely impressed. His wife, resigned, went back into the living room
with a laugh. She seems to have given up hope of getting any answer to
her problems of domestic economy. My master resumes his consideration of
the problems of Natural Man.

Having succeeded in driving off his wife with his scourge of nostril
hair, he appears to feel relieved, and, while continuing that
depilation, struggles to get on with his article. But his brush remains
unmoving.

``That `eats baked yams' is also superfluous. Out with it.'' He deletes
the phrase. ``And `incense burns' is somewhat over-abrupt, so let's
cross that out too.'' His exuberant self-criticism leaves nothing on the
paper but the single sentence ``Mr.~the-late-and-sainted Natural Man is
one who studies Infinity and reads the Analects of Confucius.'' My
master thinks this statement a trifle over-simplified. ``Ah well, let's
not be bothered: let's abandon prose and just make it an inscription.''
Brandishing the brush crosswise, he paints vigorously on the writing pad
in that watercolor style so common among literary men and produces a
very poor study of an orchid. Thus all his precious efforts to write an
article have come down to this mere nothing. Turning the sheet, he
writes something that makes no sense. ``Born in Infinity, studied
Infinity, and died into Infinity. Mr.~the-late-and-sainted Natural Man.
Infinity.'' At this moment Waverhouse drifts into the room in his usual
casual fashion. He appears to make no distinction between his own and
other people's houses; unannounced and unceremoniously, he enters any
house and, what's more, will sometimes float in unexpectedly through a
kitchen door. He is one of those who, from the moment of their birth,
discaul themselves of all such tiresome things as worry, reserve,
scruple, and concern.

``\,`Giant Gravitation again?'\,'' asks Waverhouse still standing.

``How could I be always writing only about `Giant Gravitation?' I'm
trying to compose an epitaph for the tombstone of
Mr.~the-late-and-sainted Natural Man,'' replied my master with
considerable exaggeration.

``Is that some sort of posthumous Buddhist name like Accidental Child?''
inquires Waverhouse in his usual irrelevant style.

``Is there then someone called Accidental Child?''

``No, of course there isn't, but I take it that you're working on
something like that.''

``I don't think Accidental Child is anyone I know. But
Mr.~the-late-and-sainted Natural Man is a person of your own
acquaintance.''

``Who on earth could get a name like that?''

``It's Sorosaki. After he graduated from the University, he took a
post-graduate course involving study of the `theory of infinity.' But he
over-worked, got peritonitis, and died of it. Sorosaki happened to be a
very close friend of mine.''

``All right, so he was your very close friend. I'm far from criticizing
that fact. But who was responsible for converting Sorosaki into
Mr.~the-late-and-sainted Natural Man?''

``Me. I created that name. For there is really nothing more philistine
than the posthumous names conferred by Buddhist priests.'' My master
boasts as if his nomination of Natural Man were a feat of artistry.

``Anyway, let's see the epitaph,'' says Waverhouse laughingly. He picks
up my master's manuscript and reads it out aloud. ``Eh\ldots{} `Born
into infinity, studied infinity, and died into infinity.
Mr.~the-late-and-sainted Natural Man. Infinity.' I see. This is fine.
Quite appropriate for poor old Sorosaki.''

``Good, isn't it?'' says my master obviously very pleased.

``You should have this epitaph engraved on a weight-stone for pickles
and then leave it at the back of the main hall of some temple for the
practice-benefit of passing weight lifters. It's good. It's most
artistic. Mr.~the-late-and-sainted may now well rest in peace.''

``Actually, I'm thinking of doing just that,'' answers my master quite
seriously. ``But you'll have to excuse me,'' he went on, ``I won't be
long. Just play with the cat. Don't go away.'' And my master departed
like the wind without even waiting for Waverhouse to answer.

Being thus unexpectedly required to entertain the culture-vulture
Waverhouse, I cannot very well maintain my sour attitude. Accordingly, I
mew at him encouragingly and sidle up on to his knees. ``Hello,'' says
Waverhouse, ``you've grown distinctly chubby. Let's take a look at
you.''

Grabbing me impolitely by the scruff of my neck he hangs me up in
midair. ``Cats like you that let their hind legs dangle are cats that
catch no mice\ldots{} Tell me,'' he said, turning to my master's wife in
the next room, ``has he ever caught anything?''

``Far from catching so much as a single mouse, he eats rice-cakes and
then dances.'' The lady of the house unexpectedly probes my old wound,
which embarrassed me. Especially when Waverhouse still held me in midair
like a circus-performer.

``Indeed, with such a face, it's not surprising that he dances. Do you
know, this cat possesses a truly insidious physionomy. He looks like one
of those goblin-cats illustrated in the old storybooks.'' Waverhouse,
babbling whatever comes into his head, tries to make conversation with
the mistress. She reluctantly interrupts her sewing and comes into the
room.

``I do apologize. You must be bored. He won't be long now.'' And she
poured fresh tea for him.

``I wonder where he's gone.''

``Heaven only knows. He never explains where he's going. Probably to see
his doctor.''

``You mean Dr.~Amaki? What a misfortune for Amaki to be involved with
such a patient.''

Perhaps finding this comment difficult to answer, she answers briefly:

``Well, yes.''

Waverhouse takes not the slightest notice, but goes on to ask, ``How is
he lately? Is his weak stomach any better?''

``It's impossible to say whether it's better or worse. However carefully
Dr.~Amaki may look after him, I don't see how his health can ever
improve if he continues to consume such vast quantities of jam.'' She
thus works off on Waverhouse her earlier grumblings to my master.

``Does he eat all that much jam? It sounds like a child.''

``And not just jam. He's recently taken to guzzling grated radish on the
grounds that it's a sovereign cure for dyspepsia.''

``You surprise me,'' marvels Waverhouse.

``It all began when he read in some rag that grated radish contains
diastase.''

``I see. I suppose he reckons that grated radish will repair the ravages
of jam. It's certainly an ingenious equation.'' Waverhouse seems vastly
diverted by her recital of complaint.

``Then only the other day he forced some on the baby.''

``He made the baby eat jam?''

``No, grated radish! Would you believe it? He said, `Come here, my
little babykin, father'll give you something good\ldots{}' Whenever,
once in a rare while, he shows affection for the children, he always
does remarkably silly things. A few days ago he put our second daughter
on top of a chest of drawers.''

``What ingenious scheme was that?'' Waverhouse looks to discover
ingenuities in everything.

``There was no question of any ingenious scheme. He just wanted the
child to make the jump when it's quite obvious that a little girl of
three or four is incapable of such tomboy feats.''

``I see. Yes, that proposal does indeed seem somewhat lacking in
ingenuity. Still, he's a good man without an ill wish in his heart.''

``Do you think that I could bear it if, on top of everything else, he
were ill-natured?'' She seems in uncommonly high spirits.

``Surely you don't have cause for such vehement complaint? To be as
comfortably off as you are is, after all, the best way to be. Your
husband neither leads the fast life nor squanders money on dandified
clothing. He's a born family man of quiet taste.'' Waverhouse fairly
lets himself go in unaccustomed laud of an unknown way of life.

``On the contrary, he's not at all like that\ldots{}''

``Indeed? So he has secret vices? Well, one cannot be too careful in
this world.'' Waverhouse offers a nonchalantly fluffy comment.

``He has no secret vices, but he is totally abandoned in the way he buys
book after book, never to read a single one. I wouldn't mind if he used
his head and bought in moderation, but no. Whenever the mood takes him,
he ambles off to the biggest bookshop in the city and brings back home
as many books as chance to catch his fancy. Then, at the end of the
month, he adopts an attitude of complete detachment. At the end of last
year, for instance, I had a terrible time coping with the bill that had
been accumulating month after month.''

``It doesn't matter that he should bring home however many books he may
like. If, when the bill collector comes, you just say that you'll pay
some other time, he'll go away.''

``But one cannot put things off indefinitely.'' She looks cast down.

``Then you should explain the matter to your husband and ask him to cut
down expenditure on books.''

``And do you really believe he would listen to me? Why, only the other
day, he said, `You are so unlike a scholar's wife: you lack the least
understanding of the value of books. Listen carefully to this story from
ancient Rome. It will give you beneficial guidance for your future
conduct.'\,''

``That sounds interesting. What sort of story was it?'' Waverhouse
becomes enthusiastic, though he appears less sympathetic to her
predicament than prompted by sheer curiosity.

``It seems there was in ancient Rome a king named Tarukin.''

``Tarukin? That sounds odd in Japanese.''

``I can never remember the names of foreigners. It's all too difficult.
Maybe he was a barrel of gold. He was, at any rate, the seventh king of
Rome.''

``Really? The seventh barrel of gold certainly sounds queer. But, tell
me, what then happened to this seventh Tarukin.''

``You mustn't tease me like that. You quite embarrass me. If you know
this king's true name, you should teach me it. Your attitude,'' she
snaps at him, ``is really most unkind.''

``I tease you? I wouldn't dream of doing such an unkind thing. It was
simply that the seventh barrel of gold sounded so wonderful. Let's
see\ldots{} a Roman, the seventh king\ldots{} I can't be absolutely
certain but I rather think it must have been Tarquinius Superbus,
Tarquin the Proud. Well, it doesn't really matter who it was. What did
this monarch do?''

``I understand that some woman, Sibyl by name, went to this king with
nine books and invited him to buy them.''

``I see.''

``When the king asked her how much she wanted, she stated a very high
price, so high that the king asked for a modest reduction. Whereupon the
woman threw three of the nine books into the fire where they were
quickly burnt to ashes.''

``What a pity!''

``The books were said to contain prophecies, predictions, things like
that of which there was no other record anywhere.''

``Really?''

``The king, believing that six books were bound to be cheaper than nine,
asked the price of the remaining volumes. The price proved to be exactly
the same; not one penny less. When the king complained of this
outrageous development, the women threw another three books into the
fire. The king apparently still hankered for the books and he
accordingly asked the price of the last three left. The woman again
demanded the same price as she had asked for the original nine. Nine
books had shrunk to six, and then to three, but the price remained
unaltered even by a farthing. Suspecting that any attempt to bargain
would merely lead the woman to pitch the last three volumes into the
flames, the king bought them at the original staggering price. My
husband appeared confident that, having heard this story, I would begin
to appreciate the value of books, but I don't at all see what it is that
I'm supposed to have learnt to appreciate.''

Having thus stated her own position, she as good as challenges
Waverhouse to contravert her. Even the resourceful Waverhouse seems to
be at a loss. He draws a handkerchief from the sleeve of his kimono and
tempts me to play with it. Then, in a loud voice as if an idea had
suddenly struck him, he remarked, ``But you know, Mrs.~Sneaze, it is
precisely because your husband buys so many books and fills his head
with wild notions that he is occasionally mentioned as a scholar, or
something of that sort. Only the other day a comment on your husband
appeared in a literary magazine.''

``Really?'' She turns around. After all, it's only natural that his wife
should feel anxiety about comments on my master.

``What did it say?''

``Oh, only a few lines. It said that Mr.~Sneaze's prose was like a cloud
that passes in the sky, like water flowing in a stream.''

``Is that,'' she asks smiling, ``all that it said?''

``Well, it also said `it vanishes as soon as it appears and, when it
vanishes, it is forever forgetful to return.'\,''

The lady of the house looks puzzled and asks anxiously ``Was that
praise?''

``Well, yes, praise of a sort,'' says Waverhouse coolly as he jiggles
his handkerchief in front of me.

``Since books are essential to his work, I suppose one shouldn't
complain, but his eccentricity is so pronounced that\ldots{}''

Waverhouse assumes that she's adopting a new line of attack. ``True,''
he interrupts, ``he is a little eccentric, but any man who pursues
learning tends to get like that.'' His answer, excellently noncommittal,
contrives to combine ingratiation and special pleading.

``The other day, when he had to go somewhere soon after he got home from
school, he found it too troublesome to change his clothes. So do you
know, he sat down on his low desk without even taking off his overcoat
and ate his dinner just as he was. He had his tray put on the footwarmer
while I sat on the floor holding the rice container. It was really very
funny\ldots{}''

``It sounds like the old-time custom when generals sat down to identify
the severed heads of enemies killed in battle. But that would be quite
typical of Mr.~Sneaze. At any rate he's never boringly conventional.''

Waverhouse offers a somewhat strained compliment.

``A woman cannot say what's conventional or unconventional, but I do
think his conduct is often unduly odd.''

``Still, that's better than being conventional.'' As Waverhouse moves
firmly to the support of my master, her dissatisfaction deepens.

``People are always saying this or that is conventional, but would you
please tell what makes a thing conventional?'' Adopting a defiant
attitude, she demands a definition of conventionality.

``Conventional? When one says something is conventional\ldots{} It's a
bit difficult to explain\ldots{}''

``If it's so vague a thing, surely there's nothing wrong with being
conventional.'' She begins to corner Waverhouse with typically feminine
logic.

``No, it isn't vague, it's perfectly clear-cut. But it's hard to
explain.''

``I expect you call everything you don't like conventional.'' Though
totally uncalculated, her words land smack on target. Waverhouse is now
indeed cornered and can no longer dodge defining the conventional.

``I'll give you an example. A conventional man is one who would yearn
after a girl of sixteen or eighteen but, sunk in silence, never do
anything about it; a man who, whenever the weather's fine, would do no
more than stroll along the banks of the Sumida taking, of course, a
flask of saké with him.''

``Are there really such people?'' Since she cannot make heads or tails
of the twaddle vouchsafed by Waverhouse, she begins to abandon her
position, which she finally surrenders by saying, ``It's all so
complicated that it's really quite beyond me.''

``You think that complicated? Imagine fitting the head of Major
Pendennis onto Bakin's torso, wrapping it up and leaving it all for one
or two years exposed to European air.''

``Would that produce a conventional man?'' Waverhouse offers no reply
but merely laughs.

``In fact it could be produced without going to quite so much trouble.
If you added a shop assistant from a leading store to any middle school
student and divided that sum by two, then indeed you'd have a fine
example of a conventional man.''

``Do you really think so?'' She looks puzzled but certainly unconvinced.

``Are you still here?'' My master sits himself down on the floor beside
Waverhouse. We had not noticed his return.

``\,`Still here' is a bit hard. You said you wouldn't be long and you
yourself invited me to wait for you.''

``You see, he's always like that,'' remarks the lady of the house
leaning toward Waverhouse.

``While you were away I heard all sorts of tales about you.''

``The trouble with women is that they talk too much. It would be good if
human beings would keep as silent as this cat.'' And the master strokes
my head.

``I hear you've been cramming grated radish into the baby.''

``Hum,'' says my master and laughs. He then added ``Talking of the baby,
modern babies are quite intelligent. Since that time when I gave our
baby grated radish, if you ask him `where is the hot place?' he
invariably sticks out his tongue. Isn't it strange?''

``You sound as if you were teaching tricks to a dog. It's positively
cruel. By the way, Coldmoon ought to have arrived by now.''

``Is Coldmoon coming?'' asks my master in a puzzled voice.

``Yes. I sent him a postcard telling him to be here not later than one
o'clock.''

``How very like you! Without even asking us if it happened to be
convenient. What's the idea of asking Coldmoon here?''

``It's not really my idea, but Coldmoon's own request. It seems he is
going to give a lecture to the Society of Physical Science. He said he
needed to rehearse his speech and asked me to listen to it. Well, I
thought it would be obliging to let you hear it, too. Accordingly, I
suggested he should come to your house. Which should be quite convenient
since you are a man of leisure. I know you never have any engagements.
You'd do well to listen.'' Waverhouse thinks he knows how to handle the
situation.

``I wouldn't understand a lecture on physical science,'' says my master
in a voice betraying his vexation at his friend's high-handed action.

``On the contrary, his subject is no such dry-as-dust matter as, for
example, the magnetized nozzle. The transcendentally extraordinary
subject of his discourse is `The Mechanics of Hanging.' Which should be
worth listening to.''

``Inasmuch as you once only just failed to hang yourself, I can
understand your interest in the subject, but I'm\ldots{}''

``\ldots{}The man who got cold shivers over going to the theatre, so you
cannot expect not to listen to it.'' Waverhouse interjects one of his
usual flippant remarks and Mrs.~Sneaze laughs. Glancing back at her
husband, she goes off into the next room. My master, keeping silent,
strokes my head. This time, for once, he stroked me with delicious
gentleness.

Some seven minutes later in comes the anticipated Coldmoon. Since he's
due to give his lecture this same evening, he is not wearing his usual
get-up. In a fine frock-coat and with a high and exceedingly white clean
collar, he looks twenty per cent more handsome than himself. ``Sorry to
be late.'' He greets his two seated friends with perfect composure.

``It's ages that we've now been waiting for you. So we'd like you to
start right away. Wouldn't we?'' says Waverhouse, turning to look at my
master. The latter, thus forced to respond, somewhat reluctantly says,

``Hmm.'' But Coldmoon's in no hurry. He remarks, ``I think I'll have a
glass of water, please.''

``I see you are going to do it in real style. You'll be calling next for
a round of applause.'' Waverhouse, but he alone, seems to be enjoying
himself.

Coldmoon produced his text from an inside pocket and observed,

``Since it is the established practice, may I say I would welcome
criticism.'' That invitation made, he at last begins to deliver his
lecture.

``Hanging as a death penalty appears to have originated among the
Anglo-Saxons. Previously, in ancient times, hanging was mainly a method
of committing suicide. I understand that among the Hebrews it was
customary to execute criminals by stoning them to death. Study of the
Old Testament reveals that the word `hanging' is there used to mean
`suspending a criminal's body after death for wild beasts and birds of
prey to devour it.' According to Herodotus, it would seem that the Jews,
even before they departed from Egypt, abominated the mere thought that
their dead bodies might be left exposed at night. The Egyptians used to
behead a criminal, nail the torso to a cross and leave it exposed during
the night. The Persians\ldots{}''

``Steady on, Coldmoon,'' Waverhouse interrupts. ``You seem to be
drifting farther and farther away from the subject of hanging. Do you
think that wise?''

``Please be patient. I am just coming to the main subject. Now, with
respect to the Persians. They, too, seemed to have used crucifixion as a
method of criminal execution. However, whether the nailing took place
while the criminal was alive or simply after his death is not
incontrovertibly established.''

``Who cares? Such details are really of little importance,'' yawned my
master as from boredom.

``There are still many matters of which I'd like to inform you but, as
it will perhaps prove tedious for you\ldots{}''

``\,`As it might prove' would sound better than `as it will perhaps
prove.' What d'you think, Sneaze?'' Waverhouse starts carping again but
my master answers coldly, ``What difference could it make?''

``I have now come to the main subject, and will accordingly recite my
piece.''

``A storyteller `recites a piece.' An orator should use more elegant
diction.'' Waverhouse again interrupts.

``If to `recite my piece' sounds vulgar, what words should I use?'' asks
Coldmoon in a voice that showed he was somewhat nettled.

``It is never clear, when one is dealing with Waverhouse, whether he's
listening or interrupting. Pay no attention to his heckling, Coldmoon,
just keep going.'' My master seeks to find a way through the difficulty
as quickly as possible.

``So, having made your indignant recitation, now I suppose you've found
the willow tree?'' With a pun on a little known haiku, Waverhouse, as
usual, comes up with something odd. Coldmoon, in spite of himself, broke
into laughter.

``My researches reveal that the first account of the employment of
hanging as a deliberate means of execution occurs in the Odyssey, volume
twenty-two. The relevant passage records how Telemachus arranged the
execution by hanging of Penelope's twelve ladies-in-waiting. I could
read the passage aloud in its original Greek, but, since such an act
might be regarded as an affectation, I will refrain from doing so. You
will, however, find the passage between lines 465 and 473.''

``You'd better cut out all that Hellenic stuff. It sounds as if you are
just showing off your knowledge of Greek. What do you think, Sneaze?''

``On that point, I agree with you. It would be more modest, altogether
an improvement, to avoid such ostentation.'' Quite unusually my master
immediately sides with Waverhouse. The reason is, of course, that
neither can read a word of Greek.

``Very well, I will this evening omit those references. And now I will
recite\ldots{} that is to say, I will now continue. Let us consider,
then, how a hanging is actually carried out. One can envisage two
methods. The first method is that adopted by Telemachus who, with the
help of Eumaeus and Philoetios, tied one end of a rope to the top of a
pillar: next, having made several loose loops in the rope, he forced a
woman's head through each such loop, and finally hauled up hard on the
other end of the rope.''

``In short, he had the women dangling in a row like shirts hung out at a
laundry. Right?''

``Exactly. Now the second method is, as in the first case, to tie one
end of a rope to the top of a pillar and similarly to secure the other
end of the rope somewhere high up on the ceiling. Thereafter, several
other short ropes are attached to the main rope, and in each of these
subsidiary ropes a slip-knot is then tied. The women's heads are then
inserted in the slipknots. The idea is that at the crucial moment you
remove the stools on which the women have been stood.''

``They would then look something like those ball-shaped paper-lanterns
one sometimes sees suspended from the end-tips of rope curtains,
wouldn't they?'' hazarded Waverhouse.

``That I cannot say,'' answered Coldmoon cautiously. ``I have never seen
any such ball as a paper-lantern-ball, but if such balls exist, the
resemblance may be just. Now, the first method as described in the
Odyssey is, in fact, mechanically impossible; and I shall proceed, for
your benefit, to substantiate that statement.''

``How interesting,'' says Waverhouse.

``Indeed, most interesting,'' adds my master.

``Let us suppose that the women are to be hanged at intervals of an
equal distance, and that the rope between the two women nearest the
ground stretches out horizontally, right? Now α1, α2 up to α6 become the
angles between the rope and the horizon. T1, T2, and so on up to T6
represent the force exerted on each section of the rope, so that T7 = X
is the force exerted on the lowest part of the rope. W is, of course,
the weight of the women. So far so good. Are you with me?''

My master and Waverhouse exchange glances and say, ``Yes, more or
less.'' I need hardly point out that the value of this ``more or less''
is singular to Waverhouse and my master. It could possibly have a
different value for other people.

``Well, in accordance with the theory of averages as applied to the
polygon, a theory with which you must of course be well acquainted, the
following twelve equations can, in this particular case, be established:
T1 cos αl=T2 cos α2\ldots{}\ldots{}(1), T2 cos α2=T3 cos
α3\ldots{}\ldots{}(2).''

``I think that's enough of the equations,'' my master irresponsibly
remarks.

``But these equations are the very essence of my lecture.'' Coldmoon
really seems reluctant to be parted from them.

``In that case, let's hear those particular parts of its very essence at
some other time.'' Waverhouse, too, seems out of his depth.

``But if I omit the full detail of the equations, it becomes impossible
to substantiate the mechanical studies to which I have devoted so much
effort\ldots{}''

``Oh, never mind that. Cut them all out,'' came the cold-blooded comment
of my master.

``That's most unreasonable. However, since you insist, I will omit
them.''

``That's good,'' says Waverhouse, unexpectedly clapping his hands.

``Now we come to England where, in Beowulf, we find the word `gallows':
that is to say `galga.' It follows that hanging as a penalty must have
been in use as early as the period with which the book is concerned.
According to Blackstone, a convicted person who is not killed at his
first hanging by reason of some fault in the rope should simply be
hanged again. But, oddly enough, one finds it stated in The Vision of
Piers Plowman that even a murderer should not be strung up twice. I do
not know which statement is correct, but there are many melancholy
instances of victims failing to be killed outright. In 1786 the
authorities attempted to hang a notorious villain named Fitzgerald, but
when the stool was removed, by some strange chance the rope broke. At
the next attempt the rope proved so long that his legs touched ground
and he again survived. In the end, at the third attempt, he was enabled
to die with the help of the spectators.''

``Well, well,'' says Waverhouse becoming, as was only to be expected,
re-enlivened.

``A true thanatophile.'' Even my master shows signs of jollity.

``There is one other interesting fact. A hanged person grows taller by
about an inch. This is perfectly true. Doctors have measured it.''

``That's a novel notion. How about it, Sneaze?'' says Waverhouse turning
to my master. ``Try getting hanged. If you were an inch taller, you
might acquire the appearance of an ordinary human being.'' The reply,
however, was delivered with an unexpected gravity.

``Tell me, Coldmoon, is there any chance of surviving that process of
extension by one inch?''

``Absolutely none. The point is that it is the spinal cord which gets
stretched in hanging. It's more a matter of breaking than of growing
taller.''

``In that case, I won't try.'' My master abandons hope.

There was still a good deal of the lecture left to deliver and Coldmoon
had clearly been anxious to deal with the question of the physiological
function of hanging. But Waverhouse made so many and such
capriciously-phrased interjections and my master yawned so rudely and so
frequently that Coldmoon finally broke off his rehearsal in mid-flow and
took his leave. I cannot tell you what oratorical triumphs he achieved,
still less what gestures he employed that evening, because the lecture
took place miles away from me.

A few days passed uneventfully by. Then, one day about two in the
afternoon, Waverhouse dropped in with his usual casual manners and
looking as totally uninhibited as his own concept of the ``Accidental
Child.'' The minute he sat down he asked abruptly, ``Have you heard
about Beauchamp Blowlamp and the Takanawa Incident?'' He spoke
excitedly, in a tone of voice appropriate to an announcement of the fall
of Port Arthur.

``No, I haven't seen him lately.'' My master is his usual cheerless
self.

``I've come today, although I'm busy, especially to inform you of the
frightful blunder which Beauchamp has committed.''

``You're exaggerating again. Indeed you're quite impossible.''

``Impossible, never: improbable, perhaps. I must ask you to make a
distinction on this point, for it affects my honor.''

``It's the same thing,'' replied my master assuming an air of provoking
indifference. He is the very image of a Mr.~the-late-and-sainted Natural
Man.

``Last Sunday, Beauchamp went to the Sengaku Temple at Takanawa, which
was silly in this cold weather, especially when to make such a visit
nowadays stamps one as a country bumpkin out to see the sights.''

``But Beauchamp's his own master. You've no right to stop him going.''

``True, I haven't got the right, so let's not bother about that. The
point is that the temple yard contains a showroom displaying relics of
the forty-seven ronin. Do you know it?''

``N-no.''

``You don't? But surely you've been to the Temple?''

``No.''

``Well, I am surprised. No wonder you so ardently defended Beauchamp.
But it's positively shameful that a citizen of Tokyo should never have
visited the Sengaku Temple.''

``One can contrive to teach without trailing out to the ends of the
city.'' My master grows more and more like his blessed Natural Man.

``All right. Anyway, Beauchamp was examining the relics when a married
couple, Germans as it happened, entered the showroom. They began by
asking him questions in Japanese, but, as you know, Beauchamp is always
aching to practice his German so he naturally responded by rattling off
a few words in that language. Apparently he did it rather well. Indeed,
when one thinks back over the whole deplorable incident, his very
fluency was the root cause of the trouble.''

``Well, what happened?'' My master finally succumbs.

``The Germans pointed out a gold-lacquered pill-box which had belonged
to Otaka Gengo and, saying they wished to buy it, asked Beauchamp if the
object were for sale. Beauchamp's reply was not uninteresting. He said
such a purchase would be quite impossible because all Japanese people
were true gentlemen of the sternest integrity. Up to that point he was
doing fine. However, the Germans, thinking that they'd found a useful
interpreter, thereupon deluged him with questions.''

``About what?''

``That's just it. If he had understood their questions, there would have
been no trouble. But you see he was subjected to floods of such
questions, all delivered in rapid German, and he simply couldn't make
head or tail of what was being asked. When at last he chanced to
understand part of their outpourings, it was something about a fireman's
axe or a mallet---some word he couldn't translate---so again, naturally,
he was completely at a loss how to reply.''

``That I can well imagine,'' sympathizes my master, thinking of his own
difficulties as a teacher.

``Idle onlookers soon began to gather around and eventually Beauchamp
and the Germans were totally surrounded by staring eyes. In his
confusion Beauchamp fell to blushing. In contrast to his earlier
self-confidence he was now at his wit's end.''

``How did it all turn out?''

``In the end Beauchamp could stand it no longer, shouted sainara in
Japanese and came rushing home. I pointed out to him that sainara was an
odd phrase to use and inquired whether, in his home-district, people
used sainara rather than sayonara. He replied they would say sayonara
but, since he was talking to Europeans, he had used sainara in order to
maintain harmony. I must say I was much impressed to find him a man
mindful of harmony even when in difficulties.''

``So that's the bit about sainara. What did the Europeans do?''

``I hear that the Europeans looked utterly flabbergasted.'' And
Waverhouse gave vent to laughter. ``Interesting, eh?''

``Frankly, no. I really can't find anything particularly interesting in
your story. But that you should have come here specially to tell me the
tale, that I do find much more interesting.'' My master taps his
cigarette's ash into the brazier. Just at that moment the bell on the
lattice door at the entrance rang with an alarming loudness, and a
piercing woman's voice declared, ``Excuse me.'' Waverhouse and my master
look at each other in silence.

Even while I am thinking that it is unusual for my master's house to
have a female visitor, the owner of that piercing voice enters the room.

She is wearing two layers of silk crepe kimono, and looks to be a little
over forty. Her forelock towers up above the bald expanse of her brow
like the wall of a dyke and sticks out toward heaven for easily one half
the length of her face. Her eyes, set at an angle like a road cut
through a mountain, slant up symmetrically in straight lines. I speak,
of course, metaphorically. Her eyes, in fact, are even narrower than
those of a whale. But her nose is exceedingly large. It gives the
impression that it has been stolen from someone else and thereafter
fastened in the center of her face. It is as if a large, stone lantern
from some major shrine had been moved to a tiny ten-square-meter garden.

It certainly asserts its own importance, but yet looks out of place. It
could almost be termed hooked: it begins by jutting sharply out, but
then, halfway along its length, it suddenly turns shy so that its tip,
bereft of the original vigour, hangs limply down to peer into the mouth
below.

Her nose is such that, when she speaks, it is the nose rather than the
mouth which seems to be in action. Indeed, in homage to the enormity of
that organ, I shall refer hence forward to its owner as Madam Conk.

When the ceremonials of her self-introduction had been completed, she
glared around the room and remarked, ``What a nice house.''

``What a liar,'' says my master to himself, and concentrates upon his
smoking. Waverhouse studies the ceiling. ``Tell me,'' he says, ``is that
odd pattern the result of a rain leak or is it inherent in the grain of
the wood?''

``Rain leak, naturally'' replies my master. To which Waverhouse coolly
answers, ``Wonderful.''

Madam Conk clearly regards them as unsociable persons and boils quietly
with suppressed annoyance. For a time the three of them just sit there
in a triangle without saying a word.

``I've come to ask you about a certain matter.'' Madam Conk starts up
again.

``Ah.'' My master's response lacks warmth.

Madam Conk, dissatisfied with this development, bestirs herself again.
``I live nearby. In fact, at the residence on the corner of the block
across the road.''

``That large house in the European style, the one with a godown? Ah,
yes. Of course. Have I not seen `Goldfield' on the nameplate of that
dwelling?'' My master, at last, seems ready to take cognizance of
Goldfield's European house and his incorporated godown, but his attitude
toward Madam Conk displays no deepening of respect.

``Of course my husband should call upon you and seek your valued advice,
but he is always so busy with his company affairs.'' She puts on a
``that ought to shift them'' face, but my master remains entirely
unimpressed. He is, in fact, displeased by her manner of speaking,
finding it too direct in a woman met for the first time. ``And not of
just one company either. He is connected with two or three of them and
is a director of them all, as I expect you already know.'' She looks as
if saying to herself, ``Now surely he should feel small.'' In point of
fact, the master of this house behaves most humbly toward anyone who
happens to be a doctor or a professor, but, oddly enough, he offers
scant respect toward businessmen. He considers a middle school teacher
to be a more elevated person than any businessman. Even if he doesn't
really believe this, he is quite resigned, being of an unadaptable
nature, to the fact that he can never hope to be smiled upon by
businessmen or millionaires.

For he feels nothing but indifference toward any person, no matter how
rich or influential, from whom he has ceased to hope for benefits. He
consequently pays not the faintest attention to anything extraneous to
the society of scholars, and is almost actively disinterested in the
goings-on of the business world. Had he even the vaguest knowledge of
the activities of businessmen, he still could never muster the slightest
feeling of awe or respect for such abysmal persons. While, for her part,
Madam Conk could never stretch her imagination to the point of
considering that any being so eccentric as my master could actually
exist, that any corner of the world might harbor such an oddity. Her
experience has included meetings with many people and invariably, as
soon as she declares that she is wife to Goldfield, their attitude
towards her never fails immediately to alter. At any party whatsoever
and no matter how lofty the social standing of any man before whom she
happens to find herself, she has always found that Mrs.~Goldfield is
eminently acceptable. How then could she fail to impress such an obscure
old teacher? She had expected that the mere mention of the fact that her
house was the corner residence of the opposite block would startle my
master even before she added information about Mr.~Goldfield's notable
activities in the world of business.

``Do you know anyone called Goldfield?'' my master inquires of
Waverhouse with the utmost nonchalance.

``Of course I know him. He's a friend of my uncle. Only the other day he
was present at our garden party.'' Waverhouse answers in a serious
manner.

``Really?'' said my master. ``And who, may I ask, is your uncle?''

``Baron Makiyama,'' replied Waverhouse in even graver tones. My master
is obviously about to say something, but before he can bring himself to
words, Madam Conk turns abruptly toward Waverhouse and subjects him to a
piercing stare. Waverhouse, secure in a kimono of the finest silk,
remains entirely unperturbed.

``Oh, you are Baron Makiyama's\ldots{} That I didn't know. I hope you'll
excuse me\ldots{} I've heard so much about Baron Makiyama from my
husband. He tells me that the Baron has always been so helpful\ldots{}''
Madam Conk's manner of speech has suddenly become polite. She even bows.

``Ah yes,'' observes Waverhouse who is inwardly laughing. My master,
quite astonished, watches the two in silence.

``I understand he has even troubled the Baron about our daughter's
marriage\ldots{}''

``Has he indeed?'' exclaims Waverhouse as if surprised. Even Waverhouse
seems somewhat taken aback by this unexpected development.

``We are, in fact, receiving proposal after proposal in respect of
marriage to our daughter. They flood in from all over the place. You
will appreciate that, having to think seriously of our social position,
we cannot rashly marry off our daughter to just anyone\ldots{}''

``Quite so.'' Waverhouse feels relieved.

``I have, in point of fact, made this visit precisely to raise with you
a question about this marriage matter.'' Madam Conk turns back to my
master and reverts to her earlier vulgar style of speech. ``I hear that
a certain Avalon Coldmoon pays you frequent visits. What sort of a man
is he?''

``Why do you want to know about Coldmoon?'' replies my master in a
manner revealing his displeasure.

``Perhaps it is in connection with your daughter's marriage that you
wish to know something about the character of Coldmoon,'' puts in
Waverhouse tactfully.

``If you could tell me about his character, it would indeed be
helpful.''

``Then is it that you want to give your daughter in marriage to
Coldmoon?''

``It's not a question of my wanting to give her.'' Madam Conk
immediately squashes my master. ``Since there will be innumerable
proposals, we couldn't care less if he doesn't marry her.''

``In that case, you don't need any information about Coldmoon,'' my
master replies with matching heat.

``But you've no reason to withhold information.'' Madam Conk adopts an
almost defiant attitude.

Waverhouse, sitting between the two and holding his silver pipe as if it
were an umpire's instrument of office, is secretly beside himself with
glee. His gloating heart urges them on to yet more extravagant
exchanges.

``Tell me, did Coldmoon actually say he wanted to marry her?'' My master
fires a broadside pointblank.

``He didn't actually say he wanted to, but\ldots{}''

``You just think it likely that he might want to?'' My master seems to
have realized that broadsides are best in dealing with this woman.

``The matter is not yet so far advanced, but\ldots{} well, I don't think
Mr.~Coldmoon is wholly averse to the idea.'' Madam Conk rallies well in
her extremity.

``Is there any concrete evidence whatsoever that Coldmoon is enamored of
this daughter of yours?'' My master, as if to say, ``now answer me if
you can,'' sticks out his chest belligerently.

``Well, more or less, yes.'' This time my master's militance has failed
in its effect. Waverhouse has hitherto been so delighted with his
self-appointed role of umpire that he has simply sat and watched the
scrap, but now his curiosity seems suddenly to have been aroused. He
puts down the pipe and leans forward. ``Has Coldmoon sent your daughter
a love letter? What fun! One more new event since the New Year and, at
that, a splendid subject for debate.'' Waverhouse alone is pleased.

``Not a love letter. Something much more ardent than that. Are you two
really so much in the dark?'' Madam Conk adopts a disbelieving attitude.

``Are you aware of anything?'' My master, looking nonplussed, addresses
himself to Waverhouse.

Waverhouse takes refuge in banter. ``I know nothing. If anyone should
know, it would be you.'' His reaction is disappointingly modest.

``But the two of you know all about it,'' Madam Conk triumphs over both
of them.

``Oh!'' The sound expressed their simultaneous astonishment.

``In case you've forgotten, let me remind you of what happened. At the
end of last year Mr.~Coldmoon went to a concert at the Abe residence in
Mukōjima, right? That evening, on his way home, something happened at
Azuma Bridge. You remember? I won't repeat the details since that might
compromise the person in question, but what I've said is surely proof
enough. What do you think?'' She sits bolt-upright with her
diamond-ringed fingers in her lap. Her magnificent nose looks more
resplendent than ever, so much so that Waverhouse and my master seem
practically obliterated.

My master, naturally, but Waverhouse also, appear dumbfounded by this
surprise attack. For a while they just sit there in bewilderment, like
patients whose fits of ague have suddenly ceased. But as the first shock
of their astonishment subsides and they come slowly back to normality,
their sense of humor irrepressibly asserts itself and they burst into
gales of laughter. Madam Conk, baulked in her expectations and,
ill-prepared for this reaction of rude laughing, glares at both of them.

``Was that your daughter? Isn't it wonderful! You're quite right. Indeed
Coldmoon must be mad about her. I say, Sneaze, there's no point now in
trying to keep it secret. Let's make a clean breast of everything.''

My master just says ``Hum.''

``There's certainly no point in your trying to keep it secret. The cat's
already out of the bag.'' Madam Conk is once more cock-a-hoop.

``Yes, indeed, we're cornered. We'll have to make a true statement on
everything concerning Coldmoon for this lady's information. Sneaze!
you're the host here. Pull yourself together, man. Stop grinning like
that or we'll never get this business sorted out. It's extraordinary.
Secretiveness is a most mysterious matter. However well one guards a
secret, sooner or later it's bound to come out. Indeed, when you come to
think of it, it really is most extraordinary. Tell us, Mrs.~Goldfield,
how did you ever discover this secret? I am truly amazed.'' Waverhouse
rattles on.

``I've a nose for these things.'' Madam Conk declares with some
self-satisfaction.

``You must indeed be very well informed. Who on earth has told you about
this matter?''

``The wife of the rickshawman who lives just there at the back.''

``Do you mean that man who owns that vile black cat?'' My master is
wide-eyed.

``Yes, your Mr.~Coldmoon has cost me a pretty penny. Every time he comes
here I want to know what he talks about, so I've arranged for the wife
of the rickshawman to learn what happens and to report it all to me.''

``But that's terrible!'' My master raises his voice.

``Don't worry, I don't give a damn what you do or say. I'm not in the
least concerned with you, only with Mr.~Coldmoon.''

``Whether with Coldmoon or with anyone else\ldots{} Really, that
rickshaw woman is a quite disgusting creature.'' My master begins to get
angry.

``But surely she is free to stand outside your hedge. If you don't want
your conversations overheard, you should either talk less loudly or live
in a larger house.'' Madam Conk is clearly not the least ashamed of
herself. ``And that's not my only source. I've also heard a deal of
stuff from the Mistress of the two-stringed harp.''

``You mean about Coldmoon?''

``Not solely about Coldmoon.'' This sounds menacing but, far from
retreating in embarrassment, my master retorts. ``That woman gives
herself such airs. Acting as though she and she alone were the only
person of any standing in this neighborhood. A vain, an idiotic
fellow\ldots{}''

``Pardon me! It's a woman you're describing. A fellow, did you say?
Believe me, you're talking out of the back of your neck.'' Her language
more and more betrays her vulgar origin. Indeed, it now appears as if
she has only come in order to pick a quarrel. But Waverhouse, typically,
just sits listening to the quarrel as if it were being conducted for his
amusement. Indeed, he looks like a Chinese sage at a cockfight: cool and
above it all.

My master at last realizes that he can never match Madam Conk in the
exchange of scurrilities, and he lapses into a forced silence. But
eventually a bright idea occurs to him.

``You've been speaking as though it were Coldmoon who was besotted with
your daughter, but from what I've heard, the situation is quite
different. Isn't that so, Waverhouse?''

``Certainly. As we heard it, your daughter fell ill and then, we
understand, began babbling in delirium.''

``No.~You've got it all wrong.'' Madam Conk gives the lie direct.

``But Coldmoon undoubtedly said that that was what he had been told by
Dr.~O's wife.''

``That was our trap. We'd asked the Doctor's wife to play that trick on
Coldmoon precisely in order to see how he'd react.''

``Did the doctor's wife agree to this deception in full knowledge that
it was a trick?''

``Yes. Of course we couldn't expect her to help us purely for
affection's sake. As I've said, we've had to lay out a very pretty penny
on one thing and another.''

``You are quite determined to impose yourself upon us and quiz us in
detail about Coldmoon, eh?'' Even Waverhouse seems to be getting annoyed
for he uses some sharpish turns of phrase quite unlike his usual manner.

``Ah well, Sneaze,'' he continues, ``what do we lose if we talk? Let's
tell her everything. Now, Mrs.~Goldfield, both Sneaze and I will tell
you anything within reason about Coldmoon. But it would be more
convenient for us if you'd present your questions one at a time.''

Madam Conk was thus at last brought to see reason. And when she began to
pose her questions, her style of speech, only recently so coarsely
violent, acquired a certain civil polish, at least when she spoke to
Waverhouse. ``I understand,'' she opens, ``that Mr.~Coldmoon is a
bachelor of science. Now please tell me in what sort of subject has he
specialized?''

``In his post-graduate course, he's studying terrestrial magnetism,''
answers my master seriously.

Unfortunately, Madam Conk does not understand this answer.

Therefore, though she says, ``Ah,'' she looks dubious and asks: ``If one
studies that, could one obtain a doctor's degree?''

``Are you seriously suggesting that you wouldn't allow your daughter to
marry him unless he held a doctorate?'' The tone of my master's inquiry
discloses his deep displeasure.

``That's right. After all, if it's just a bachelor's degree, there are
so many of them!'' Madam Conk replies with complete unconcern.

My master's glance at Waverhouse reveals a deepening disgust.

``Since we cannot be sure whether or not he'll gain a doctorate, you'll
have to ask us something else.'' Waverhouse seems equally displeased.

``Is he still just studying that terrestrial something?''

``A few days ago,'' my master quite innocently offers, ``he made a
speech on the results of his investigation of the mechanics of
hanging.''

``Hanging? How dreadful! He must be peculiar. I don't suppose he could
ever become a doctor by devoting himself to hanging.''

``It would of course be difficult for him to gain a doctorate if he
actually hanged himself, but it is not impossible to become a doctor
through study of the mechanics of hanging.''

``Is that so?'' she answers, trying to read my master's expression. It's
a sad, sad thing but, since she does not know what mechanics are, she
cannot help feeling uneasy. She probably thinks that to ask the meaning
of such a trifling matter might involve her in loss of face. Like a
fortuneteller, she tries to guess the truth from facial expressions. My
master's face is glum. ``Is he studying anything else, something more
easy to understand?''

``He once wrote a treatise entitled `A Discussion of the Stability of
Acorns in Relation to the Movements of Heavenly Bodies.'\,''

``Does one really study such things as acorns at a university?''

``Not being a member of any university, I cannot answer your question
with complete certainty, but since Coldmoon is engaged in such studies,
the subject must undoubtedly be worth studying.'' With a dead-pan face,
Waverhouse makes fun of her.

Madam Conk seems to have realized that her questions about matters of
scholarship have carried her out of her depth, for she changes the
subject. ``By the way,'' she says, ``I hear that he broke two of his
front teeth when eating mushrooms during the New Year season.''

``True, and a rice-cake became fixed on the broken part.''

Waverhouse, feeling that this question is indeed up his street, suddenly
becomes light-hearted.

``How unromantic! I wonder why he doesn't use a toothpick!''

``Next time I see him, I'll pass on your sage advice,'' says my master
with a chuckle.

``If his teeth can be snapped on mushrooms, they must be in very poor
condition. What do you think?''

``One could hardly say such teeth were good. Could one, Waverhouse?''

``Of course they can't be good, but they do provide a certain humor.
It's odd that he hasn't had them filled. It really is an extraordinary
sight when a man just leaves his teeth to become mere hooks for snagging
rice-cakes.''

``Is it because he lacks the money to get them filled or because he's
just so odd that he leaves them unattended to?''

``Ah, you needn't worry. I don't suppose he will continue as Mr.~Broken
Front Tooth for any long time.'' Waverhouse is evidently regaining his
usual bouyancy.

Madam Conk again changes the subject. ``If you should have some letter
or anything which he's written, I'd like to see it.''

``I have masses of postcards from him. Please have a look at them,'' and
my master produces some thirty or forty postcards from his study.

``Oh, I don't have to look at so many of them\ldots{} perhaps two or
three would do\ldots{}''

``Let me choose some for you,'' offers Waverhouse, adding as he selects
a picture postcard, ``Here's an interesting one.''

``Gracious! So he paints pictures as well? Rather clever that,'' she
exclaims. But after examining the picture she remarks ``How very silly!
It's a badger! Why on earth does he have to paint a badger of all
things! Strange. But it does indeed look like a badger.'' She is, albeit
reluctantly, mildly impressed.

``Read what he's written beside it,'' suggests my master with a laugh.

Madam Conk begins to read aloud like a servant-girl deciphering a
newspaper.

``On New Year's Eve, as calculated under the ancient calendar, the
mountain badgers hold a garden party at which they dance excessively.

Their song says, `This evening, being New Year's Eve, no mountain hikers
will come this way.' And bom-bom-bom they thump upon their bellies. What
is he writing about? Is he not being a trifle frivolous?'' Madam Conk
seems seriously dissatisfied.

``Doesn't this heavenly maiden please you?'' Waverhouse picks out
another card on which a kind of angel in celestial raiment is depicted
as playing upon a lute.

``The nose of this heavenly maiden seems rather too small.''

``Oh no, that's about the average size for an angel. But forget the nose
for the moment and read what it says,'' urges Waverhouse.

``It says `Once upon a time there was an astronomer. One night he went
as was his wont high up into his observatory, and, as he was intently
watching the stars, a beautiful heavenly maiden appeared in the sky and
began to play some music; music too delicate ever to be heard on earth.
The astronomer was so entranced by the music that he quite forgot the
dark night's bitter cold. Next morning the dead body of the astronomer
was found covered with pure white frost. An old man, a liar, told me
that this story was all true.' What the hell is this? It makes no sense,
no nothing. Can Coldmoon really be a bachelor of science? Perhaps he
should read a few literary magazines.'' Thus mercilessly does Madam Conk
lambaste the defenseless Coldmoon.

Waverhouse for fun selects a third postcard and says, ``Well then, what
about this one?'' The card has a sailing boat printed on it and, as
usual, there is something scribbled underneath the picture.

Last night a tiny whore of sixteen summers

Declared she had no parents.

Like a plover on a reefy coast,

She wept on waking in the early morning.

Her parents, sailors both, lie at the bottom of the sea.

``Oh, that's good. How very clever! He's got real feeling,'' erupted
Madam Conk.

``Feeling?'' says Waverhouse.

``Oh yes,'' says Madam Conk. ``That would go well on a samisen.

``If it could be played on the samisen, then it's the real McCoy. Well,
how about these?'' asks Waverhouse picking out postcard after postcard.

``Thank you, but I've seen enough. For now, at least I know that
Coldmoon's not a straight-laced prude.'' She thinks she has achieved
some real understanding and appears to have no more queries about
Coldmoon, for she remarks, ``I'm sorry to have troubled you. Please do
not report my visit to Mr.~Coldmoon.'' Her request reflects her selfish
nature in that she seems to feel entitled to make a thorough
investigation of Coldmoon whilst expecting that none of her activities
should be revealed to him. Both Waverhouse and my master concede a
half-hearted ``Y-es,'' but as Madam Conk gets up to leave, she
consolidates their assent by saying, ``I shall, of course, at some later
date repay you for your services.''

The two men showed her out and, as they resumed their seats, Waverhouse
exclaimed, ``What on earth is that?'' At the very same moment my master
also ejaculated, ``Whatever's that?'' I suppose my master's wife could
not restrain her laughter any longer, for we heard her gurgling in the
inner-room.

Waverhouse thereupon addressed her in a loud voice through the sliding
door. ``That, Mrs.~Sneaze, was a remarkable specimen of all that is
conventional, of all that is `common or garden.' But when such
characteristics become developed to that incredible degree the result is
positively staggering. Such quintessence of the common approximates to
the unique. Don't seek to restrain yourself. Laugh to your heart's
content.''

With evident disgust my master speaks in tones of the deepest revulsion.
``To begin with,'' he says, ``her face is unattractive.''

Waverhouse immediately takes the cue. ``And that nose, squatting, as it
were, in the middle of that phiz, seems affectedly unreal.''

``Not only that, it's crooked.''

``Hunchbacked, one might say. A hunchbacked nose! Quite extraordinary.''
And Waverhouse laughs in genuine delight.

``It is the face of a woman who keeps her husband under her bottom.''

My master still looks resentful.

``It is a sort of physiognomy that, left unsold in the nineteenth
century, becomes in the twentieth shop-soiled.'' Waverhouse produces
another of his invariably bizarre remarks. At which juncture my master's
wife emerges from the inner-room and, being a woman and thus aware of
the ways of women, quietly warns them, ``If you talk such scandal, the
rickshaw-owner's wife will snitch on you again.''

``But, Mrs.~Sneaze, to hear such tattle will do that Goldfield woman no
end of good.''

``But it's self-demeaning to calumniate a person's face. No one sports
that sort of nose as a matter of choice. Besides, she is a woman. You're
going a little too far.'' Her defense of the nose of Madam Conk is
simultaneously an indirect defense of her own indifferent looks.

``We're not unkind at all. That creature isn't a woman. She's just an
oaf. Waverhouse, am I not right?''

``Maybe an oaf, but a formidable character nonetheless. She gave you
quite a tousling, didn't she just?''

``What does she take a teacher for, anyway?''

``She ranks a teacher on roughly the same level as a rickshaw-owner.

To earn the respect of such viragoes one needs to have at least a
doctor's degree. You were ill-advised not to have taken your doctorate.
Don't you agree, Mrs.~Sneaze?'' Waverhouse looks at her with a smile.

``A doctorate? Quite impossible.'' Even his wife despairs of my master.

``You never know. I might become one, one of these days. You mustn't
always doubt my worth. You may well be ignorant of the fact, but in
ancient times a certain Greek, Isocrates, produced major literary works
at the age of ninety-four. Similarly, Sophocles was almost a centenarian
when he shook the world with his masterpiece. Simonides was writing
wonderful poetry in his eighties. I, too\ldots{}''

``Don't be silly. How can you possibly expect, you with your stomach
troubles, to live that long.'' Mrs.~Sneaze has already determined my
master's span of life.

``How dare you! Just go and talk to Dr.~Amaki. Anyway, it's all your
fault. It's because you make me wear this crumpled black cotton surcoat
and this patched-up kimono that I am despised by women like
Mrs.~Goldfield. Very well then. From tomorrow I shall rig myself out in
such fineries as Waverhouse is wearing. So get them ready.''

``You may well say `get them ready,' but we don't possess any such
elegant clothes. Anyway, Mrs.~Goldfield only grew civil to Waverhouse
after he'd mentioned his uncle's name. Her attitude was in no way
conditioned by the ill-condition of your kimono.'' Mrs.~Sneaze has
neatly dodged the charge against her.

The mention of that uncle appears to trigger my master's memory, for he
turns to Waverhouse and says, ``That was the first I ever heard of your
uncle. You never spoke of him before. Does he, in fact, exist?''

Waverhouse has obviously been expecting this question, and he jumps to
answer it. ``Yes, that uncle of mine, a remarkably stubborn man. He,
too, is a survival from the nineteenth century.'' He looks at husband
and wife.

``You do say the quaintest things. Where does this uncle live?'' asks
Mr.

Sneaze with a titter.

``In Shizuoka. But he doesn't just live. He lives with a top-knot still
on his head. Can you beat it? When we suggest he should wear a hat, he
proudly answers that he has never found the weather cold enough to don
such gear. And when we hint that he might be wise to stay abed when the
weather's freezing, he replies that four hour's sleep is sufficient for
any man. He is convinced that to sleep more than four hours is sheer
extravagance, so he gets up while it's still pitch-dark. It is his boast
that it took many long years of training so to minimize his sleeping
hours. `When I was young,' he says, `it was indeed hard because I felt
sleepy, but recently I have at last achieved that wonderful condition
where I can sleep or wake, anywhere, anytime, just as I happen to wish.'
It is of course natural that a man of sixty-seven should need less
sleep. It has nothing to do with early training, but my uncle is happy
in the belief that he has succeeded in attaining his present condition
entirely as a result of rigorous self-discipline. And when he goes out,
he always carries an iron fan.''

``Whatever for?'' asks my master.

``I haven't the faintest idea. He just carries it. Perhaps he prefers a
fan to a walking stick. As a matter of fact an odd thing happened only
the other day.'' Waverhouse speaks to Mrs.~Sneaze.

``Ah yes?'' she noncommittally responds.

``In the spring this year he wrote to me out of the blue with a request
that I should send him a bowler hat and a frock-coat. I was somewhat
surprised and wrote back asking for further clarification. I received an
answer stating that the old man himself intended to wear both items on
the occasion of the Shizuoka celebration of the war victory, and that I
should therefore send them quickly. It was an order. But the quaintness
of his letter was that it enjoined me `to choose a hat of suitable size
and, as for the suit, to go and order one from Daimaru of whatever size
you think appropriate.'\,''

``Can one get suits made at Daimaru?''

``No.~I think he'd got confused and meant to say at Shirokiya's.''

``Isn't it a little unhelpful to say `of whatever size you think
appropriate'?''

``That's just my uncle all over.''

``What did you do?''

``What could I do? I ordered a suit which I thought appropriate and sent
it to him.''

``How very irresponsible! And did it fit?''

``More or less, I think. For I later noticed in my home-town newspaper
that the venerable Mr.~Makiyama had created something of a sensation by
appearing at the said celebration in a frock coat carrying, as usual,
his famous iron fan.''

``It seems difficult to part him from that object.''

``When he's buried, I shall ensure that the fan is placed within the
coffin.''

``Still it was fortunate that the coat and bowler fitted him.''

``But they didn't. Just when I was congratulating myself that everything
had gone off smoothly, a parcel came from Shizuoka. I opened it
expecting some token of his gratitude, but it proved only to contain the
bowler. An accompanying letter stated, `Though you have taken the
trouble of making this purchase for me, I find the hat too large. Please
be so kind as to take it back to the hatter's and have it shrunk. I will
of course defray your consequent expenses by postal order.'\,''

``Peculiar, one must admit.'' My master seems greatly pleased to
discover that there is someone even more peculiar than himself. ``So
what did you do?'' he asks.

``What did I do? I could do nothing. I'm wearing the hat myself.''

``And is that the very hat?'' says my master with a smirk.

``And he's a Baron?'' asks my master's wife from her mystification.

``Is who?''

``Your uncle with the iron fan.''

``Oh, no. He's a scholar of the Chinese classics. When he was young he
studied at that shrine dedicated to Confucius in Yushima and became so
absorbed in the teachings of Chu-Tzu that, most reverentially, he
continues to wear a top-knot in these days of the electric light.
There's nothing one can do about it.'' Waverhouse rubs his chin.

``But I have the impression that in speaking just now to that awful
woman you mentioned a Baron Makiyama.''

``Indeed you did. I heard you quite distinctly, even in the other
room.''

Mrs.~Sneaze for once supports her husband.

``Oh, did I?'' Waverhouse permits himself a snigger. ``Fancy that. Well,
it wasn't true. Had I a Baron for an uncle I would by now be a senior
civil servant.'' Waverhouse is not in the least embarrassed.

``I thought it was somehow queer,'' says my master with an expression
half-pleased, half-worried.

``It's astonishing how calmly you can lie. I must say you're a past
master at the game.'' Mrs.~Sneaze is deeply impressed.

``You flatter me. That woman quite outclasses me.''

``I don't think she could match you.''

``But, Mrs.~Sneaze, my lies are merely tarrydiddles. That woman's lies,
every one of them, have hooks inside them. They're tricky lies. Lies
loaded with malice aforethought. They are the spawn of craftiness.
Please never confuse such calculated monkey-minded wickedness with my
heaven-sent taste for the comicality of things. Should such confusion
prevail, the God of Comedy would have no choice but to weep for
mankind's lack of perspicacity.''

``I wonder,'' says my master, lowering his eyes, while Mrs.~Sneaze,
still laughing, remarks that it all comes down to the same thing in the
end.

Up until now I have never so much as crossed the road to investigate the
block opposite. I have never clapped eyes on the Goldfield's corner
residence so I naturally have no idea what it looks like. Indeed today
is the first time that I've even heard of its existence. No one in this
house has ever previously talked about a businessman and consequently I,
who am my master's cat, have shared his total disinterest in the world
of business and his equally total indifference to businessmen. However,
having just been present during the colloquy with Madam Conk, having
overheard her talk, having imagined her daughter's beauty and charm, and
also having given some thought to that family's wealth and power, I have
come to realize that, though no more than a cat, I should not idle all
my days away lying on the veranda. Nor only that, I cannot help but feel
deep sympathy with Coldmoon. His opponent has already bribed a doctor's
wife, bribed the wife of the rickshaw-owner, bribed even that
high-falutin mistress of the two-stringed harp. She has so spied upon
poor Coldmoon that even his broken teeth have been disclosed, while he
has done no more than fiddle with the fastenings of his surcoat and, on
occasion, grin. He is guileless even for a bachelor of science just out
of the university. And it's not just anyone who can cope with a woman
equipped with such a jut of nose. My master not only lacks the heart for
dealing with matters of this sort, but he lacks the money, too.

Waverhouse has sufficient money, but is such an inconsequential being
that he'd never go out of his way merely to help Coldmoon. How isolated,
then, is that unfortunate person who lectures on the mechanics of
hanging. It would be less than fair if I failed at least to try and
insinuate myself into the enemy fortress and, for Coldmoon's sake, pick
up news of their activities. Though but a cat, I am not quite as other
cats. I differ from the general run of idiot cats and stupid cats. I am
a cat that lodges in the house of a scholar who, having read it, can
bang down any book by Epictetus on his desk. Concentrated in the tip of
my tail there is sufficient of the spirit of chivalry for me to take it
upon myself to venture upon knight-errantry. It is not that I am in any
way beholden to poor Coldmoon, nor am I engaging in foolhardy action for
the sake of any single individual. If I may be allowed to blow my own
trumpet, I am proposing to take magnificent unself-interested action
simply in order to realize the will of Heaven that smiles upon
impartiality and blesses the happy medium. Since Madam Conk makes
impermissible use of such things as the happenings at Azuma Bridge;
since she hires underlings to spy and eavesdrop on us; since she
triumphantly retails to all and sundry the products of her espionage;
since by the employment of rickshaw-folk, mere grooms, plain rogues,
student riff-raff, crone daily-help, midwives, witches, masseurs, and
other trouble-makers she seeks to trouble a man of talent; for all these
reasons even a cat must do what can be done to prevent her getting away
with it.

The weather, fortunately, is fine. The thaw is something of nuisance,
but one must be prepared to sacrifice one's life in the cause of
justice. If my feet get muddy and stamp plum blossom patterns on the
veranda, O-san may be narked but that won't worry me. For I have come to
the superlatively courageous, firm decision that I will not put off
until tomorrow what needs to be done today. Accordingly, I whisk off
around to the kitchen, but, having arrived there, pause for further
thought.

``Softly, softly,'' I say to myself. It's not simply that I've attained
the highest degree of evolution that can occur in cats, but I make bold
to believe my brain is as well-developed as that of any boy in his third
year at a middle school. Nevertheless, alas, the construction of my
throat is still only that of a cat, and I cannot therefore speak the
babbles of mankind. Thus, even if I succeed in sneaking into the
Goldfield's citadel and there discovering matters of moment, I shall
remain unable to communicate my discoveries to that Coldmoon who so
needs them. Neither shall I be able to communicate my gleanings to my
master or to Waverhouse. Such incommunicable knowledge would, like a
buried diamond, be denied its brilliance and my hard-won wisdom would
all be won for nothing.

Which would be stupid. Perhaps I should scrap my plan. So thinking, I
hesitated on the very doorstep.

But to abandon a project halfway through breeds a kind of regret, that
sense of unfulfillment which one feels when the slower one had so
confidently expected drifts away under inky clouds into some other part
of the countryside. Of course, to persist when one is in the wrong is an
altogether different matter, but to press on for the sake of so-called
justice and humanity, even at the risk of death uncrowned by success,
that, for a man who knows his duty, can be a source of the deepest
satisfaction. Accordingly, to engage in fruitless effort and to muddy
one's paws on a fool's errand would seem about right for a cat. Since it
is my misfortune to have been born a cat, I cannot by turns of the tip
of my tail convey, as I can to cats, my thinking to such scholars as
Coldmoon, Sneaze, and Waverhouse. However, by virtue of felinity, I can,
better than all such bookmen, make myself invisible. To do what no one
else can do is, of itself, delightful. That I alone should know the
inner workings of the Goldfield household is better than if nobody
should know.

Though I cannot pass my knowledge on, it is still cause for delight that
I may make the Goldfields conscious that someone knows their secrets.

In the light of this succession of delights, I boldly make to believe my
brain is as delightful as well. All right then. I will go.

Coming to the side street in the opposite block, there, sure enough, I
find a Western-style house dominating the crossroads as if it owned the
whole area. Thinking that the master of such a house must be no less
stuck up than his building, I slide past the gate and examine the
edifice. Its construction has no merit. Its two stories rear up into the
air for no purpose whatever but to impress, even to coerce, the
passersby. This, I suppose, is what Waverhouse means when he calls
things common or garden. I slink through some bushes, take note of the
main entrance to my right, and so find my way round to the kitchen. As
might be expected, the kitchen is large---at least ten times as large as
that in my master's dwelling.

Everything is in such apple pie order, all so clean and shining, that it
cannot be less splendid than that fabulous kitchen of Count Okuma so
ful-somely described in a recent product of the national press. I tell
myself, as I slip inside on silent muddy paws, that this must be ``a
model kitchen.''

On the plastered part of its floor the wife of the rickshaw-owner is
standing in earnest discussion with a kitchen-maid and a
rickshaw-runner.

Realizing the dangers of this situation, I hide behind a water-tub.

``That teacher, doesn't he really even know our master's name?'' the
kitchen-maid demands.

``Of course he knows it. Anyone in this district who doesn't know the
Goldfield residence must be a deaf cripple without eyes,'' snaps the man
who pulls the Goldfield's private rickshaw.

``Well, you never know. That teacher's one of those cranks who know
nothing at all except what it says in books. If he knew even the least
little thing about Mr.~Goldfield he might be scared out of his wits. But
he hasn't the wits to be scared out of. Why,'' snorts Blacky's
bloody-minded mistress, ``he doesn't even know the ages of his own
mis-managed children.''

``So he's not afraid of our Mr.~Goldfield! What a cussed clot he is!
There's no call to show him the least consideration. Let's go around and
give him something to be scared about.''

``Good idea! He says such dreadful things. He was telling his crackpot
cronies that, since Madam's nose is far too big for her face, he finds
her unattractive. No doubt he thinks himself a proper picture, but his
mug's the spitting image of a terra-cotta badger. What can be done, I
ask you, with such an animal?''

``And it isn't only his face. The way he saunters down to the public
bathhouse carrying a hand towel is far too high and mighty. He thinks
he's the cat's whiskers.'' My master Sneaze seems notably unpopular,
even with this kitchen-maid.

``Let's all go and call him names as loud as we can from just outside
his hedge.''

``That'll bring him down a peg.''

``But we mustn't let ourselves be seen. We must spoil his studying just
with shouting, getting him riled as much as we can. Those are Madam's
latest orders.''

``I know all that,'' says the rickshaw wife in a voice that makes it
clear that she's only too ready to undertake one-third of their
scurrilous assignment. Thinking to myself, ``So that's the gang who're
going to ridicule my master,'' I drift quietly past the noisesome trio
and penetrate yet further into the enemy fortress.

Cat's paws are as if they do not exist. Wheresoever they may go, they
never make clumsy noises. Cats walk as if on air, as if they trod the
clouds, as quietly as a stone going light-tapped under water, as an
ancient Chinese harp touched in a sunken cave. The walking of a cat is
the instinctive realization of all that is most delicate. For such as I
am concerned, this vulgar Western house simply is not there. Nor do I
take cognizance of the rickshaw-woman, manservant, kitchen-maids, the
daughter of the house, Madam Conk, her parlor-maids or even her ghastly
husband. For me they do not exist. I go where I like and I listen to
whatever talk it interests me to hear. Thereafter, sticking out my
tongue and frisking my tail, I walk home self-composedly with my
whiskers proudly stiff. In this particular field of endeavor there's not
a cat in all Japan so gifted as am I. Indeed, I sometimes think I really
must be blood-kin to that monster cat one sees in ancient picture books.
They say that every toad carries in its forehead a gem that in the
darkness utters light, but packed within my tail I carry not only the
power of God, Buddha, Confucius, Love, and even Death, but also an
infallible panacea for all ills that could bewitch the entire human
race. I can as easily move unnoticed through the corridors of
Goldfield's awful mansion as a giant god of stone could squash a
milk-blancmange.

At this point, I become so impressed by my own powers and so conscious
of the reverence I consequently owe to my own most precious tail that I
feel unable to withhold immediate recognition of its divinity. I desire
to pray for success in war by worshiping my honored Great Tail Gracious
Deity, so I lower my head a little, only to find I am not facing in the
right direction. When I make the three appropriate obeisances I should,
of course, as far as it is possible, be facing toward my tail. But as I
turn my body to fulfill that requirement, my tail moves away from me.

In an effort to catch up with myself, I twist my neck. But still my tail
eludes me. Being a thing so sacred, containing as it does the entire
universe in its three-inch length, my tail is inevitably beyond my power
to control. I spun round in pursuit of it seven and a half times but,
feeling quite exhausted, I finally gave up. I feel a trifle giddy. For a
moment I lose all sense of where I am and, deciding that my whereabouts
are totally unimportant, I start to walk about at random. Then I hear
the voice of Madam Conk. It comes from the far side of a paper-window.
My ears prick up in sharp diagonals and, once more fully alert, I hold
my breath.

This is the place which I set out to find.

``He's far too cocky for a penny-pinching usher,'' she's screaming in
that parrot's voice.

``Sure, he's a cocky fellow. I'll have a bit of the bounce taken out of
him, just to teach him a lesson. There are one or two fellows I know,
fellows from my own province, teaching at his school.''

``What fellows are those?''

``Well, there's Tsuki Pinsuke and Fukuchi Kishago for a start. I'll
arrange with them for him to be ragged in class.''

I don't know from what province old man Goldfield comes, but I'm rather
surprised to find it stiff with such outlandish names.

``Is he a teacher of English?'' her husband asks.

``Yes. According to the wife of the rickshaw-owner, his teaching
specializes in an English Reader or something like that.''

``In any case, he's gotta be a rotten teacher.''

I'm also struck by the vulgarity of that ``gotta be'' phraseology.

``When I saw Pinsuke the other day he mentioned that there was some
crackpot at his school. When asked the English word for bancha, this
fathead answered that the English called it, not `coarse tea' as they
actually do, but `savage tea.' He's now the laughing stock of all his
teaching colleagues. Pinsuke added that all the other teachers suffer
for this one's follies. Very likely it's the self-same loon.''

``It's bound to be. He's got the face you'd expect on a fool who thinks
that tea can be savage. And to think he has the nerve to sport such a
dashing mustache!''

``Saucy bastard.''

If whiskers establish sauciness, every cat is impudent.

``As for that man Waverhouse---Staggering Drunk I'd call him---he's an
obstreperous freak if ever I saw one. Baron Makiyama, his uncle indeed!
I was sure that no one with a face like his could have a baron for an
uncle.''

``You, too, are at fault for believing anything which a man of such
dubious origins might say.''

``Maybe I was at fault. But really there's a limit and he's gone much
too far.'' Madam Conk sounds singularly vexed. The odd thing is that
neither mentions Coldmoon. I wonder if they concluded their discussion
about him before I sneaked up on them or whether perhaps they had
earlier decided to block his marriage suit and had therefore already
forgotten all about him. I remain disturbed about this question, but
there's nothing I can do about it. For a little while I lay crouched
down in silence but then I heard a bell ring at the far end of the
corridor. What's up down there? Determined this time not to be late on
the scene, I set out smartly in the direction of the sound.

I arrived to find some female yattering away by herself in a loud
unpleasant voice. Since her tones resemble those of Madam Conk, I deduce
that this must be that darling daughter, that delicious charmer for
whose sake Coldmoon has already risked death by drowning.

Unfortunately, the paper-windows between us make it impossible for me to
observe her beauty and I cannot therefore be sure whether she, too, has
a massive nose plonked down in the center of her face. But I infer from
her mannerisms, such as the way she sounds to be turning up her nose
when she talks, that that organ is unlikely to be an inconspicuous
pug-nose. Though she talks continuously, nobody seems to be answering,
and I deduce that she must be using one of those modern telephones.

``Is that the Yamato? I want to reserve, for tomorrow, the third box in
the lower gallery. All right? Got it? What's that? You can't? But you
must. Why should I be joking? Don't be such a fool. Who the devil are
you? Chōkichi? Well, Chōkichi, you're not doing very well. Ask the
proprietress to come to the phone. What's that? Did you say you were
able to cope with any possible inquiries? How dare you speak to me like
that? D'you know who I am? This is Miss Goldfield speaking. Oh, you're
well aware of that, are you? You really are a fathead. Don't you
understand, this is the Goldfield. Again? You thank us for being regular
patrons? I don't want your stupid thanks. I want the third box in the
lower gallery. Don't laugh, you idiot. You must be terribly stupid. You
are, you say? If you don't stop being insolent, I shall just ring off.
You understand? I can promise you you'll be sorry. Hello. Are you still
there? Hello, hello. Speak up. Answer me. Hello, hello, hello.''
Chōkichi seems to have hung up, for no answer is forthcoming. The girl
is now in something of a tizzy and she grinds away at the telephone
handle as though she's gone off her head. A lapdog somewhere around her
feet suddenly starts to yap, and, realizing I'd better keep my wits
about me, I quickly hop off the veranda and creep in under the house.

Just then I hear approaching footsteps and the sound of a paper-door
being slid aside. I tilt my head to listen.

``Your father and mother are asking for you, Miss.'' It sounds like a
parlor-maid.

``So who cares?'' was the vulgar answer.

``They sent me to fetch you because they've something they want to tell
you.''

``You're being a nuisance. I said I just don't care.'' She snubs the
maid once more.

``They said it's something to do with Mr.~Coldmoon.'' The maid tries
tactfully to put this young vixen into a better humor.

``I couldn't care less if they want to talk about Coldmoon or
Piddlemoon. I abominate that man with his daft face looking like a
bewildered gourd.'' Her third sour outburst is directed at the absent
Coldmoon. ``Hello,'' she suddenly goes on, ``when did you start dressing
your hair in the Western style?''

The parlor-maid gulps and then replies as briefly as she can ``Today.''

``What sauce. A mere parlor-maid, what's more.'' Her fourth attack comes
in from a different direction. ``And isn't that a brand new collar
you've got on?''

``Yes, it's the one you gave me recently. I've been keeping it in my box
because it seemed too good for the likes of me, but my other collar
became so grubby I thought I'd make the change.''

``When did I give it you?''

``It was January you bought it. At Shirokiya's. It's got the ranks of
sumō wrestlers set out as decoration on the greeny-brown material. You
said it was too somber for your style. So you gave it me.''

``Did I? Well, it certainly looks nice on you. How very provoking!''

``I'm much obliged!''

``I didn't intend a compliment. I'm very much put out.''

``Yes, Miss.''

``Why did you accept something which so very much becomes you without
letting me know that it would?''

``But Miss\ldots{}''

``Since it looks that nice on you, it couldn't fail, could it, to look
more nice on me?''

``I'm sure it would have looked delightful on you.''

``Then why didn't you say so? Instead of that, you just stand there
wearing it when you know I'd like it back. You little beast.'' Her
vituperations seem to have no end. I was wondering what would happen
when, from the room at the other end of the house, old man Goldfield
himself suddenly roared out for his daughter. ``Opula,'' he bellowed.

``Opula, come here.'' She had no choice but to obey and mooched sulkily
out of the room containing the telephone. Her lapdog, slightly bigger
than myself with its eyes and mouth all bunched together in the middle
of its revolting mug, slopped along behind her.

Thereupon, with my usual stealthy steps, I tiptoed back to the kitchen
and, through the kitchen-door, found my way to the street, and so back
home. My expedition has been notably successful.

Coming thus suddenly from a beautiful mansion to our dirty little
dwelling, I felt as though I had descended from a sunlit mountaintop to
some dark dismal grot. Whilst on my spying mission. I'd been far too
busy to take any notice of the ornaments in the rooms, of the decoration
of the sliding-doors and paper-windows or of any similar features, but
as soon as I returned and became conscious of the shabbiness of home, I
found myself yearning for what Waverhouse claims to despise. I am
inclined to think that, after all, there's a good deal more to a
businessman than there is to a teacher. Uncertain of the soundness of
this thinking, I consult my infallible tail. The oracle confirms that my
thinking is correct.

I am surprised to find Waverhouse still sitting in my master's room.

His cigarette stubs, stuffed into the brazier, make it look like a
beehive.

Comfortably cross-legged on the floor, he is, as usual, talking. It
appears, moreover, that during my absence Coldmoon has dropped in.

My master, his head pillowed on his arms, lies flat on his back rapt in
contemplation of the pattern of the rainmarks on the ceiling. It is
another of those meetings of hermits in a peaceful reign.

``Coldmoon, my dear fellow, I seem to remember that you insisted upon
maintaining as the darkest of dark secrets the name of that young lady
who called your name from the depths of her delirium. But surely the
time has now come when you could reveal her identity?''

Waverhouse begins to niggle Coldmoon.

``Were it just solely my concern, I wouldn't mind telling you, but since
any such disclosure might compromise the other party\ldots{}''

``So you still won't tell?''

``Besides, I promised the Doctor's wife\ldots{}''

``Promised never to tell anyone?''

``Yes,'' says Coldmoon back at his usual fiddling with the strings of
his surcoat. The strings are a bright purple, objects of a color one
could never nowadays find in any shop.

``The color of those strings is early nineteenth century'' remarks my
supine master. He is genuinely quite indifferent to anything that
concerns the Goldfields.

``Quite. It couldn't possibly belong to these times of the
Russo-Japanese War. That kind of string would be appropriate only to the
garments worn by the rank and file of soldiers under the Shogunate. It
is said that on the occasion of his marriage, nearly four hundred years
ago, Oda Nobunaga dressed his hair back in the fashion of a tea whisk,
and I have no doubt his projecting top-knot was bound with precisely
such a string.'' Waverhouse goes, as usual, all around the houses to
make his little point.

``As a matter of fact, my grandfather wore these strings at the time,
not forty years back, when the Tokugawa were putting down the last
rebellion before the restoration of the Emperor.'' Coldmoon takes it all
dead seriously.

``Isn't it then about time you presented those strings to a museum? For
that well-known lecturer on the mechanics of hanging, that leading
bachelor of science, Mr.~Avalon Coldmoon to go around looking like a
relic of mediaevalism would scarcely help his reputation.''

``I myself would be only too ready to follow your advice. However,
there's a certain person who says that these strings do specially become
me\ldots{}''

``Who on earth could have made such an imperceptive comment?''

asks my master in a loud voice as he rolls over onto his side.

``A person not of your acquaintance.''

``Never mind that. Who was it?''

``A certain lady.''

``Gracious me, what delicacy! Shall I guess who it is? I think it's the
lady who whimpered for you from the bottom of the Sumida River. Why
don't you tie up your surcoat with those nice purple strings and go on
out and get drowned again?'' Waverhouse offers a helpful suggestion.

Coldmoon laughs at the sally. ``As a matter of fact she no longer calls
me from the riverbed. She is now, as it were, in the Pure Land, a little
northwest from here\ldots{}''

``Don't hope for too much purity. That ghastly nose looks singularly
unwholesome.''

``Eh?'' says Coldmoon, looking puzzled.

``The Archnose from over the way has just been round to see us. Yes,
right here. I can tell you we had quite a surprise. Hadn't we, Sneaze?''

``We had,'' replies my master still lying on his side but now sipping
tea.

``Whom do you mean by the Archnose?''

``We mean the honorable mother of your ever-darling lady.''

``Oh!''

``A woman calling herself Mrs.~Goldfield came round here asking all
sorts of questions about you.'' My master, clarifying the situation,
speaks quite seriously.

I watch poor Coldmoon, wondering if he will be surprised or pleased or
embarrassed, but in fact he looks exactly as he always does. And in his
accustomed quiet tones he comments ``I suppose she's asking if I'll
marry the daughter? Was that it?'' and he goes on twisting and
untwisting his purple strings.

``Far from it! That mother happens to own the most enormous
nose\ldots{}'' But before Waverhouse could finish his sentence my master
interrupted him with a sudden irrelevance.

``Listen,'' he chirps, ``I've been trying to compose a new-style haiku
on that snout of hers.'' Mrs.~Sneaze begins to giggle in the next room.

``You're taking it all extremely lightly! And have you composed your
poem?''

``I've made a start. The first line goes `A Conker Festival takes place
in this face.'\,''

``And then?''

``\,`At which one offers sacred wine.'\,''

``And the concluding line?''

``I've not yet got to that.''

``Interesting,'' says Coldmoon with a grin.

``How about this for the missing line?'' improvises Waverhouse. ``\,`Two
orifices dim.'\,''

Whereupon Coldmoon offers, ``\,`So deep no hairs appear.'\,''

They were thus thoroughly enjoying themselves by proposing wilder and
wilder lines when from the street beyond the hedge came the voices of
several people shouting ``Where's that terra-cotta badger? Come on out,
you terra-cotta badger. Terra-cotta badger! Yah!''

Both my master and Waverhouse look somewhat startled and they peer out
through the hedge. Loud hoots of derisive laughter are followed by the
sound of footsteps running away.

``Whatever can they mean by a terra-cotta badger?'' Waverhouse asks in
puzzled tones.

``I've no idea,'' replies my master.

``An unusual occurrence,'' says Coldmoon.

Waverhouse suddenly gets to his feet as if he had remembered something.
``For some several lustra,'' he declaims in parody of the style of
public lecturers, ``I have devoted myself to the study of aesthetic
nasofrontology and I would accordingly now like to trespass on your time
and patience in order to present certain interim conclusions at which I
have arrived.'' His initiative has been so suddenly taken that my master
just stares up at him in silent blank amazement.

Coldmoon's tiniest voice observes, ``I'd love to hear your interim
conclusions.''

``Though I have made a thorough study of this matter, the origin of the
nose remains, alas, still deeply obfuscated. The first question that
arises reflects the assumption that the nose is intended for use. The
functional approach. If that premise is valid, would not two mere
vent-holes meet the case? There is no obvious need either for such
arrogant profusion or for the nasal arrogation of a median position in
the human physiognomy. Why then should the nasal organ thus,'' and he
paused to pinch his own, ``thrust itself forward?''

``Yours doesn't stick out much,'' cuts in my master rather rudely.

``At any rate it has no indentations, no incurvations; still less could
it be described as countersunk or infundibular. I draw your attention to
these facts because if you fail to make the necessary distinction
between having two holes in the medio-frontal area of the face and
having two such holes in some form of protuberance, you will inevitably
be unable to follow the quintessential drift of my dissertation. Now, it
is at least my own, albeit humble, opinion that it is by an accumulation
of human actions trifling in themselves, for who could attach major
importance to the blowing of one's nose, that the organ in question has
developed into its present phenomenal form.''

``How very humbly you do hold your humble views,'' interjects my master.

``As you will know, the act of blowing the nose involves the coarctation
of that organ. Such stenosis of the nose, such astrictive and, one might
even venture to say, pleonastic stimulation of so localized an area
results, by response to that stimulus and in accordance with the
well-established principles of Lamarckian evolutionary theory, in the
development of that specific area to a degree disproportionate to the
development of other areas. The epidermis of the affected area
inevitably indurates and the subcutaneous material so coagulates as
eventually to ossify.''

``That's a bit extreme. Surely you can't turn flesh to bone just by
blowing your nose.'' Coldmoon, as behoves a bachelor of science, lodges
a protest. Waverhouse continues to deliver his speech with the utmost
nonchalance.

``I can well appreciate your natural dubieties, but the proof of the
pudding is the eating. For, behold, there is bone there, and that bone
has demonstrably been molded. Nevertheless, and despite that bone, one
snivels. If one snivels one has to blow the nose, and in the course of
that action both sides of the bone get worn away until the nose itself
acquires the shape of a high and narrow bulge. It is indeed a terrifying
process. But just as little taps of dropping water will eventually bore
through granite, so has the high, straight ridge of the nasal organ been
smithied by incessant nose-blows. Thus painfully was fangled the hard
straight line on one's face.''

``But yours is flabby.''

``I deliberately refrain from any discussion of this particular feature
as it may be observed in the physiognomy of the lecturer himself; for
such a purely personal approach involves the dangers of
self-exculpation, the temptation to gloss over, even to defend, one's
individual defects or deficiencies. But the nose of the honorable
Mrs.~Goldfield is such that I would wish to bring it to your attention
as the most highly developed of its kind, the most egregiously rare
object, in the world.''

Coldmoon cries out in spontaneous admiration. ``Hear, hear.''

``But anything whatever that develops to an extreme degree becomes
thereby intimidating. Even terrifying. Spectacular it may be, but
simultaneously awesome, unapproachable. Thus the bridge of that lady's
nose, though certainly magnificent, appears to me unduly rigid,
unacceptably steep. If one pauses to consider the nature of the noses of
the ancients, it seems probable that those of Socrates, Oliver
Goldsmith, and William Thackeray were strikingly imperfect from the
structural point of view, but those very imperfections had their own
peculiar charms. This is, no doubt, the intellection behind the saying
that a nose, like a mountain, is not significant because it is high but
because it is odd. Similarly, the popular catch-phrase that `dumplings
are better than nosegays' is no doubt a corruption of some yet more
ancient adage to the effect that dumplings are better than noses. From
which it follows that, viewed aesthetically, the nose of Citizen
Waverhouse is just about right.''

Coldmoon and my master greet this fantastication with peals of
appreciative laughter, and even Waverhouse joins in.

``Now, the piece I have just been reciting\ldots{}''

``Distinguished speaker, I must object to your use of the phrase
`reciting a piece': a somewhat vulgar word one would only expect from a
storyteller.'' Coldmoon, catching Waverhouse in the use of language
which only recently Waverhouse had criticized, feels himself revenged.

``In which case, sir, and having with your gracious permission purged
myself of error, I would now like to touch upon the matter of the proper
proportion between the nose and its associated face. If I were simply to
discuss noses in disregard of their relation to other entities, then I
would declare without fear of contradiction that the nose of Mrs.

Goldfield is superb, superlative, and, though possibly supervacaneous,
one well-placed to win first prize at any exhibition of nasal
development which might be organized by the long-nosed goblins on Mount
Kurama.

But alas! And even alack! That nose appears to have been formed,
fashioned, dare I say fabricated, without any regard for the
configuration of such other major items as the eyes and mouth. Julius
Caesar was undoubtedly dowered with a very fine nose. But what do you
think would be the result if one scissored off that Julian beak and
fixed it on the face of this cat here? Cats' foreheads are proverbially
diminutive. To raise the tower of Caesar's boned proboscis on such a
tiny site would be like plonking down on a chessboard the giant image of
Buddha now to be seen at Nara. The juxtaposing of disproportionate
elements destroys aesthetic value. Mrs.~Goldfield's nose, like that of
Caesar's, is, as a thing in itself, a most dignified and majestic
protuberance. But how does it appear in relation to its surroundings? Of
course those circumjacent areas are not quite so barren of aesthetic
merit as the face of this cat.

Nevertheless, it is a bloated face, the face of an epileptic skivvy
whose eyebrows meet in a sharp-pitched gable above thin tilted eyes.

Gentlemen, I ask you, what sort of nose could ever survive so lamentable
a face?''

As Waverhouse paused, a voice could be heard from the back of the house.
``He's still going on about noses. What a spiteful bore he is.''

``That's the wife of the rickshaw-owner,'' my master explains to
Waverhouse.

Waverhouse resumes. ``It is a great, if unexpected, honor for this
present lecturer to discover at, as it were, the back of the hall an
interested listener of the gentle sex. I am especially gratified that a
gleam of charm should be added to my arid lecture by the bell-sweet
voice of this new participant. It is, indeed, a happiness unlooked for,
a serendipity. To be worthy of our beautiful lady's patronage I would
gladly alter the academic style of this discourse into a more popular
mode, but, as I am just about to discuss a problem in mechanics, the
unavoidably technical terminology may prove a trifle difficult for the
ladies to comprehend. I must therefore beg them to be patient.''

Coldmoon responds to the mention of mechanics with his usual grin.

``The point I wish to establish is that such a nose and such a face will
never harmonize. In brief, they cannot conform to Zeising's rule of the
Golden Section, a fact which I propose to prove by use of a mechanical
formula. We should first designate H as the height of the nose, and α as
the angle between the nose and the level surface of the face. Please
note that W is, of course, the weight of the nose. Are you with me thus
far?''

``Hardly,'' breathes my master.

``Coldmoon, what about you?''

``I, too, am slightly at a loss.''

``You distress me, Coldmoon. Sneaze doesn't matter, but I'm shocked that
you, a bachelor of science, should fail to understand. This formula is a
key part of my lecture. To abandon this portion of my argument must
render the whole endeavor pointless. However, such things can't be
helped. I'll omit the formula and merely deliver the peroration.''

``Is there a peroration?'' asks my master in genuine curiosity.

``Why, naturally! A lecture without a peroration is like a Western
dinner shorn of the dessert. Now, listen, both of you, carefully. I am
launching on my peroration. Gentlemen, if one reflects upon the theory
which I have advanced on this occasion and gives due weight to the
related theories of Virchow and of Wisemen, one is bound also to take
appropriate account of the problem of the heredity of congenital form.
Furthermore, though there is a substantial body of evidence to support
the contention that acquired characteristics are not hereditarily
transmissible, one cannot lightly dismiss the view that the mental
conditions associated with hereditarily transmissible forms are
themselves also transmissible. It is consequently reasonable to assume
that a child born to the possessor of a nose of such enormity will have
an abnormal nose. Because Coldmoon is still young, he has not noticed
any particular abnormality in the structure of Miss Goldfield's nasal
organ. But the genes lurk. The products of heredity take long to
incubate. One never knows. Perhaps it would need no more than a sharp
change of climate for the daughter's snout suddenly to germinate and, in
a mere instant, to tumesce into a replica of that of her most honorable
mother. In sum, I believe that in the light of my theoretical
demonstration, it would seem prudent to forswear any idea of this
marriage. Now, while it is still possible to do so. I would go so far as
to claim that, quite apart from the master of this house, even his
monstrous cat asleep among us, would not dissent my conclusions.''

My master sits up at last. ``Of course,'' he says ``no one in his senses
would ever marry a daughter of that creature! Really, my dear
Coldmoon,'' he insists in real earnest, ``you simply must not marry
her.''

I seek in my own humble way to second all these sentiments by mewing
twice. Coldmoon, however, does not seem to be particularly alarmed. ``If
you two sages share that opinion, I would be prepared to give her up,
but it would be cruel if the consequent distress brought the person in
question into poor health.''

``That,'' burbled Waverhouse happily, ``might even be regarded as a sort
of sex crime.''

Only my master continues to take the matter seriously. ``Don't joke
about such things. That girl wouldn't wither away, not if she's the
daughter of that forward and presumptuous creature who strove to
humiliate me from the moment she set an uninvited foot in my house.'' My
master again works himself up into a great huff.

At which point there is a further outbreak of laughter from, by the
sound of it, three or four people on the far side of the hedge. A voice
says, ``You're a stuck-up blockhead.'' Another jeers, ``I bet you'd like
to live in a bigger house.'' A third loud voice announces, ``Ain't it a
pity! You swagger around but you're only a silly old windbag.''

My master goes out on to the veranda and shouts with matching violence,
``Hold your tongues. What do you think you're doing making this sort of
disturbance so close to my property?''

The laughter gets even louder. ``Hark at him. It's silly old Savage Tea.
Savage Tea. Savage Tea.'' They set up an abusive chant.

My master, looking furious, turns abruptly, snatches up his stick and
rushes out into the street.

Waverhouse claps his hands in pure delight. ``Up guards and at 'em'' he
shouts, urging my master on.

Coldmoon sits and grins, twisting his purple fastening-strings.

I follow my master and, as I crawl out through a gap in the hedge, find
him standing in the middle of the street with his stick held awkwardly
in his hand. Apart from him, the street is empty. I cannot help but feel
that he's been made to make a ninny of himself.
\end{document}
